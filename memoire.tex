\documentclass[hidelinks,12pt,a4paper]{report}
\usepackage[T1]{fontenc}
\usepackage[utf8]{inputenc}    %mettre utf8 à la place de latin si vous utilisez utf8
\usepackage{textcomp}
\usepackage{setspace}
\renewcommand{\baselinestretch}{1.75}
\usepackage{sectsty}
\allsectionsfont{\singlespacing}
\widowpenalty=9999
\clubpenalty=9999
\setlength{\emergencystretch}{3em}
\usepackage{csquotes}
\usepackage[style=verbose-trad1,isbn=false,language=french,backend=biber]{biblatex}
\DefineBibliographyStrings{french}{in={dans},inseries={dans}}
\AtEveryCitekey{\clearfield{pagetotal}} %AtEveryBibitem (biblio8aphie)
\AtEveryCitekey{\clearfield{pages}} %pages = quand chapitre d'un livre
\AtEveryBibitem{\clearfield{pagetotal}} %pages = quand chapitre d'un livre
\AtEveryCitekey{\clearfield{url}}
\renewcommand*{\mkibid}{\emph}
\DeclareNameAlias{sortname}{last-first}
\usepackage[cyr]{aeguill}
\usepackage{geometry}
\geometry{verbose,letterpaper,tmargin=27.5mm,bmargin=27.5mm,lmargin=27.5mm,rmargin=27.5mm}
\usepackage{epigraph}
\setlength\epigraphwidth{8cm}
\usepackage{url}
\urlstyle{same}
\usepackage{hyperref}
\usepackage[normalem]{ulem}

\newcommand{\guil}[1]{«~{#1}~»}    %guillemets
\newcommand{\guill}[1]{``{#1}''}     %guillemets dans les guillemets
\addbibresource{bibliographie/memoirelatex.bib}
\setcounter{tocdepth}{1} %(3 = jusqu'à subsubsection)
\setcounter{secnumdepth}{1}
\usepackage[french]{babel}

\DeclareFieldFormat
  [book,article,inbook,collection,incollection,thesis]
  {series}{coll. \mkbibquote{#1}}
\DeclareFieldFormat[article]{volume}{{vol.}~#1}

\usepackage{titling}
\renewcommand{\thechapter}{\Roman{chapter}}
\renewcommand{\thesection}{\Alph{section})}
\renewcommand{\thesubsection}{\roman{subsection})}
\renewcommand{\thesubsubsection}{\alph{subsubsection})}

\begin{document}
\selectlanguage{french}
\pagenumbering{gobble} %bouffe la numérotation des pages

%
\null
\newpage
\sloppy
\begin{titlepage}
  \begin{singlespace}

\begin{center}
{Université de Montréal} \vspace{1.5 cm}\\
\end{center}

\begin{center}
\Large{{\bf{Revisiter le mythe d'Alice}}\\Le personnage d'enfant-femme chez \\Raymond Queneau et Marie-Sissi Labrèche}
\end{center}

\vspace{1.5 cm}

\begin{center}
\normalsize{par Emilie Drouin}
\end{center}

\vspace{1.5 cm}

\begin{center}
Département des littératures de langue française\\
Faculté des arts et des sciences
\end{center}

\vspace{1.5 cm}

\begin{center}
\bf{Mémoire présenté en vue de l'obtention du grade \\ de maîtrise en littératures de langue française}
\end{center}
\vspace{1.5 cm}

\begin{center}
Mai 2017\\
\vspace{3 cm}
{\textcircled{c}}~Emilie Drouin, 2017
\end{center}


%\pagebreak

%\vspace{3 cm}

%\begin{center}
%  {Université de Montréal \\ Faculté des études supérieures et postdoctorales}
%\end{center}

%\vspace{3cm}

%\begin{center}
%{Ce mémoire intitulé}
%\end{center}

%\vspace{0.5 cm}

%\begin{center}
%{{\bf{Revisiter le mythe d'Alice}}\\Le personnage d'enfant-femme chez Raymond Queneau et Marie-Sissi Labrèche}
%\end{center}

%\vspace{3 cm}

%\begin{center}
%\normalsize{présenté par \\ Emilie Drouin}
%\end{center}

%\vspace{3 cm}


%\begin{center}
%{a été évalué par un jury composé des personnes suivantes:} \\
%\vspace{1 cm}
%{Marie-Pascale Huglo\\ \small{président-rapporteur}} \\
%\vspace{0.5 cm}
%{Gilles Dupuis\\ \small{directeur de recherche}} \\
%\vspace{0.5 cm}
%{Élisabeth Nardout-Lafarge\\ \small{membre du jury}}
%\end{center}

  \end{singlespace}
 \newpage
\end{titlepage}


\pagenumbering{roman}
\setcounter{page}{1}
%\addcontentsline{toc}{chapter}{Résumé}
\chapter*{Résumé}

\begin{spacing}{1.4}

Établi en littérature française depuis le XIX\up{e} siècle, l'enfant-personnage se présente sous plusieurs figures, aussi appelées \guil{typologies}.
Autonome, imperturbable, parfois même perturbant: c'est la figure de l'\guil{enfant du monde} qui semble le mieux représenter les petits personnages modernes ou contemporains.
Encore plus problématique lorsqu'il s'agit d'une petite fille, cet enfant adaptatif n'est pas sans rappeler le personnage anglais d'Alice, présenté par Lewis Carroll: plongée dans un monde insensé, la jeune fille cherche ses repères et s'adapte afin de survivre.
Nous avons choisi, afin de représenter l'héritage de cette \textit{lignée}, les personnages de Zazie (\textit{Zazie dans le métro} de Raymond Queneau) et de Sissi (\textit{Borderline} de Marie-Sissi Labrèche).
\par
Dans une perspective théorique relevant de la sémiologie du personnage de Philippe Hamon, nous avons dégagé la figure de l'\guil{enfant-femme}, qui constitue une façon de revisiter le mythe d'Alice.
Poussée à l'autonomisation par son contexte d'existence, l'enfant-femme adopte un comportement générateur d'ambiguïté à propos de son âge: elle est enfant mais agit parfois comme une adulte.
Le langage constituant l'une de ses caractéristiques les plus distinctives, les théories linguistiques de Roman Jakobson permettent d'établir en quoi son usage du langage relève d'une poétique ludique.
Le rapport de l'enfant à la langue des adultes s'analyse en deux étapes: d'abord sous l'angle de sa compréhension du code des adultes, ensuite par le biais de son usage détourné de ce même code.

\end{spacing}

\bigskip

\begin{singlespace}
\textbf{Mots clés~:}
littérature contemporaine, représentation de l'enfance, personnage d'enfant-femme, jeux de langage, Raymond Queneau, Marie-Sissi Labrèche
\end{singlespace}

%\addcontentsline{toc}{chapter}{Abstract}
\chapter*{Abstract}
%\selectlanguage{english}

\begin{spacing}{1.4}

Established in French literature since the 19th century, the child character presents itself under several figures, also called \guil{typologies}.
Autonomous, imperturbable, sometimes even disturbing: its the \guil{enfant du monde} (child of the world) that seems to represent best the young moderne or contemporary characters.
Even more problematic when it is a little girl, this adaptive child is reminiscent of the English character of Alice, presented by Lewis Carroll: diving into a world of nonsense, the young girl seeks her landmarks and adapts in order to survive.
In order to represent the legacy of this \textit{lineage}, we chose the characters of Zazie (\textit{Zazie dans le métro} by Raymond Queneau) and Sissi (\textit{Borderline} by Marie-Sissi Labrèche).
\par
From a theoretical perspective which is the \textit{sémiologie du personnage} (\guil{typology of the character}) of Philippe Hamon, we have created the new figure of the \guil{enfant-femme} (woman-child, not to confuse with child-like woman), which is a way of revisiting the myth of Alice.
Pushed to empowerment by its context of existence, the \textit{enfant-femme} adopts an ambiguous behaviour about her age: she is a child but sometimes acts as an adult.
Since language is one of its most distinctive characteristics, linguistic theories by Roman Jakobson help to define how its use of language establishes a playful poetic.
Thus, the child's relation to the adult language can be analyzed in two stages: from the point of view of her understanding of the adult code, and through her subversive misuse of the same code.

\end{spacing}

\bigskip

\begin{singlespace}
\textbf{Keywords ~:}
contemporary literature, childhood representation, woman-child figure, play-on-words, Raymond Queneau, Marie-Sissi Labrèche
\end{singlespace}
%\selectlanguage{french}


\begin{singlespace}
\tableofcontents
\end{singlespace}
%\addcontentsline{toc}{chapter}{Remerciements}
\chapter*{Remerciements}
J'aimerais d'abord exprimer ma gratitude envers Gilles Dupuis qui a, dès nos premières rencontres, accueilli mes idées incertaines avec un grand enthousiasme et un intérêt sincère, et qui m'a soutenue tout au long du processus de rédaction. Merci de m'avoir guidée, bien plus que dirigée.
\par
Côté robots, merci aux logiciels libres qui ont permis à ce mémoire de prendre forme: \LaTeX pour la mise en page ainsi que Zotero pour la gestion de mon chaos bibliographique.
\par
Côté humains, je tiens à remercier Thèsez-vous? et la \guil{clique des épingles à linge}, qui ont brisé ma solitude et m'ont offert l'encadrement \sout{militaire} nécessaire pour avancer dans la pénombre rédactionnelle.
Je remercie aussi mon papa ainsi que tous ceux -- membres de ma famille, ami.e.s, collègues de travail et d'université -- qui, au détour d'une conversation, ont eu envers moi des mots d'encouragement.
\bigskip
\par
Finalement, merci merci merci! à mes inconditionnels: ceux qui m'ont lue et relue, supportée et encouragée tout au long de cette maîtrise,
\par
\textbf{Nico, Fliflo, Tipou et Mamantine, merci pour tout!}

\newpage

\section*{Notes préliminaires}

\subsection*{Liste des abbréviations}
Afin d'alléger le texte, les références aux \oe{}uvres de notre corpus seront faites à même le texte en utilisant les abbréviations suivantes, suivies du numéro de la page:

\begin{singlespace}
\noindent
\textit{Z}: \cite{Queneau1959}.
\end{singlespace}

\begin{singlespace}
\noindent
\textit{B}: \cite{Labreche2003}.
\end{singlespace}

\subsection*{Éditions utilisées}
Pour \textit{Zazie dans le métro} de Raymond Queneau, nous avons employé une version imprimée en 1965 dans la collection \guil{Le Livre de poche}, aux éditions Gallimard; la première de couverture représente Zazie griffonée sur du papier quadrillé, le corps remplacé par un billet de métro.
\bigskip
\par
Pour les romans de Marie-Sissi Labrèche (\textit{Borderline}, \textit{La Brèche} et \textit{La lune dans un HLM}), toutes nos notes font référence aux publications dans le format \guil{Compact} chez les Éditions du Boréal.

\newpage

\pagenumbering{gobble}

\pagenumbering{arabic}
\setcounter{page}{1}
\chapter*{Introduction}
\addcontentsline{toc}{chapter}{Introduction}

\begin{flushright}
                                    \begin{singlespace}
                                    \epigraph{
                                    Un enfant, c'est un insurgé $\left[ \dots \right]$. Exubérant, désinvolte, insolent, au collège il lui arrivait souvent de chahuter; il me montra en riant sur son carnet une obervation qui lui reprochait des \guil{bruits divers en espagnol}; il ne posait pas au petit garçon modèle: c'était un adulte à qui sa maturité permettait d'enfreindre une trop puérile discipline.}
                                    \par
                                    Simone de Beauvoir, \textit{Mémoires d'une jeune fille rangée}\footcite[276]{Beauvoir1958}
                                    \end{singlespace}
\end{flushright}

Insoumis, indomptable, inflexible: l'enfant est un insurgé, nous dit Beauvoir.
Réel ou littéraire, l'enfant est complexe.
En tant que personnage, il est riche, en ce sens qu'il est avenir et aventures, espoir et espièglerie, innocence et rêverie.
Si l'enfance contemporaine se déroule sous le regard attentif de tous -- ne faut-il pas tout un village pour élever un enfant? --, la présence en littérature du jeune personnage n'a pas toujours été aussi marquée.
Ainsi la représentation de l'enfance en littérature a-t-elle grandement évolué au gré des courants de pensée philosophiques et du développement des connaissances scientifiques: de la créature plus proche de l'animal que de l'humain, l'enfant est devenu \textit{small adult} à développer, en passant par la dichotomie entre l'être parfois pur, parfois entaché du péché originel.
Au regard du changement de perception à son égard, il ne fait plus de doute que l'enfant est désormais un être à part entière dont les affects sont tout aussi complexes que ceux des adultes.
Les philosophes lui ayant rendu son humanité et les juristes lui ayant donné des droits, c'est au tour des littéraires d'accorder à l'enfant un droit de parole.
\par
Il y a plus de deux siècles que l'enfant a fait son entrée dans la littérature et qu'il est devenu un personnage digne d'intérêt, que ce soit dans la littérature légitime ou dans la littérature enfantine.
Si le personnage de garçon ne semble pas avoir posé problème -- on lui fait vivre des aventures desquelles il ressort grandi, sorte d'odyssée vers l'\textit{être homme} --, le \guil{cas} de la jeune fille n'est pas aussi simple.
Vouée à un rôle d'épouse puis de mère, la petite fille, réelle comme littéraire, se veut idéalement délicate et féminine.
On lui refuse donc l'action et les aventures, que l'on applaudit pourtant chez les garçons.
Dans les livres aux visées pédagogiques, elle est sage et bien élevée, sinon l'histoire tend à prouver que son bonheur ne peut passer que par son assagissement.
C'est la situation présentée par le livre \textit{Les Malheurs de Sophie}\footnote{Écrit par la comtesse de Ségur, née Sophie Rostopchine, c'est un livre pour enfants publié en 1858 aux éditions Hachette et encore lu de nos jours. Les livres \textit{Les Petites filles modèles} et \textit{Les Vacances} en sont la suite.} dans lequel les bêtises de Sophie, enfant aventurière, sont dépeintes négativement, et les tempéraments plus calmes de ses amies Camille et Madeleine, petites filles modèles, valorisés.
Tentant d'expliquer pourquoi c'est Alice qui est \guil{devenue \textit{la} petite fille, plutôt que Sophie en ses malheurs\footcite[7]{Lecercle1998}}, Jean-Jacques Lecercle propose qu'elle soit \guil{la première des petites filles\footcite[7]{Lecercle1998}}.
Alors que l'enfant était, jusque là, soit petit ange sage, soit petit diable à discipliner, l'enfant décrite par Lewis Carroll se pose différemment.
Se détachant de cette \guil{littérature pour enfants chrétienne et édifiante}, elle appartient à une nouvelle \guil{littérature enfantine où l'enfant, au lieu d'être objet de correction, est héros d'aventures\footcite[9]{Lecercle1998}.}
Ni petite fille sage, ni bon petit diable, elle est simplement \textit{petite fille}, rejetant à la fois moralisme et mièvrerie\footcite[10]{Lecercle1998}: \guil{Alors naît effectivement la petite fille moderne, délivrée des nécessités de la correction et de la conversion, qui peut être ce qu'est encore pour nous aujourd'hui la petite fille: un espace (mythique) de liberté\footcite[11]{Lecercle1998}.}
\par
Cette conception de la petite fille comme personnage \textit{libre} s'accompagne cependant des présupposés intrinsèquement liberticides selon lesquels \guil{l'innocence de la petite fille doit être protégée contre les dangers du vaste monde, où rôdent des loups déguisés en grand-mères\footcite[13]{Lecercle1998}.}
Une petite fille comme Alice est donc \textit{problématique} pour la littérature, en ce sens que son existence en tant que personnage n'est envisageable qu'au prix du non-respect des diktats sociaux sur les genres.
Cette perspective féministe -- d'ailleurs fort actuelle -- nous semble intéressante, notamment à la lumière des théories sur le genre de Monique Wittig, revisitées par Judith Butler\footcite[222-247]{Butler2012}.
\par
Or, ayant choisi de nous intéresser à \textit{Zazie dans le métro} de Raymond Queneau ainsi qu'à l'enfance représentée dans \textit{Borderline} de Marie-Sissi Labrèche\footnote{Puisque nos recherches portent sur l'enfance, nous nous intéresserons exclusivement aux chapitres pairs, soit ceux qui présentent une narratrice enfant.}, nous avons vite repéré un autre fil conducteur: si la petite fille pose encore problème, c'est d'abord et avant tout par son langage.
De là notre idée de rapprocher deux romans que tout oppose sur un autre plan: l'un est un roman parodique publié en 1959 par un homme de lettres français extrêmement respecté; l'autre est une autofiction publiée en 2000 et tirée de la partie \guil{création} du mémoire de maîtrise d'une jeune québécoise.
Puisque \textit{Zazie dans le métro} a déjà été beaucoup étudié -- plusieurs ont d'ailleurs fait le lien entre les écrits de Queneau et ceux de Lewis Carroll, et même posé Zazie comme \guil{héritière} naturelle d'Alice --, l'originalité de notre travail réside surtout au niveau de notre analyse du personnage d'enfant chez Labrèche.
Bien que \textit{Borderline} ait déjà été analysé, personne n'a, à notre connaissance, fait d'analyse spécifique sur le personnage d'enfant.
Ainsi, beaucoup ont examiné le rapport mère-fille ou le discours sur le corps, la sexualité et le désir, mais toujours en mettant l'accent sur le personnage adulte et en évacuant presque totalement les chapitres narrés par l'enfant Sissi.
Ceux qui se sont intéressés à l'enfant l'ont fait de concert avec le personnage adulte, utilisant une approche psychanalytique afin d'étudier principalement le rapport à la mère: ainsi \textit{Borderline} est-il parfois perçu, dans une perspective freudienne, comme un \guil{roman familial des névrosés\footcite[9-10]{Dion2010}}.
\par
Avec ce mémoire, nous comptons mettre en lumière une confrontation: non pas celle entre deux générations, mais plutôt celle entre deux \textit{mondes}, deux univers qui sont aux antipodes mais qui se nourrissent l'un et l'autre, en ce sens qu'ils se doivent de coexister et même de cohabiter.
Nous souhaitons donc dépasser la simple étude sur l'enfance et le personnage d'enfant, afin d'observer, dans les romans \textit{Zazie dans le métro} et \textit{Borderline}, la \guil{collision} entre les mondes des adultes et des enfants.
En plus de porter attention au conflit entre ces univers si différents, nous nous intéresserons à la médiation de ce conflit, à savoir comment ces deux mondes qui s'entrechoquent essaient -- et parviennent, ou pas -- à s'arrimer ou à s'harmoniser.
\par
Nous convoquerons d'abord la figure d'Alice de Lewis Carroll\footcite{Carroll2012}, enfant confrontée à un monde inquiétant, vaste métaphore du monde adulte, afin de définir une \textit{typologie} au sens où l'a entendu Philippe Hamon dans sa théorie sur la sémiologie du personnage\footcite{Hamon1977, Hamon1983}.
Puis, nous établirons des liens entre Zazie et Sissi, soit les jeunes filles des romans de notre corpus, et notre typologie nouvellement établie.
Enfin, dans une approche qui relève d'une \textit{linguistique du locuteur}, nous examinerons le rapport qu'ont ces personnages avec le langage, à savoir quel usage elles en font et leurs raisons.
Par ce mémoire, nous espérons dresser le portrait d'une nouvelle \textit{figure} de personnage d'enfant, que nous appellerons l'\textit{enfant-femme}.

\chapter{Le personnage d'enfant-femme}

\begin{flushright}
                                    \begin{singlespace}
                                    \epigraph{
                                    \guill{What--is--this?} he
                                    said at last.\\
                                    \guill{This is a child! $\left[ \dots \right]$}
                                    %Haigha replied eagerly, coming in front of Alice to introduce her, and spreading out both of his hands towards her in an Anglo-Saxon attitude.
                                    \guill{ We only found it
                                    to-day. It's as large as life, and twice as natural!} \\
                                    \guill{I always thought
                                    they were fabulous monsters!} said the Unicorn. \guill{Is it alive?}}
                                    \par
                                    Lewis Carroll, \textit{Through the looking-glass}\footcite[189]{Carroll2012}
                                    \end{singlespace}
\end{flushright}


Constituant une relecture du personnage d'Alice Liddell, née de la plume de
Lewis Carroll en 1865, nous estimons que le personnage d'enfant-femme trouve ses
racines les plus profondes dans les grands visages de l'enfance tels qu'élaborés
par la littérature du XIXe siècle, par ailleurs très riche en personnages
d'enfants. Afin de bien établir notre propre typologie, nous reviendrons dans un
premier temps aux grands types d'enfants dans la littérature française du XIXe
siècle en reprenant la classification de Marina Bethlenfalvay. Une fois ces
ancêtres de l'enfant moderne bien décrits, nous nous attacherons en second lieu à
élaborer la typologie d'un personnage qui, à notre connaissance, n'a pas encore
été défini: l'enfant-femme.

\section{L'historique de la représentation de l'enfance en littérature} Si la
représentation de l'enfant dans la littérature a grandement varié selon les
époques, les courants, de même que les cultures, il nous apparaît être devenu un
personnage à part entière sous la plume des auteurs pré-romantiques et
romantiques. Littérairement mis au monde par Jean-Jacques Rousseau sous le nom
d'Émile, l'enfant envahit la littérature française à l'apogée du courant
romantique, lequel \guil{instaur[e] un véritable culte de
l'enfance\footcite[90]{Vurm2014}}.

\subsection{Les visages de l'enfant: la typologie de Bethlenfalvay} Selon la
classification de base de l'enfant en littérature, deux figures d'enfant se
voient opposées du tout au tout. Le premier est doux et sentimental, \guil{la
figure de l'enfant permet[ant] à l'auteur de se remémorer avec nostalgie
l'enfance perdue\footcite[90]{Vurm2014}}; le second est, au contraire,
\guil{inversé, \guill{paradoxal} -- le plus souvent un enfant avancé sur son âge
biologique, génie et révolté\footcite[90]{Vurm2014}.} Cette classification de
base est toutefois raffinée par la typologie esquissée par Marina Bethlenfalvay à
propos de l'enfant dans la littérature du XIX\up{e} siècle, dans laquelle elle
distingue trois grandes figures d'enfants.

\subsubsection{L'enfant venu d'ailleurs: l'enfant-ange de la poésie romantique}
Radicalement différent de l'adulte, essentiellement meilleur, l'enfant venu
d'ailleurs a un physique idéalisé et correspond à la métaphore de l'enfant-ange,
fréquente en poésie\footcite[20]{Bethlenfalvay1979}. Ainsi, dans ce que
Bethlenfalvay appelle le \guil{complexe de l'Enfant Romantique}, l'enfant est vu
comme un messager entre Dieu et les hommes\footcite[21-22]{Bethlenfalvay1979}.
Cet enfant, représenté surtout dans l'\oe{}uvre des poètes romantiques\footcite[19]{Bethlenfalvay1979}, suggère un monde idéal en référant à la joie édénique de la scène pastorale.
Chargé des rêves brisés du monde adulte, l'enfant appelle autant la métaphore de l'île exotique que la dichotomie entre le berceau et le tombeau\footcite[24-25, 33]{Bethlenfalvay1979}.
Il se retrouve surtout dans la poésie de Victor Hugo, de Marceline Desbordes-Valmore et celle d'Alphonse de Lamartine\footcite[19, 34]{Bethlenfalvay1979}.
En tant que personnage, c'est au \textit{Petit Prince} de Saint-Exupéry qu'il fait penser tandis que \textit{Le Grand Meaulnes} d'Alain-Fournier représente sa survivance au XX\up{e} siècle\footcite[44-47]{Bethlenfalvay1979}.

\subsubsection{L'enfant victime: le petit martyr du roman réaliste}
Bien que toujours aussi pur, innocent et aimant que l'enfant venu d'ailleurs, l'enfant victime s'en distingue par la façon dont il est traité par l'auteur, lequel ne le protège pas de la misère du monde et s'attache intentionnellement à sa misérable condition. C'est donc un enfant souffrant, victime ou martyr, souffre-douleur plutôt qu'ange ou messie\footcite[53]{Bethlenfalvay1979}.
Enfant par excellence du roman réaliste, il a un rôle didactique et moral, voire polémique, dont l'objectif est de susciter pitié et indignation relativement à l'humanité souffrante et à l'injustice sociale qu'il symbolise\footcite[53, 64, 85]{Bethlenfalvay1979}.
\par
C'est bien souvent l'enfant d'une prostituée, l'enfant priant auprès de sa mère morte, celui mis en pension, abandonné, délaissé ou maltraité. Il correspond aux personnages hugoliens de Cosette dans \textit{Les Misérables} et de Gwynplaine dans \textit{L'Homme qui rit}, à ceux mis en scène dans \textit{Jack} d'Alphonse Daudet et \textit{Poil de Carotte} de Jules Renard\footcite[56, 71, 76-77]{Bethlenfalvay1979}.
Cependant, le personnage le plus représentatif de l'enfant victime se trouve du côté de la littérature anglaise: il s'agit du protagoniste d'\textit{Oliver Twist} de Charles Dickens\footcite[110]{Bethlenfalvay1979}.

\subsubsection{L'enfant du monde: le \guil{bon sauvage} du roman naturaliste}
Bien enraciné dans la vie terrestre, l'enfant du monde est robuste et vif. Optimiste et débrouillard, il se tourne vers l'avenir et s'adapte à sa vie de misère.
S'il lui arrive de souffrir, il n'est jamais victimisé, subissant un sort ni pire ni meilleur que celui des grandes personnes qui l'entourent\footcite[85-86]{Bethlenfalvay1979}.
Découlant de la foi dans le progrès, du déterminisme et de la glorification de la force vitale, cette vision place l'enfant comme un être originel, pré-civilisé voire animal. Cet enfant, généralement rattaché à un cadre rustique, incarne le mythe du \guil{bon sauvage} duquel est évacuée la dimension mystique\footcite[85, 87-88, 92]{Bethlenfalvay1979}.
\par
Correspondant à une promesse d'avenir optimiste quant à la réalisation de l'harmonie entre l'homme et l'univers\footcite[92]{Bethlenfalvay1979}, l'enfant du monde s'épanouit avant tout dans l'\oe{}uvre d'Émile Zola, tous les enfants des \textit{Rougon-Macquart}, surtout Jeanlin, semblant correspondre à ce type\footcite[85-89]{Bethlenfalvay1979}.
C'est aussi le Gavroche des \textit{Misérables} chez Victor Hugo, lequel \guil{ne se sentait jamais si bien que dans la rue [puisque] le pavé lui était moins dur que le c\oe{}ur de sa mère\footcite[90-91]{Bethlenfalvay1979}.}
\par
Ce personnage mène, au siècle suivant, au roman d'apprentissage tel \textit{Jean-Christophe} de Romain Rolland\footcite[112]{Bethlenfalvay1979}.
La Zazie de Raymond Queneau est également mentionnée comme faisant partie de ces \guil{petits personnages que l'imagination ne trouble guère, et qui prolongent à leur manière la lignée du gamin, s'adaptant à tous les milieux, sans perdre son équilibre} en tant qu'\guil{enfant terrible et irrépressible}, moins désorientée que les adultes par tous les événements qu'elle traverse\footcite[117]{Bethlenfalvay1979}.

\subsubsection{L'enfant des autres: l'enfant aliéné par la modernité}
Dans sa typologie des enfants du XIX\up{e} siècle, Marina Bethlenfalvay déborde sur le XX\up{e} siècle pour dresser un portrait de l'enfant de la modernité.
Elle décrit un enfant faible et vulnérable, dépendant du \guil{milieu humain} auquel il est soumis et représenté surtout chez Marcel Proust, chez Jean-Paul Sartre et chez Nathalie Sarraute\footcite[128-129]{Bethlenfalvay1979}.
Voyant en lui un enfant hautement \guil{sociologisé}, voire aliéné,
Bethlenfalvay postule que les sentiments d'infériorité, d'impuissance et d'abandon sont la condition existentielle de cette figure d'enfant\footcite[128, 134-135, 138]{Bethlenfalvay1979}.


\subsection{Un enfant multiple: l'éclatement des modèles}
Dès l'évocation par Bethlenfalvay de cet \guil{enfant des autres} préfigurant le XX\up{e} siècle, il est évident que la typologie du personnage d'enfant n'ira qu'en se complexifiant.
Ainsi, avec la modernité surviennent des personnages qui débordent des modèles jusqu'alors conçus.
C'est de ce constat d'inadéquation des \textit{types} préexistants que naîtront notamment l'enfant révolté puis l'enfant-adulte, lesquels nous amènent au plus près de cette enfant-femme qui nous intéresse.

\subsubsection{La nécessité d'un nouveau modèle: l'enfant révolté}
Cherchant à comprendre l'enfant chez l'auteur québécois Réjean Ducharme, Petr Vurm se tourne d'abord vers l'ouvrage de Bethlenfalvay, en portant une attention particulière à l'épilogue et en explicitant certaines nouvelles figures propres au XX\up{e} siècle.
D'une part, il y a l'enfant-génie ou enfant-révolté, qu'il associe aux \textit{Mots} et à \guil{L'enfance d'un chef} de Jean-Paul Sartre et à la boutade \guil{Familles, je vous hais!} lancée par André Gide dans \textit{Les nourritures terrestres}, et qui est un enfant qui remet constamment en cause les valeurs du monde adulte\footcite[93]{Vurm2014}.
D'autre part, il y a l'enfant nouveau, \guil{repérable chez les surréalistes, qui confèrent à l'enfant le pouvoir de l'imagination illimitée et d'un regard nouveau et insolite, doué d'une immense potentialité créatrice, voisine de celle de l'inconscient}, incluant l'évasion vers des mondes imaginaires et dont l'exemple donné est \textit{Les enfants terribles} de Jean Cocteau\footcite[93]{Vurm2014}.
Suite à cette analyse, Vurm propose de rajouter à la typologie de Bethlenfalvay la figure de l'\guil{enfant révolté}, à la fois aux antipodes et \guil{corollaire} de l'enfant qui subit passivement ses peines\footcite[94]{Vurm2014}, figure qui l'aidera à poser un regard critique sur l'enfant dans l'\oe{}uvre de Ducharme.

\subsubsection{L'ambiguïté identitaire: l'enfant-adulte de Vurm}
Parlant de l'enfant ducharmien, Vurm exprime la difficulté de son identité, notamment \guil{le mélange d'incertitudes concernant son âge physique, mental ou émotionnel et linguistique\footcite[97]{Vurm2014}}.
À ses yeux, l'indétermination de l'âge linguistique est la plus intéressante au plan littéraire puisqu'elle amène une confusion entre le discours des adultes et celui des enfants.
Ainsi, \guil{l'ambiguïté qui réside au c\oe{}ur des romans sur l'enfance, consiste dans l'impossibilité de dire: \guill{Je suis un enfant} ou \guill{Je suis un adulte.} }, ambiguïté qui pousse le critique à qualifier le personnage d'\guil{enfant-adulte}\footcite[99-100]{Vurm2014}.
Par la création de ce nouveau type de personnage, Vurm traduit en un néologisme toute la complexité que représente l'enfant ducharmien et qui réside dans la question identitaire de l'appartenance à un monde d'âge, que ce soit adulte ou enfant:
\begin{quote}
  \begin{singlespace}
    \small
    Malgré l'impossibilité de dire qui est celle de l'enfant, l'écrivain travaille le paradoxe littéraire du dialogue entre les mondes enfantin et adulte. Cela lui permet de situer l'enfant littéraire à la croisée de ces deux mondes et de bénéficier de ce que ces deux mondes offrent: la culture et le jeu sophistiqué de la subversion chez l'adulte, la spontanéité et le jeu chez l'enfant\footcite[103]{Vurm2014}.
    \normalsize
  \end{singlespace}
\end{quote}
Le personnage d'\textit{enfant-adulte} tel qu'élaboré par Vurm apparaît crucial pour nos recherches puisque c'est essentiellement à partir de ce dernier que nous souhaitons construire et particulariser notre propre typologie d'\textit{enfant-femme}. Nous reprendrons donc aux fins de la théorie cette idée de dialogue entre l'enfance et l'âge adulte.

\section{Le personnage d'enfant-femme: notre typologie}
Puisque le concept d'enfant-femme est de notre création, nous nous devons d'abord de le définir.
Nous reviendrons à cette fin à ce qu'on appelle le \guil{mythe} d'Alice, lequel jette les bases de notre typologie.
Puis, nous reprendrons ce que Petr Vurm a défini comme étant l'enfant-adulte et nous nous emploierons à élaborer plus profondément cette forme de cohabitation de l'enfant et de l'adulte.
Enfin, nous expliquerons en quoi l'enfant-femme est différente de la femme-enfant, principalement sur le plan des intentions en ce qui a trait à la sexualité.

\subsection{Un retour au mythe d'Alice: l'enfant-femme comme \textit{topos} moderne?}
Tel qu'évoqué dans notre titre, l'enfant-femme comme \textit{type} littéraire se veut une façon de revisiter la figure d'Alice de Lewis Carroll.
Nous affirmons à cette fin qu'Alice, tant comme personnage que par le récit de ses aventures dans le terrier du lapin, constitue un mythe littéraire moderne, voire que ce personnage représente une \guil{nouvelle mythologie de l'enfance\footcite[140]{Jousni2005}}.
Nous avançons même qu'au-delà de ce mythe, l'enfant-femme qui en découle est à son tour une figure récurrente de la littérature, à savoir un \textit{topos} contemporain.

\subsubsection{Élever Alice au rang d'une mythologie}
Comme l'a dit Marie-Hélène Inglin-Routisseau au terme de sa thèse de doctorat, la figure d'Alice \guil{gouverne incontestablement l'imaginaire littéraire du 20\up{ième} siècle}, arrivant en France avec le surréalisme et disparaissant en même temps que les auteurs de ce courant\footcite[329]{Inglin-Routisseau2006}. De la même façon, Jean-Jacques Lecercle affirme qu'Alice survit au temps \guil{parce qu'elle est devenue mythe, parce qu'elle incarne l'archétype de ce personnage éternel, la petite fille\footcite[7]{Lecercle1998}.}
Alice est donc bien ancrée dans l'imaginaire littéraire, même en France: \guil{Cette présence poétique indéfiniment prolongée l'érige au rang de mythe\footcite[330]{Inglin-Routisseau2006}.}
\par
Si la reprise de la figure d'Alice s'est quelque peu atténuée après les années 1960 et 1970 dans la littérature au sens strict, tel que le prétend Inglin-Routisseau, nous estimons qu'elle est toujours aussi présente dans la culture populaire et dans l'imaginaire social,
notamment au cinéma\footnote{Il n'y a qu'à penser à l'adaptation de Tim Burton, \textit{Alice au pays des merveilles} (2010) et à sa suite de James Bobin, \textit{Alice de l'autre côté du miroir} (2016). Du côté québécois, \textit{L'Odyssée d'Alice Tremblay} (2002) se veut une parodie du conte.},
à la télévision\footnote{Outre les adaptations plus ou moins fidèles à l'histoire originale, notamment la série \textit{Once Upon A Time In Wonderland} (2013-2014) du Studio ABC, il y a également la série \textit{Lost} (2004-2010), également de la chaîne ABC, qui s'inspire librement d'Alice, notamment dans les titres de certains épisodes.}
et dans la chanson populaire\footnote{En plus des chansons \textit{I Am the Walrus} (1967) des Beatles et \textit{White Rabbit} (1967) de Jefferson Airplane, notons les plus récentes \textit{Tweedle Dee \& Tweedle Dum} (2001) de Bob Dylan, \textit{Sunshine} (2001) d'Aerosmith, \textit{What You Waiting For?} (2004) de Gwen Stefani, \textit{Wonderland} (2014) de Taylor Swift et le vidéoclip de \textit{Brick By Boring Brick} (2009) du groupe Paramore. Il y a également les albums \textit{Cheshire Cat} (1994) de Blink-182, \textit{Alice and June} (2005) d'Indochine (album duquel est tiré la chanson du même titre), \textit{Le Cheshire Cat et moi} (2009) de Nolwenn Leroy ainsi que \textit{Eat Me, Drink Me} (2007) de Marylin Manson, dont est tirée une chanson intitulée \textit{Are You The Rabbit?}.},
sans oublier la paralittérature\footnote{Au Québec, le roman fantastique \textit{Aliss} (2000) de Patrick Sénécal est inspiré du personnage carrollien. Il y a également de nombreux mangas japonais légèrement ou fortement inspirés des aventures d'Alice.} ainsi que divers jeux vidéo.
En somme, bien que le personnage d'Alice telle qu'exploité en France par les surréalistes se soit dissipé, sa mémoire reste profondément vivante dans la culture populaire occidentale.

\subsubsection{Les autres Alice de la modernité}
Après la traduction par Louis Aragon de \textit{La Chasse au Snark} en 1929, Alice est introduite en France dans l'\oe{}uvre des surréalistes où elle devient \guil{poupée désarticulée, fragmentée, hachée\footcite[175]{Inglin-Routisseau2006}}, notamment femme-homme ou femme-mannequin et même \guil{femmes-enfants, femmes-fleurs, femmes-étoiles, femmes-flammes, flots de la mer, grandes vagues de l'amour et du rêve} selon Eluard\footcite[176]{Inglin-Routisseau2006}.
Redevenue poupée pour le poète, elle est, toujours pour Eluard, \guil{objet manipulable}, captive et fantasmatique\footcite[285]{Inglin-Routisseau2006}. \guil{Enfant brisée, misérable, et domptée}, Alice-poupée ouvre la voie à la nymphette, à la \textit{Lolita} de Vladimir Nabokov, par le biais du fantasme de l'inceste\footcite[287]{Inglin-Routisseau2006}.
C'est une \guil{Alice au pays \guill{des tristes merveilles}\footnote{\mancite \cite{Parrot1944} cité dans \cite[287]{Inglin-Routisseau2006}}}.
Véritable \textit{incarnation} de l'inconscient, cette dernière subit un \guil{morcellement sadique [de son] corps} qui mènera à la femme-enfant, icône surréaliste et \guil{ultime avatar d'Alice\footcite[177]{Inglin-Routisseau2006}}.
\par
Par la suite, Alice est reprise par de nombreux auteurs, souvent via le pastiche ou un rappel du personnage\footnote{Notamment dans la parodie \textit{Alice en France} (1945) de Raymond Queneau, dans le livre illustré \textit{De l'autre côté de la page. Alice au pays des lettres} (1968) de Roland Topor, dans \textit{Les Enfants de l'été} (1978) de Robert Sabatier où plusieurs personnages connus de la littérature jeunesse se rencontrent, dans plusieurs poèmes de Paul Gilson, entre autres dans \guil{Alerte aux rêves} du recueil \textit{À la vie à l'amour} (1943) où Alice croise Peter Pan.}.
Cependant, cette reprise de l'enfant carrollien est limitée en ce sens qu'elle ne rend pas bien compte de l'entière postérité du personnage: \guil{L'exercice de style, pour être brillant et ingénieux, oblitère en effet la spécificité de l'\oe{}uvre originale\footcite[281]{Inglin-Routisseau2006}.}


%Année 1959: parution de \textit{Zazie dans le métro}, traduction française de \textit{Lolita} (Nabokov), \textit{Laissez-moi tranquille} de Lise Deharme: \guil{Les héroïnes de ces deux romans, Lolita et Carole, ont à peu près l'âge de Zazie, et ne sont pas plus qu'elle tourmentées par les scrupules d'une conscience morale exigeante ni par le souci des bienséances. Mais la comparaison s'arrête là. [...] Ces rapprochements nous indiquent pourtant que, d'une certaine manière, l'heure de telles héroïnes avait sonné.\footcite[15]{Bigot1994}}



\subsection{Une déclinaison féminine de l'enfant-adulte}
Le personnage d'enfant-femme étant d'abord la déclinaison féminine de l'enfant-adulte, lequel a été brièvement décrit par Vurm, nous estimons utile de reprendre son travail aux fins de notre typologie, plus particulièrement son idée d'ambiguïté identitaire qui est au c\oe{}ur du concept même d'\textit{enfant-adulte}.

\subsubsection{\guil{Les âges} du personnage: quand âge biologique et âge social ne concordent pas}
À la base même du personnage d'enfant-adulte et de son ambiguïté identitaire se trouve une discordance entre \textit{les} âges d'un personnage.
Ainsi, la typologie que nous esquissons se fonde sur une discordance entre l'âge biologique et l'âge social d'un personnage: l'enfant-femme se devra d'être biologiquement une enfant, en ce sens qu'elle aura obligatoirement la physionomie d'une gamine, tout en adoptant des comportements qui rappellent inévitablement la maturité d'une adulte.
C'est le concept de dédoublement des âges d'un même personnage qui nous permet d'élaborer le concept d'enfant-femme.

\subsubsection{Tout à la fois ou ni l'un ni l'autre}
D'abord, nous tenons à poser une prémisse essentielle à la construction de notre modèle selon laquelle l'enfant-femme, ou plutôt l'enfant-adulte, n'est ni un enfant, ni un adulte, tout en réunissant à la fois des caractéristiques de l'un et de l'autre.
Si l'enfant-femme n'est ni une adulte, ni tout à fait une enfant, il est tout aussi clair pour nous qu'elle ne correspond pas à la \guil{médiane} de ces deux âges, c'est-à-dire à l'adolescente. Personnage combinant des caractéristiques appartenant aux deux extrêmes, elle n'est pas la figure de l'entre-deux, le personnage mitoyen, car nous tenons à poser comme élément essentiel de son identité cet état du ni l'un ni l'autre.
\par
L'enfant-femme est également un personnage \guil{contre} ou \textit{anti}, en ce sens qu'il implique un rejet du monde adulte: \guil{La révolte toujours placée entre les mains d'Alice, incarne à présent la rébellion contre la morale et les valeurs adultes. Le non-sens ouvre les portes du royaume de l'enfance, et l'enfance celles de la liberté\footcite[169]{Inglin-Routisseau2006}.}
Ce positionnement nous permet d'affirmer que l'enfant-femme est un personnage qui se construit par rapport à une altérité, laquelle correspond à l'adulte, ou plus précisément à l'\textit{adulte-adulte}, c'est-à-dire à celui dont à la fois les âges biologique et comportemental sont ceux d'un adulte. Ainsi, l'enfant-adulte se positionne par rapport à un \guil{eux}, dans une attitude de détachement par rapport à ces \guil{fonctionnaires} que représentent pour lui les adultes\footcite[100]{Vurm2014}, s'établissant hors de la portée de leur autorité.

\subsubsection{Un enfant adaptatif: l'autonomisation comme outil de survie}
Un peu à la façon de l'\guil{enfant du monde} qu'a décrit Marina Bethlenfalvay en référant au personnage du \guil{bon sauvage} dans le roman naturaliste, nous estimons que l'enfant-femme est un enfant qui s'adapte bien aux aléas de la vie.
Alors que l'enfant du monde du XIX\up{e} siècle est décrit comme acceptant son sort, nous croyons bon de nuancer cette capacité d'\guil{acceptation} pour en faire plutôt une qualité d'\guil{adaptation}: l'enfant-femme est à nos yeux un enfant extrêmement adaptatif qui, sans se complaire dans ses difficultés, s'efforcera de les surmonter ou de \guil{faire avec}.
\par
À l'origine de cette haute capacité d'adaptation chez l'enfant, nous retrouverons toujours une situation personnelle complexe.
Si l'on revient au mythe d'Alice, l'acclimatation concerne surtout le \textit{nonsense} des personnages et de l'environnement avec qui elle doit interagir suite à son entrée dans le terrier:
\begin{quote}
  \begin{singlespace}
    \small
    \textit{On the one hand, Alice encounters characters, in relation to whom she is like a child in the dogmatic and perplexing adult world, and on the other hand, the characters she meets behave like self-centred, stubborn children in pursuance of their enthusiasm for childish word playing\footnote{\guil{D'un côté, Alice rencontre des personnages face auxquels elle est comme une enfant dans le monde dogmatique et déconcertant des adultes, et d'un autre côté, les personnages qu'elle rencontre se comportent comme des enfants têtus et égoïstes par leur enthousiasme pour les jeux de mots enfantins.}~(notre traduction) dans \mancite \cite[39]{Katajamaki2005}.}.}
    \normalsize
  \end{singlespace}
\end{quote}
Dans le cas de l'enfant-femme, nous pensons davantage à un enfant laissé à lui-même et qui pallie cette absence d'autorité parentale en s'autonomisant à sa façon, soit en vieillissant prématurément.
Dépourvu de repères dans le monde adulte, cet enfant adapte son comportement à sa condition et se mue en enfant-adulte afin de survivre, développant des mécanismes de ruse et même de rejet de l'autorité dont il a été privé.

\subsection{Un enfant de langage: la parole comme condition d'existence}
Outre les caractéristiques psychologiques et comportementales que nous avons déjà dégagées, le langage constitue chez l'enfant-femme un trait hautement distinctif.
Ainsi, nous ne croyons pas faire erreur en affirmant que l'enfant-femme est un personnage de langage en ce sens qu'elle est déterminée par son usage du verbe, trait qu'elle hérite d'ailleurs de son ancêtre anglaise Alice, chez qui on voit \guil{pour la première fois dans la littérature, l'affirmation d'une véritable inscription de l'enfant dans l'ordre du langage, un enfant qui par là accède au statut d'être singulier et à part entière\footcite[144]{Jousni2005}}.

\subsubsection{L'Alice française, subversive plus qu'onirique}
Bien que l'écriture de Lewis Carroll ait inspiré maints auteurs français au XX\up{e} siècle, il semblerait que certains aspects de l'\oe{}uvre aient été préférés à d'autres par les surréalistes: \guil{les aventures d'\textit{Alice} en France intéressent plus par ce qu'elles comportent de subversif que par ce qu'elles véhiculent d'onirique\footcite[167]{Inglin-Routisseau2006}}.
C'est donc davantage le \textit{nonsense} anglais, que nous apparentons au ludisme langagier, qui place Alice comme précurseur de l'enfant-femme.
Ce recours aux jeux de langage rend même possible une lecture dédoublée d'Alice puisque le ludisme mène à la coexistence de discours à la fois enfant et adulte:
\begin{quote}
  \begin{singlespace}
    \small
    \textit{And it is important to recognize that the audience for Alice, at least in the English-speaking world, consists of two groups -- children and adults. Thus far, I have been speaking of Alice as a child's book. But Alice in Wonderland is, in effect, two books: a book for children and a book for adults. Its interest, its fantasy, its humor, and its logic all operate at two levels. I know that adults often wonder why and how Alice can appeal to children. I suspect that children wonder why adults like it\footnote{\guil{C'est important de reconnaître que le lectorat d'\textit{Alice}, au moins dans le monde anglophone, consiste en deux groupes -- enfants et adultes. Jusqu'ici, j'ai parlé d'\textit{Alice} comme d'un livre pour enfants. Mais \textit{Alice au Pays des Merveilles} est, en fait, deux livres: un livre pour enfants et un livre pour adultes. Son intérêt, sa fantaisie, son humour et sa logique fonctionnent tous sur deux niveaux. Je sais que les adultes se demandent souvent pourquoi et comment \textit{Alice} peut plaire aux enfants. Je soupçonne que les enfants se demandent aussi pourquoi les adultes l'aiment.}~(notre traduction) dans \mancite \cite[7]{Weaver1964}.}.}
    \normalsize
  \end{singlespace}
\end{quote}
Cette dualité dans l'écriture constitue une forme de pervertissement du langage, subversion qui demeure tout de même assez douce dans le cas du Pays des merveilles.


\subsubsection{Le ludisme langagier: au-delà de la fonction poétique}
Dans la création de son enfant-adulte, Vurm conclut que ce personnage est hautement langagier, jumelant \guil{la culture et le jeu sophistiqué de la subversion chez l'adulte} ainsi que \guil{la spontanéité et le jeu chez l'enfant\footcite[103]{Vurm2014}}.
Cette affirmation nous renvoie au concept du ludisme langagier, notamment à ce que Brigitte Seyfrid a nommé la \guil{stratégie ludique}, qui consiste en la mise en place d'\guil{une rhétorique du jeu, rattachable à la \guill{fonction poétique} de Jakobson centrée sur le message en tant que tel\footcite[65]{Seyfrid-Bommertz1999}.}
Cette stratégie ludique consiste en l'exacerbation de la fonction poétique laquelle, poussée à l'extrême, se mue en fonction purement ludique:
\begin{quote}
  \begin{singlespace}
    \small
    À la théorie de l'adéquation absolue du vers à l'esprit de la langue, de la non-résistance de la forme au matériau, nous opposons la théorie de la violence organisée exercée par la forme poétique sur la langue. (\guil{Ce n'est pas la langue qui est maîtresse du poète, mais le poète qui est maître de la langue} selon la formule du professeur Brandt.) La forme tient compte du matériau auquel elle a affaire, mais ne peut être donnée tout entière dans ce matériau (ne peut en être tout entière déduite, coïncider avec)\footcite[40]{Jakobson1973}.
    \normalsize
  \end{singlespace}
\end{quote}
Usant allégrement de cette \guil{violence organisée} dans son expression, l'enfant-femme semble être une \guil{créature de langage}, voire un personnage purement langagier et dont l'unicité est subordonnée à son usage de la langue.
\par
Ainsi, nous estimons que la discordance dans les âges, déjà évoquée au niveau comportemental, se manifeste également dans l'expression verbale de l'enfant-adulte, qui combine un vocabulaire mature et élaboré dans un arrangement naïf et joueur typique de l'enfance.
Nous avançons même que cette stratégie ludique, soit l'usage des jeux de langage, constitue pour l'enfant-adulte une autre façon de résister au monde adulte tout en en intégrant profondément certaines particularités: si l'enfant maîtrise, en tout ou en partie, le code des adultes, il s'emploie lorsqu'il est l'émetteur à brouiller le message en subvertissant la fonction poétique jusqu'à verser dans le ludisme.

\subsection{Au contraire de la femme-enfant: le paradigme de la sexualisation}
Au c\oe{}ur de notre conception de l'enfant-femme se trouve la nécessité de la distinguer et de l'opposer au personnage de la femme-enfant.
Bien que ces figures soient toutes les deux du genre féminin, nous tenons à les poser comme des antonymes sur deux plans:  d'une part, les âges d'un personnage réfèrent à la fois à son âge réel et à son âge perçu ou représenté; d'autre part, la sexualisation d'un personnage peut être intrinsèque ou extrinsèque, volontaire ou involontaire.

\subsubsection{La femme-enfant comme séductrice}
À notre avis, la différence entre les deux figures est fondamentale puisque la femme-enfant est un personnage d'âge relativement adulte qui utilise volontairement l'enfance à des fins de séduction.
À cet effet, le Petit Robert la définit comme étant une \guil{femme qui semble avoir conservé les attributs de l'enfance, qui cultive un comportement enfantin\footcite[Entrée: « Femme-enfant »]{Robert2013}}.
\par
D'abord née sous la plume de Catulle Mendès qui publie en 1891 le roman \textit{La Femme-enfant}, elle est par la suite étudiée par le psychanalyste autrichien Fritz Wittels\footcite{Wittels1999} qui la dépeint comme hautement sexualisée, \guil{convaincu d'avoir découvert \guill{la femme primitive} au pouvoir sexuel illimité\footcite[302]{Inglin-Routisseau2006}}:
\begin{quote}
  \begin{singlespace}
    \small
    Wittels décrit, dans ce texte, une fille dont \guil{l'intense pouvoir de séduction sexuelle se déclare si tôt qu'elle est forcée de commencer à vivre sa sexualité alors qu'à tout autre égard elle n'est encore qu'une enfant. Toute sa vie durant, elle demeure surdéveloppée sexuellement et incapable de comprendre le monde adulte. Pas plus que ce monde n'est capable de la comprendre.} [...] Ce type de femme \guil{est nécessairement perverse polymorphe, sadique, lesbienne et tout ce qui s'ensuit.}
    \normalsize
  \end{singlespace}
\end{quote}
C'est cependant avec le surréalisme que la femme-enfant connaîtra son apogée: \guil{Cette jeune femme capricieuse, très belle et futile, stupide et sans tact, totalement infidèle ressemble au prototype féminin idéal de la transgression sexuelle plus tard prônée par le surréalisme\footcite[302]{Inglin-Routisseau2006}.}
Cependant, quoique personnage central du surréalisme, la femme-enfant n'y constitue pas une figure à la définition stable. Son visage est multiple: vierge immatérielle chez René Crevel (\textit{Babylone}), jeunesse éternelle aux accents \oe{}dipiens chez André Breton (\textit{Arcane 17}), prisme unificateur des genres, des espèces et des âges dans une sculpture de Salvador Dali où elle surgit au centre d'un buste aux têtes de vieillard, de chérubin, d'oiseau, d'homme et de chat à la fois\footcite[303]{Inglin-Routisseau2006}. Elle est opposée à la sorcière par Breton mais aussi par Benjamin Péret, \guil{porteuse d'un amour rédempteur [...] quand la sorcière est une femme fatale qui déchaîne les passions\footcite[306]{Inglin-Routisseau2006}}.
Malgré sa multiplicité, la femme-enfant est sacralisée en raison \guil{de sa beauté merveilleuse et poétique, de son pouvoir de subversion, de la nature ou de la vie\footcite[307]{Inglin-Routisseau2006}}.
\par
Hors du surréalisme, la femme-enfant s'est également développée en mythe social et désigne un ensemble de traits de personnalité chez la femme, dont l'actrice Brigitte Bardot constitue l'exemple vivant le plus équivoque:
\begin{quote}
  \begin{singlespace}
    \small
    Mêlant innocence et sensualité, B.B. condensera malicieusement l'enfant et la femme désirable, indomptable et inaccessible. La femme-enfant se voit, à ce stade, investie d'un pouvoir de séduction et d'une attraction érotique inégalables\footcite[307]{Inglin-Routisseau2006}.
    \normalsize
  \end{singlespace}
\end{quote}
C'est donc une femme qui joue à l'enfant et qui utilise la gaminerie pour séduire, et dont le rappel de la sexualité est essentiel à la construction du personnage.
Pour la femme-enfant, le recours à la sensualité et à la sexualité est intrinsèque et volontaire et constitue un outil dans son arsenal de séduction.
Davantage qu'à la nymphe surréaliste, c'est à cette vision de la femme-enfant que nous désirons plus que tout opposer notre propre conception du personnage d'enfant-femme.

\subsubsection{L'enfant-femme comme adulte involontaire}
Au contraire de la femme-enfant, l'enfant-femme est un personnage d'enfant à l'âge clairement défini comme tel et qui comporte certaines caractéristiques d'adulte.
C'est donc un enfant dont la précocité rappelle sur certains points la maturité d'un adulte.
Cette précocité n'est toutefois pas de nature sexuelle puisque, pour ce personnage, la sexualité est toujours apportée par l'Autre.
À l'époque victorienne, contexte dans lequel évoluait Alice, alors que le petit garçon est un homme en devenir, \guil{la petite fille $\left[ \dots \right]$ n'est pas une petite femme\footcite[14]{Lecercle1998}}.
Ainsi, la figure de l'enfant \guil{n'est pas sexuée: elle concerne autant le petit garçon que la petite fille\footcite[11]{Lecercle1998}.}
Par lui-même, le personnage d'enfant-femme est donc indistinct de celui de l'enfant-adulte puisqu'il n'est ni masculin, ni féminin: c'est simplement un enfant, sans distinction de genre.
\par
Ce caractère non-genré que l'on présuppose à l'enfant est cependant impossible chez la petite fille dès que l'on prend en compte la façon dont elle est perçue, puisqu'elle est invariablement et bien involontairement sexuée par l'Autre, qui voit en elle une femme:
\guil{La fillette est érotisée par le désir de l'homme. Voici Alice transformée en une enfant avertie et aguicheuse, annonçant \textit{Zazie}, \textit{Lolita}, et la femme-enfant\footcite[328]{Inglin-Routisseau2006}.}
Alors que l'enfant est dépourvu de genre, la petite fille est un personnage invariablement genré.
\par
D'ailleurs, Dissard exprime dans son texte \textit{Alice contre les garçons} que l'enfant plongé dans le terrier du lapin ne peut être que petite fille puisque \guil{le garçon tue le texte carrollien\footcite[116]{Dissard2005}}.
En réponse, Chiara Lagani émet l'idée que l'enfant carrollien échoue à être un garçon parce que ce dernier est considéré comme un adulte dès l'enfance\footcite[204]{Lagani2005}.
Elle reprend l'explication de Lecercle selon laquelle la petite fille victorienne, contrairement à ses frères, n'avait pas accès à la scolarisation d'état et était éduquée par une gouvernante, ce qui la préservait du monde adulte et du \guil{futur regard plein de désir de l'adulte\footcite[204]{Lagani2005}}.
Par cette éducation en vase clos qui la rend \guil{à la fois plus réprimée et plus libre}, la fillette victorienne évoque une conception de l'enfant dont on doit à tout prix protéger l'innocence, voire plus crûment l'ignorance, à savoir celle de la sexualité\footcite[13-14]{Lecercle1998}.
\par
C'est surtout sur le plan de la volonté sexuelle que l'enfant-femme se distingue le plus de la femme-enfant et qu'elle correspond en quelque sorte à son inverse puisque la sensualité n'est en rien susceptible d'intéresser l'enfant-femme, si ce n'est que pour la curiosité terminologique d'ajouter de nouveaux mots à son vocabulaire.
L'enfant-femme est donc tout l'opposé de la nymphette, \guil{très jeune fille  au physique attrayant, aux manières aguicheuses, à l'air faussement candide\footcite[Entrée: « Nymphette »]{Robert2013}}, que l'on nomme également \guil{lolita\footnote{Cette synonymie tient son origine d'une lecture particulière du roman \textit{Lolita}, de Vladimir Nabokov, voulant que la jeune Dolorès \textit{séduise} Humbert Humbert, plutôt que de considérer ce dernier comme un \guil{nympholepte}, à savoir un pédophile.}}.
\par
Enfin, nous estimons que l'existence même de l'enfant-femme est tributaire de sa distinction face aux personnages littéraires de la femme-enfant et de la lolita.
La figure de l'enfant-femme se construit ainsi par opposition à ces grandes figures dont elle se distingue complètement en matière de maturité et de volonté sexuelle.
Dès lors s'ajoutera aux particularités du personnage le rapport qu'elle entretient avec le langage: d'abord celui des adultes, puis celui, hybride entre l'enfance et l'âge adulte, qui lui est propre.

\chapter{Au croisement des âges: les personnages d'enfant-femme}
%
\begin{flushright}
                                    \begin{singlespace}
                                    \epigraph{
                                    L'adulte est mou. L'enfant est dur. Il faut éviter l'adulte comme on évite le sable mouvant. Un baiser qu'on met sur un adulte s'y enfonce, y germe, y fait éclore des tentacules qui prennent et ne vous lâchent plus. Rien ne pénètre un enfant; une aiguille s'y briserait, une francisque s'y briserait, une hache s'y briserait. L'enfant n'est pas mou, visqueux et fertile, il est dur, sec et stérile comme un bloc de granit.}
                                    \par
                                    Réjean Ducharme, \textit{L'Avalée des avalés}\footcite[336-337]{Ducharme1982}
                                    \end{singlespace}
\end{flushright}

Tel que nous l'avons explicité précédemment, le personnage d'enfant-femme comme nous le définissons est avant tout une figure d'enfant. Ainsi, ses traits adultes ne constituent pas l'essence même du personnage et existent en toute innocence, d'abord et surtout dans le regard de l'Autre, à savoir dans sa façon de la percevoir.
Façon de revisiter Alice Liddell et ses aventures au Pays des Merveilles, l'enfant-femme est certes plus autonome que ce que l'on attend généralement d'un enfant de son groupe d'âge, mais il s'agit tout de même d'une autonomie enfantine, en ce sens qu'elle est contextuelle, constituée de méfiance et d'adaptation plutôt que de réelle volonté de liberté.
\par
Personnage dont la filiation avec la jeune fille des écrits carrolliens est déjà établie\footnote{D'autant plus que Queneau avait auparavant fait paraître le pastiche \textit{Alice en France}.}, il ne fait aucun doute que \textit{Zazie dans le métro\footcite{Queneau1959}} de Raymond Queneau contribue à enrichir notre typologie d'enfant-femme, voire même qu'elle en constitue l'emblème dans la littérature française.
Si d'autres personnages de jeunes filles pourraient à notre avis être rattachés, à notre typologie\footnote{Nous pensons ici au personnage de Bérénice Einberg de \textit{L'avalée des avalés} de Réjean Ducharme, laquelle a été régulièrement associée à Zazie par la critique.}, nous croyons que la narratrice enfant du roman \textit{Borderline\footcite{Labreche2003}} de Marie-Sissi Labrèche constitue l'une des plus fortes manifestations récentes de cette figure en littérature québécoise.
Cette lecture peut sembler hautement improbable, nous en convenons, surtout que l'\oe{}uvre de Labrèche a d'abord été analysée en fonction de son contexte de publication\footnote{En tant qu'autofiction impudique, Labrèche s'inscrit dans la lignée des françaises Christine Angot (\textit{L'inceste}, 1999), Catherine Millet (\textit{La Vie sexuelle de Catherine M.}, 2001) et Chloé Delaume (\textit{Le Cri du sablier}, 2001), et précède la québécoise Nelly Arcan (\textit{Putain}, 2001 et \textit{Folle}, 2004).} sous les angles de la théorie psychanalytique ou des études féministes, perspectives mettant en lumière soit les rapports au corps, à la sexualité et à la mère.
Si ces théories se prêtent effectivement très bien à l'étude des romans de Labrèche en tant qu'autofictions narrées par une adulte, nous croyons que le roman \textit{Borderline} ouvre également la porte à une toute autre analyse, celle-ci portée exclusivement sur la narratrice enfant.
Ainsi, profitant de cette \textit{brèche} inexplorée par la critique, nous proposons d'établir l'appartenance de la narratrice enfant de \textit{Borderline} au grand type de l'enfant-femme, jusqu'à en faire à son tour un modèle de cette typologie dans la littérature québécoise du nouveau millénaire.


\section{Zazie, l'enfance à tout vent}
Petite fille d'une dizaine d'années\footcite[88]{Maurin2007}, Zazie est un personnage à la fois archifictionnel et hautement réaliste.
D'abord fictionnelle, parce qu'elle est un personnage très \guil{construit}, particulièrement sur le plan linguistique, elle est néanmoins rattachable au monde réel par le réalisme de son comportement, qui rappelle sans nul doute celui d'un  véritable enfant:
\guil{\textit{Zazie's behavior is an exaggerated but recognizable form of what one might expect from a child her age placed (dépaysée) in an unfamiliar adult world; her frustrations are familiar. $\left[ \dots \right]$ Zazie displays a true child's wonder\footnote{\guil{Le comportement de Zazie est une forme exagérée mais reconnaissable de ce qu'on pourrait s'attendre de la part d'une enfant de son âge placée (\textit{dépaysée}) dans le monde inconnu des adultes; ses frustrations sont familières.}~(notre traduction) dans \mancite \cite[121-122]{Bernofsky1994}.}.}}
De la même façon, Aurélie Maurin décrit Zazie comme une enfant \guil{de son temps}, représentante de sa génération:
\guil{En digne enfant du baby-boom, elle aime les \guill{bloudjins} et le soda, autrement dit le rêve américain, mais elle est provinciale, issue d'un milieu modeste qu'on imagine ouvrier\footcite[88]{Maurin2007}.}
Ainsi, Zazie est à la fois caricaturale et fidèle à la réalité des enfances françaises de l'après-guerre.


\subsection{Zazie la mouflette}
Quoique son comportement et sa façon de parler nous en fassent parfois douter, Zazie est d'abord et avant tout une enfant et son \guil{jeune âge} lui permet notamment d'être instantanément excusée pour une mauvaise plaisanterie~(\textit{Z}:~98).
Tel que nous l'avons déjà évoqué au chapitre précédent, elle s'inscrit, selon la typologie de Bethlenfalvay, dans la \guil{lignée du gamin}, héritier direct de l'enfant du monde du XIX\up{e} siècle\footcite[117]{Bethlenfalvay1979}.
D'un point de vue physique, autant sa démarche galopante~(\textit{Z}:~10) que le fait que Trouscaillon et la veuve Mouaque la soulèvent chacun par un bras pour la traîner~(\textit{Z}:~106) trahissent sa petite taille.

\subsubsection{Zazie affublée de noms: la ptite, la gosse, la mouflette}
Tout au long du roman, Zazie est affublée de surnoms ou désignations, dont plusieurs la réduisent à son jeune âge: la petite ou la \guil{ptite}, l'enfant, la mouflette, la gosse, la fillette ou la petite fille, la môme, la gamine et autres variations sur un même thème.
On dit d'elle qu'elle est mineure~(\textit{Z}:~89) et on l'appelle \guil{mademoiselle}~(\textit{Z}:~44, 95, 126, 127, 128, 143, 161, 174), \guil{petite}~(\textit{Z}:~11, 13, 20, 22, 30, 110) voire \guil{petite demoiselle}~(\textit{Z}:~126) ou même \guil{princesse des djinns bleus}~(\textit{Z}:~76).
On la traite de \guil{gosseline}~(\textit{Z}:~151), de \guil{petite voleuse}~(\textit{Z}:~53), de \guil{petite salope}~(\textit{Z}:~18, 25) ou encore de \guil{petit être stupide}~(\textit{Z}:~99).
Son oncle Gabriel, qui lui est malgré tout plutôt sympathique, la qualifie pour sa part de \guil{ptite vache}~(\textit{Z}:~92), de \guil{vraie petite mule}~(\textit{Z}:~98) et, non sans ironie, de \guil{petit ange}~(\textit{Z}:~23).
En somme, Zazie est constamment ramenée à son statut de \guil{petite}, c'est-à-dire à son enfance: que ce soit en la qualifiant d'enfant ou en l'interpellant par un surnom infantilisant -- comme le \guil{Eh petite}~(\textit{Z}:~30) répété par le satyre --, tout ce qui la désigne est linguistiquement connoté, soit de façon péjorative, soit dans une perspective réductrice.
Elle n'est pas une fille et encore moins une femme, comme le lui rappelle rapidement Charles lorsque Zazie se compare à elles: \guil{Toutes les femmes, voyez-vous ça, toutes les femmes. Mais tu n'es qu'une mouflette.}~(\textit{Z}:~83) Incidemment à son statut d'enfant et dans un rappel constant de sa petitesse, Zazie est avant tout désignée comme étant une fillette ou une \textit{petite} fille.

\subsubsection{Zazie cinématographique: l'\guil{in-désirable}}
Lors de sa transposition au cinéma sous la direction de Louis Malle\footcite{Malle1960} et avec l'accord de l'auteur, Zazie est rajeunie de quatre ans afin d'éviter toute ambiguïté sur le fait qu'elle soit une enfant:
\begin{quote}
  \begin{singlespace}
    \small
    Nous avons carrément rajeuni le personnage de quatre ans. Je voulais éviter tout côté \guil{Lolita}. Notre Zazie est donc une petite fille de dix ans qui dit n'importe quoi, sans équivoque, qui est absolument hors du monde des adultes et qui n'a jamais tort devant lui\footcite{Gilson1960}.
    \normalsize
  \end{singlespace}
\end{quote}
Dans le roman, sous son alias d'inspecteur Bertin Poirée, le \textit{type} nie avoir de l'intérêt pour Zazie afin de séduire Marceline: \guil{On s'en fout de Zazie. Les gosselines, ça m'éc\oe{}ure, c'est aigrelet, beuhh. Tandis qu'une belle personne comme vous\dots \ crènom.}~(\textit{Z}:~151)
L'opposition entre Zazie et Marceline exacerbe la jeunesse de la première et dénote qu'elle soit tout le contraire d'une femme, qu'elle ne soit pas \textit{désirable} comme l'est une femme.
Comme le réitère Louis Malle en entrevue, c'est dans l'objectif d'éviter l'effet \guil{lolita} ou femme-enfant que l'actrice Catherine Demongeot, âgée de dix ans lors du tournage, a été sélectionnée:
\begin{quote}
  \begin{singlespace}
    \small
    Dans le livre, Zazie est plus âgée que dans le film. J'ai demandé à Queneau la permission de la rajeunir. Au cinéma, on n'a guère le choix entre la petite fille et le petit monstre: une starlette de quatorze ans a l'air d'une femme. Pour moi, Zazie n'a plus le côté un peu trouble, un peu \guill{Lolita} de la Zazie de Queneau\footcite[227]{Bigot1994}.
    \normalsize
  \end{singlespace}
\end{quote}
Ce choix d'une actrice plus jeune démontre une volonté très claire de montrer un personnage dont l'allure physique rappelle résolument son statut d'enfant, avant tout dans l'objectif d'évacuer toute ambiguïté sexuelle qui serait exacerbée par l'importance de l'aspect visuel du médium qu'est le cinéma.

\subsubsection{Zazie et l'\textit{asessualité}: curieuse mais pas provoc'}
Bien qu'elle soit quelque peu confuse et surtout curieuse à propos de la sexualité, Zazie n'est définitivement pas une séductrice: \guil{\textit{This barely pubescent fictional creature is no seductress $\left[ \dots \right]$; she's as confused about sex and sexuality as about most everything else\footnote{\guil{Cette créature fictive à peine pubère n'est aucunement séductrice $\left[ \dots \right]$; elle est aussi confuse à propos du sexe et de la sexualité qu'à propos d'à peu près tout le reste.}~(notre traduction) dans \mancite \cite[119]{Bernofsky1994}.}.}}
Elle a certes acquis un certain vocabulaire à propos de la sexualité en écoutant les adultes, mais elle ne connaît pas forcément \guil{la chose} et ne maîtrise pas l'usage de ce vocabulaire qu'elle ne fait que reproduire: \guil{\textit{But Zazie knows only the words for things, not what they signify\footnote{\guil{Zazie connaît seulement les mots, pas leur signification.}~(notre traduction) dans \mancite \cite[119]{Bernofsky1994}.}.}}
Elle ne correspond donc définitivement pas au mythe de l'enfant provocatrice\footcite[90]{Maurin2007} et ne devrait pas être assimilée à une lolita\footnote{\cite[118-119]{Bernofsky1994}; \cite[90]{Maurin2007}}. En fait, \guil{Zazie petite fille aurait aussi bien pu être née garçon\footcite[90]{Maurin2007}} puisqu'elle \guil{semble encore loin d'être \guill{sexuée}\footcite[90]{Maurin2007}}.
En ce qui a trait à son apparence physique asexuée, la surprise d'un concitoyen sanctimontronais des Lalochère lorsqu'il rencontre Zazie est sans équivoque: \guil{Tiens, dit le conducteur, mais c'est la fille de Jeanne Lalochère. Je l'avais pas reconnue, déguisée en garçon.}~(\textit{Z}:~106)
\par
Chez Zazie, la sexualité n'est qu'un autre sujet à découvrir, objet de curiosité à l'instar du métro: \guil{Elle les pioche avant tout dans un insatiable appétit de la nouveauté. Zazie est curieuse de tout, ses principaux intérêts portent sur les choses de la modernité: le métro, les produits américains, mais aussi et surtout sur les questions de sexualité\footcite[89-90]{Maurin2007}.}
Ayant lu dans le journal que \guil{les chauffeurs de taxi izan voyaient sous tous les aspects et dans tous les genres, de la sessualité}, Zazie demande à Charles si ça lui est arrivé souvent de voir des \guil{clientes qui veulent payer en nature.}~(\textit{Z}:~83)
Se posant du côté du questionnement curieux plutôt qu'indécent, Zazie nous semble avant tout avoir une grande soif de savoir, plus précisément de \textit{tout} savoir.
Ainsi, lorsque Zazie questionne le taximan pour savoir si elle lui plaît -- \guil{Et moi, dit Zazie, je vous plairais?}~(\textit{Z}:~82) --, c'est davantage par provocation que par intention de séduction\footcite[90]{Maurin2007}, surtout qu'elle nie tout naturellement la simple éventualité qu'il puisse lui-même lui plaire lorsque Charles lui retourne la question: \guil{Alors? moi? je te plairais? / Bien sûr que non, répondit Zazie avec simplicité.}~(\textit{Z}:~82)
Enfin, la curiosité de Zazie s'inscrit en particulier dans sa poétique langagière, à savoir son usage démesuré de l'interrogation lorsqu'elle se fait Grande Inquisitrice au sujet de l'hormosessualité.


\subsection{Au Pays des grandes personnes: quand l'enfance mène le monde}
Bien que Zazie soit hors de tout doute une enfant, il demeure qu'elle possède plusieurs traits rappelant l'adulte. Mature et peu impressionnable, elle a une bonne compréhension du monde qui l'entoure, notamment des nombreux sujets de discussion des adultes.

\subsubsection{Les Aventures de Zazie au Pays du métro}
\guil{Remarquablement vive pour son âge\footcite[191]{Ajame1981}}, Zazie fait parfois figure d'enfant-adulte posée dans un monde d'adultes-enfants.
Décrite comme une petite fille \guil{effrontée, dégourdie et curieuse}, elle n'hésite pas à s'affirmer, comme lorsqu'elle détourne la question de Gabriel qui cherche à savoir l'heure de son couvre-feu chez sa mère: \guil{Ici et là-bas ça fait deux, j'espère.}~(\textit{Z}:~20)
D'une nature peu craintive, Zazie ne perd pas une seconde et, dès son réveil dans l'appartement silencieux de Gabriel, part à l'aventure dans Paris.
Soucieuse de ne pas se faire prendre, elle est prudente dans sa fugue, employant des \guil{précautions considérables} puis de \guil{non moins grandes précautions que précédemment} afin d'éviter de faire du bruit~(\textit{Z}:~28-29).
\par
Rusée, Zazie utilise parfois l'enfance afin de parvenir à ses fins.
Bonne comédienne, capable de \guil{doubl[er] le volume de ses larmes}~(\textit{Z}:~41) pour émouvoir, elle expose une vulnérabilité feinte, notamment lorsqu'elle requiert l'intervention de son oncle: \guil{Faut  mdéfendre, tonton Gabriel. Faut mdéfendre.}~(\textit{Z}:~62)
Fière de sa ruse, elle se félicite d'ailleurs d'avoir été \guil{aussi bonne que Michèle Morgan dans \textit{La Dame aux Camélias}.}~(\textit{Z}:~62)
Cependant, son plus impressionnant fait d'armes comme actrice est probablement la feinte qu'elle emploie pour se défaire de Turandot, lequel tentait simplement de rattraper la petite fugueuse.
Jouant la carte de l'enfance, Zazie détourne l'attention d'elle en accusant le \guil{meussieu}, Turandot, de lui avoir \guil{dit des choses sales}, se posant en fillette trop gênée pour répéter ce qu'on lui a dit: \guil{C'est trop sale, murmure Zazie.}~(\textit{Z}:~31)
La foule de badauds étant trop occupée à discuter de cette histoire scandaleuse, personne ne se rend compte que Zazie a fui à nouveau.
\par
Décisionnaire, autonome et opportuniste\footcite[88]{Maurin2007}, Zazie proteste lorsqu'il le faut, hurlant et glapissant, \guil{folle de rage}~(\textit{Z}:~11, 90, 106).
Elle estime ne pas avoir besoin de son oncle, représentant de l'autorité parentale, et affirme que sa tante et elle peuvent sortir seules et se passer de lui~(\textit{Z}:~22).
Elle revendique ses droits -- \guil{J'aime pas qu'on me traite comme ça}~(\textit{Z}:~106) -- et demande ce qu'elle veut avant même qu'on le lui offre, que ça soit une glace fraise-chocolat~(\textit{Z}:~114) ou son café qu'elle prend \guil{avec de la peau dessus}~(\textit{Z}:~174).
Ayant déterminé que l'assiette qu'on lui avait servie \guil{était de la merde}, elle tient à sa liberté d'expression -- \guil{Vous m'empêcherez tout de même pas de dire, dit Zazie, que c' (geste) est dégueulasse.} Ainsi, refusant de bouffer \guil{cette saloperie} et réclamant impérieusement \guil{ottchose} puisque ce qu'on lui avait servi était \guil{tout simplement de la merde}~(\textit{Z}:~125-127), Zazie revendique en toute impolitesse son autonomie et son droit de décider pour elle-même.

\subsubsection{Le monde adulte, Pays des cauchemars?}
Parfois qualifiée de \guil{gamine délurée, sans vergogne dans sa curiosité insolente\footcite[114]{Pestureau1983}}, Zazie est comparée à Alice et à Ulysse, d'abord pour son audace mais surtout pour son \guil{goût de la découverte} qui l'entraîne dans une \guil{descente aux Enfers} vers le métro, \guil{terrier moderne pour lapins humains\footcite[113]{Pestureau1983}}.
Dans la lignée des \guil{fééries inquiétantes} de Lewis Carroll, Raymond Queneau \guil{écrit le monde à partir d'un regard enfantin qui en révèle l'envers, l'absurde, le burlesque ou l'horreur\footcite[113]{Pestureau1983}.}
Le Paris de Zazie, sorte de Pays des merveilles urbain et moderne, est avant tout un monde d'adultes, féroce, instable, violent et ambigu\footcite[113]{Pestureau1983}.
Comme Alice avant elle, Zazie se retrouve plongée ou même perdue dans un univers inquiétant qui n'a rien d'enfantin: \guil{\textit{Like \textit{Alice's Adventures in Wonderland}, \textit{Zazie dans le métro} centers around a single child in the midst of adults whose doings bewilder her\footnote{\guil{Comme \textit{Les Aventures d'Alice au Pays des Merveilles}, \textit{Zazie dans le métro} tourne autour d'une enfant seule au milieu d'adultes dont les agissements la déconcertent.}~(notre traduction) dans \mancite \cite[119]{Bernofsky1994}.}.}}
Entraînée dans ce \guil{\textit{clearly unreal environment} (environnement clairement irréel)\footcite[119]{Bernofsky1994}}, elle échoue à trouver sa voie, en ce sens qu'elle ne parvient pas à son but puisqu'elle est endormie lorsqu'elle prend enfin le métro et qu'elle n'obtient pas de réponse sur ce qu'est un \textit{hormosessuel}.
Elle quitte donc la ville aussi désorientée qu'à son arrivée\footcite[119]{Bernofsky1994}, sans réponse ni à sa quête du métro, ni à son questionnement sur les choses adultes de la sexualité et du langage.

\subsubsection{\textit{Essméfie}, en aventurière avertie}
Plongée dans un tel monde, Zazie montre maints traits de caractère que l'on retrouve normalement chez les personnes plus âgées: au-delà de sa curiosité, elle est aussi méfiante, questionnant longuement les adultes et doutant de ce qu'ils lui disent:
\begin{quote}
  \begin{singlespace}
    \small
    On peut $\left[ \dots \right]$ affirmer que la jeune héroïne n'a de l'enfance que l'âge et le questionnement. Elle possède des caractéristiques attribuées habituellement aux adultes: elle est décisionnaire et revendique une certaine autonomie, elle n'est en rien naïve et sait même se montrer soupçonneuse; ses prises de risques sont le plus souvent opportunistes. Du haut de ses 10 ans, elle mène une troupe d'adultes en mal de repères. Sans montrer aucune faiblesse, elle clame à qui veut bien l'entendre qu'elle n'a peur de rien, car, elle, Zazie, en a vu d'autres\footcite[88]{Maurin2007}!
    \normalsize
  \end{singlespace}
\end{quote}
Elle est aussi difficile à berner, surtout en ce qui concerne son objet fétiche qu'est le métro:
\begin{quote}
  \begin{singlespace}
    \small
    Zazie fronce le sourcil. Essméfie.
    \guil{Le métro? qu'elle répète. Le métro, ajoute-t-elle avec mépris, le métro, c'est sous terre, le métro. Non mais.\\
    -- Çui-là, dit Gabriel, c'est l'aérien.\\
    -- Alors, c'est pas le métro.\\
    -- Je vais t'esspliquer, dit Gabriel. Quelquefois, il sort de terre et ensuite il y rerentre.\\
    -- Des histoires.} ~(\textit{Z}:~12)
    \normalsize
  \end{singlespace}
\end{quote}
Suspicieuse et rationnelle, elle \guil{réserv[e] son opinion}~(\textit{Z}:~17), un peu comme le ferait un adulte.
\par
Au-delà de son seul caractère en avance sur son âge, Zazie a également une compréhension étonnamment profonde du monde des adultes: \guil{Elle fait parfois des raccourcis mais cerne bien la complexité des \guill{choses de la vie}\footcite[90]{Maurin2007}.}
Elle ne correspond ainsi pas au standard de l'enfant naïf, faible et pur\footcite[90]{Maurin2007}, pour l'essentiel ignorant des choses du monde des \guil{grandes personnes}.
Au contraire, Zazie possède un imaginaire des choses adultes particulièrement étoffé pour son âge: \guil{C'est hun dégueulasse qui m'a fait des propositions sales, alors on ira devant les juges tout flic qu'il est, et les juges je les connais moi, ils aiment les petites filles, alors le flic dégueulasse, il sera condamné à mort et guillotiné et moi j'irai chercher sa tête dans le panier de son et je lui cracherai sur sa sale gueule, na.}~(\textit{Z}:~62)
Pédophilie, système judiciaire, peine de mort: rien ne lui échappe!
\par
Ainsi, Zazie est précoce et lucide, voire \guil{affranchie, jetée dans un monde féroce\footcite[113]{Pestureau1983}}.
C'est également un personnage d'une grande curiosité, \guil{\textit{a fictional creature filled with curiosity, desire and a profound, never-relieved perplexity\footnote{\guil{une créature fictive remplie de curiosité, de désir et d'une profonde perplexité inassouvie}~(notre traduction) dans \mancite \cite[114-115]{Bernofsky1994}.}.}}
Ayant en commun avec son ancêtre anglaise \guil{l'audace et le goût de la découverte\footcite[113]{Pestureau1983}}, Zazie est présentée par Pestureau comme une sorte d'\guil{Alice amère et insolente, $\left[ \dots \right]$, à la fois rêveuse et cynique, voulant pénétrer le souterrain des adultes, soumise à leur jeu mais retournant leurs armes contre eux, [qui] tente de les déchiffrer au cours du bref cycle de ses aventures, d'un train l'autre\footcite[114]{Pestureau1983}.}
\par
Outre sa grande curiosité envers le monde des adultes et ce qu'il représente, Zazie n'est pas dupe des intentions de ces derniers: \guil{En effet, elle semble avoir une conscience aiguë des désirs malveillants de certains adultes, elle en use parfois, mais ne se laisse pas abuser par leurs intentions\footcite[90]{Maurin2007}.}
Elle flaire donc les dangers potentiels et y oppose une attitude méfiante.
Au satyre qui lui demande de lui faire confiance, elle répond sournoisement par la question \guil{Pourquoi que j'aurais confiance en vous?}, l'accusant ensuite d'être \guil{un vieux salaud} lorsqu'il affirme aimer les enfants, petits garçons comme petites filles~(\textit{Z}:~42).
Quoiqu'elle soit alléchée à l'idée d'avoir enfin des \textit{bloudjinnzes}, Zazie ne se laisse pas leurrer par les fausses bonnes intentions du \textit{type}: \guil{C'est sûrement un sale type, pas un dégoûtant sans défense, mais un vrai sale type. Faut sméfier, faut sméfier, faut sméfier.}~(\textit{Z}:~45-46)\
Quand ce dernier raconte à Gabriel le délit de Zazie, elle met en garde son oncle: \guil{Méfie-toi, tonton Gabriel.}~(\textit{Z}:~56)
Ayant préalablement conseillé à son oncle de ne répondre que devant son avocat~(\textit{Z}:~56), Zazie réitère son conseil lorsque le même \textit{type} questionne Marceline: \guil{Faut parler que devant ton avocat, dit Zazie. Tonton a pas voulu m'écouter, tu vois comme il est emmerdé maintenant.}~(\textit{Z}:~63)
\par
Outre sa méfiance prudente, Zazie possède également~(\textit{Z}:~63) une franche lucidité face à l'insolente bouffonnerie des autres personnages du roman.
Étant extérieure au monde des adultes, l'héroine de Queneau en met ainsi au jour le caractère ridiculement burlesque, stylistiquement amplifié au cinéma par le choix de la trame sonore:
\begin{quote}
  \begin{singlespace}
    \small
    Ce monde lui paraît rigoureusement absurde, fait de gens qui ne savent rien d'eux-mêmes et qui vivent dans le chaos. Son entrée se fait sur une musique de western et elle a un côté justicier de western : elle arrive dans la ville et se désolidarise de ses habitants. Plus elle les provoque, se moque d'eux, les injurie et par là augmente encore le chaos. $\left[ \textit{sic} \right]$ Elle les juge sévèrement. Jamais elle ne joue leur jeu; mais, à la fin, il était temps, elle allait commencer à se laisser prendre: « J'ai vieilli ! »\footcite{Gilson1960}
    \normalsize
  \end{singlespace}
\end{quote}
Ainsi, Zazie serait, selon Bernofsky, un personnage anti-anti-réel, en ce sens qu'elle n'est pas purement réaliste, mais qu'elle est plutôt en combat contre l'anti-réel:
\begin{quote}
  \begin{singlespace}
    \small
    \textit{In fact she is unique among the characters in the novel, a realist figure placed within a subversively self-reflective anti-realist milieu. This double inversion within the novel has considerable ontological and aesthetic implications both in the context of Queneau's work and for the novel in general; when the anti-real is taken as the norm, the realist tendency becomes itself a subversion: the anti-anti-real\footnote{\guil{En fait, elle est unique parmi les personnages du roman, figure réaliste placée dans un milieu anti-réaliste subversivement auto-réflexif. Cette double inversion a des implications ontologiques et esthétiques considérables, à la fois dans le contexte de l'\oe{}uvre de Queneau et du roman en général; lorsque l'anti-réel est pris comme norme, le réalisme devient lui-même une subversion: l'anti-anti-réel.}~(notre traduction) dans \mancite \cite[115]{Bernofsky1994}.}.}
    \normalsize
  \end{singlespace}
\end{quote}
Shérif de la Raison à la poursuite des cowboys du \textit{nonsense}, Zazie est donc, comme Alice, posée en gardienne de la rationalité dans un monde en déroute:
\begin{quote}
  \begin{singlespace}
    \small
    Zazie n'est pas dupe. Et elle évolue également dans une société où les adultes ont vécu la guerre, l'occupation, et la privation. Ils prennent, tant qu'ils peuvent, leur revanche sur la liberté. Il reconstruisent leurs mythes à grand renfort de frasques. Seule parmi des adultes fantaisistes, et parfois absents à eux-mêmes, Zazie doit trouver son ancrage, son étayage dans la réalité, par elle-même\footcite[89]{Maurin2007}.
    \normalsize
  \end{singlespace}
\end{quote}
D'où notre hypothèse selon laquelle Zazie est sa propre bouée de sauvetage, ne pouvant se fier que sur elle-même dans ce monde d'adultes pour la plupart dépourvus de bon sens.

\subsubsection{Les adultes, tous des cons!}
Aux yeux de Zazie, la plupart des adultes sont méprisables, d'où son attitude globalement peu respectueuse envers eux.
Ainsi, tandis que Charles le chauffeur de taxi conduit une \guil{charrette dégueulasse} dont il faudrait selon la jeune fille faire baisser le tarif~(\textit{Z}:~14), le bar de Turandot est minable puisqu'on n'y danse pas~(\textit{Z}:~26), de même que l'est le \textit{type} pour sa méconnaissance d'un journal régional~(\textit{Z}:~48).
Quand ce dernier lui dit s'intéresser au sport, et plus particulièrement au catch, Zazie ricane en se moquant de lui, jugeant à la vue de son \guil{gabarit médiocre} qu'il doit être \guil{dans la catégorie spectateurs}~(\textit{Z}:~49).
De la même façon, elle s'esclaffe lorsque Gabriel prétend avoir déjà eu à \guil{repousser les assauts de ces gens-là\footcite[93]{Queneau1959}}.
Enfin, à son oncle Gabriel qui la questionne, Zazie rétorque, à la limite de la condescendance, qu'il \guil{trouverai[t] pas tout seul, hein?}~(\textit{Z}:~20)
\par
Devant la déraison généralisée des adultes qui l'entourent, Zazie en vient presqu'à douter qu'ils puissent ne pas tous être complètement idiots: \guil{Dis donc, tonton, demande Zazie, quand tu déconnes comme ça, tu le fais esprès ou c'est sans le vouloir?}~(\textit{Z}:~14)
Acquiéscant sereinement aux propos de Charles qui décrète que \guil{[l]es passants $\left[ \dots \right]$ c'est tous des cons}~(\textit{Z}:~12-13), Zazie affirme que la foule qui fréquente la rue est constituée de \guil{braves gens avec des têtes de cons}~(\textit{Z}:~53), se demandant plus tard à propos des gens jugeant le parfum de Gabriel \guil{ce que ces cons-là peuvent bien y connaître}~(\textit{Z}:~165).
À son oncle qui décrète que Saint-Germain-des-Prés est démodé, Zazie rétorque promptement qu'elle pourrait lui répondre à lui qu'il \guil{n'es[t] qu'un vieux con}~(\textit{Z}:~15), de la même façon que l'avocat parisien célèbre de sa mère était lui aussi \guil{un con, quoi}~(\textit{Z}:~47).
La veuve Mouaque est pour sa part \guil{moins conne que [Zazie] le croyai[t]}~(\textit{Z}:~100), quoiqu'on se doute bien qu'elle le soit malgré tout encore un peu aux yeux de la jeune fille: \guil{Ce qu'elle peut déconner, dit Zazie faiblement.}~(\textit{Z}:~169)
On pourrait donc résumer la pensée zazique en cette formule choc: les adultes, tous des cons!
\par
Traitant Gabriel et Charles de \guil{ptits marants}~(\textit{Z}:~13), Zazie se pose parfois en adulte face à eux, les rappelant à l'ordre en leur intimant de ne pas recommencer~(\textit{Z}:~79, 91) et soulignant leur immaturité d'un commentaire incisif -- \guil{Les petits farceurs de votre âge, dit Zazie, ils me font de la peine.}~(\textit{Z}:~80)
Dans son rejet des enfantillages des adultes, elle fait preuve de maturité et figure en personne raisonnée face à des adultes en perdition.
Ainsi, face au Sanctimontronais qui prétend qu'il y a le métro à Saint-Montron, la fillette est incisive: \guil{Ça alors, dit Zazie, c'est le genre de déconnances qui m'éc\oe{}urent particulièrement.}~(\textit{Z}:~110)
Démasquant le mensonge et remarquant l'incohérence des adultes, la jeune héroïne confirme son statut d'enfant-adulte par le renversement de perspectives suivant: \guil{Zazie, par sa précocité, met en relief le fait qu'elle soit une enfant se comportant comme une adulte, dans un monde d'adultes se comportant comme des enfants\footcite[91]{Maurin2007}.}

\subsubsection{En-deça de l'amour: hors des enfantillages des grands}
Lucide face aux technicalités de la vie des adultes, notamment à propos de la raison qui pousse sa mère à la confier à Gabriel -- \guil{C'est comme ça qu'elle est quand elle a un jules, dit Zazie, la famille ça compte plus pour elle}~(\textit{Z}:~9) --, Zazie n'en est pas moins dédaigneuse du sentiment amoureux en général, qu'elle juge absurde et dont elle se moque sans gêne.
Aussi raille-t-elle les émois de la veuve Mouaque qui attend Trouscaillon:
\begin{quote}
  \begin{singlespace}
    \small
    \guil{- C'est le flicard qui vous met dans cet état? \\
    - Ah! l'amour\dots \ quand tu connaîtras\dots \\
    - Je me disais bien qu'au bout du compte vous alliez me débiter des cochonneries. Si vous continuez, j'appelle un flic\dots \ un autre\dots \\
    - Cest cruel}, dit la veuve Mouaque amèrement. \\
    Zazie haussa les épaules. \\
    \guil{Pauv'vieille\dots \ $\left[ \dots \right]$} \\
    Avant que la Mouaque utu le temps de répondre, Zazie avait ajouté: \\
    \guil{Tout de même\dots \ un flicard. Moi, ça me débecterait.}~(\textit{Z}:~121)
    \normalsize
  \end{singlespace}
\end{quote}
Loin de mal comprendre le sentiment amoureux comme on le supposerait de la part d'une enfant, Zazie parvient à \guil{lire} l'agitation de la veuve Mouaque et à verbaliser avec justesse l'émotion qui habite cette dernière.
Elle se place cependant au-dessus de cette futilité qu'est l'amour en rejetant rapidement le postulat bienveillant de la veuve qui suppose que la petite ne le connaît pas encore: ce n'est pas qu'elle ne comprend pas le sentiment, c'est plutôt qu'elle ne l'estime pas digne de son intérêt et qu'il ne constitue surtout pas, à ses yeux, une raison justifiable de se mettre \guil{dans cet état}.
\par
Directe et sans pitié, Zazie signale finalement sa lassitude par rapport à la discussion et son mépris de l'amour à la veuve Mouaque, mettant abruptement fin aux doléances de cette dernière:
\begin{quote}
  \begin{singlespace}
    \small
    Zazie haussa les épaules. \\
    \guil{Tout ça, c'est du cinéma, qu'elle dit. Vous auriez pas un autre sujet de conversation? \\
    - Non, dit énergiquement la veuve Mouaque. \\
    - Eh bien alors, dit non moins énergiquement Zazie, je vous annonce que la semaine de bonté est terminée. A rvoir.}~(\textit{Z}:~122)
    \normalsize
  \end{singlespace}
\end{quote}
Indélicate mais non moins diplomate, Zazie signifie ainsi à la veuve Mouaque son désintérêt pour les choses de l'amour, s'élevant au-delà de ce sujet futile et le mettant à distance puisqu'il est sans aucun doute indigne d'elle.
\par
Enfin, à la limite du mépris, Zazie se moque des effets de l'amour sur Mme Mouaque: \guil{Elle est suprême, celle-là. $\left[ \dots \right]$ C'est l'amour qui rend comme ça?}~(\textit{Z}:~122)
Soulignant sans complexe l'ignorance de la veuve par l'interrogation rhétorique \guil{Qu'est-ce que vous en savez? Il vous a fait des confidences? Déjà?}~(\textit{Z}:~123), Zazie-moqueuse en rajoute avec l'adverbe \guil{[d]éjà?} qui a pour effet de décupler la portée de son mépris pour les \guil{enfantillages} des adultes.
Enfin, revenue à de meilleures dispositions envers la veuve Mouaque sur le point de partir rejoindre son Trouscaillon, Zazie lui souhaite en toute simplicité de \guil{bonnes fleurs bleues}~(\textit{Z}:~124), montrant par son usage de cette expression toute l'étendue de sa compréhension des relations amoureuses des adultes, qu'elle résume même à son oncle Gabriel en affirmant que la dame a \guil{un fleurte terrible avec le flicmane}~(\textit{Z}:~124).
De façon générale, Zazie est donc plutôt dédaigneuse face à l'irrationnel et surtout aux histoires de princesses. Après avoir divagué jusqu'à s'en être fait un conte de fées à elle-même, Zazie, \guil{récupérant son esprit critique, $\left[ \dots \right]$ finit par se déclarer que c'est drôlement con les contes de fées et décide de sortir.}~(\textit{Z}:~29)

\subsubsection{Une République zazique: hors de l'autorité parentale}
Si Zazie peut aller et venir aussi librement dans Paris, c'est avant tout parce qu'elle ne subit pas les contraintes de l'autorité parentale: \guil{Plutôt que l'émanation d'un modèle, il semblerait qu'elle incarne le fait que des enfants puissent être acteurs de leur quotidien et vivre en dehors de l'autocratie parentale\footcite[91]{Maurin2007}.}
Dès les premières pages du roman, Zazie est confiée à son oncle Gabriel par une mère \guil{désinvolte mais émancipatrice\footcite[114]{Pestureau1983}}, davantage pressée d'aller rejoindre son prétendant que de se soucier du bien-être de son enfant: \guil{Tu vois l'objet, dit Jeanne Lalochère s'amenant enfin. T'as bien voulu t'en charger, eh bien, le voilà.}~(\textit{Z}:~9)
Qualifiée d'\guil{objet} par sa mère, qui réitère son détachement envers son enfant lorsqu'elle vient \guil{récupérer la fille} plutôt que \guil{sa} fille~(\textit{Z}:~180), Zazie semble être aux yeux des adultes un simple encombrement: \guil{Et qu'est-ce que c'est que cette môme que tu trimbales avec toi?}~(\textit{Z}:~88)
Bien que l'oncle Gabriel prenne théoriquement le relais de Jeanne Lalochère et qu'il fasse figure de substitut de parent, il n'a en réalité que très peu d'ascendant sur sa nièce, que ce soit sur ses déplacements, son comportement ou son langage:
\begin{quote}
  \begin{singlespace}
    \small
    On pourrait plutôt parler d'un enfant sujet, qui semble mener sa vie en dehors de toute autorité ou considérations parentales. Zazie fait son éducation, seule, par elle-même et pour elle-même. Les figures masculines du roman (son oncle, Charles le taximan, Pedro-surplus le flicman, le cafetier) revendiqueront ce souci d'éducation, mais ne parviendront jamais à imposer à la demoiselle aucune marque d'obéissance. Zazie est, certes, indisciplinée, mais personne ne tente réellement de maintenir auprès d'elle quelques règles qui en soient vraiment. $\left[ \dots \right]$ Zazie et son impétuosité inquiètent les rangs de l'autorité, mais elle divertit aussi\footcite[89]{Maurin2007}.
    \normalsize
  \end{singlespace}
\end{quote}
D'ailleurs, il ne semble pas réellement souhaiter avoir sur elle une quelconque autorité, notamment lorsqu'il apprend sa fugue et qu'il suppose qu'elle \guil{se retrouvera bien toute seule}~(\textit{Z}:~36).
Ainsi, bien qu'il revendique le droit de l'élever comme il le veut -- \guil{Vous, dit Gabriel au flicmane, je vous prie de me laisser élever cette môme comme je l'entends. C'est moi qui en ai la responsibilitas. Pas vrai, Zazie?}~(\textit{Z}:~125) --, la question finale ajoutée à son affirmation, à savoir qu'il demande à Zazie d'approuver son propos, inverse à notre avis la relation d'autorité entre l'oncle et la nièce.
N'est-ce pas Zazie qui est de cette façon posée comme responsable de Gabriel et, par le fait même, de sa propre personne?
\par
Au final, Zazie évolue hors de toute autorité sans même avoir eu à se libérer d'une quelconque forme de tutelle puisque personne ne semble en fait avoir de contrôle sur elle ou même chercher à en avoir.
Devant cette absence quasi-totale d'encadrement, Zazie \guil{grandit} en s'auto-responsabilisant: \guil{L'autonomie et la rébellion deviennent alors une question de survie, ne pouvant compter sur la fiabilité des adultes, elle doit faire preuve d'autres ressources\footcite[89]{Maurin2007}.}
Disposant de sa liberté afin d'assouvir sa soif de découvertes et affirmant son autonomie par sa curiosité\footcite[91]{Maurin2007}, Zazie démontre tout au long du roman l'étendue de sa débrouillardise.
Conséquemment, elle s'autonomise, sans qu'il ne soit possible de déterminer si cette précocité est innée ou acquise chez elle: \guil{Devant le peu de constance du monde adulte, on ne sait donc pas si Zazie est autonome par disposition ou par obligation\footcite[89]{Maurin2007}.}


\subsection{Un enfant adaptatif: la philosophie du \guil{tout va de soi!}}
En plus de sa très grande disposition pour la liberté et l'aventure, ce qui distingue Zazie de la plupart des autres enfants est sa capacité d'adaptation et sa résilience face aux événements qu'elle vit.
À la façon de Gavroche ou des \guil{enfants du monde} zoliens, elle représente la \guil{vraie vie} et ne correspond pas au mythe de l'enfant martyr\footcite[90]{Maurin2007}.
Inébranlable malgré son histoire familiale particulière, Zazie n'est également pas décontenancée par les choses rationnelles de la vie, à l'égard desquelles elle nourrit généralement une certaine curiosité.
Ainsi, même si elle ne comprend pas le mot \guil{hormosessuel}, elle n'a pas de réaction particulière lorsqu'elle rencontre Marceline ou bien Marcel, compagne-compagnon de son oncle Gabriel~(\textit{Z}:~181).
Malgré leur aspect inédit pour elle, ce n'est ni l'homosexualité ni le travestissement de son oncle qui arriveront à perturber l'inébranlable Zazie.
\par
Ayant échappé de peu à son père qui tentait de l'agresser, pour ensuite être témoin de l'assassinat de ce dernier par sa mère, Zazie semble très peu affectée par son histoire familiale, la racontant comme s'il s'agissait du récit d'événements anodins:
\begin{quote}
  \begin{singlespace}
    \small
    Zazie a une histoire de vie complexe et douloureuse, mais n'en fait ni étalage ni gloriole. Pour elle, tout semble aller de soi, il lui suffit d'une tirade pour exposer les faits, presque sans affects: après que son père, alcoolique, tente de la violer, sa mère l'assassine d'un coup de hache, s'ensuivent d'innombrables procédures judiciaires et pénales\footcite[88]{Maurin2007}.
    \normalsize
  \end{singlespace}
\end{quote}
C'est donc avec une attitude détachée que Zazie entame son récit, davantage intéressée par sa bière que par le drame qu'elle raconte:
\begin{quote}
  \begin{singlespace}
    \small
    \guil{Vous en mettez du temps pour écluser votre godet. Papa, lui, il en avalait dix comme ça en autant de temps.\\
    -- Il boit beaucoup ton papa?\\
    -- I buvait, qu'il faut dire. Il est mort.\\
    -- Tu as été bien triste quand il est mort?\\
    -- Pensez-vous (geste). J'ai pas eu le temps avec tout ce qui se passait (silence).\\
    -- Et qu'est-ce qui se passait?\\
    -- Je boirais bien un autre demi, mais pas panaché, un vrai demi de vraie bière.}~(\textit{Z}:~47)
    \normalsize
  \end{singlespace}
\end{quote}
Poursuivant son histoire, Zazie explicite la mort de son père jusque dans les moindres détails, la hache comprise:
\begin{quote}
  \begin{singlespace}
    \small
    \guil{Vous lisez les journaux?\\
    -- Des fois.\\
    -- Vous vous souvenez de la couturière de Saint-Montron qu'a fendu le crâne de son mari d'un coup de hache? En bien, c'était maman. Et le mari, naturellement, c'était papa.\\
    -- Ah! dit le type.\\
    -- Vous vous en souvenez pas?}\\
    Il n'en a pas l'air très sur. Zazie est indignée.\\
    \guil{Merde, pourtant, ça a fait assez de foin.}~(\textit{Z}:~47)
    \normalsize
  \end{singlespace}
\end{quote}
Avec un naturel improbable, Zazie semble davantage perturbée par la possibilité que le \textit{type} n'ait pas entendu parler de son histoire que par l'histoire elle-même.
Toujours aussi imperturbable, elle raconte par la suite la quasi-agression à l'origine de toute la séquence d'événements:
\begin{quote}
  \begin{singlespace}
    \small
    Papa, il était donc tout seul à la maison, tout seul qu'il attendait, il attendait rien de spécial, il attendait tout de même, et il était tout seul, ou plutôt il se croyait tout seul, attendez, vous allez comprendre. Je rentre donc, faut dire qu'il était noir comme une vache, papa, il commence donc à m'embrasser ce qu'était normal puisque c'était mon papa, mais voilà qu'il se met à me faire des papouilles zozées, alors je dis ah! non parce que je comprenais où c'est qu'il voulait en arriver le salaud, mais quand je lui ai dit ah! non ça jamais, lui il saute sur la porte et il la ferme à clef et il met la clef dans sa poche et il roule les yeux en faisant ah ah ah tout à fait comme au cinéma, c'était du tonnerre. Tu y passeras à la casserole qu'il déclamait, tu y passeras à la casserole, il bavait même un peu quand il proférait ces immondes menaces et finalement immbondit dssus. J'ai pas de mal à l'éviter. Comme il était rétamé, il se fout la gueule par terre. Isrelève. Ircommence à me courser, enfin bref, une vraie corrida. Et voilà qu'il finit par m'attraper. Et les papouilles zozées de recommencer.~(\textit{Z}:~50)
    \normalsize
  \end{singlespace}
\end{quote}
De façon enfantine, notamment par l'emploi de l'expression \guil{papouilles zozées}, qui relève d'une terminologie approximative de la sexualité, Zazie expose toujours aussi calmement le drame de sa famille.
Ne parlant pas d'agression sexuelle -- saurait-elle seulement ce que c'est? --, elle compare les événements à une \guil{corrida} afin d'illustrer l'assaut, dont elle sait toutefois qu'il n'était pas \guil{normal} comme pouvait l'être l'embrassade initiale que lui a faite son père, qu'elle qualifie ultimement de \guil{salaud}.
\par
Dans le détachement le plus complet et après quelques considérations plutôt techniques sur le prix exorbitant de l'avocat qui a défendu sa mère, Zazie explique que c'est Georges, le \guil{coquin de maman}, qui \guil{avait refilé [à cette dernière] la hache (silence) pour couper son bois (léger rire).}~(\textit{Z}:~48)
Elle salue d'ailleurs la ruse de sa mère -- \guil{Pas bête, la guêpe, hein?} --, soulignant avec dégoût l'horreur visuelle de la scène du crime:
\begin{quote}
  \begin{singlespace}
    \small
    -- Mais, à ce moment, la porte s'ouvre tout doucement, parce qu'il faut vous dire que maman elle lui avait dit comme ça, je sors, je vais acheter des spaghetti et des côtes de porc, mais c'était pas vrai, c'était pour le feinter, elle s'était planquée dans la buanderie où c'est que c'est qu'elle avait garé la hache et elle s'était ramenée en douce et naturellement elle avait avec elle son trousseau de clefs. Pas bête, la guêpe, hein?\\
    -- Eh oui, dit le type.\\
    -- Alors donc elle ouvre la porte en douce et elle entre tout tranquillement, papa lui il pensait à autre chose le pauvre mec, il faisait pas attention quoi, et c'est comme ça qu'il a eu le crâne fendu. Faut reconnaître, maman elle avait mis la bonne mesure. C'était pas beau à voir. Dégueulasse même. De quoi mdonner des complexes. Et c'est comme ça qu'elle a été acquittée. J'ai eu beau dire que c'était Georges qui lui avait refilé la hache, ça n'a rien fait, ils ont dit que quand on a un mari qu'est un salaud de skalibre, y a qu'une chose à faire, qu'à lbousiller. Jvous ai dit, même qu'on l'a félicitée. Un comble, vous trouvez pas?\\
    $\left[ \dots \right]$\\
    -- Et après? demanda le type.\\
    -- Bin après c'est Georges qui s'est mis à tourner autour de moi. Alors maman a dit comme ça qu'elle pouvait tout de même pas les tuer tous quand même, ça finirait par avoir l'air drôle, alors elle l'a foutu à la porte, elle s'est privée de son jules à cause de moi. C'est pas bien, ça? C'est pas une bonne mère?~(\textit{Z}:~50-51)
    \normalsize
  \end{singlespace}
\end{quote}
Enfant de l'immédiat, si Zazie est d'abord indignée de la clémence du juge, elle s'accommode rapidement de l'acquittement de sa mère, allant jusqu'à être reconnaissante envers cette dernière d'avoir quitté Georges pour la protéger.
Avec ce récit, Zazie ne cherche aucunement la pitié et semble accepter son sort sans broncher:
\begin{quote}
  \begin{singlespace}
    \small
    On peut surprendre Zazie comme faisant preuve de subjectivité et d'objectivité, elle est un sujet, elle est partie prenante de l'action qu'elle vit. Elle ne se déresponsabilise pas de son vécu, ne se pose pas en victime de son drame familial, pas plus que ses parents ne passent pour des bourreaux\footcite[90]{Maurin2007}.
    \normalsize
  \end{singlespace}
\end{quote}
C'est donc son acceptation de tout ce qui lui arrive, sorte de philosophie du \guil{tout va de soi!}, qui justifie à nos yeux l'appartenance de Zazie au type de l'enfant-femme tel que nous l'avons précédemment décrit.


\section{Sissi, une enfance à réparer}
Dans \textit{Borderline}\footcite{Labreche2003}, l'auteure québécoise Marie-Sissi Labrèche fictionnalise son \guil{enfance de coquerelles\footcite{Labreche2008b}}, vécue dans la pauvreté entre une mère souffrant de maladie mentale et une grand-mère autoritaire.
Alternant de chapitre en chapitre entre une narratrice enfant et la jeune-adulte hautement sexualisée qu'elle est devenue, Labrèche met au monde le personnage de Sissi Labrèche~(\textit{B}:~63).
Dans \textit{La Brèche}\footcite{Labreche2008}, la narratrice Émilie-Kiki raconte l'époque entourant l'écriture du premier roman. L'enfance n'y est que très peu évoquée, mais reprend, lorsqu'elle l'est, les mêmes marqueurs, à savoir la pauvreté matérielle, la folie de la mère et les reproches incessants de la grand-mère.
\par
En 2008, Lyne Charlebois et Marie-Sissi Labrèche cosignent le scénario du film \textit{Borderline}\footcite{Charlebois2008}, adaptation cinématographique des deux romans amalgamés et dont le personnage principal, Kiki, est montrée à trois différents moments de sa vie: sous les traits de l'enfant de dix ans élevée dans la misère et la folie, puis de la jeune femme archisexuelle du roman \textit{Borderline}, ainsi que de la quasi-trentenaire qui rédige ce qui sera son premier roman à l'instar de la narratrice dans \textit{La Brèche}\footcite[85]{Ledoux-Beaugrand2009}.
Clôturant ce que nous qualifierions de \guil{grand cycle romanesque des enfances délabrées}, \textit{La lune dans un HLM}\footcite{Labreche2008b} a été écrit pendant la période de production du film.
Y alternent des lettres de l'auteure à sa mère et des chapitres purement fictionnels mettant en scène Léa, jeune adulte dont le contexte familial n'est pas sans rappeler les personnages précédents -- mère psychiatrisée et décès d'une grand-mère responsabilisante --, à l'exception notable toutefois que son comportement ne verse pas dans l'excès, ni de substances, ni de sexe, ni d'amour.


\subsection{Sissi la \guil{fillette de l'est}}
Si la construction du roman \textit{Borderline} départage clairement deux temporalités différentes dans la vie d'une même narratrice, la phrase liminaire de chacun des chapitres pairs rend non-équivoque le fait qu'il y soit question d'une enfant puisqu'on y fait mention de son âge exact: \guil{J'ai onze ans}~(\textit{B}:~29), \guil{J'ai huit ans et je suis en deuxième année}~(\textit{B}:~55), \guil{J'ai sept ans}~(\textit{B}:~87) et \guil{J'ai cinq ans}~(\textit{B}:~115).
La narratrice parle d'ailleurs d'elle-même comme d'une \guil{petite fille}~(\textit{B}:~30-31, 121), se décrivant comme une enfant maigre avec des \guil{cheveux comme des spaghettis} et des \guil{yeux trop grands, grands comme des yeux de chien piteux}~(\textit{B}:~63).
La Sissi des chapitres pairs est ainsi un personnage d'enfant, d'autant plus qu'elle se reconnaît lorsqu'il est question de jeunesse: \guil{Et l'enfant, c'est moi, je le sais $\left[ \dots \right]$.}~(\textit{B}:~94)
\par
De son enfance dans le quartier Centre-Sud, \guil{à l'est de la rue Papineau}~(\textit{B}:~62), la Sissi de huit ans, en deuxième année du primaire, tire une illustration plutôt sombre, la comparant au \guil{\textit{Russian way of life}, le \textit{concentration camp way of life}, version rue Dorion} et l'opposant à l'enfance de type \guil{\textit{American way of life}, version québécoise} que dessinent, au sens propre du terme lorsque la maîtresse leur fait faire l'exercice artistique de représenter leurs familles en dessin, ses camarades de classe~(\textit{B}:~56).
C'est, selon Virginie Doucet, une enfance \guil{déterminée par la peur, le manque, le solitude (\textit{sic})\footcite[79]{Doucet2007}.}
Reprenant à la façon d'une comptine un court poème de Lucie Delarue-Mardrus\footnote{Les vers \guil{\textit{Je suis une petite fille / Et tout le monde m'aime bien / Quelquefois je pleure et je rage / On ne peut pas toujours être sage comme une image / C'est tout!}}, quoique retranscrits de façon approximative avec certains mots inversés, sont tirés du poème \guil{Une petite fille} du recueil \textit{Poèmes mignons} de Lucie Delarue-Mardrus, publié en 1929.}, Émilie-Kiki évoque son passé de \guil{fillette de l'est\footcite[58]{Labreche2008}}, dans un récit qui n'est pas sans rappeler celui de Sissi et qui se déroule dans le même Montréal défavorisé, dans l'appartement rempli de coquerelles~(\textit{B}:~34-35, 38, 119) du \guil{2020 de la rue Dorion où [s]a grand-mère habitait et où [elle] demeurai[t] avec elle huit mois par année parce que [s]a mère était internée\footcite[149]{Labreche2008}}.
\par
Décrite par la travailleuse sociale comme étant hyperactive, agitée et très créative~(\textit{B}:~99), Sissi est une \guil{petite fille nerveuse}~(\textit{B}:~121), trop \guil{paquet de nerfs} pour qu'on lui laisse tenir les alliances lors du mariage imaginé de sa mère~(\textit{B}:~116).
Enfant au tempérament nerveux, elle a des gestes brusques et pousse des cris aigus, constamment à la recherche d'attention: \guil{qu'on se taise et m'écoute}~(\textit{B}:~118).
Comme enfant, Sissi est donc énervée et volontairement énervante: \guil{C'est [s]a marque de commerce.}~(\textit{B}:~117)

\subsubsection{À l'âge de la contradiction}
Pour l'enfant Sissi, l'enfance semble se construire de façon complètement antagoniste à l'âge adulte, qu'elle refuse.
Ainsi, elle exprime sous quelques formes son rejet de l'adulte, \textit{a priori} de sa mère: \guil{Je ne veux pas lui ressembler et je me bats. Tout ce qu'elle aime, je ne l'aime pas. Tout ce qu'elle fait, je ne le fais pas. Je ne veux pas être elle. Niet. No. Non. Je ne suis pas elle.}~(\textit{B}:~30-31)
Moins brutal que la haine, Sissi utilise également la contradiction comme mécanisme de mise à distance de l'adulte: \guil{Moi, quand je serai plus vieille, je contredirai ma grand-mère. Je me marierai avec tous les hommes de la terre juste pour la faire chier.}~(\textit{B}:~32-33)
Elle retire donc une satisfaction de dire ou de faire l'opposé de ce que dit ou fait sa grand-mère.
\par
Son rejet de l'adulte, bien qu'il soit principalement dirigé à l'encontre de ces deux figures maternelles, est également présent envers d'autres représentants de l'autorité:  \guil{Alors là, je l'ai traumatisée, la maîtresse, avec mon dessin. Je l'ai traumatisée, sauf qu'elle a été obligée de l'accrocher quand même, la vieille chipie. $\left[ \dots \right]$ Traumatisée, la prof. Je m'en fiche! $\left[ \dots \right]$ Même la directrice, qui me serre toujours dans ses bras lorsqu'elle voit mes beaux dessins, a été traumatisée. Quand elle l'a vu, elle n'a pas su quoi dire.}~(\textit{B}:~59)
Non seulement Sissi ne semble pas avoir envie de plaire aux intervenants adultes, mais elle semble également retirer une fierté de leur déplaire ou, selon ses mots, de les traumatiser.
Cette constante contradiction des adultes semble être, pour elle, une façon bien réfléchie de se positionner hors du monde des adultes et d'y échapper.

\subsubsection{Sissi et l'asexualité: ne sait ni ne veut savoir}
Pour la jeune Sissi, la sexualité n'est pas le sujet d'un grand savoir, ni même un sujet suscitant un quelconque intérêt.
Quoique l'on suppose qu'elle prononcerait, au contraire de Zazie et de ses innombrables questions sur la \guil{sessualité}, le terme \guil{sexualité} correctement, le fait est que ce n'est pas un terme qu'on l'imagine vraiment utiliser; Sissi verse dans l'asexualité la plus totale, en ce que le sexe ne semble même pas piquer sa curiosité.
Ainsi, lorsque sa grand-mère lui décrit des sévices que subissent certains enfants et bifurque vers les agressions de nature sexuelle, Sissi semble ne pas trop comprendre, ou peut-être ne pas vouloir comprendre, où elle veut en venir:
\begin{quote}
  \begin{singlespace}
    \small
    -- S'amuser avec eux? C'est le fun, s'amuser avec quelqu'un? \\
    -- Pas s'amuser avec des bébelles, s'amuser avec quelqu'un. Tu sais toucher quelqu'un partout, même où c'est pas supposé. \\
    -- C'est où que c'est pas supposé? \\
    -- Dans les fesses!~(\textit{B}:~129-130)
    \normalsize
  \end{singlespace}
\end{quote}
Lasse des insinuations de sa grand-mère avant même que cette dernière en vienne à ce qu'elle voulait véritablement savoir, Sissi plaide à nouveau l'ignorance enfantine afin d'éviter le sujet:
\begin{quote}
  \begin{singlespace}
    \small
    Mais elle revient à la charge en s'accrochant après mes mots comme une journaliste en plein milieu d'une entrevue qui veut faire cracher le morceau. Elle veut poursuivre jusqu'à ce que j'avoue son intention à elle. \\
    -- Est-ce qu'il t'a déjà grattée, le nouveau chum de ta mère? \\
    -- Bien oui, il m'a déjà grattée! \\
    -- Hon oui, comment? \\
    -- Bien comme ça, avec les doigts. \\
    -- Oh oui, et où ça? \\
    -- Dans le dos pis sur la tête aussi. \\
    -- T'a-t-il déjà grattée ailleurs? \\
    -- Je ne me rappelle pas. \\
    -- Ou plutôt tu ne veux pas me le dire. \\
    -- T'es tannante! Je viens de te le dire, il m'a déjà grattée dans le dos et sur la tête. Je ne raconte pas de mensonges, moi. ~(\textit{B}:~130-131)
    \normalsize
  \end{singlespace}
\end{quote}
Hésitante et mal à l'aise face au sujet de la sexualité abordé par sa grand-mère, la Sissi de cinq ans semble comprendre ce qui est attendu d'elle, à savoir qu'elle accuse son beau-père de l'avoir attouchée, sans cependant manifester de curiosité face à la sexualité.
L'hésitation de Sissi semble d'ailleurs exacerbée par son ambivalence entre une dualité de valeurs: d'une part, il y a sa volonté de ne pas mentir et d'autre part son désir de contenter sa grand-mère en lui disant ce qu'elle veut entendre.
Sissi semble donc en mesure de conceptualiser grossièrement la sexualité, quoiqu'elle ne s'y intéresse pas vraiment et ne serait pas en mesure de l'expliquer dans les détails; si elle sait qu'il y a des touchers \guil{pas supposés}, le reste de son éducation sexuelle reste à faire.
\par
Malgré la mise en garde virulente de sa grand-mère contre les attouchements de nature sexuelle, le contact physique peut tout de même, pour la jeune Sissi, prendre les traits d'un vecteur d'apaisement.
Il n'est alors porteur ni de sensualité ni de sexualité, apportant plutôt un état de bien-être.
Ainsi, associé à sa grande gentillesse, le toucher du professeur d'éducation physique contribue à apaiser la douleur que cause à Sissi son sentiment de solitude:
\begin{quote}
  \begin{singlespace}
    \small
    Je ne voulais pas rater mon cours d'éducation physique avec mon professeur barbu. Mon professeur qui est si gentil avec moi. Je ne voulais pas arriver en retard parce que je voulais courir en rond et que mon professeur barbu vienne attacher mon lacet. Qu'il s'approche de moi et qu'il me regarde avec ses grands yeux noirs gentils, encadrés par ses gros sourcils noirs gentils. Qu'il se penche vers moi et que son épaule frôle mon petit corps. Que je sois moins seule, pour quelques secondes.~(\textit{B}:~63)
    \normalsize
  \end{singlespace}
\end{quote}
Ayant rapidement compris que Sissi ne se sentait pas bien en la voyant insulter sa mère, le professeur d'éducation physique tente de la rassurer en établissant une proximité physique de type paternel avec elle: il la tient par la main, l'aide à se changer au vestiaire puis lui fait un câlin.
\begin{quote}
  \begin{singlespace}
    \small
    Une fois que la maîtresse et la classe de deuxième année ont commencé à s'éloigner, il m'a regardée longtemps dans les yeux, s'est penché vers moi puis m'a serrée dans ses bras fort, fort, fort. Et il a dit: \textit{Je sais que c'est dur pour toi, Sissi. Je le sais.} Mon estomac vide a grimpé le long de mon corps et s'est remis à sa place, tout comme mon c\oe{}ur, mon pancréas, mes intestins, mon foie et mes reins. Les choses ont repris un semblant de normalité.~(\textit{B}:~69)
    \normalsize
  \end{singlespace}
\end{quote}
Cette douceur est également celle de la main \guil{douce et tiède}, \guil{comme une caresse dans [s]a main} d'Aline, la jeune intervenante sociale qui lui \guil{tient tout doucement la main sans la comprimer, sans [lui] écraser les doigts} et qu'elle oppose à la main \guil{bourrée d'arthrite et prête à [lui] broyer les jointures}~(\textit{B}:~98) de sa grand-mère.
Chez le personnage d'enfant créé par Labrèche, le toucher peut appartenir à diverses catégories.
Il y a bien évidemment la proximité physique dangereuse que pourrait imposer un agresseur à l'enfant, qui prend racine dans l'imaginaire de Sissi par les mises en garde de sa grand-mère.
Tout à l'opposé, il y a également le contact physique réconfortant, agréable et non-sexuel qui prend la forme d'une main tendue ou d'une caresse de la part d'un adulte compréhensif face aux difficultés familiales de l'enfant.
Ainsi, bien que les rapports entre Sissi et son professeur barbu aient une dimension physique, ils sont, en raison du jeune âge de la fillette, dépourvus de tout ce qui touche de près ou de loin à la sexualité.
Cette asexualité dans les rapports entre Sissi et son professeur, alors que le câlin se distingue de la caresse et le déshabillage de l'effeuillage, est génératrice d'enfance en ce sens que de tels rapports physiques entre adultes supposent forcément une connotation affective différente, ce qui n'est pas le cas avec un enfant.


\subsection{Au Pays des adultes: une enfant seule au monde}
Enfant unique dont la famille se résume à sa mère et à sa grand-mère, Sissi construit sa conception du \guil{nous} autour de ces deux femmes et à l'exclusion du reste du monde: \guil{Contre nous, moi, ma grand-mère et ma mère.}~(\textit{B}:~100)
Isolée par ses deux mamans~(\textit{B}:~40, 88, 124-125), lesquelles exercent l'autorité dans une sorte de co-parentalité -- \guil{mes parents, mes deux mamans}~(\textit{B}:~119) --, Sissi est seule au monde: \guil{De toute façon, si je meurs, qui ça va déranger? Je suis toute seule au monde. Toute seule. Je vois déjà mon épitaphe d'ici: \textit{Ci-gît Sissi, la plus toute seule des toutes seules. Va en paix, petite souris sans queue. Fin.}}~(\textit{B}:~61)
Cette solitude s'alourdit également du \guil{perpétuel silence qui [l]'entoure}~(\textit{B}:~93) imposé par la folie de la mère, auquel la petite tente de remédier en le brisant de sa propre parole, \guil{pass[ant] [s]es journées à soliloquer}~(\textit{B}:~93).
\par
Face à ce climat familial claustrophobique, Sissi développe une aversion pour la vie, ou plutôt pour \textit{cette} vie qui est la sienne:
\guil{À vrai dire, j'ai plutôt envie de la dégueuler, la vie. J'ai huit ans et j'en fais déjà une indigestion, de la vie.}~(\textit{B}:~59)
En découle d'abord un désintérêt envers les membres de sa famille, à commencer par sa grand-mère qui \guil{dit tout le temps des niaiseries}~(\textit{B}:~92) et qui \guil{marmonne des choses} incompréhensibles qui n'ont \guil{pas d'importance} pour elle~(\textit{B}:~88).
Sa solitude d'enfant égarée dans le monde des adultes, accrue par le peu d'intérêt qu'elle porte aux adultes qui l'entourent, développe chez Sissi une haine envers sa famille, sentiment qu'elle extériorise par un violent rejet: \guil{Oui, si je le pouvais, je détruirais tout. Je piétinerais cette maudite maison de carton remplie de coquerelles. J'écrabouillerais la chambre de ma mère qui renferme un drame. Je réduirais en mille miettes tous ces vilains meubles décatis qui m'empêchent de courir jusqu'à l'infini.}~(\textit{B}:~119)
Isolée du reste du monde, avec pour seule compagnie celle de ses \guil{deux folles de mamans}~(\textit{B}:~88) qu'elle honnit, Sissi est en quelque sorte orpheline, dépourvue de sa famille qu'elle a rejetée violemment.

\subsubsection{Deux mamans, pas de maman}
Dans l'\oe{}uvre de Marie-Sissi Labrèche, l'autonomisation de l'enfant passe avant tout par le délaissement et le blâme qu'elle subit.
Chez la petite Sissi, l'autorité parentale est surtout assumée par la grand-mère, vu l'incapacité récurrente de la mère d'assurer son rôle maternel en raison de ses problèmes mentaux:
\begin{quote}
  \begin{singlespace}
    \small
    Ma mère, c'était une folle. Une vraie folle avec des yeux qui fixent, un comportemnet désaxé et des milliers de pilules à prendre tous les jours. Une vraie folle avec un vrai certificat médical en bonne et due forme, qui devait passer le test de Rorschach très souvent, si souvent qu'à la vue d'une tache elle ne pouvait s'empêcher de dire à quoi ça lui faisait penser: \textit{Une tulipe! Un éléphant! Un nuage! Un utérus éventré! Des Chinois qui mangent du riz!}~(\textit{B}:~16)
    \normalsize
  \end{singlespace}
\end{quote}
Même dans ses moments de lucidité, la mère de Sissi est inapte à assumer son rôle, notamment parce qu'elle craint absolument tout, sa fille incluse: \guil{Ma mère a peur de moi. Faut dire que ma mère a peur de tout. $\left[ \dots \right]$ Elle a peur de mes cris et de mes pleurs. Quand je pleure et je crie, elle a peur que les autres disent qu'elle n'est pas une bonne mère et qu'elle me bat. Elle a peur que les voisins envoient la DPJ et que la DPJ me mette dans son fourgon d'enfants maltraités.}~(\textit{B}:~120)
Souvent déconnectée de la réalité, la mère de la narratrice est absente bien que présente physiquement, ce qui cause chez l'enfant un sentiment d'abandon:
\begin{quote}
  \begin{singlespace}
    \small
    Ma mère, j'ai toujours pensé qu'elle ne tenait pas à moi. J'ai toujours pensé que, parce qu'elle se réfugiait trop souvent quelque part dans sa tête où je n'avais pas accès, elle ne tenait pas à moi. Ma mère pouvait passer des semaines comme ça, dans sa tête, à me regarder avec ses yeux bleus braqués sur moi, sans expression, ses yeux remplis de dépression qui me rendaient malade.~(\textit{B}:~19-20)
    \normalsize
  \end{singlespace}
\end{quote}
C'est ce que Virginie Doucet appelle la \guil{non-vie} de la mère\footcite[79]{Doucet2007}, qui fait d'elle une mère instable incapable d'endosser son rôle: \guil{Une mère instable est défaillante en ceci qu'elle est incapable d'offrir à ses proches -- et notamment à ceux qui dépendent d'elle -- des réactions suffisamment prévisibles pour faire fonction de référence, de repère, d'appui\footnote{\cite[212]{Eliacheff2003} cité dans \cite[31]{Leduc2010}}.}
Or, la mère de Sissi ne la \guil{comprenait même pas à cause de ses maudites hormones défectueuses, ses maudites hormones passées date}~(\textit{B}:~20) et était pour ainsi dire une figure absente en raison de sa maladie.
Katherine Dion abonde dans le même sens, à savoir que la folie de la mère la \guil{tue} symboliquement: \guil{Absente à sa fille autant qu'elle l'est à elle-même, la mère folle est telle une mère morte\footcite[33]{Dion2010}.}
Cette \guil{mère-spectrale} n'assure donc, sur le plan affectif, ni stabilité ni sécurité\footcite[31]{Leduc2010}, et l'enfant doit apprendre à vivre malgré l'absentéisme maternel.
Outre cette absence mentale et affective de la mère qui se retranche fréquemment dans sa tête, il ne faut pas négliger les périodes d'hospitalisation qui l'amènent parfois à s'absenter totalement, à la fois sur les plans physique, mental et affectif, accentuant le deuil symbolique dont a parlé Dion\footcite[36]{Dion2010}.
\par
En l'absence d'une véritable figure maternelle, Sissi est prise en charge par sa grand-mère maternelle qui fait office d'autorité parentale.
Cette substitution de la mère par la \textit{mémé} opère un déplacement symbolique\footcite[33]{Leduc2010} alors que Sissi et sa mère se retrouvent au même niveau générationnel: \guil{Tout se mélange dans ma tête, comme les rôles familiaux dans la maison: ma mère est ma soeur, ma grand-mère est ma mère $\left[ \dots \right]$.}~(\textit{B}:~63)

\subsubsection{Une main de fer sans le gant de velours}
Cependant, la grand-mère est une figure très peu maternelle et représente encore moins la douceur associée traditionnellement à la mère, éduquant Sissi dans la crainte de tout, à grands coups d'histoires d'horreur et de récits des malheurs d'enfants du monde entier: \guil{Il y a des enfants qui se font maltraiter. On ne leur donne pas à manger. On ne les habille pas. On les attache et on les brûle avec des cigarettes.}~(\textit{B}:~129)
Ainsi, dès son plus jeune âge, Sissi est consciente qu'elle \guil{ne doi[t] pas trop faire chier [s]a grand-mère}~(\textit{B}:~116) sous peine de représailles:
\begin{quote}
  \begin{singlespace}
    \small
    Par exemple, quand j'étais tannante, elle avait coutume de me dire: \textit{Si t'es pas gentille, un fifi va entrer par la fenêtre et te violer} ou \textit{Je vais te vendre à un vilain qui fera la traite des Blanches avec toi} ou encore \textit{Un assassin va venir te découper en petits morceaux avec un scalpel, c'est ça que tu veux? Hein?}~(\textit{B}:~11)
    \normalsize
  \end{singlespace}
\end{quote}
Bien au-delà de la seule incapacité maternelle de la mère biologique, la violence verbale subie par Sissi est, à notre avis, le principal vecteur de la dépossession de l'enfance de la narratrice. Cette violence verbale à laquelle se livre la \textit{Mémé} se manifeste dans l'hybridation des trois formes que sont le reproche, la menace et l'insulte.
\par
Laissée à elle-même sur le plan affectif, Sissi est très jeune accablée par ce blâme quasi-litanique que fait peser sur elle une grand-mère toute-puissante, inflexible et même despotique\footcite[34]{Leduc2010}: \guil{D'ailleurs, combien de fois ma grand-mère m'a cassé les oreilles avec ça? \textit{T'es bonne pour dire des niaiseries, toi. T'es bonne en crisse pour dire des niaiseries qui inquiètent ta mère}.}~(\textit{B}:~15)
Abreuvant constamment Sissi de ses reproches, sa grand-mère lui impose la responsabilité de ne pas inquiéter sa mère: \guil{\textit{Sissi, n'oublie pas ta bombonne de Ventolin, sinon tu risques de faire une crise d'asthme, et ça va inquiéter ta mère. Et quand ta mère s'inquiète, elle devient folle et on doit l'enfermer. Tu ne veux pas rendre ta mère folle, hein?}}~(\textit{B}:~60)
Et si ce n'est pas l'asthme, c'est autre chose: la grand-mère trouve rapidement un autre motif pour lui faire porter la culpabilité de la folie de sa mère -- \guil{Ma grand-mère me l'a toujours dit: \textit{T'es bonne en crisse pour raconter des histoires. T'es bonne en crisse pour raconter des histoires qui rendent ta mère folle.} Des histoires qui rendent folle ma mère folle\dots}~(\textit{B}:~76) --, reproches qui font par ailleurs également partie des souvenirs d'enfance d'Émilie-Kiki, narratrice du second livre: \guil{\textit{ÉMILIE-KIKI, TU RENDS FOLLE TA MÈRE! C'EST DE TA FAUTE SI ON DOIT TOUT LE TEMPS L'INTERNER!}}~(\textit{B}:~40)
\par
Au reproche s'ajoute parfois la menace, notamment dans la formule répétitive du \guil{si tu fais ceci il arrivera tel malheur}, la réalisation du malheur étant toujours conditionnelle et imputée au mauvais comportement de Sissi: \guil{Elle me dit: \textit{Si t'arrête pas de courir comme ça, les voisins d'en bas vont se plaindre et on va nous jeter à la porte. On n'aura plus de maison et on devra rester dans la rue. C'est-tu ça que tu veux?}}~(\textit{B}:~118)
Notons que pour culpabiliser l'enfant davantage et renforcer sa menace, la grand-mère utilise une formulation qui fait porter le blâme à l'enfant, en lui demandant si c'est vraiment ce qu'elle veut.
Accablant Sissi d'une responsabilité supplémentaire, la grand-mère l'accuse de ne pas vouloir le bien de sa mère:
\guil{Ça va faire ton affaire si ta mère meurt. Mais je vais te dire \dots si elle meurt, je vais peut-être être obligée de te placer dans une famille d'accueil, et je t'ai conté ce qui arrive dans ces familles, c'est pas drôle.}~(\textit{B}:~35)
Ainsi, toute son enfance durant, il plane sur Sissi une perpétuelle menace d'abandon:
\begin{quote}
  \begin{singlespace}
    \small
    Si tu n'es pas gentille, je vais te placer. Si tu ne manges pas toute ton assiette, je vais te placer. Si tu racontes des menteries, je vais te placer. Si tu te fouilles dans le nez, je vais te placer! Si tu déplaces le pot de jus, je vais te placer. Je vais te placer! Je vais te placer! Je vais te placer! Je vais te placer comme ça ne se peut pas! Je vais te placer jusque sur une autre planète. Tiens, Pluton, c'est la plus loin!~(\textit{B}:~36)
    \normalsize
  \end{singlespace}
\end{quote}
De façon récurrente et toujours sous la formule du \guil{si tu fais ceci ou celà}, la grand-mère menace d'imposer à Sissi des conséquences en raison de son comportement jugé répréhensible: \guil{Si t'es méchante comme ça, je vais appeler ton vrai papa, Papa Méchant. Il va t'emmener chez lui, là où il reste avec sa pute, et ça ne va pas être drôle.}~(\textit{B}:~123)
Le rituel de menace que fait subir la grand-mère à Sissi implique généralement son abandon et sa prise en charge par une famille de remplacement, famille dont elle a une grande crainte en raison de ce qui lui a été raconté à son propos.
\par
Ajoutant l'injure à la menace d'abandon, la grand-mère rappelle constamment à Sissi qu'elle est une mauvaise fille:
\begin{quote}
  \begin{singlespace}
    \small
    Je suis fatiguée de la voir s'énerver, parce que quand elle s'énerve, elle s'en prend toujours à moi et j'en prends plein la gueule. Elle me dit que je suis méchante, que je ne pense qu'à faire mal aux autres, que je suis une petite débauchée et qu'un jour elle va me placer.~(\textit{B}:~36)
    \normalsize
  \end{singlespace}
\end{quote}
Tout devient ainsi prétexte au reproche, que ça soit envers ce que Sissi dit ou fait, ne dit pas ou ne fait pas, voire même le fait qu'elle n'aime pas porter certains vêtements:
\begin{quote}
  \begin{singlespace}
    \small
    Que l'imperméable lui a coûté un bras et que si je ne le porte pas, c'est parce que je suis méchante et que je veux juste lui faire de la peine, que je veux juste lui faire gaspiller son argent, que l'argent ne pousse pas dans les arbres, que j'exagère tout le temps sur le pain pis le beurre, que j'en demande toujours trop, que je ne suis jamais contente de rien, que je suis paresseuse, que je suis traîneuse, que je suis débauchée, que je n'arriverai à rien dans la vie, que je vais finir sur l'aide sociale avec un mari qui me bat et quatre enfants sur les bras, et blablabla.~(\textit{B}:~90)
    \normalsize
  \end{singlespace}
\end{quote}
L'imperméable semble ici n'être qu'un point de départ pour une cascade de reproches tout aussi variés les uns que les autres et qui s'accumulent.
Partant de la seule ingratitude d'une enfant qui ne veut pas porter un vêtement qu'elle trouve laid, la grand-mère termine sa pluie de reproches par une mention prophétique à l'effet que Sissi aura une vie de misère.
Les accusations de débauche ne sont d'ailleurs pas isolées et font partie du rituel dénigrant de la grand-mère: \guil{Ça aussi ma grand-mère me l'a souvent dit: \textit{T'es juste une petite vicieuse et une petite cochonne!}}~(\textit{B}:~76)
Bref, aux encouragements normalement attendus de la part d'un parent se substituent pour la petite Sissi les récriminations et injures de sa \textit{Mémé}.
Élevée dans la menace et le reproche constants, et bien que sa grand-mère ne la \guil{place} pas vraiment en famille d'accueil, Sissi est victime d'une véritable forme d'abandon sur les plans affectif et émotionnel dont résultent \guil{un moi criblé de trous\footnote{\cite[63]{Haineault2006} cité dans \cite[33]{Dion2010}.}} et un vieillissement émotionnel précoce: \guil{Crisse! Je n'ai même pas besoin de me grandir, malgré ma petite taille, j'ai déjà mille ans.}~(\textit{B}:~68)
Enfin, si la grand-mère remplace effectivement la mère et représente à ce titre l'autorité parentale, il faut souligner qu'il s'agit davantage d'autorité que de parentalité: la main de fer est raide et nue, sans gant de velours pour l'adoucir.

\subsubsection{De l'autre côté du miroir du monde adulte}
Réduisant chez Sissi la part laissée à l'enfance, sa compréhension de la \guil{chose adulte} est surtout illustrée dans un spectre précis de sa vie, soit sa conscience de sa situation familiale peu orthodoxe: \guil{\textit{Bien oui, ma mère n'est pas comme la vôtre! Bien oui, ma mère est folle! Qu'est-ce que ça peut bien vous foutre! Allez donc tous chier!}}~(\textit{B}:~70)
À seulement cinq ans, la jeune fille a conscience que sa mère prend des antidépresseurs qui la rendent \guil{peace and love}~(\textit{B}:~116):
\guil{J'ai beau avoir cinq ans, mais je m'en aperçois, que tout est au ralenti, je ne suis pas con, je ne suis pas stupide. Je ne suis pas sur les antidépresseurs, moi. J'ai un plafond dans la tête, moi.}~(\textit{B}:~118)
Puis, à sept ans, elle explique à sa façon la maladie de sa mère:
\begin{quote}
  \begin{singlespace}
    \small
    Elle est froide, ma mère. Froide et effacée. Mais ce n'est pas de sa faute. C'est à cause de son manque de petits ponts dans le cerveau. C'est ce qu'un médecin m'a raconté. Ça a l'air qu'il y a plein de petits ponts dans notre tête qui font passer les mots d'un endroit à l'autre. Ma mère, elle, quelques fois durant l'année, il lui en manque.~(\textit{B}:~91)
    \normalsize
  \end{singlespace}
\end{quote}
Exposant métaphoriquement son interprétation enfantine de la réalité adulte, Sissi démontre sa compréhension précise de la maladie mentale de sa mère.
Malgré son jeune âge, Sissi semble donc bien saisir la complexité de la vie des adultes qui l'entourent, monoparentalité et maladie mentale incluses.
\par
En plus de la compréhension strictement matérielle de certaines réalités, Sissi fait également montre d'une perspicacité quant il s'agit d'évaluer les motivations des agissements des adultes.
Fine observatrice, l'enfant décode rapidement la crainte de la travailleuse sociale: \guil{Vu le ton qu'elle emploie pour me parler, je sens qu'elle a peur que je me mette à crier $\left[ \dots \right]$.}~(\textit{B}:~96)
Sissi, pas dupe, comprend que les questions de la jeune femme ne sont pas anodines et visent avant tout à lui \guil{occuper l'esprit}~(\textit{B}:~95) afin de la distraire, relevant d'ailleurs à un moment qu'elle trouve \guil{louche} de ne pas avoir vu sa grand-mère depuis un moment~(\textit{B}:~97).
D'un naturel méfiant, elle ne croit donc pas tout ce qu'on lui dit, notamment lorsque les adultes lui ont raconté que son chien s'était sauvé pour retrouver sa maman.
Cynique, elle estime qu'il s'agissait là d'un pur mensonge: \guil{Façon détournée de me dire qu'ils l'avaient fait tuer à la SPCA $\left[ \dots \right]$.}~(\textit{B}:~118)
Sissi décèle souvent les intentions détournées des adultes, y compris celles de sa propre grand-mère: \guil{Là, ma grand-mère commence son baratin. Elle me tartine, elle me cuisine, si elle pouvait me faire cuire, elle le ferait. $\left[ \dots \right]$ Ma grand-mère veut me faire dire des choses, elle veut que je lui avoue des choses qu'elle a inventées.}~(\textit{B}:~130)
Comprenant dans une certaine mesure le \guil{code} des adultes, elle n'a pas, du moins pas entièrement, la naïveté que l'on suppose généralement aux enfants.
Bien que ne faisant pas explicitement partie du monde des adultes, elle n'y est pas tout à fait étrangère et y trouve une certaine aisance puisqu'elle en saisit une partie du fonctionnement.
\par
Outre sa bonne compréhension de certains agissements des adultes, Sissi est également au fait des désirs moins nobles qui motivent certains d'entre eux puisqu'elle y a été initiée dès son plus jeune âge par les histoires que lui racontait sa grand-mère: \guil{À quatre ans, je n'avais pas droit au croque-mitaine ou au Bonhomme Sept-Heures, mais au serial killer.}~(\textit{B}:~11)
Elle est donc conditionnée à craindre un peu tout, et surtout les hommes.
Par exemple, elle ne peut pas se rendre à l'école seule: \guil{Ma grand-mère dit que, parce que je suis petite, les vieux ivrognes de la taverne vont me rentrer de force dans les toilettes pour que je touche leur pipi, et après on ne va plus jamais me revoir\footcite[56]{Labreche2003}.}
Cette même grand-mère qui menaçait Sissi de l'abandonner lui fait miroiter le danger qu'elle se fasse enlever: \guil{Si on ne l'arrête pas, il va vous enlever, toi et ta mère, vous emmener très loin pour vous violer et vous tuer. Et personne ne va vous retrouver, puisqu'il va vous laisser dans un champ, la nuit, sans lumière.}~(\textit{B}:~133)
À entendre de telles histoires, Sissi en développe un vaste imaginaire quant à la méchanceté des adultes et les sévices qu'ils peuvent faire subir aux enfants.
La maltraitance est d'ailleurs une thématique qu'elle réutilise afin d'impressionner Céline lorsqu'elle évoque la possibilité d'être prise en charge par une famille d'accueil suite à la tentative de suicide de sa mère.
Fort bien informée sur la méchanceté potentielle des adultes à l'égard des enfants, Sissi dresse alors un portrait sans nuances des familles d'accueil:
\begin{quote}
  \begin{singlespace}
    \small
    -- Ma grand-mère m'a dit que si je tombe sur une bonne famille, ça peut aller. J'aurai plein de belles robes. Une limousine viendra me reconduire à l'école et j'aurai toutes les Barbies de la terre. Par contre, si je tombe sur une bande de salauds qui adoptent des enfants uniquement pour l'argent, ils vont me donner à manger des toasts pas de beurre. Et je porterai le vieux linge des autres enfants, tout usé, plein de trous. On me laissera me laver juste à l'eau froide et pas de savon. Peut-être aussi que le papa de la maison voudra s'amuser avec moi. \\
    -- S'amuser avec toi? \\
    -- Oui, tu sais\dots \ il va me montrer son machin\dots \ sa bitte. \\
    -- Oh! sa bitte\dots \\
    -- Il va vouloir que je la mette dans ma bouche, au complet dans ma petite bouche. Et lui, le gros dégueulasse, il va l'enfoncer si loin dans ma gorge qu'il va m'étouffer. Et moi, je ne pourrai pas me libérer parce qu'il va tenir ma tête solidement avec ses deux grosses mains sales aux ongles crottés. Comme il fera noir, il ne verra pas que je suis toute bleue, en train de mourir.~(\textit{B}:~38)
    \normalsize
  \end{singlespace}
\end{quote}
Dissertant ensuite sur ce qu'elle ferait subir à son agresseur -- \guil{la lui mordre, sa grosse bitte}, et \guil{même la lui arracher au complet} --, elle en conclut qu'elle devrait vivre recluse dans les bois, \guil{à cause des flics qui seront à [s]es trousses parce qu'[elle] aur[a] mutilé le papa de la famille}~(\textit{B}:~39).
Tout ça, bien évidemment, afin d'éviter la prison pour délinquants: \guil{Ma grand-mère m'a dit que c'est horrible ce qui se passe là, c'est pire que dans les familles d'accueil!}~(\textit{B}:~39)
Du test de résistance imposé par les gardiens aux petites filles, de l'appartenance des fillettes ayant réussi le test à certains de leurs gardiens, jusqu'à l'échec du test qui mène à l'éclatement du ventre qui libère par le nombril les objets qu'on y avait mis, Sissi répète ce que sa grand-mère lui a appris~(\textit{B}:~39-40).
Ainsi, du haut de ses onze ans, elle possède un imaginaire qui inclut déjà de nombreuses notions du monde adulte, quoique sa perception en soit exagérément cauchemardesque.
Elle a conscience du danger que représentent les adultes et vit dans une crainte immodérée des abus sexuels et physiques.


\subsection{Un enfant adaptatif: malgré tout, on survit}
Lorsque la psychologue demande à Émilie-Kiki, narratrice de \textit{La Brèche}, de parler de sa naissance ou plutôt de sa conception, la jeune femme expose et résume à la fois tout le drame de son enfance:
\begin{quote}
  \begin{singlespace}
    \small
    Un viol rue Wurtele, d'après ce qu'elle m'a raconté, ma mère folle à lier, un viol pour ne pas avoir à assumer l'envie de se faire tripoter, les vêtements déchirés, qu'elle m'a déjà dit, ma mère, une femme à qui la sensualité est refusée, qu'elle a essayé de me faire croire, ma mère. Et mes premiers mois, à moi! \textit{Ce n'est pas ma fille!} criait-elle sur tous les toits, ma mère. Petite puce en plein milieu d'une cuisine sale remplie de coquerelles, \textit{Ce n'est pas ma fille! C'est la fille d'une autre!} et ses pleurs à ma mère, ses longs pleurs comme des guimauves au-dessus d'un feu de tristesse \textit{CE N'EST PAS MA FILLE! J'AVAIS PRIS DES VALIUMS, CRISSE!}\footcite[38]{Labreche2008}
    \normalsize
  \end{singlespace}
\end{quote}
La monoparentalité, la pauvreté\footnote{La rue Wurtele est située dans le quartier Centre-Sud à Montréal, faubourg populaire abritant une population majoritairement défavorisée.}, la maladie mentale de la mère et son rejet de sa maternité: tout y passe!
Quelques lignes suffisent à mettre en place la tragédie des enfances racontées dans les romans de Marie-Sissi Labrèche.
\par
Si le seul rejet de Sissi par sa mère a déjà le potentiel de participer à son autonomisation et à sa transition accélérée vers l'âge adulte, son exposition à des événements dramatiques y contribue d'autant plus.
Ainsi, lorsque la narratrice de \textit{Borderline} a onze ans, sa mère tente de se suicider: \guil{Ma mère vient de se suicider. Elle a pris du Lithium Carbonate, des Luvox, des Dalmane et des Valium; toutes ses pilules en même temps. Puis, elle a crié: \textit{JE VOUS AIME TOUS!}}~(\textit{B}:~30)
Cynique, Sissi estime qu'il s'agit là d'une \guil{drôle de façon d'aimer le monde}~(\textit{B}:~30).
Réagissant de façon mature, elle compose le numéro d'urgence afin d'obtenir des secours, exposant ironiquement la situation dans toute sa complexité:
\begin{quote}
  \begin{singlespace}
    \small
    J'ai appelé à l'urgence de l'hôpital Notre-Dame. Je ne sais pas comment j'ai fait. Je ne me rappelle rien. En fait, je pense que c'est une autre petite fille qui l'a fait pour moi. Une autre petite fille blonde comme moi qui m'a souri et qui a pris sa main pour composer le numéro. Elle a parlé aussi: \textit{Bonjour, suis-je bien à l'hôpital Notre-Dame? Oui, bon. J'aurais besoin d'une ambulance $\left[ \dots \right]$.}~(\textit{B}:~31)
    \normalsize
  \end{singlespace}
\end{quote}
Derrière cet humour se cache cependant une détresse évoquée par l'amnésie et l'idée du dédoublement de la personnalité dans cette \guil{autre petite fille}.
Réexposant son intérêt pour la télévision, l'enfant démontre surtout sa volonté d'avoir une vie \guil{normale}, à tout le moins dépourvue de drames: \guil{Ensuite, la petite fille m'a dit: \textit{Viens, on va aller voir qui a gagné.} Alors moi et la petite fille, on a regardé la télé.}~(\textit{B}:~32)
Ainsi, si l'enfant réagit de façon mature et avec beaucoup de sang-froid, il n'en est pas moins perturbé par la tragédie qui se joue autour de lui: \guil{J'ai beaucoup de misère à me concentrer. Il y a comme une boule dans ma gorge, une boule qui est en train de devenir une pastèque tellement elle grossit. Je ne sais pas si j'ai envie de pleurer ou de vomir, tout est confus.}~(\textit{B}:~29-30)
Quand un policier parle à Sissi, cette dernière a \guil{envie de [s]e faire toute petite}~(\textit{B}:~32) afin qu'il l'avale, indiquant une volonté de disparition.
Point de départ de son autonomisation, le drame familial de Sissi est également une grande source de solitude:
\begin{quote}
  \begin{singlespace}
    \small
    Je ne sais pas si ma mère est morte. Je ne sais pas ce qui va m'arriver. $\left[ \dots \right]$ J'ai faim, mais je ne mangerai pas. Il n'y a personne pour me faire la bouffe, et de toute façon mon ventre est déjà trop rempli. Le vide m'habite. Il s'infiltre dans chacune de mes cellules à une vitesse vertigineuse $\left[ \dots \right]$. Je suis couchée par terre dans le salon, le plancher est froid et me glace le dos. Je m'en fous. Je ne me lèverai pas d'ici. Je n'ai plus envie de bouger. Le vide est tellement lourd.~(\textit{B}:~33)
    \normalsize
  \end{singlespace}
\end{quote}
Si la jeune fille comprend bien ce qui se passe, à savoir que sa mère a tenté de se suicider, elle est également placée dans un état d'ignorance puisqu'on ne l'informe pas de la suite des événements et qu'elle ne sait pas si sa mère est toujours vivante ou non.
D'abord autonomisée par la gravité intrinsèque des événements -- il est peu commun qu'une enfant de cet âge soit témoin d'une tentative de suicide --, Sissi est doublement laissée à elle-même par le fait que personne ne la prenne en charge pour lui offrir un soutien psychologique.
S'il est mention de son oncle, supposé venir la garder, et d'un policier qui lui \guil{dit des mots} qu'elle ne comprend pas~(\textit{B}:~32), elle est finalement laissée à elle-même: \guil{Moi, on m'a laissée là, seule.}~(\textit{B}:~33)
\par
Avec toute cette tragédie qui l'entoure, l'enfant semble parfois adopter une attitude d'ouverture et d'acceptation des événements.
Elle dit par exemple avoir hâte de raconter le suicide de sa mère: \guil{Un événement pareil, ça me donne de l'importance, ça fait de moi un point de mire.}~(\textit{B}:~37)
Comédienne, elle utilise ses talents de conteuse afin de bien divertir le public auquel elle fait son récit de l'histoire:  \guil{Pour l'annoncer à Céline, je prends mon air tragique. Je dis ça parce que j'ai l'impression de vivre dans un film.}~(\textit{B}:~37)
Cette fausse sérénité est cependant un simple mécanisme de protection servant à camoufler l'angoisse de l'enfant: \guil{C'est tellement gros ce qui arrive qu'il faut que je me force pour avoir l'air dedans. Quand des choses comme ça m'arrivent, je me divise en deux: une partie fait semblant, pendant que l'autre se cache et tremble.}~(\textit{B}:~37)
Cette scission de l'enfant en deux parties, l'une adulte et l'autre enfant, relève à notre avis de l'instinct de survie: la façade de calme a pour but d'éviter l'écroulement autrement inévitable.
\par
Ainsi, les caractéristiques de l'adulte présentes chez Sissi sont sollicitées et développées par le drame qui se joue autour d'elle; puis, elles sont exacerbées à des fins de protection de ses caractéristiques d'enfant.
Le développement de son autonomie sert alors à masquer la solitude qui lui est imposée tandis que sa méfiance à l'égard de l'autre tend à protéger ce qu'elle a de pureté face au monde des adultes.
Nous estimons que l'autonomisation de l'enfant chez Marie-Sissi Labrèche n'est pas exclusivement dûe à sa volonté, mais qu'elle est surtout induite par son environnement et le contexte de son existence.
Sissi n'est pas une enfant-adulte parce qu'elle apprécie la liberté et l'autonomie, mais plutôt parce que les événements qui forment sa vie relèvent à ce point de l'expérience adulte qu'elle en ressort mûrie voire flétrie.
Du point de vue strict du personnage, bien que Sissi soit assurément une enfant-femme, il ne s'agit pas pour elle d'un constat positif: elle appartient à ce type parce que sa vie est une succession de difficultés et que sa maturation se place dans la lignée d'une philosophie de simple survie.


\section{Retour à l'enfant-femme: Zazie et Sissi comme adultes involontaires}
Suite à nos lectures des romans \textit{Zazie dans le métro} et \textit{Borderline}, nous évoquons l'hypothèse qu'il n'y ait pas qu'une seule et unique façon de correspondre au modèle de l'enfant-femme.
Au-delà des évidentes distinctions de genre entre Zazie et Sissi -- la première est l'héroïne d'une parodie burlesque publiée à l'issue de la Seconde Guerre mondiale, tandis que la seconde est une des multiples présences d'une même narratrice dans un roman autofictionnel contemporain --, la différence la plus marquante au regard de notre analyse est dans la manière dont chacune combine les traits de l'enfance et de l'âge adulte: comme une aventure ou comme une épreuve, avec détachement ou dans la souffrance, avec facilité ou difficilement.
Ainsi, si Zazie et Sissi sont toutes les deux des enfant-adultes et des enfant-femmes, il demeure qu'elles ne vivent pas l'expérience de la même façon: l'une en sortira grandie -- \guil{J'ai vieilli}, dira-t-elle~(\textit{Z}:~181) -- et l'autre meurtrie: \guil{Crisse! Je n'ai même pas besoin de me grandir, malgré ma petite taille, j'ai déjà mille ans.}~(\textit{B}:~68)


\subsection{Zazie comme enfant-adulte absolu: le contre-mythe de Barthes}
Dans son court article \textit{Zazie et la littérature\footcite{Barthes1964}}, le sémiologue et critique français Roland Barthes présente Zazie comme personnage à vocation contre-mythique, à savoir qu'elle défait le mythe de la jeunesse de par son inadéquation à ce même mythe.
Outre son analyse détaillée sur le rapport zazique au langage, également de nature à déconstruire la mythologie et qui constitue un sujet qui en lui seul mérite d'être traité en profondeur, Barthes propose une lecture du personnage de Zazie sous l'angle de la subversion des âges.
\par
Zazie, comme personnage, pose dès le début la question de l'appartenance à un âge ou à un autre: \guil{De cette propension à mener, tout en déroutant, le monde adulte, Barthes élèvera Zazie au rang de contre-mythe en évoquant sa jeunesse comme une abstraction, puisqu'elle réunit l'enfance et la maturité, et en postulant que sa fonction est de \guill{dégonfler} la mythologie qui l'environne\footcite[88-89]{Maurin2007}.}
L'incapacité de Zazie d'être une enfant ou une adulte mène donc Barthes à lui accoler le statut de contre-mythe.
Sa réflexion repose sur le constat que Zazie est à la fois réaliste tout en étant impossible à imaginer dans la réalité, en se posant comme \guil{un être irréel, magique, faustien, puisqu'il est contraction surhumaine de l'enfance et de la maturité, du \guill{Je suis jeune, hors du monde des adultes} et du \guill{J'ai énormément vécu}\footcite[129]{Barthes1964}.}
\par
Dans ce sens, Zazie nous apparaît comme représentant l'enfant-adulte absolue, parfaite incarnation de cette créature qui joint l'enfance à l'âge adulte.
Le choix du terme \guil{enfant-adulte} au détriment d'\guil{enfant-femme} n'est pas ici anodin et soutient cette idée que Zazie n'est à proprement parler \guil{genrée} que par autrui et que cette assignation de genre que lui impose l'Autre ne la définit ni ne consacre son appartenance à un genre en particulier.
Zazie demeure donc, à notre avis, un personnage d'enfant bien avant d'en être un de femme, et est donc par excellence l'enfant-adulte.

\subsection{Sissi comme enfant-femme contextuelle: la résilience}
La situation ne se présente pas tout à fait de la même façon chez le personnage de Sissi, pour qui le jumelage de l'enfance et de l'âge adulte ne se fait pas aussi facilement.
Nous avançons l'hypothèse que Sissi n'est une enfant-adulte et une enfant-femme que par \textit{obligation}, à savoir que son autonomisation serait le résultat d'un déterminisme socio-familial et constituerait une forme de résilience.
\par
Désignant étymologiquement la résistance au choc des matériaux, le terme \guil{résilience} s'est avec le temps transporté dans les sciences sociales, pour y prendre une définition plus adaptée à la nature de ces disciplines: \guil{Capacité d’une personne ou d’un groupe à se développer bien, à continuer à se projeter dans l’avenir en dépit d’événements déstabilisants, de conditions de vie difficiles, de traumatismes sévères\footcite{Manciaux2001a}.}
Cette résilience se fonde sur plusieurs facteurs, lesquels se regroupent en trois grandes catégories: d'abord les facteurs individuels, comme l'autonomie, la capacité de distanciation ou la sociabilité; ensuite les facteurs familiaux, référant à la qualité des interactions avec la famille proche; enfin les facteurs sociaux, lesquels touchent au support extérieur à la famille, notamment à l'école et aux ressources communautaires\footcite{Anaut2005}.
Psychiatre et éthologue français ayant vulgarisé le concept de résilience, Boris Cyrulnik accorde une grande importance à l'environnement familial -- biologique ou substitutif -- dans la construction de la résilience chez l'enfant, allant même jusqu'à parler de \guil{tuteurs de résilience\footcite{Cyrulnik1998}} comme manifestations actives de ces facteurs familiaux et sociaux.
Dans ses recherches, lesquelles portaient initialement sur les enfants ayant vécu la guerre, Cyrulnik évoque le tempérament de l'enfant en tant que facteur individuel mais l'explique comme étant inévitablement forgé par la famille et l'entourage\footnote{Il évoque que l'aptitude à la résilience d'un individu s'acquiert par l'affection qu'il reçoit lorsqu'il est nouveau-né.}.
Ainsi, la personnalité résiliente ne peut qu'être acquise, puis développée.
En l'absence d'un environnement soutenant et d'un milieu affectif adéquat, l'enfant développe donc un tempérament résilient, lequel se manifeste par les attitudes de protection que sont la révolte, le rêve, la mégalomanie, le déni ou même l'humour\footcite[3]{Taubes2001}.
\par
C'est à la lumière de ces théories sur la résilience que nous parvenons à établir la véritable nature de l'enfant-femme qu'incarne Sissi.
Nous estimons que la coexistence chez elle de caractéristiques adultes et enfantines relève du développement d'une personnalité résiliente.
Nous retrouvons d'ailleurs chez Sissi plusieurs de ces éléments constitutifs du tempérament résilient évoqués par Cyrulnik: elle se révolte contre l'autorité notamment en l'insultant, rêve d'une vraie famille, fait le projet plutôt ambitieux d'un \guil{come-back démentiel}~(\textit{B}:~129), nie être une victime et formule un grand nombre de jeux de langage.
Sissi emploie ces mécanismes de protection et parvient à se développer dans une relative normalité malgré tous les événements perturbants qui marquent son enfance.
Puisque ni sa mère -- folle -- ni sa grand-mère -- contrôlante -- ne lui offrent un véritable milieu affectif, nous croyons que sa résilience relève quasi-exclusivement de facteurs individuels, quoique certains adultes de son environnement scolaire, notamment son professeur d'éducation physique, puissent être considérés comme un élément extérieur propice au développement de la résilience.
\par
Puisqu'il s'agit d'une réaction face à un traumatisme, ce qui implique forcément une certaine forme de souffrance, l'acquisition du statut d'enfant-femme n'est pas, pour Sissi, une chose naturelle qui \guil{va de soi}.
Si cette résilience et la grande adaptabilité qui en découle sont pour nous des facteurs qui confirment le statut d'enfant-adulte de Sissi, il est indéniable qu'elle appartient au modèle d'abord et avant tout parce que son environnement familial défavorable lui a imposé un parcours de vie difficile.
Elle n'est donc une enfant-femme que parce que son contexte de vie le requiert, en ce sens que son autonomie et sa maturité ne sont pas le fruit de simples prédispositions, mais bien des caractéristiques développées face à divers facteurs familiaux, économiques et sociaux.
\par
Enfin, tout comme Zazie, il est clair pour nous que Sissi n'a de \guil{femme} que l'étiquette qui lui est donnée par l'Autre: bien qu'elle se perçoive \guil{fille} -- à savoir de genre féminin -- dans la crainte des hommes que lui a inculquée sa grand-mère, elle n'a d'\guil{adulte} que quelques caractéristiques induites par des facteurs externes.
Ainsi, Sissi se positionne selon nous comme une enfant-femme contextuelle, dont l'adéquation au modèle est conditionnelle à ce qu'elle évolue dans un certain milieu.

\chapter{Ces enfants de langage: une existence conditionnée par la parole}

\begin{flushright}
                                    \begin{singlespace}
                                    \epigraph{
                                    Je hais tellement l'adulte, le renie avec tant de colère, que j'ai dû jeter les fondements d'une nouvelle langue.}
                                    \par
                                    Réjean Ducharme, \textit{L'Avalée des avalés}\footcite[336-337]{Ducharme1982}
                                    \end{singlespace}
\end{flushright}

Chez l'enfant-femme, le paradigme de l'âge ne se résume pas au seul caractère du personnage.
Son langage rapprochera également des caractéristiques contradictoires, jusqu'à constituer la pierre d'assise de la spécificité de la typologie.
Ainsi, l'enfant-femme est paradoxale jusqu'au bout de la langue, en ce sens qu'elle existe et se distingue par son rapport au langage et par son usage des mots.
\par
En ce sens, nous convoquons le concept d'\guil{enfants de langage\footcite[95-103]{Vurm2014}}, mécanisme qui permet au personnage d'accéder au statut d'enfant-adulte, lequel n'est pleinement atteint que par le maniement de la langue.
C'est donc par l'agencement joueur et enfantin d'un vocabulaire avancé et adulte que se manifestera brutalement la discordance entre les divers âges du personnage d'enfant-femme, à savoir son âge biologique et son âge psychologique.
Par ce ludisme, qui consiste en fait en un jeu sur le code, l'enfant-adulte exposera à la fois son appartenance au monde adulte et sa résistance envers lui.
Ainsi, l'ambiguïté née de cette incapacité à situer le narrateur sur le plan de l'âge -- il n'est ni un enfant, ni un adulte -- transpose le paradoxe de l'âge sur un second plan, soit celui du langage, lequel est à son tour fort ambivalent quant à l'âge du personnage.
\par
Petr Vurm, dont la thèse porte sur les enfants chez Réjean Ducharme, voit l'\guil{indétermination de l'âge linguistique} -- lorsque \guil{des enfants parlent comme des adultes et les adultes glissent souvent vers le discours des enfants} --, comme élément générateur de paradoxes et donc d'un grand intérêt au niveau littéraire\footcite[97]{Vurm2014}.
L'existence en soi d'\guil{enfants de langage} est d'autant plus paradoxale que le terme d'\guil{enfant} signifie étymologiquement \guil{qui ne parle pas\footnote{Le mot enfant provient du latin \textit{infantem}, dérivé accusatif de l'\textit{infans}.}}.
\par
Le \guil{caractère ambigu de l'enfant-adulte $\left[ \dots \right]$ nous mène à une deuxième métamorphose de l'enfant, celle de l'enfant littéraire\footcite[100]{Vurm2014}}, dont l'une des particularités sera sa \guil{lutte contre le monde des adultes\footcite[102]{Vurm2014}.}
Alors que le combat entre l'enfance et l'âge adulte se transpose sur le plan du langage, l'enfant se doit d'utiliser la langue des adultes s'il veut se positionner face à lui à armes égales.
Vurm insiste d'ailleurs sur le fait que même si les enfants sont en révolte, \guil{ils utilisent en même temps la langue, la logique et le raisonnement des adultes $\left[ \dots \right]$\footcite[99]{Vurm2014}.}
Leur usage du langage des adultes relève cependant de la ruse de guerre, sorte de cryptographie militaire mise en \oe{}uvre avec les moyens dont ils disposent.


\section{Zazie et le dégonflage du mythe langagier}
Dans \textit{Zazie dans le métro}, il est évident dès l'entrée en scène du personnage de Zazie que cette dernière ne correspond pas au modèle de l'enfant sage et naïf: \guil{Son langage et son comportement ne relèvent en rien des bonnes manières, elle maîtrise parfaitement l'argot et les grossièretés\footcite[88]{Maurin2007}.}
Cependant, si la critique mentionne systématiquement la \guil{\textit{foul tongue}\footcite[118]{Bernofsky1994}} de Zazie, c'est Roland Barthes qui a véritablement mis au jour la mécanique derrière le langage zazique, y repérant un système de réactions fort différentes envers les deux catégories que sont le langage-objet et le métalangage:
\begin{quote}
  \begin{singlespace}
    \small
    L'innocence de Zazie n'est pas fraîcheur, virginité fragile, valeurs qui ne pourraient appartenir qu'au méta-langage romantique ou édifiant: elle est refus du langage chanté, science du langage transitif; Zazie circule dans son roman à la façon d'un génie ménager, sa fonction est hygiénique, contre-mythique: elle rappelle à l'ordre\footcite[129]{Barthes1964}.
    \normalsize
  \end{singlespace}
\end{quote}
Selon Barthes, la \guil{vocation} du personnage de Zazie serait de déconstruire le langage des adultes en y opposant la vulgarité afin de le remplacer par un langage-jouet enfantin.
C'est ce qui correspond pour nous à son rapport au langage, à la fois démonstratif et constitutif de son statut d'enfant-adulte.

\subsection{Les zazismes: l'empressement phonétique}
Dans le vocabulaire de Zazie, maints mots font l'objet de graphies peu orthodoxes, détournant l'orthographe conventionnelle des mots au profit d'un phonétisme enfantin qui crée un \guil{climat sonore\footcite[79]{Leon1962}} particulier.
Ainsi, Zazie veut un \guil{cacocalo} ou des \guil{bloudjinnzes} et se demande ce que peut bien être un \guil{hormosessuel}.
Version écrite de la prononciation parfois déformée des mots, les \textit{zazismes} relèvent de cette \guil{ortografe fonétik} chère à Raymond Queneau, sorte de \guil{professeur de baragouin\footcite[13]{Queneau1975}} qui avait déjà prédit depuis un moment une révolution dans l'orthographe afin de tendre vers un néo-français constituant en la \guil{notation correcte du français parlé\footcite[20]{Queneau1965}}.
Si l'usage de la langue vernaculaire et de l'oral-écrit est linguistiquement marqueur de réalisme, il semble que cette affirmation soit inexacte en ce qui concerne \textit{Zazie dans le métro}\footnote{Gillian Lane-Mercier estime d'ailleurs que \guil{la logique du texte exclut $\left[ \dots \right]$ toute \guill{logique phonématique} réaliste} puisque les déviations de la norme orthographique ne sont pas constantes. Dans \cite[30]{Lane-Mercier1989}.} alors que la malléabilité du langage y constitue le bourgeon d'une poétique singulière chez un auteur qui apprécie divers types de mots et même les \guil{mots savoureux d'être des mots}: \guil{En vérité, pour Queneau, le graphisme n'est donc pas seulement la clé du réalisme mais aussi celle de la poésie. Et cette poésie est surtout celle de la fantaisie verbale\footcite[79]{Leon1962}.}
Dans son \guil{délire verbal\footcite[82]{Leon1962}}, ce qui importe à Queneau \guil{dans [s]es personnages, ce sont moins les actes qu'ils ont à accomplir que le climat poétique dans lequel ils ont à se mouvoir\footnote{\cite[1]{Hulot1952} cité dans \cite[83]{Leon1962}.}.}
\par
Ainsi, plusieurs digressions à la grammaire semblent relever du simple plaisir de l'écrivain, du \guil{jeu de la graphie pour la graphie} dont la vocation est de détruire le \guil{système établi} afin d'en rebâtir un nouveau\footcite[80]{Leon1962}, notamment l'\guil{agglutination syllabique} -- aussi appelé \guil{\textit{concertina-words}\footcite[45]{Redfern1959}} --, dont le fameux \guil{Doukipudonktan\footnote{\guil{D'où est-ce qu'il pue donc tant.}}}~(\textit{Z}:~7) de l'incipit ou l'exclamation \guil{Lagoçamilébou\footnote{\guil{La gosse à mis les bouts}, signifiant que l'enfant est partie.}}~(\textit{Z}:~33).
Si elles semblent parfois être laissées au hasard lorsqu'elles relèvent de la narration ou de paroles prononcées par les adultes -- à cet effet, Pierre R. Léon souligne que la phonétisation opérée par Queneau est non-systématique et relève \guil{de multiples contradictions} au sein du texte, parfois même à quelques pages ou quelques lignes\footcite[72-73]{Leon1962} --, les fantaisies grammaticales attribuées à Zazie opèrent comme un engrenage bien huilé, mécanisme de réactions induites par le langage et les écarts au langage des adultes.

\subsubsection{L'apport étranger: comme un langage de proximité}
Dans sa réinterprétation orthographique des mots, plusieurs variations concernent des termes étrangers qui sont orthographiés phonétiquement selon leur prononciation avec l'accent français par un auteur qui \guil{aime lui aussi habiller à la française les mots d'emprunt étranger\footcite[80]{Leon1962}}.
Tout d'abord le mot servant à désigner ces langues étrangères, les langues \guil{forestières}~(\textit{Z}:~87, 94), constitue selon Pierre Léon un bricolage adjectival \textit{à la française} à partir du mot anglais \textit{Far East} -- \guil{langues \textit{far-eastières}\footcite[82]{Leon1962}} --, utilisé ici de façon généralisatrice pour désigner tout ce qui est étranger à la France et au français \footnote{Nous pourrions ajouter que le mot \guil{\textit{forestiero}} désigne un étranger en italien.
La manie encyclopédique de Queneau ne nous étant pas étrangère -- ou \guil{forestière}! --, nous estimons qu'il s'agit là d'un autre trésor linguistique enfoui par l'auteur à l'intention des lecteurs curieux.}.
S'ajoutent le \guil{guidenappeur\footnote{Mot-valise entre \guil{guide} (touristique) et \textit{kidnapping}, mot anglais désignant un enlèvement (d'enfant).}}, le \guil{flicmane\footnote{De \textit{policeman}, mot anglais désignant un officier de police.}}, les \guil{vécés\footnote{D'après l'anglais W-C, pour \textit{water closet}, à savoir un cabinet d'eau (et de toilette).}} et même l'expression \guil{apibeursdè touillou\footnote{Souhait d'anniversaire -- \textit{Happy birthday to you} -- phonétisé en incluant un fort accent français dans la prononciation.}}.
Cette appropriation par la \guil{noble langue françouèze\footcite[127]{Barthes1964}} de concepts américains accentue l'impression de familiarité et induit un sentiment de proximité puisqu'il s'agit d'une reproduction relativement fidèle de la sonorité que prennent les mots lorsqu'ils sont prononcés par les Français et les Parisiens.
\par
Pour le personnage de Zazie, cette idée de \guil{langage de proximité} est encore plus concrète puisqu'il s'agit en réalité de son terrain de jeu:
\begin{quote}
  \begin{singlespace}
    \small
    Le langage-objet, c'est le langage qui se fonde dans l'action même, qui \textit{agit} les choses, c'est le premier langage transitif, celui dont on peut parler mais qui lui-même transforme plus qu'il ne parle. C'est exactement dans ce langage-objet que vit Zazie, ce n'est donc jamais lui qu'elle distance ou détruit. Ce que Zazie parle, c'est le contact transitif du réel: Zazie \textit{veut} son coca-cola, son blue-jean, son métro, elle ne parle que l'impératif ou l'optatif, et c'est pour cela que son langage est à l'abri de toute dérision\footcite[128]{Barthes1964}.
    \normalsize
  \end{singlespace}
\end{quote}
Le langage-objet tel que défini par Barthes, langue du concret et du réel, suscite chez Zazie le respect puisqu'il s'agit de la langue de ce qu'elle veut.
D'un point de vue sémiologique, le mot-objet, sorte de signe linguistique zazique, se construit directement du signifiant au référent, évacuant le concept médian de signifié: Zazie ne pense pas l'objet, elle le veut et le nomme.
\par
Si elle ne la met à distance ni ne la détruit, il demeure que l'enfant profite de la malléabilité de la langue et qu'elle en dévoile une prononciation nouvelle hautement phonétique.
Constituant des accrocs à l'orthographe usuelle, les \guil{zazismes} consistent en la mise à l'écrit de prononciations enfantines ou erronées attribuées au personnage de Zazie.
Il y a d'abord le terme \guil{bloudjinnzes}, françisation accentuée de \textit{blue jeans}, mot que Zazie se refuse de prononcer devant un commerçant de la foire aux puces: \guil{Elle n'ose pas énoncer le mot disyllabique et anglo-saxon qui voudrait dire ce qu'elle veut dire.}~(\textit{Z}:~43)
Pour revenir brièvement à la sémiotique, le signe \guil{bloudjinnzes}, intégré par Zazie à son vocabulaire en tant que mot-objet désiré, semble susciter chez elle de très vives émotions, comme si le seul fait de prononcer le mot (le signifiant) impliquait directement l'objet (le référent) sans même passer par la représentation mentale (le signifié) de cet objet.
Enfin, l'inversion des voyelles mène à l'invention du mot \guil{caco-calo}, également objet du désir, et dans lequel le linguiste et phonéticien Pierre Léon lit un calembour dont le second sens serait \guil{à l'eau\footcite[80]{Leon1962}}.
Les zazismes incluent également le phonétisme erroné \guil{hormosessuel}, objet de toutes les questions et qui, en plus du jeu sur l'orthographe, relève d'une forme particulière d'humour portant sur le double-sens.

\subsubsection{L'invention de nouveaux mots: l'insidieuse moquerie}
Au-delà du zazisme, le discours sur l'homosexualité prend parfois une tournure connotée par la liaison dans la prononciation des termes \guil{un hormosessuel}, syntagme homophone d'\guil{un normosessuel}, sorte de néologisme concédant une \textit{norme sexuelle}.
C'est le personnage de Fédor Balanovitch, lequel \guil{connaissait à fond la langue française} qui relève le jeu zazique sur la sonorité.
Lorsque Zazie lui demande si Gabriel est \guil{un hormo}, l'homme lui répond: \guil{Tu veux dire un normal.}~(\textit{Z}:~117)
Paradoxalement, Gabriel serait ainsi désigné à la fois comme homosexuel et comme \guil{normosexuel}, terme dont l'actualisation en fonction des théories \textit{queer} désignerait possiblement un hétérosexuel cisgenre.
Cette norme est donc définie par un personnage dont le physique est décrit comme éminement masculin, qui pratique un métier féminisé -- danseuse de charme -- sous un prénom également féminin, Gabriella -- et dont la conjointe semble à la fin du roman être plutôt un conjoint.
La normosexualité semble associée au maître de l'inversion sexuelle par excellence, au point où la blague semble plus qu'évidente.
De même, si le questionnement de Zazie à savoir ce que peut bien être un \guil{hormosessuel} relève possiblement de la naïveté enfantine qui lui est généralement supposée, la double lecture que nous proposons va à l'opposé: Zazie, posée comme être de raison servant selon Barthes à dégonfler la \textit{doxa} et le mythe de la jeunesse\footcite{Barthes1964}, ne serait-elle pas celle qui suggère la fin de l'hétéronormativité et de la catégorisation en \guil{normosessuels} et \guil{anormaux-sessuels}?
\par
Par le biais de Zazie, sorte de marionnette qui permet à son auteur d'écrire ce que lui-même ne peut pas dire, Raymond Queneau laisse également transparaître sa propre moquerie.
De façon tout aussi humoristique que les autres zazismes, l'agglutination syllabique \guil{Singermindépré}~(\textit{Z}:~26) est écrite ainsi comme parole l'enfant, alors que le nom du quartier avait pourtant été bien orthographié lorsque mentionné par Zazie auparavant~(\textit{Z}:~15).
Contenant le nom \guil{singe} de même que le verbe \guil{singer\footnote{\guil{Imiter maladroitement ou d'une manière caricaturale, pour se moquer.} dans \cite[Entrée: « Singer »]{Robert2013}}}, nous estimons qu'il pourrait s'agir d'une critique moqueuse de l'élite intellectuelle et culturelle parisienne habitant le quartier Saint-Germain-des-Prés, où sont d'ailleurs situées plusieurs maisons d'édition, dont Gallimard qui a édité \textit{Zazie dans le métro}.
Il nous semble, au-travers de cette moquerie zazique, apercevoir l'humour de Raymond Queneau, lequel a possiblement transposé dans les paroles de sa jeune créature une part de sa propre critique envers la société bourgeoise de son époque.

\subsection{La vulgarité comme violence langagière}
Qualifiée par le tenancier du bar de \guil{petite salope qui di[t] des cochoncetés}~(\textit{Z}:~18), Zazie maîtrise habilement le langage des adultes, y compris (et surtout) les grossièretés et l'argot\footcite[88]{Maurin2007}.
Elle offense pour ainsi dire les adultes par sa vulgarité, comme l'exprime la veuve Mouaque: \guil{La preuve, vous n'avez qu'à l'écouter parler (geste), elle est d'une grossièreté, dit la dame en manifestant tous les signes d'un vif dégoût.}~(\textit{Z}:~95)
Si nous ne remettons aucunement en question le fait que Zazie soit, linguistiquement parlant, vulgaire et grossière, nous émettons l'hypothèse qu'elle ne le soit pas gratuitement.
Il nous semble plutôt que le langage soit pour elle une façon de combattre les adultes et que son l'emploi de la vulgarité et d'un vocabulaire adulte lui permette d'aborder ce combat langagier à armes égales: \oe{}il pour \oe{}il, dent pour dent, mot pour mot!

\subsubsection{Le langage de brousse comme réaction à la contrariété}
Pour Zazie, l'utilisation d'un langage subversif semble avant tout constituer une réaction face à la contrariété.
Ainsi, l'exclamation qui lui échappe est d'une sincérité sans équivoque lorsqu'elle apprend que le métro est en grève: \guil{Ah! les salauds, s'écrie Zazie, ah! les vaches. $\left[ \dots \right]$ Sacrebleu, merde alors.}~(\textit{Z}:~10)
Devant la plus grande des déceptions -- le métro étant son principal objet de convoitise --, la jeune fille expose d'un seul coup toute l'étendue de son vocabulaire vulgaire afin d'exprimer sa frustration.
\par
Ce même vocabulaire sera d'ailleurs repris lors de contrariétés subséquentes.
Par exemple, l'expression \guil{le salaud} servira à désigner le \textit{type} lorsqu'il lui reprendra \guil{ses} bloudjinnzes qu'elle avait volés~(\textit{Z}:~56) ou encore le sergent de ville dont elle se méfie et qu'elle \guil{voi[t] venir avec ses gros yéyés}~(\textit{Z}:~101).
Parlant du tarif de l'avocat de sa mère, elle commentera: \guil{Il a été gourmand, la vache.}~(\textit{Z}:~48)
L'expression revient alors qu'elle s'indigne de ce que dit le \textit{type} à son propos -- \guil{C'est vache, ça, dit Zazie}~(\textit{Z}:~108) --, comme une ritournelle face à la contrariété.
Enfin, l'expression \guil{merde alors} sera reprise comme marqueur d'impatience ou de contrariété: d'impatience d'abord, que ce soit face à la lenteur des discussions futiles entre adultes -- \guil{Alors quoi, merde, dit Zazie, on va le boire, ce verre?}~(\textit{Z}:~15) -- ou face au refus des adultes de répondre à ses questions sur la sexualité -- \guil{Tu réponds, oui ou merde, cria Zazie}~(\textit{Z}:~87); de contrariété lorsque les enfantillages des adultes l'empêchent de dormir: \guil{Alors quoi merde, on peut plus dormir?}~(\textit{Z}:~25)
Ne reprenant qu'une seule des expressions à la fois, Zazie paraît évidemment moins irritée par ces désagréments que face à la grève du métro.
Que son utilisation d'un vocabulaire vulgaire soit faite avec parcimonie ou immodérément en fonction de l'ampleur de sa contrariété, l'objectif reste toutefois le même: il s'agit pour Zazie d'un exutoire à sa frustration, laquelle est souvent causée par l'embourbement bureaucratique des adultes.
Le langage grossier constitue alors une nouvelle modalité du rejet global du monde adulte.
\par
Se \guil{foutant} tout d'abord de l'explication selon laquelle il n'y a pas que pour elle que le métro soit en grève~(\textit{Z}:~10), Zazie exprime le même détachement nonchalant face à la belle ville que Gabriel lui montre: \guil{Je m'en fous, dit Zazie, moi ce que j'aurais voulu c'est aller dans le métro.}~(\textit{Z}:~11)
Elle n'a d'intérêt que pour une sélection restreinte de choses -- la sexualité, les \textit{jeans} vendus dans les surplus américains et surtout le métro parisien -- et en fait presque une obsession, exprimant envers tout le reste la plus grande indifférence qui soit.
Que sa fixation porte sur le métro ou plus tard sur les bloudjinnzes, face auxquels elle déclarera se foutre des boussoles~(\textit{Z}:~43), Zazie manifeste un rejet de ce qui est autre d'une expression vaguement grossière: \guil{Je m'en fous.}
Plus violemment encore, son seul désintérêt laisse place à un rejet pur et simple alors qu'elle \guil{emmerde} le monde adulte en général: \guil{Les pas marants, dit Zazie, je les emmerde. $\left[ \dots \right]$ Parce que moi, le billard, ça m'emmerde.}~(\textit{Z}:~120)
Auparavant qualifiés de \guil{petits marants}~(\textit{Z}:~13), l'oncle Gabriel et son ami Charles sont possiblement exclus de cette boutade méprisante envers le monde adulte.
De l'expression un peu brusque jusqu'à l'affront direct, Zazie élève par le langage une barrière entre elle et le monde adulte, jusqu'à dire à Charles: \guil{La nouvelle génération, dit Zazie, elle t'...}~(\textit{Z}:~15)

\subsubsection{La clausule zazique: mon cul!}
Si ce \guil{langage de brousse} est régulièrement utilisé par Zazie en réaction à une action (ou une inaction) des adultes, c'est face à leur langage que sa violence atteint son paroxysme et qu'elle se pose le plus concrètement comme enfant-adulte.
Selon Barthes, c'est en opposant à certaines prises de parole des adultes une \guil{clausule zazique\footcite{Barthes1964}} que Zazie dégonfle véritablement toute la mythologie et qu'elle acquiert son statut de contre-mythe.
Si tous sont d'accord à l'effet que le langage de Zazie est inconvenant -- Charles dira d'ailleurs qu'\guil{Elle peut pas dire un mot, cette gosse, sans ajouter mon cul après.}~(\textit{Z}:~18) --, l'analyse de Barthes va beaucoup plus loin en constatant qu'il s'agit là d'une véritable \guil{technique du dégonflage} systématiquement opérée sur le méta-langage tout en épargnant le langage-objet\footcite[128]{Barthes1964}. Ainsi, ce que rejette Zazie est le \guil{méta-langage des grandes personnes}:
\begin{quote}
  \begin{singlespace}
    \small
    Ce méta-langage est celui dont on parle, non pas les choses, mais \textit{à propos} des choses (ou \textit{à propos} du premier langage). C'est un langage parasite, immobile, de fond sentencieux, qui double l'acte comme la mouche accompagne le coche; face à l'impératif et à l'optatif du langage-objet, son mode principiel est l'indicatif, sorte de degré zéro de l'acte destiné à \textit{représenter} le réel, non à le modifier. Ce méta-langage développe autour de la lettre du discours un sens complémentaire, éthique, ou plaintif, ou sentimental, ou magistral, etc.; bref, c'est un \textit{chant}: on reconnaît en lui l'être même de la Littérature\footcite[128]{Barthes1964}.
    \normalsize
  \end{singlespace}
\end{quote}
Complètement opposé au langage-objet dont font notamment partie les zazismes et autres fantaisies grammaticales, ce méta-langage représente la \guil{langue des idées}, sorte de réflexion à haute voix qu'abhorre plus que tout Zazie.
Aussi, bien au-delà de la simple grossièreté ou impolitesse, la vulgarité est-elle, pour l'enfant, un moyen de rejeter en deux petits mots toute la \textit{doxa}.
\par
Concrètement, Zazie appose d'abord sa clausule vulgaire lorsque l'adulte tente de la définir par la parole.
Par exemple, elle réagit promptement lorsque Gabriel lui ordonne de ne pas être snob -- \guil{Snob mon cul}~(\textit{Z}:~11) -- ou que \guil{Turandot, d'un air supérieur} sous-entend qu'elle est gentille -- \guil{Gentille mon cul}~(\textit{Z}:~26).
Lorsque la veuve Mouaque suggère que l'enfant puisse avoir des \guil{qualités}, la principale intéressée réplique aussitôt: \guil{Qualités mon cul, grommela Zazie.}~(\textit{Z}:~98)
La réplique est quasi-automatique lorsque l'on parle d'elle pour la qualifier: Zazie refuse de sa \guil{clausule assassine} tout étiquetage de sa personne.
\par
Sorte de tentative renouvelée de la définir par l'impératif, les règles de bienséance sont rejetées tout aussi violemment par Zazie, qui opère un rejet tout aussi catégorique lorsque l'on tente de lui faire la morale: \guil{Politesse mon cul}~(\textit{Z}:~125), décrètera-t-elle.
Par son refus de respecter ces règles que tentent de lui imposer les adultes, Zazie mène selon Barthes une importante opération contre-mythique puisqu'elle dégonfle le mythe de l'enfant sage que cachent les propos des adultes.
C'est par la violence de sa ritournelle \guil{mon cul} que l'enfant réussira à rejeter ce mythe, à \textit{ne pas être} l'enfant bien élevée que l'on attend d'elle:
\begin{quote}
  \begin{singlespace}
    \small
    -- Il ne faut pas brutaliser comme ça les grandes personnes. \\
    -- Grandes personnes mon cul, répliqua Zazie. $\left[ \dots \right]$ \\
    -- [La violence] est éminemment condamnable. \\
    -- Condamnable mon cul, répliqua Zazie, je ne vous demande pas l'heure qu'il est.\\
    -- Seize heures quinze, dit la bourgeoise. \\
    -- Vous n'allez pas laisser cette petite tranquille, dit Gabriel qui s'était assis sur un banc. \\
    -- Vous m'avez encore l'air d'être un drôle d'éducateur, vous, dit la dame. \\
    -- Éducateur mon cul, tel fut le commentaire de Zazie.~(\textit{Z}:~95)
    \normalsize
  \end{singlespace}
\end{quote}
Par sa clausule, Zazie dégonfle coup sur coup les grands mythes que sont l'autorité, la morale et l'éducation, lesquels représentent tous l'existence d'un certain ascendant des adultes sur les enfants.
Le \guil{mon cul} zazique est donc d'une efficacité remarquable en ce sens qu'il détruit tout le rapport de subordination entre l'enfant et l'adulte, libérant l'espace requis pour la naissance d'un enfant-adulte.
\par
À cet effet, le projet de carrière de Zazie démontre sa compréhension manifeste du rapport d'autorité entre les adultes et les enfants -- elle anticipe le moment où ce sera elle l'autorité --, tout en manifestant tous les signes langagiers d'un refus d'obéir à cette autorité.
Ainsi, après l'avoir dégonflé par le langage, Zazie reconstruit le grand mythe sur l'éducation:
\begin{quote}
  \begin{singlespace}
    \small
    -- Moi, déclara Zazie, je veux aller à l'école jusqu'à soixante-cinq ans. \\
    -- Jusqu'à soixante-cinq ans? répéta Gabriel un chouïa surpris. \\
    -- Oui, dit Zazie, je veux être institutrice. \\
    -- Ce n'est pas un mauvais métier, dit doucement Marceline. Y a la retraite. \\
    $\left[ \dots \right]$\\
    -- Retraite mon cul, dit Zazie. Moi c'est pas pour la retraite que je veux être institutrice. \\
    $\left[ \dots \right]$ \\
    -- Alors? Pourquoi que tu veux l'être, institutrice? \\
    -- Pour faire chier les mômes, répondit Zazie. Ceux qu'auront mon âge dans dix ans, dans vingt ans, dans cinquante ans, dans cent ans, dans mille ans, toujours des gosses à emmerder. \\
    -- Eh bien, dit Gabriel. \\
    -- Je serai vache comme tout avec elles. Je leur ferai lécher le parquet. Je leur ferai manger l'éponge du tableau noir. Je leur enfoncerai des compas dans le derrière. Je leur botterai les fesses. Parce que je porterai des bottes. En hiver. Hautes comme ça (geste). Avec des grands éperons pour leur larder la chair du derche. \\
    $\left[ \dots \right]$ \\
    -- D'ailleurs, dit Gabriel, dans vingt ans, y aura plus d'institutrices: elle seront remplacées par le cinéma, la tévé, l'électronique, des trucs comme ça. $\left[ \dots \right]$ \\
    Zazie envisagea cet avenir un instant. \\
    \guil{Alors, déclara-t-elle, je serai astronaute. \\
    -- Voilà, dit Gabriel approbattivement. Voilà, faut être de son temps. \\
    -- Oui, continua Zazie, je serai astronaute pour aller faire chier les Martiens.}~(\textit{Z}:~20-21)
    \normalsize
  \end{singlespace}
\end{quote}
Désintéressée par le monde adulte, c'est encore par sa clausule que Zazie porte le coup fatal à la Grande histoire. À Gabriel qui lui propose de l'emmener \guil{voir vraiment les Invalides et le tombeau véritable du vrai Napoléon}~(\textit{Z}:~13), elle rétorque, sans appel: \guil{Napoléon mon cul, réplique Zazie. Il m'intéresse pas du tout, cet enflé, avec son chapeau à la con.}~(\textit{Z}:~13)
Véritable mécanisme de rejet de la littérarité adulte [et] du méta-langage, la clausule zazique \guil{est comme une détonation finale qui surprend la phrase mythique $\left[ \dots \right]$, la dépouille rétroactivement, en un tour de main, de sa \textit{bonne conscience}\footcite[129]{Barthes1964}.}
Autant qu'elle déstabilise les adultes par son comportement, Zazie s'attaque cette fois-ci à leur langage, qu'elle détourne par sa courte expression vulgaire.
Les adultes étant, aux yeux des enfants, de vulgaires \guil{fonctionnaires\footcite[100]{Vurm2014}}, se pourrait-il que la vocation contre-mythique de Zazie soit en fait de démontrer la \guil{dérive bureaucratique}?
Il est permis d'y croire, puisqu'en détournant invariablement la langue de l'\guil{administration}, c'est tout le système des adultes qu'elle tourne en ridicule: \guil{Cette clausule zazique résume tous les procédés du contre-mythe, dès lors qu'il renonce à l'explication directe et se veut lui-même traîtreusement littérature\footcite[129]{Barthes1964}.}


\subsection{Zazie, Grande Inquisitrice: la poétique de la question}
S'il est possible d'alléguer la grande curiosité de Zazie, il nous apparaît pertinent, afin de comprendre le rapport du personnage à la langue, de réfléchir plus longuement sur la façon dont cette curiosité se présente puisque sa principale manifestation consiste en la construction d'une poétique de la question.
Ainsi, Zazie n'est pas \textit{décrite} comme étant curieuse; elle l'\textit{est}, principalement par le langage, lequel opère souvent sur le mode interrogatif.
\par
Selon Dominique Rioux, dont le mémoire de maîtrise porte spécifiquement sur la pratique interrogative chez les personnages de jeunes filles de Raymond Queneau, le concept de la question se définit comme une \guil{appréhension de l'altérité de ce que le sujet cherche à comprendre et [est], par conséquent, début de la levée des préjugés $\left[ \dots \right]$\footcite[20]{Rioux2006}.}
Les questions des jeunes filles \guil{port[a]nt essentiellement sur la langue et le sexe, deux puissants tabous dans la littérature quenienne\footcite[22]{Rioux2006}}, elles sont représentées parfaitement par le terme \textit{hormosessualité}, lequel réfère à l'anormalité à la fois langagière et sexuelle, allant à l'encontre autant de la grammaire française que de l'hétéronormativité.
\par
Outre le métro, l'hormosessualité supposée de son oncle semble être le principal sujet d'intérêt de Zazie, et la quête initiale du métro cédera bien vite la place à une quête plus globale, soit celle d'une réponse à sa question: \guil{Au fond, dit Zazie, je voudrais bien savoir ce xé. $\left[ \dots \right]$ Ce xé qu'un hormosessuel.}~(\textit{Z}:~97)
Bien sûr, quelques adultes tentent de la contenter en lui offrant de vagues réponses: \guil{Qu'est-ce que c'est un hormosessuel? demanda Zazie. / - C'est un homme qui met des bloudjinnzes, dit doucement Marceline.}~(\textit{Z}:~61)
Zazie n'est cependant pas rassasiée de cette réponse et poursuit sa quête.
Questionnant Charles à propos du fait que Gabriel \guil{pratique l'hormosessualité}~(\textit{Z}:~81), elle lui rapporte les propos du \textit{type} disant qu'\guil{on p[eut] aller en tôle pour ça}~(\textit{Z}:~81) mais demeure toutefois sans réponse sur la signification du mot:
\begin{quote}
  \begin{singlespace}
    \small
    -- Qu'il soit hormosessuel? Mais qu'est-ce que ça veut dire? Qu'il se mette du parfum? \\
    -- Voilà. T'as compris. \\
    -- Y a pas de quoi aller en prison. \\
    $\left[ \dots \right]$ \\
    Et vous? demanda Zazie. Vous l'êtes, hormosessuel?\\
    -- Est-ce que j'ai l'air d'une pédale?\\
    -- Non, pisque vzêtes chauffeur.~(\textit{Z}:~81)
    \normalsize
  \end{singlespace}
\end{quote}
Relevant seulement la définition d'usage du mot \guil{pédale}, Zazie souligne l'absurdité du propos de Charles  -- un chauffeur ne peut évidemment pas être une pédale! --, tout en emmagasinant dans son vocabulaire le double-sens du mot, à savoir qu'il est également synonyme d'homosexuel.
Ainsi, après avoir acquis un vaste vocabulaire de l'homosexualité et de ses synonymes, mais n'en maîtrisant toujours pas le sens, Zazie se montre plus déterminée que jamais à obtenir une réponse, abandonnant les questions délicates et déballant toutes ses interrogations d'un coup à la veuve Mouaque, en reprenant notamment le synonyme \guil{pédale} récemment appris: \guil{Qu'est-ce que c'est au juste qu'une tante? lui demanda familièrement Zazie en vieille copine. Une pédale? une lope? un pédé? un hormosessuel? Y'a des nuances?}~(\textit{Z}:~123)
En plus de vouloir savoir \guil{ce xé} qu'un hormosessuel, Zazie questionne également incessemment pour savoir si son tonton en est un ~(\textit{Z}:~87, 94, 96, 114, 117).
\par
À cette grande question relative à l'hormosessualité de son oncle s'ajoutent les presqu'aussi grandes interrogations de Zazie concernant les langues étrangères: \guil{Tonton Gabriel, dit Zazie paisiblement, tu m'as pas encore espliqué si tu étais un hormosessuel ou pas, primo, et deuzio où t'avais été pêcher toutes les belles choses en langue forestière que tu dégoisais tout à l'heure? Réponds.}~(\textit{Z}:~94)
Ces \guil{langues forestières} référant, comme nous l'avons déjà évoqué, aux langues étrangères, Zazie questionne la capacité de son oncle à les parler: \guil{C'est justement ça, ma deuxième question, dit Zazie. Quand je t'ai retrouvé aux pieds de la tour Eiffel, tu parlais l'étranger aussi bien que lui [Fédor Balanovitch]. Qu'est-ce qui t'avait pris? Et pourquoi que tu recommences plus?}~(\textit{Z}:~118)
\par
Revendiquant le droit imprescriptible de poser des questions, Zazie semble avoir une interrogation adaptée à chaque situation possible.
Ainsi, rejoignant son oncle au bas de la tour Eiffel, elle constate l'absence du taximan et l'interroge:
\begin{quote}
  \begin{singlespace}
    \small
    Elle n'y vit point Charles et le fit remarquer. \\
    \guil{Il s'est tiré, dit Gabriel. \\
    -- Pourquoi? \\
    -- Pour rien. \\
    -- Pour rien, c'est pas une réponse. \\
    -- Oh bin, il est parti comme ça. \\
    -- Il avait une raison. \\
    -- Tu sais, Charles. (geste) \\
    -- Tu veux pas me le dire?
    -- Tu le sais aussi bien que moi.} \\
    $\left[ \dots \right]$ \\
    Zazie revint à son point de départ. \\
    \guil{Tout ça ne me dit pas pourquoi charlamilébou.}~(\textit{Z}:~86-87)
    \normalsize
  \end{singlespace}
\end{quote}
Tenace, la jeune fille pose et repose les mêmes questions, insistant pour que l'on assouvisse son besoin d'être informée: \guil{Et ma question à moi, demanda-t-elle mignardement. On y répond pas?}~(\textit{Z}:~95)
Revenant incessamment à la charge avec ses interrogations -- \guil{Alors tonton, et cette réponse?}~(\textit{Z}:~98) --, Zazie jumelle le mode impératif à la violence physique: les ordres -- \guil{Réponds} et \guil{Réponds donc} -- sont ainsi entrecoupés d'\guil{un bon coup de pied sur la cheville}~(\textit{Z}:~94) d'un Gabriel peu collaboratif.
Elle va jusqu'à justifier sa propre violence envers son oncle par le fait qu'il ne voulait pas lui répondre et exige de lui qu'il le fasse avant de s'en aller.
\par
De par son usage constant du mode interrogatif, nous estimons que le personnage de Zazie se construit, du moins en partie, par le biais de ses questionnements et qu'elle se définit donc d'un point de vue langagier par une poétique de la question.
De tous les objets de son intérêt, c'est à notre avis le terme \guil{hormosessuel} qui résume le mieux cette poétique puisqu'il relève à la fois de l'anomalie langagière et de la sexualité, tous deux sujets de prédilection de la curiosité zazique.
Nous estimons donc que l'hormosessualité -- et tous les questionnements qu'elle suscite -- est le principal vecteur de la poétique langagière de Zazie.
\par
L'enfant-femme faisant montre d'une curiosité hors de l'ordinaire, sa poétique de la question s'inscrit à la limite d'une politique de la torture envers son oncle.
Zazie est Grande Inquisitrice: elle a tous les droits, celui de poser des questions certes, mais surtout celui d'obtenir des réponses.
Elle est tenace devant le silence des adultes, exigeant la vérité devant leurs réponses évasives ou manifestement fausses.
Autant dégonfle-t-elle le code des adultes, autant tient-elle à le posséder: Zazie est sans complexes, revendiquant son droit d'être informée comme une adulte, sans égards à son âge.


\section{Sissi et le langage, ce jouet comme un autre}
Dans \textit{Borderline}, que l'on rappelle être une autofiction dans laquelle la narratrice revisite -- entre autres -- son enfance, le langage est en adéquation avec le milieu défavorisé dans lequel Sissi grandit.
Ainsi, la langue y est populaire, voire vernaculaire, et ce n'est qu'au prix de cours de diction que l'enfant développe une prononciation -- académiquement -- \guil{adéquate}, laquelle sera par ailleurs raillée par ses camarades de classe:
\begin{quote}
  \begin{singlespace}
    \small
    On rit de moi parce que j'ai un accent, parce que je dis \guil{moooâââ} et que j'ar-ti-cu-leuuu quand je parle. On me traite de petite fraîche: \textit{Saleuuu, petiteuuu fraîcheuuu! Saleuuu, petiteuuu fraîcheuuu!} On rit de moi et de mes cours de diction, le mardi et le jeudi, à l'heure du midi. Mes cours de diction, pour m'empêcher de dire un \guil{sien}, un \guil{sat}, des \guil{bisous}, des \guil{zenoux}. Tous les \guil{s}, les \guil{che}, les \guil{s} et les \guil{z} se mélangent.~(\textit{B}:~62-63)
    \normalsize
  \end{singlespace}
\end{quote}
Amenant l'enfant à bien détacher les syllabes et à prononcer correctement les mots, les cours de diction contribuent également à la \guil{construction} de son arsenal langagier, à savoir sa maîtrise de la langue.
C'est une fois la langue ainsi \guil{domptée} que se déploie la poétique langagière de Sissi, mettant de l'avant une \guil{déconstruction} de ce langage à peine construit.
Concrétisant cette \guil{théorie de la violence organisée} évoquée par Jakobson\footcite[40]{Jakobson1973} et présentée comme correspondant à l'exacerbation de la fonction poétique au nom d'une \guil{stratégie ludique\footcite[65]{Seyfrid-Bommertz1999}}, Sissi s'emploie donc, à partir de sa maîtrise du code du langage, à détourner ce dernier à des fins d'amusement.
\par
Si les jeux de langage peuvent sembler à prime abord relever simplement du jeu, Laure Hesbois estime au contraire qu'ils constituent une démonstration de choix de la mécanique de la langue:
\begin{quote}
  \begin{singlespace}
    \small
    Les jeux de langage sont incontestablement un matériau de choix pour étudier le fonctionnement du langage dans toute sa complexité, c'est-à-dire sous l'angle à la fois de la communication et de l'expression, comme système de signes et comme discours symbolique\footcite[18]{Hesbois1986}.
    \normalsize
  \end{singlespace}
\end{quote}
Elle prétend que les jeux de langage résultent de l'amalgame entre l'intérêt pour la linguistique au sens strict du terme\footcite[17]{Hesbois1986}, qui voit le langage comme instrument de communication\footnote{Cette approche découle de la linguistique structuraliste de Ferdinand de Saussure et fait du \textit{signe} l'unité de base; elle se perpétue par les travaux des héritiers fidèles de la linguistique saussurienne, de même que par ceux de Noam Chomsky, qui voit la langue comme une compétence.}, et la \textit{linguistique des locuteurs}\footcite[17-18]{Hesbois1986}, qui tient compte de la présence en langue du sujet parlant\footnote{C'est une approche qui n'est pas sans lien avec la psychanalyse de Freud, laquelle ouvre la porte à l'étude linguistique du \textit{sujet}; elle est portée entre autres par Roman Jakobson ou, plus récemment, par Julia Kristeva, de même que par tous ceux qui s'intéressent aux liens entre linguistique et littérature, dont Roland Barthes.}.
\par
C'est cette mécanique du jeu -- terme qui échappe d'ailleurs à toute définition chez Hesbois -- qui se déploie discrètement dans \textit{Borderline}, où les jeux de mots de tout acabit se côtoient et finissent inévitablement par être générateurs de sens.
Enfin, les jeux de mots laissent souvent échapper quelques vulgarités -- qui ne sont pas totalement étrangères à un certain ludisme --, qu'il s'agisse de jurons ou d'insultes, et qui marquent également le discours de la narratrice en tant que moyen d'exprimer ses émotions.

\subsection{Princesse Sissi et ses jeux de langage}
Dans la bouche de Sissi, plusieurs mots prennent rapidement une tournure enfantine: elle appelle sa grand-mère \guil{mémé}~(\textit{B}:~34, 35, 87, 88, 91, 92, 116, 122, 129, 130, 132), sa mère \guil{môman}~(\textit{B}:~37, 65-67, 91, 134), \guil{[s]on chien Ponpon, [s]on chat Magamarou}~(\textit{B}:~116) et son père biologique \guil{Papa Méchant}~(\textit{B}:~123-125).
Affection ou crainte, les sentiments mis en mots par ces surnoms ont en commun d'être tous exprimés de façon enfantine: par le redoublement de la syllabe dans \guil{mémé} et dans \guil{Ponpon}; par la prononciation fautive dans \guil{môman}; par un procédé qui n'est pas sans rappeler l'agglutination syllabique zazique dans \guil{Magamarou}; et par l'emploi de la majuscule -- ayant un effet d'agrandissement -- dans \guil{Papa Méchant}.
\par
Il en va de même lorsqu'elle explique à son amie Céline qu'elle risque de faire abuser d'elle si elle est \guil{placée} en famille d'accueil.
Réticente à employer un vocabulaire adulte, elle substitue d'abord aux véritables mots une terminologie enfantine: \guil{son machin \dots sa bitte}~(\textit{B}:~38); ce n'est qu'après avoir hésité qu'elle reformule plus expressément son propos en utilisant les mots d'adultes.
Toujours aussi peu à l'aise devant le vocabulaire sexuel explicite, elle réitère sa crainte -- que lui a inculquée sa grand-mère -- que des hommes la forcent à \guil{touche[r] leur pipi}~(\textit{B}:~56).
Cette hésitation est d'autant plus symbolique que Sissi connaît les véritables mots pour désigner les organes génitaux masculins: ainsi, lorsqu'on lui demande de dessiner un mouton, elle ne dessine pas autre chose, surtout pas \guil{un pilier, un pénis ou un Vinier}~(\textit{B}:~58); elle remarque également que sa maîtresse de classe mal à l'aise \guil{se frott[e] le nez, les sourcils, le soutien-gorge}~(\textit{B}:~58).
Les employant de la façon la plus banale dans des énumérations, Sissi expose sa connaissance des mots \guil{pénis} et \guil{soutien-gorge}; son hésitation ne peut donc être que fort signifiante, en ce sens que l'enfant \textit{connaît} mais n'\textit{ose} pas prononcer les véritables mots pour désigner les choses du sexe.
\par
Si cette hésitation devant le vocabulaire sexuel est révélatrice d'enfance -- malgré une connaissance des choses qui démontre une maturité plus grande --, c'est lorsqu'elle exprime combien son père biologique suscite en elle la crainte que Sissi se révèle linguistiquement comme une enfant-adulte: \guil{Papa Méchant, Papa Méchant, Papa Méchant, et qu'il me fait plus peur que le Bonhomme Sept-Heures, plus peur que King Kong, plus peur que le streptocoque de type A.}~(\textit{B}:~123)
Jonglant aisément entre la crainte enfantine suscitée par un monstre imaginaire et celle, strictement adulte, qu'induit la mention d'une maladie au nom rare, la narratrice adjoint indistinctement le jeune âge à la maturité dans un même syntagme.

\subsubsection{L'entrelacement métaphorique des âges}
Outil langagier de prédilection de Sissi, la métaphore, comparaison informelle construite sur une analogie entre deux termes, est par ailleurs la figure de style par excellence de l'enfant-adulte puisqu'elle permet de bâtir des ponts linguistiques entre l'enfance et l'âge adulte.
C'est donc par une mise en langage imagée que se rejoindront la maturité et la jeunesse dont peut faire preuve l'enfant-femme.
Comme de raison, la lucidité démontrée par Sissi concernera surtout sa situation familiale, dont elle sait qu'elle est toxique: \guil{Chez moi, la vie n'est pas un long fleuve tranquille, mais un lac artificiel rempli de BPC. Stagnant, le lac.}~(\textit{B}:~119)
Désireuse d'échapper à cet univers, elle élabore une métaphore filée pour décrire sa volonté d'avoir une famille normale et son incapacité à y arriver: \guil{Je construis des maisons avec des cubes Lego et j'y loge mes familles de bonshommes Fisher Price et ils parlent et ils parlent! $\left[ \dots \right]$ L'autre jour, j'ai voulu me construire une maison pour moi avec mes cubes Lego. Mais je n'ai pu me rendre qu'aux chevilles. Je n'avais pas assez de cubes. Je n'ai pas été capable de me construire une maison.}~(\textit{B}:~93)
Cet échec, beaucoup plus signifiant au sens symbolique qu'au sens strict, a pour sens profond le constat métaphorique par l'enfant de l'inadéquation de son environnement familial.
Voulant illustrer le peu de respect qu'elle a pour l'inertie de sa mère, Sissi la compare aux \guil{victimes} provenant de champs les plus divers:
\begin{quote}
  \begin{singlespace}
    \small
    Ma mère est une victime. Si elle était un animal, elle serait l'agneau qui se fait bouffer par un lion. Si elle était une Russe, elle resterait à Tchernobyl à deux pas de la centrale nucléaire. Dans un film d'horreur, elle serait la première à se faire couper la tête et les quatre membres, et à se faire sortir les intestins par l'énorme monstre vert et gluant. Elle est comme ça, ma mère, elle a autant de personnalité qu'une débarbouillette.~(\textit{B}:~120)
    \normalsize
  \end{singlespace}
\end{quote}
Devant ce constat hautement déceptif, Sissi adopte une attitude de rejet envers sa mère, dont la chambre est nommée \guil{chambre des lamentations}~(\textit{B}:~126) pour illustrer qu'elle pleure tout le temps, figure anti-maternelle à laquelle elle ne veut d'ailleurs pas ressembler: \guil{Niet. No. Non.}~(\textit{B}:~31)
Comme si l'usage de langues étrangères renforçait le refus.
Son rejet est d'ailleurs exprimé lui aussi de façon imagée: \guil{J'ai regardé ma mère comme ça, jusqu'à tant qu'on tourne dans le couloir et que je ne la voie plus. J'aurais souhaité que ce soit le tournant de ma vie. Que ce soit pour de bon.}~(\textit{B}:~68)
Manifestant le désir que ce coin de corridor représente davantage qu'un simple mur, Sissi transpose à l'échelle de sa vie l'idée que sa mère s'éloigne jusqu'à être complètement hors de sa vue.
\par
Elle est parfois ouvertement hostile à sa mère, osant même lui dire: \guil{Je t'haïs, Môman! Je t'haïs, Môman! Si tu savais combien je t'haïs! J'ai mal à mon nombril tellement que je t'haïs! J'ai mal à ma naissance tellement que je t'haïs! J'aurais dû me pendre avec le cordon ombilical dès que je suis sortie de ton ventre de folle! J'aurais dû!}~(\textit{B}:~67)
En plus de la répétition des mots \guil{je t'haïs}, la métaphore \guil{J'ai mal à mon nombril}, quoique immédiatement reformulée par la narratrice elle-même, s'adjoint efficacement à son regret de ne pas s'être \guil{pend[ue] avec le cordon ombilical} dès sa naissance pour exprimer toute la douleur qu'implique la relation de l'enfant à sa mère.
Ce lien de naissance est linguistiquement illustré par le nombril, trace du cordon qui reliait la mère au nourrisson.
Naïdza Leduc estime que ce \guil{véritable ressentiment} éprouvé par Sissi est une répercussion de la maladie de sa mère, de cette \guil{psychose en héritage\footcite[47-48]{Leduc2010}}.
C'est au nom de ce ressentiment que l'enfant manifeste sa colère envers sa mère et qu'elle lui lance des \guil{insectes}, que l'on peut aisément imaginer être des paroles blessantes: \guil{Ma mère, c'est ma cible. Je m'en sers pour sortir mes bibites. Je lance mes coquerelles sur elle, mes maringouins sur elle, mes araignées sur elle. Ma mère, c'est mon Insectarium de Montréal.}~(\textit{B}:~30)
Cette colère, expression d'une ranc\oe{}ur profonde, peut également être analysée en tant que manifestation langagière de l'autonomisation de la jeune fille.
Le langage est ici instrumentalisé sous le motif de la fuite et c'est par ses mots blessants que l'enfant s'éloigne irrémédiablement de sa mère.
\par
Enfant autonome mais aussi très adaptative, Sissi ne fait pas grand cas des événements dramatiques qui se déroulent autour d'elle.
C'est donc par l'humour qu'elle raconte la tentative de suicide de sa mère, représentant l'événement comme une \guil{imitation de Marilyn Monroe}~(\textit{B}:~31), fort réussie d'ailleurs:
\begin{quote}
  \begin{singlespace}
    \small
    J'ai appelé à l'urgence de l'hôpital Notre-Dame. $\left[ \dots \right]$ \textit{Bonjour, suis-je bien à l'hôpital Notre-Dame? Oui, bon. J'aurais besoin d'une ambulance, c'est que ma mère vient tout juste de faire une imitation de Marilyn Monroe. Et c'était très réussi. On a tous applaudi. Mais là, elle ne veut plus débarquer de la scène. Alors vite, envoyez-nous une ambulance ou une équipe de tournage $\left[ \dots \right]$}.~(\textit{B}:~31)
    \normalsize
  \end{singlespace}
\end{quote}
Par cette métaphore filée à propos du cinéma hollywoodien, Sissi dédramatise la situation et laisse croire qu'elle a été peu affectée par les événements.
L'effet est comique, dégonflant dans sa quasi-totalité la tension dramatique inhérente aux circonstances.
Par le ludisme, l'enfant parvient à nous montrer sa maîtrise d'elle-même, face à laquelle on n'a d'autre choix que de déduire un certain niveau de maturité.

\subsubsection{Le langage, un jeu de pauvres}
Il y a également dans l'écriture de Marie-Sissi Labrèche un ludisme que l'on pourrait qualifier d'\guil{immotivé}, en ce qu'il ne nous semble pas sémantiquement justifié.
C'est le jeu de langage pour lui-même, sorte de corollaire linguistique du slogan parnassien de \guil{l'Art pour l'art}.
La narratrice joue donc avec les mots -- dont elle détourne l'usage ou propose des associations farfelues -- afin de s'amuser, ainsi que les autres.
Cette forme d'humour, qui met en scène le langage, nous semble relever au point de vue formel de l'enfance la plus pure puisqu'il s'agit d'un \guil{jeu} auquel Marie-Sissi Labrèche accorde une véritable valeur ludique:
\begin{quote}
  \begin{singlespace}
    \small
    $\left[\textit{Extrait de la neuvième lettre adressée par l'auteure à sa mère:}\right]$
    Je pense aussi à nos jeux, à toutes les deux, quand j'avais quatre ans, puisqu'on était pauvres et que tu ne pouvais pas m'acheter tous les jouets de la terre, on s'amusait avec les mots, \guil{Finis toutes tes phrases en A, en E, en I.} Quand je réussissais, tu me félicitais à m'en rendre toute molle. À certains moments, je me dis que ma vocation d'écrivaine a pris racine dans ces jeux de pauvres\footcite[186]{Labreche2008a}.
    \normalsize
  \end{singlespace}
\end{quote}
Elle lance avec humour avoir \guil{le cerveau en compote de pommes}~(\textit{B}:~34) et décrit ainsi son amie Céline, celle qui a de la \guil{fidélité de caniche qui coule dans [l]es veines}~(\textit{B}:~37): \guil{Céline, mon amie Raisin Bran. Mon gage de régularité dans tout ce chaos.}~(\textit{B}:~36)
Revenant à la thématique des céréales matinales et de leur effet sur le transit intestinal, elle affirme son exaspération envers sa grand-mère avec le jeu sur le double sens de \guil{faire chier}: \guil{Elle, elle me fait assez chier avec ses Raisin Bran qu'elle me force à avaler tous les matins.}~(\textit{B}:~33)
La mémoire lui faisant défaut, elle rebaptise à de multiples reprises le centre sportif où sa mère va voir de la lutte:
\begin{quote}
  \begin{singlespace}
    \small
    Je me fais garder. Ma mère est partie voir un combat de lutte au centre Guy-Robillard ou Marcel-Robillard ou Jean-Marc-Robillard. Je ne sais pas trop. Je ne retiens pas ces choses-là. $\left[ \dots \right]$ Je retiens aussi que je ne dois pas trop faire chier ma grand-mère quand ma mère va voir des combats de lutte au centre Pierre, Jean, Jacques-Robillard $\left[ \dots \right]$.~(\textit{B}:~115-116)
    \normalsize
  \end{singlespace}
\end{quote}
\par
Bien plus que la mémoire qui lui fait défaut, auquel cas elle emploierait un nom erroné (mais un seul nom), Sissi démontre d'abord, par la première énumération, le peu d'importance qu'elle accorde au véritable nom de ce \guil{centre Roland-Robitaille}~(\textit{B}:~133).
C'est cependant à la seconde énumération que l'enfant fait preuve d'une véritable bouffonnerie: en le nommant le \guil{centre Pierre, Jean, Jacques-Robillard}, elle intègre dans son discours un concept d'onomastique générique.
Pierre, Jean, Jacques: ce sont les autres, à la façon de M. Dupont chez les français ou de Mr Smith pour les anglo-saxons.
Ainsi, en plus d'être linguistiquement générateur de banalité, son jeu a un aspect purement ludique, en ce sens qu'il \textit{amuse}.
\par
De façon tout aussi drôle, Sissi résume l'évolution de ses aptitudes en dessin en référant aux grands courants artistiques, dont elle détourne les noms en réduisant les mots à leur sens le plus commun:
\begin{quote}
  \begin{singlespace}
    \small
    Après ma période cubiste, où je dessinais les cubes stupides qu'on doit introduire dans un moule stupide, je suis devenue une impressionniste. À trois ans, j'étais une impressionniste qui impressionnait tout le monde avec ses dessins. $\left[ \dots \right]$ Aujourd'hui, je suis devenue une surréaliste. J'ai dépassé les bornes de l'audace artistique pour les deuxième année, et là, ils ont peur.~(\textit{B}:~57-58)
    \normalsize
  \end{singlespace}
\end{quote}
En réduisant le cubisme aux cubes ou l'impressionnisme au fait d'être impressionnante, Sissi joue avec les mots et se moque du code qui les sous-tend.
Ce ludisme non-utilitariste -- en ce sens qu'il n'a pas de vocation autre que l'amusement -- a tout de même pour effet d'affirmer l'appartenance de Sissi à la figure de l'enfant-femme puisque son maniement du langage implique forcément une maîtrise du code; elle ne pourrait détourner ainsi les noms des grands courants en peinture si elle n'en connaissait pas la définition ni la valeur symbolique, et c'est ce savoir adulte qui lui permet d'accomplir son jeu enfantin.
\par
D'une façon qui nous semble cette fois tout aussi ludique mais également délibérée, la jeune Sissi invente la perle qu'est le néologisme suivant:
\begin{quote}
  \begin{singlespace}
    \small
    Ma grand-mère est comme Dieu. Elle est partout à la fois. Elle est omniprésente. En fait, je devrais dire femniprésente, parce que tout ce qui est homme, elle le réduit en bouillie.~(\textit{B}:~32)
    \normalsize
  \end{singlespace}
\end{quote}
Le mot \guil{femniprésente} nouvellement inventé est porteur de sens puisqu'il résume à lui seul tout le dégoût des hommes qu'a la grand-mère.
La création par Sissi d'un néologisme aussi significatif révèle d'une part une bonne compréhension du monde, et d'autre part une tout aussi bonne maîtrise du langage, deux attributs valorisés chez l'enfant-femme.
Ainsi, même dans le jeu de langage, Sissi ne peut échapper à sa \guil{nature} d'enfant-femme et aux manifestations langagières de celle-ci.

\subsubsection{Sissi Labrèche, elle-même jeu de langage}
Si sa propension à faire des jeux de mots place indiscutablement la narratrice de \textit{Borderline} comme importante génératrice de ludisme langagier, nous estimons que c'est par son nom qu'elle se positionne définitivement comme reine du langage ludique.
D'abord par son prénom, lequel est en lui-même hautement connoté comme le souligne une travailleuse sociale: \guil{Elle me demande c'est quoi mon nom. Sissi. Elle trouve ça beau, elle dit que c'est comme l'impératrice, que c'est un nom de princesse. Tout le monde le dit.}~(\textit{B}:~96)
À force d'en entendre parler, l'enfant en vient à intégrer cette sémantisation de son propre prénom.
Habituée à recevoir des compliments de sa maîtresse d'école pour ses dessins, elle joue sur la connotation royale de son prénom: \guil{J'attendais qu'elle me lance ces phrases qui grossissent ma couronne invisible de petite princesse Sissi $\left[ \dots \right]$.}~(\textit{B}:~59)
Son prénom est également source de distraction pour ses camarades de classe en raison de certaines assonances qu'ils y trouvent: \guil{On rit de moi à cause de mon prénom: Sissi. On me crie: \textit{Heille Oui! Oui! Heille Pipi!} ou encore \textit{Heille Suce!} ça, c'est les cinquième année.}~(\textit{B}:~63)
\par
C'est cependant par son nom de famille que le poids sémantique de l'onomastique du personnage est décuplé:
\begin{quote}
  \begin{singlespace}
    \small
    On rit aussi de mon nom de famille dysfonctionnelle; mon nom de famille laissé, oublié par mon grand-père Labrèche mort d'un cancer du poumon à l'hôpital Notre-Dame, un dimanche après-midi ensoleillé. On m'appelle la Broche, la Brosse, la Poche. Mais ils ne comprennent pas. Ils ne rient pas pour la bonne affaire. En fait, mon nom, c'est le trou, c'est la brèche, c'est la fente de mon petit corps.~(\textit{B}:~63)
    \normalsize
  \end{singlespace}
\end{quote}
Outre les jeux de mots basés sur la sonorité ou sur le sens du mot \guil{brèche} -- les premiers étant amusants et les seconds étant davantage signifiants pour la narratrice devenue adulte que pour celle, enfant, qui nous intéresse --, c'est par le syntagme \guil{mon nom de famille dysfonctionnelle} que se sémantise toute l'onomastique du personnage de Sissi.
Le jeu de mot, sorte d'\guil{expression valise}, est d'autant plus lourd de sens que la grammaire nie la possibilité qu'il ne s'agisse pas d'une manifestation ludique: ainsi, l'accord féminin du mot \guil{dysfonctionnel} implique qu'il se rattache obligatoirement à la famille plutôt qu'au nom, scindant du même coup le syntagme entre les idées de \guil{nom de famille} et de \guil{famille dysfonctionnelle}.
La narratrice exprime de la sorte tout son ressentiment face à sa famille, négativité qu'elle rejette sur son patronyme.
De son prénom jusqu'à son nom de famille, Sissi Labrèche \textit{est} un jeu de langage!

\subsection{L'insulte et la grossièreté}
Lorsqu'il n'est pas ludique, le langage employé par Sissi concorde généralement avec le milieu dans lequel elle vit, en ce sens qu'il est familier, caractéristique du quartier populaire d'où elle vient.
Ainsi, ses jeux de mots sont parfois entrecoupés de jurons et d'insultes, lesquels sont systématiquement proférés en réaction face à la frustration de ne pas être écoutée ni comprise.

\subsubsection{La familiarité et les jurons}
Bien que Sissi sache \guil{jouer} avec les mots et qu'elle en fasse parfois un usage relativement savant, elle maîtrise également le langage populaire.
Elle sait notamment qu'elle \guil{ne doi[t] pas trop faire chier [s]a grand-mère}~(\textit{B}:~116).
Cet emploi de l'expression \guil{faire chier} est par ailleurs repris, qu'il concerne la grand-mère~(\textit{B}:~33), Cendrillon \guil{qui se fait chier par sa famille}~(\textit{B}:~129) et même lorsque l'enfant proclame que \guil{tous ceux qui [l]'ont fait chier}~(\textit{B}:~129) paieront.
Lorsque sa grand-mère dit des \guil{niaiseries}, la jeune fille est catégorique et glisse un \guil{Je m'en fous}~(\textit{B}:~88) bien senti.
Déclinant l'expression populaire sous diverses variations, elle s'en \guil{fout} ou s'en fiche: \guil{Bah, je m'en fiche de ce qu'elle me dit, ma grand-mère $\left[ \dots \right]$.}~(\textit{B}:~118)
Si elle emploie également des expressions populaires dépourvues de vulgarité -- qu'elle dise que \guil{ça ne devait pas être jojo}~(\textit{B}:~94) ou que quelque chose \guil{fait dur en titi!}~(\textit{B}:~132) --, il demeure que sa maîtrise du langage vernaculaire passe essentiellement par le langage \guil{malpoli}; ainsi déclarera-t-elle se \guil{sen[tir] moins seule dans [s]a merde}~(\textit{B}:~40) ou lancera-t-elle sans retenue l'exclamation \guil{Une grosse journée de cul!}~(\textit{B}:~36)
\par
Si Zazie ne semblait pas inquiéter outre-mesure \guil{les rangs de l'autorité\footcite[89]{Maurin2007}}, il en va de même pour Sissi, dont le langage est tout aussi inadéquat que l'était l'attitude verbale de la première.
Elle manie le juron comme d'autres enfants récitent des comptines et aucun adulte ne semble en faire de cas: \guil{Tout le long, je sacrais: \textit{Câlice-d'ostie-de-maudit-tabarnak-de-crisse-de-sacramant-de-calvaire!} $\left[ \dots \right]$ et je resacrais: \textit{Câlice-de-saint-ciboire-de-saint-calvaire-de-maudite-marde-de-crisse!}}~(\textit{B}:~66)
On ne l'écoute pas plus lorsqu'elle jure que lorsqu'elle s'exprime normalement: \guil{Elle me fait mal et je me plains. \textit{Mémé, j'ai mal!} Mais elle n'entend rien. \textit{Mémé, tu me fais bobo, câlice!} Même si j'ai sacré, elle ne m'entend toujours pas.}~(\textit{B}:~87)
Sissi fait ainsi un usage répété du juron, élément langagier négativement connoté, socialement inconvenant même pour l'adulte, d'autant plus malséant lorsque prononcé par un enfant.
\par
Si sa maîtrise d'un tel vocabulaire exprime d'une part son acclimatation et même son appartenance au monde adulte, elle est également symptomatique du peu d'autorité qu'ont sur elle les adultes, ces derniers ne s'inquiétant aucunement de la voir employer ces mots \guil{interdits}, trop occupés qu'ils sont à vaquer à leurs propres occupations: \guil{C'est à croire qu'elles pleurent pour passer le temps, câlice!}~(\textit{B}:~62)
Ainsi, le juron est pour Sissi bien plus qu'un simple défouloir; elle l'emploie afin d'attirer l'attention, voire d'\guil{alerter} (sans grand succès) l'autorité.
Il lui sert également à établir son positionnement en regard de l'adulte, cet Autre face auquel elle n'a qu'une volonté, soit celle d'\guil{être contre}.
Par exemple, à la travailleuse sociale qui remarque qu'elle est pâle et qui lui demande ce qu'elle a, Sissi réplique férocement: \guil{C'est parce que j'ai mangé du papier de toilette quand j'avais trois ans, câlice!}~(\textit{B}:~98)
Loin de se laisser amadouer, la jeune fille appose un juron à sa phrase, laquelle était déjà très \guil{mordante} en elle-même -- animal qui refuse d'être docile, elle grogne et montre les crocs.
\par
Le juron sert aussi de marque exclamative, que ça soit un \guil{Crisse!}~(\textit{B}:~67, 68) placé avant la phrase ou un \guil{câlice}~(\textit{B}:~128) qui la termine.
Sissi l'utilise pour manifester son mécontentement à l'effet qu'\guil{on marche trop vite, calvaire!}~(\textit{B}:~92) ou pour dire à sa grand-mère combien elle est méchante: \guil{Câlice! T'as pas le droit de dire ça! C'est toi qui es méchante! Méchante! Méchante! Aussi méchante que la belle-mère de Cendrillon! Aussi méchante que la bonne femme Olson dans \textit{La Petite Maison dans la prairie}!}~(\textit{B}:~127)
Si le juron appartient évidemment au vocabulaire des adultes, il nous semble que l'usage qu'en fait Sissi soit -- paradoxalement -- enfantin: en tant que marqueur de mécontentement, il démontre bien toute l'impuissance de l'enfant face à l'adulte.
Incompris, l'enfant exprime sa contrariété par la langue:
\begin{quote}
  \begin{singlespace}
    \small
    Câlice, elle devrait le savoir. J'ai de la misère à rester concentrée plus d'une minute sur la même affaire, elle devrait se le rappeler. Maudit! Là, couchée dans le lit de ma grand-mère, je suis super fâchée, je suis hyper en câlice! Je suis bleue, je suis bleu blanc rouge de colère. J'ai mon drapeau de la France sorti, alors c'est à mon tour de bougonner. D'ailleurs, si elle se repointe, la Mémé, je vais lui vomir mon camembert sur la tête.~(\textit{B}:~122)
    \normalsize
  \end{singlespace}
\end{quote}
Empêchée de regarder la télé par sa grand-mère qui \guil{n'arrête pas de chialer}~(\textit{B}:~121), la petite Sissi de cinq ans explose de colère et intime à sa grand-mère l'ordre de se taire:
\begin{quote}
  \begin{singlespace}
    \small
    - Ferme ta gueule! \\
    - Quoi?  \\
    - J'ai dit ferme ta gueule, vieille câlice! \\
    - Quoi? est-ce que j'ai bien compris? \\
    - J'ai dit ferme ta gueule, vieille câlice!~(\textit{B}:~123)
    \normalsize
  \end{singlespace}
\end{quote}
Par un amalgame entre l'insulte et la vulgarité, Sissi s'impose langagièrement afin de revendiquer son droit de parole et d'être écoutée; employant le langage comme une arme, elle exprime sa frustration et se place comme égale à l'adulte.
Dans la guerre communicationnelle qu'elle livre à sa grand-mère -- guerre qui se joue sur le front de la fonction phatique --, Sissi ne peut s'imposer par ses armes enfantines et doit user du juron, arme d'adulte.

\subsubsection{Des insultes à volonté}
Envers sa grand-mère qui ne l'écoute pas, Sissi n'est pas toujours tendre: \guil{Elle dit tout le temps plein de niaiseries, la vieille crisse!}~(\textit{B}:~56)
Elle la traite de \guil{vieille bitche}~(\textit{B}:~91) ou de \guil{vieille peau}~(\textit{B}:~89), la désigne comme étant \guil{la vieille maudite}~(\textit{B}:~121) ou simplement \guil{la maudite}~(\textit{B}:~89), la traite de \guil{vieille pas fine}~(\textit{B}:~123) ou de \guil{vieille débile}~(\textit{B}:~87); elle exprime d'ailleurs à nouveau combien elle la \guil{trouve tellement débile}~(\textit{B}:~96).
Autant que le juron, le propos insultant est souvent un moyen d'exprimer sa frustration de ne pas être écoutée: \guil{Elle ne me répond toujours pas, la vieille pas fine. Aussi bien parler à un plat de pâté chinois, câlice!}~(\textit{B}:~92)
Pour Sissi, l'insulte sert de moyen de se faire entendre ou, à défaut d'être écoutée, permet d'évacuer le mécontentement qui découle de cette impossibilité de communiquer.
\par
L'insulte est également dirigée vers les camardes de classe de Sissi.
Si le mécanisme d'expression de la frustration est similaire à celui qu'elle déploie envers sa grand-mère, la cause de son mécontentement est renouvelée: plutôt que de vivre une situation de communication ratée, Sissi subit l'incompréhension de la part des autres écoliers quant à sa réalité familiale peu commune.
Sa réaction est toutefois la même, qu'il s'agisse de mauvaise communication ou d'incompréhension: face à la frustration, elle attaque par l'insulte la plus crue.
Ainsi, elle traite les enfants traumatisés par son dessin des yeux de sa mère de \guil{peureux}~(\textit{B}:~57), exposant leur soi-disant \guil{faiblesse} devant la folie de sa mère, qui est pour elle une réalité du quotidien: \guil{Petites natures, va! Gang de brouteux de luzerne, va!}~(\textit{B}:~56)
Devant leur incompréhension de sa réalité à elle, Sissi réagit par un langage brutal: \guil{Je voudrais leur crier: \textit{Fermez-la, gang de chieux! Gang de petits crisses pas bons en dessin. $\left[ \dots \right]$ Allez donc tous chier! Allez donc tous vous faire enculer par les mouches!}}~(\textit{B}:~70)
À ses yeux, autant les adultes avec leurs précautions que les autres enfants sont une \guil{gang de chieux}~(\textit{B}:~61) qu'elle rejette.
Même sa mère et son beau-père n'y échappent pas: \guil{C'est moi qui serai censée tenir les anneaux de mariage, c'est ce qu'ils m'auront dit, les morons, sauf que lorsque ce sera le temps de tenir les maudits anneaux de mariage, ils me les enlèveront. Des morons, que je dis.}~(\textit{B}:~116)
Tous y passent: de sa \guil{mère qui pleurait comme une vraie nulle}~(\textit{B}:~64) aux enfants de l'aile psychiatrique de l'hôpital.
\guil{Bande de petits cons}, dira-t-elle~(\textit{B}:~99).
Sissi n'oublie personne et chacun reçoit son épithète insultante.

\section{À la guerre comme à la guerre: l'arme langagière}
Point de friction par excellence entre l'enfance et l'âge adulte, le langage démontre, autant chez Zazie que chez Sissi, toutes les difficultés existentielles de l'enfant-femme.
Chacune à sa façon, les deux jeunes filles manient le langage comme une arme: d'une part afin de se définir et d'autre part dans une tentative de combattre les adultes.
Toute cette mécanique fait indiscutablement du langage un vecteur majeur d'ambiguïté quant à l'âge du personnage.
C'est toutefois sur le plan du combat contre les adultes que Zazie et Sissi se distinguent langagièrement l'une de l'autre: d'abord parce que leurs combats ne se livrent pas sur le même front, l'une menant le combat de toute une génération et l'autre menant une bataille individuelle; ensuite parce que leurs quêtes n'ont pas la même issue, l'une médusant tous les adultes dans leur conception de la jeunesse, l'autre peinant simplement à être entendue.

\subsection{Le langage comme ultime consécration de la valeur contre-mythique de Zazie}
Chez Zazie, le langage s'emploie comme une arme de précision afin de dégonfler toute la mythologie entourant la jeunesse: il se veut aiguille affutée, suffisamment précise pour atteindre et crever le mythe de la \textit{doxa}.
Les attaques de l'enfant se portent surtout contre les Grandes Idées avec une majuscule, celles des Grandes Personnes (elles aussi avec la majuscule).
\guil{Attei[gnant] la pensée ou l'absence de pensée\footcite[38]{Hoja-Lacki1963}}, ce sont des attaques de front, sans complexes et sournoises.
Comme l'a démontré Barthes, Zazie est incisive: elle maîtrise bien les concepts linguistiques et son découpage se fait au scalpel, attaquant le métalangage en portant le langage-objet comme étendard.
Le succès de son entreprise est total et l'effet est grand: sous les traits de Zazie, l'enfant-femme décontenance les adultes par son usage de la langue et acquiert, selon Roland Barthes, le statut de contre-mythe.
Elle ébranle et fait vaciller tout l'édifice des Idées, en se posant comme plus philosophe que cette pseudo-philosophie émanant des adultes.
Ce succès -- dans son usage de la langue à des fins de guerre contre les adultes -- consacre par le fait même son ascendant sur toute la typologie de l'enfant-femme: précédemment désignée comme \guil{enfant-adulte absolu}, Zazie renouvelle cette consécration lorsqu'il s'agit du langage.

\subsection{Le langage à la fois jouet et arme chez Sissi}
Chez Sissi, la langue se veut à la fois jouet et arme: jouet par ce ludisme omniprésent et arme par cette multiplication des tentatives d'atteindre l'adulte par l'usage du langage.
Si le premier objectif est définitivement atteint -- Sissi fait rire et \textit{se} fait rire --, le second usage de la langue obtient pour sa part un succès mitigé puisque cet Autre, qu'elle tente d'atteindre par ses insultes et son vocabulaire vulgaire, est hors de portée; il ne l'écoute pas et n'est donc pas dérangé par elle et par ses paroles.
Pour reprendre la métaphore guerrière, c'est comme un coup d'épée dans l'eau.
\par
Ainsi, nous revenons au constat déceptif émis par Vurm à propos des enfants ducharmiens:
\begin{quote}
  \begin{singlespace}
    \small
    Mais la déception et la chute arrivent surtout avec l'impuissance des mots. Car l'éloquence semble rompue très souvent lorsque les personnages s'aperçoivent que le langage ne suffit plus dans leur lutte contre le monde. À ce moment même, les enfants perdent leur double, symboliquement et littéralement, malgré les injures, tout juste créées pour être lancées aux adultes, malgré leurs créations lexicales, qui sont leur refuge, ils sont voués à l'échec, car les mots ne peuvent plus rien pour éviter leur isolement dans ce monde fou $\left[ \dots \right]$\footcite[101]{Vurm2014}.
    \normalsize
  \end{singlespace}
\end{quote}
Loin de remédier aux tensions entre l'enfance et le monde des adultes, le langage, en ses composantes communicationnelles, constitue pour Sissi un autre vecteur de souffrance.
Isolée malgré elle, incapable d'instaurer un dialogue convenable -- et de se faire entendre et comprendre --, elle trouve remède dans cette exacerbation de la fonction poétique que représente la \textit{fonction ludique}.
\par
Faute de mieux, elle s'amuse avec les mots.
C'est sur ce point que son rapport au langage rejoint à notre avis le caractère du personnage, nous permettant d'émettre une hypothèse audacieuse: le ludisme langagier ne pourrait-il pas être une application linguistique du concept de résilience?
Ou le langage, dans sa dynamique poétique, ne serait-il pas, pour le type d'enfant-femme qu'est Sissi, un vecteur d'apaisement?
Nous estimons que oui en nous permettant d'affirmer que la capacité d'adaptation et de résilience chez Sissi est définitivement tributaire de son rapport ludique au langage.

\chapter*{Conclusion}
\addcontentsline{toc}{chapter}{Conclusion}

\begin{flushright}
                                    \begin{singlespace}
                                    \epigraph{
                                    Regardez tous les délais comme des avantages: c’est gagner beaucoup que d’avancer vers le terme sans rien perdre; laissez mûrir l’enfance dans les enfants. Enfin, quelque leçon leur devient-elle nécessaire? gardez-vous de la donner aujourd’hui, si vous pouvez différer jusqu’à demain sans danger.}
                                    \par
                                    Jean-Jacques Rousseau, \textit{Émile, ou De l'éducation}\footcite[441]{Rousseau1852}
                                    \end{singlespace}
\end{flushright}

Enfant-femme ou enfant-adulte, mais fondamentalement enfant avant tout.
Nous avons examiné comment s'entrecroisaient l'enfance et l'âge adulte dans \textit{Zazie dans le métro} de Raymond Queneau et \textit{Borderline} de Marie-Sissi Labrèche, mais avons-nous véritablement étudié l'âge adulte?
N'avons-nous pas étudié que l'enfance, en réalité?
Il nous semble que ces apparitions de traits adultes touchent encore au sujet de l'enfance en ce qu'ils la froissent, l'abîment et la marquent à jamais.
Autonomisée par les événements, l'enfant-femme s'acclimate en observant et en analysant les adultes, pour ensuite adopter un comportement qui les imite ou qui les offusque.
Les réalités sont donc les mêmes pour Zazie et pour Sissi: elles sont toutes deux plongées dans un monde peu fiable, entourées d'adultes sur lesquels elles ne peuvent pas compter.
Toutefois, les enjeux diffèrent: alors que Zazie livre un combat ouvert contre les adultes, Sissi tente avant tout de survivre et de se protéger du monde adulte.
Ces différences sont visibles autant sur le plan du caractère que celui du langage: l'une est brutale tandis que l'autre est résiliente; l'une manie l'arme langagière dans un \guil{combat de brousse}, l'autre n'usant de cette arme-parole que pour se protéger.
\par
Avec notre typologie, nous avons tenté de présenter un type d'enfant plutôt moderne, que nous estimons être le résultat d'une société et d'une époque données.
Selon Denise Lemieux, qui a étudié l'enfant dans la littérature québécoise, la littérature sur l'enfance est \guil{étroitement lié[e] aux transformations du statut de l'enfant dans les sociétés en voie de modernisation\footcite[11]{Lemieux1984}.}
Ainsi, le personnage enfantin est tributaire de \guil{l'avènement de modes de vie où l'enfant est de plus en plus retiré de la société des adultes\footcite[9]{Lemieux1984}}, alors que l'on ne s'efforce plus autant de l'intégrer à la vie quotidienne des adultes\footcite[9]{Lemieux1984}.
C'est un constat adéquat en ce qui concerne Zazie et Sissi, lesquelles vivent certes dans le même monde que les adultes, mais ne vivent pas \textit{avec} eux.
\par
Conciliation travail-famille ardue, parents et enseignants débordés, multiplication des problèmes de santé mentale infantiles et ressources insuffisantes: si la négligence envers les enfants est loin d'être un fait nouveau, elle est désormais plus insidieuse et souvent involontaire, en ne produisant pas des petits martyrs aux blessures visibles mais plutôt des enfants carencés dont les stigmates sont toutefois tout aussi graves.
D'ailleurs, comme l'avait annoncé Brigitte Seyfrid-Bommertz il y a plus de vingt ans, l'enfance en littérature ne va pas en se simplifiant:
\begin{quote}
  \begin{singlespace}
    \small
    De façon générale, il semble que plus on s'avance vers la période contemporaine, plus le profil passionnel de l'enfant se fait complexe, déroutant, voire inquiétant. L'enfant acquiert de multiples visages, se mue en un être polymorphe plus difficile à saisir. [...] L'enfant est aussi saisi à travers ses zones d'ombre, son côté satanique, ou encore il se fait étrange, irréel, fantastique, être incompréhensible sur lequel on n'a plus de prise \footcite[20]{Seyfrid-Bommertz1999}.
    \normalsize
  \end{singlespace}
\end{quote}
Plusieurs fictions de l'extrême contemporain se rattachent aisément à cette complexification de l'enfant en littérature\footnote{Nous pensons notamment aux romans \textit{Et au pire, on se mariera} (2011), \textit{Chercher Sam} (2014) et \textit{Autour d'elle} (2016) de Sophie Bienvenu, \textit{La déesse des mouches à feu} (2014) de Geneviève Pettersen, \textit{À l'abri des hommes et des choses} (2016) de Stéphanie Boulay, ainsi qu'à la trilogie \textit{La bête à sa mère} (2015), \textit{La bête et sa cage} (2016) et \textit{Abattre la bête} (2017) de David Goudreault.}, alors que les jeunes personnages semblent de moins en moins \textit{aimés}\footnote{En ce sens que les parents, bien qu'ils comblent leurs besoins physiques, ne comblent pas leurs besoins affectifs.} et qu'ils deviennent de plus en plus violents, de plus en plus troublés.
Ils sont perturbés, mais également perturbants et percutants, et leur rejet de ce monde adulte peu accueillant n'est pas sans nous rappeler les \textit{prisons nostalgiques}, que l'on rattache aux enfances anglaises de \textit{Peter Pan}\footcite{Barrie1987} et d'\textit{Alice au pays des merveilles}\footcite{Carroll2012} et qui correspondent à un détachement et à un retrait volontaires du monde adulte.
Ainsi, la nostalgie s'installe du fait de l'échec ou du refus volontaire d'arriver à un état d'harmonie quant aux réalités culturelles de l'époque\footcite[241]{Coveney1967}.
Tandis qu'Alice s'acclimate quelque peu à la folie inhérente au Pays des Merveilles et en ressort \textit{grandie} ou \textit{vieillie} -- d'où l'idée que l'enfant-femme en constitue une actualisation --, Peter Pan refuse catégoriquement de vieillir et s'emmure dans le monde de l'enfance.
Si la prophétie de Brigitte Seyfrid-Bommertz s'avère vraie, et que cette complexification de l'enfant passe également par son éloignement et son rejet de l'adulte, l'enfant contemporain ne peut que se rapprocher de la figure de Peter Pan.
Ce déplacement de la problématique de l'enfance s'accompagne aussi d'un changement de paradigme: alors qu'Alice est \textit{tombée} dans le terrier et qu'elle se retrouve bien involontairement au Pays des Merveilles, Peter Pan se rend, volontairement et consciemment, au Pays Imaginaire\footnote{Aussi appelé Pays de Nulle part ou Pays du Jamais, d'après son nom anglais \textit{Neverland}. Il s'agit d'un lieu qui n'est pas soumis au temps et où le jeu est la seule activité possible.}.
D'Alice, Zazie et Sissi qui subissent le monde adulte et tentent d'y trouver leurs repères, nous passons ainsi à Peter Pan et aux enfants contemporains qui rejettent le monde adulte afin de créer leur propre univers. \guil{[L]aissez mûrir l'enfance dans les enfants}, disait Jean-Jacques Rousseau; ce à quoi nous aimerions ajouter: \guil{aimez les enfants, aimez-les bien et assez}.



\nocite{Carroll1979}
\nocite{Carroll1994}
\nocite{Charbonnier1962}
\nocite{Pattison1978}
\nocite{Hyatte1982}
\nocite{Bastin2002}
\nocite{Boivin2009}
\nocite{Calvet1930}
\nocite{Dupuy1931}
\nocite{David1994}
\nocite{Girard2008}
\nocite{Hale1989}
\nocite{Iche2015}

\begin{singlespace}
\addcontentsline{toc}{chapter}{Bibliographie}
\setlength\bibitemsep{1.5\itemsep}


\defbibheading{subbibliography}[\refname]{\subsection*{#1}}
\defbibheading{subsubbibliography}[\refname]{\subsubsection*{#1}}

\printbibheading

\printbibheading[heading=subbibliography, title={Textes littéraires et adaptations cinématographiques}]
\printbibliography[keyword=corpusprinc, heading=subsubbibliography, title={Corpus principal}]
\printbibliography[keyword=corpussec, heading=subsubbibliography, title={Corpus secondaire}]

\printbibheading[heading=subbibliography, title={Théorie littéraire}]
\printbibliography[keyword=perso, heading=subsubbibliography, title={Typologie du personnage}]
\printbibliography[keyword=langage, heading=subsubbibliography, title={Linguistique et ludisme langagier}]
\printbibliography[keyword=enfant, heading=subsubbibliography, title={Représentations de l'enfance en littérature}]

\printbibheading[heading=subbibliography, title={Ouvrages critiques, articles et entrevues}]
\printbibliography[keyword=alice, heading=subsubbibliography, title={Sur les actualisations d'Alice dans la littérature française}]
\printbibliography[keyword=queneau, heading=subsubbibliography, title={Sur \textit{Zazie dans le métro} et l'écriture de Raymond Queneau}]
\printbibliography[keyword=labreche, heading=subsubbibliography, title={Sur \textit{Borderline} et l'écriture de Marie-Sissi Labrèche}]

\end{singlespace}
\end{document}
