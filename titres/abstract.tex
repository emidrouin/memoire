%\addcontentsline{toc}{chapter}{Abstract}
\chapter*{Abstract}
%\selectlanguage{english}

\begin{spacing}{1.4}

Established in French literature since the 19th century, the child character presents itself under several figures, also called \guil{typologies}.
Autonomous, imperturbable, sometimes even disturbing: its the \guil{enfant du monde} (child of the world) that seems to represent best the young moderne or contemporary characters.
Even more problematic when it is a little girl, this adaptive child is reminiscent of the English character of Alice, presented by Lewis Carroll: diving into a world of nonsense, the young girl seeks her landmarks and adapts in order to survive.
In order to represent the legacy of this \textit{lineage}, we chose the characters of Zazie (\textit{Zazie dans le métro} by Raymond Queneau) and Sissi (\textit{Borderline} by Marie-Sissi Labrèche).
\par
From a theoretical perspective which is the \textit{sémiologie du personnage} (\guil{typology of the character}) of Philippe Hamon, we have created the new figure of the \guil{enfant-femme} (woman-child, not to confuse with child-like woman), which is a way of revisiting the myth of Alice.
Pushed to empowerment by its context of existence, the \textit{enfant-femme} adopts an ambiguous behaviour about her age: she is a child but sometimes acts as an adult.
Since language is one of its most distinctive characteristics, linguistic theories by Roman Jakobson help to define how its use of language establishes a playful poetic.
Thus, the child's relation to the adult language can be analyzed in two stages: from the point of view of her understanding of the adult code, and through her subversive misuse of the same code.

\end{spacing}

\bigskip

\begin{singlespace}
\textbf{Keywords ~:}
contemporary literature, childhood representation, woman-child figure, play-on-words, Raymond Queneau, Marie-Sissi Labrèche
\end{singlespace}
%\selectlanguage{french}
