\newpage

\section*{Notes préliminaires}

\subsection*{Liste des abbréviations}
Afin d'alléger le texte, les références aux \oe{}uvres de notre corpus seront faites à même le texte en utilisant les abbréviations suivantes, suivies du numéro de la page:

\begin{singlespace}
\noindent
\textit{Z}: \cite{Queneau1959}.
\end{singlespace}

\begin{singlespace}
\noindent
\textit{B}: \cite{Labreche2003}.
\end{singlespace}

\subsection*{Éditions utilisées}
Pour \textit{Zazie dans le métro} de Raymond Queneau, nous avons employé une version imprimée en 1965 dans la collection \guil{Le Livre de poche}, aux éditions Gallimard; la première de couverture représente Zazie griffonée sur du papier quadrillé, le corps remplacé par un billet de métro.
\bigskip
\par
Pour les romans de Marie-Sissi Labrèche (\textit{Borderline}, \textit{La Brèche} et \textit{La lune dans un HLM}), toutes nos notes font référence aux publications dans le format \guil{Compact} chez les Éditions du Boréal.

\newpage
