\sloppy
\begin{titlepage}
  \begin{singlespace}

\begin{center}
{Université de Montréal} \vspace{1.5 cm}\\
\end{center}

\begin{center}
\Large{{\bf{Revisiter le mythe d'Alice}}\\Le personnage d'enfant-femme chez \\Raymond Queneau et Marie-Sissi Labrèche}
\end{center}

\vspace{1.5 cm}

\begin{center}
\normalsize{par Emilie Drouin}
\end{center}

\vspace{1.5 cm}

\begin{center}
Département des littératures de langue française\\
Faculté des arts et des sciences
\end{center}

\vspace{1.5 cm}

\begin{center}
\bf{Mémoire présenté en vue de l'obtention du grade \\ de maîtrise en littératures de langue française}
\end{center}
\vspace{1.5 cm}

\begin{center}
Mai 2017\\
\vspace{3 cm}
{\textcircled{c}}~Emilie Drouin, 2017
\end{center}


%\pagebreak

%\vspace{3 cm}

%\begin{center}
%  {Université de Montréal \\ Faculté des études supérieures et postdoctorales}
%\end{center}

%\vspace{3cm}

%\begin{center}
%{Ce mémoire intitulé}
%\end{center}

%\vspace{0.5 cm}

%\begin{center}
%{{\bf{Revisiter le mythe d'Alice}}\\Le personnage d'enfant-femme chez Raymond Queneau et Marie-Sissi Labrèche}
%\end{center}

%\vspace{3 cm}

%\begin{center}
%\normalsize{présenté par \\ Emilie Drouin}
%\end{center}

%\vspace{3 cm}


%\begin{center}
%{a été évalué par un jury composé des personnes suivantes:} \\
%\vspace{1 cm}
%{Marie-Pascale Huglo\\ \small{président-rapporteur}} \\
%\vspace{0.5 cm}
%{Gilles Dupuis\\ \small{directeur de recherche}} \\
%\vspace{0.5 cm}
%{Élisabeth Nardout-Lafarge\\ \small{membre du jury}}
%\end{center}

  \end{singlespace}
 \newpage
\end{titlepage}
