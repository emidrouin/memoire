%\addcontentsline{toc}{chapter}{Résumé}
\chapter*{Résumé}

\begin{spacing}{1.4}

Établi en littérature française depuis le XIX\up{e} siècle, l'enfant-personnage se présente sous plusieurs figures, aussi appelées \guil{typologies}.
Autonome, imperturbable, parfois même perturbant: c'est la figure de l'\guil{enfant du monde} qui semble le mieux représenter les petits personnages modernes ou contemporains.
Encore plus problématique lorsqu'il s'agit d'une petite fille, cet enfant adaptatif n'est pas sans rappeler le personnage anglais d'Alice, présenté par Lewis Carroll: plongée dans un monde insensé, la jeune fille cherche ses repères et s'adapte afin de survivre.
Nous avons choisi, afin de représenter l'héritage de cette \textit{lignée}, les personnages de Zazie (\textit{Zazie dans le métro} de Raymond Queneau) et de Sissi (\textit{Borderline} de Marie-Sissi Labrèche).
\par
Dans une perspective théorique relevant de la sémiologie du personnage de Philippe Hamon, nous avons dégagé la figure de l'\guil{enfant-femme}, qui constitue une façon de revisiter le mythe d'Alice.
Poussée à l'autonomisation par son contexte d'existence, l'enfant-femme adopte un comportement générateur d'ambiguïté à propos de son âge: elle est enfant mais agit parfois comme une adulte.
Le langage constituant l'une de ses caractéristiques les plus distinctives, les théories linguistiques de Roman Jakobson permettent d'établir en quoi son usage du langage relève d'une poétique ludique.
Le rapport de l'enfant à la langue des adultes s'analyse en deux étapes: d'abord sous l'angle de sa compréhension du code des adultes, ensuite par le biais de son usage détourné de ce même code.

\end{spacing}

\bigskip

\begin{singlespace}
\textbf{Mots clés~:}
littérature contemporaine, représentation de l'enfance, personnage d'enfant-femme, jeux de langage, Raymond Queneau, Marie-Sissi Labrèche
\end{singlespace}
