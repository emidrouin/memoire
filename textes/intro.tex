\chapter*{Introduction}
\addcontentsline{toc}{chapter}{Introduction}

\begin{flushright}
                                    \begin{singlespace}
                                    \epigraph{
                                    Un enfant, c'est un insurgé $\left[ \dots \right]$. Exubérant, désinvolte, insolent, au collège il lui arrivait souvent de chahuter; il me montra en riant sur son carnet une obervation qui lui reprochait des \guil{bruits divers en espagnol}; il ne posait pas au petit garçon modèle: c'était un adulte à qui sa maturité permettait d'enfreindre une trop puérile discipline.}
                                    \par
                                    Simone de Beauvoir, \textit{Mémoires d'une jeune fille rangée}\footcite[276]{Beauvoir1958}
                                    \end{singlespace}
\end{flushright}

Insoumis, indomptable, inflexible: l'enfant est un insurgé, nous dit Beauvoir.
Réel ou littéraire, l'enfant est complexe.
En tant que personnage, il est riche, en ce sens qu'il est avenir et aventures, espoir et espièglerie, innocence et rêverie.
Si l'enfance contemporaine se déroule sous le regard attentif de tous -- ne faut-il pas tout un village pour élever un enfant? --, la présence en littérature du jeune personnage n'a pas toujours été aussi marquée.
Ainsi la représentation de l'enfance en littérature a-t-elle grandement évolué au gré des courants de pensée philosophiques et du développement des connaissances scientifiques: de la créature plus proche de l'animal que de l'humain, l'enfant est devenu \textit{small adult} à développer, en passant par la dichotomie entre l'être parfois pur, parfois entaché du péché originel.
Au regard du changement de perception à son égard, il ne fait plus de doute que l'enfant est désormais un être à part entière dont les affects sont tout aussi complexes que ceux des adultes.
Les philosophes lui ayant rendu son humanité et les juristes lui ayant donné des droits, c'est au tour des littéraires d'accorder à l'enfant un droit de parole.
\par
Il y a plus de deux siècles que l'enfant a fait son entrée dans la littérature et qu'il est devenu un personnage digne d'intérêt, que ce soit dans la littérature légitime ou dans la littérature enfantine.
Si le personnage de garçon ne semble pas avoir posé problème -- on lui fait vivre des aventures desquelles il ressort grandi, sorte d'odyssée vers l'\textit{être homme} --, le \guil{cas} de la jeune fille n'est pas aussi simple.
Vouée à un rôle d'épouse puis de mère, la petite fille, réelle comme littéraire, se veut idéalement délicate et féminine.
On lui refuse donc l'action et les aventures, que l'on applaudit pourtant chez les garçons.
Dans les livres aux visées pédagogiques, elle est sage et bien élevée, sinon l'histoire tend à prouver que son bonheur ne peut passer que par son assagissement.
C'est la situation présentée par le livre \textit{Les Malheurs de Sophie}\footnote{Écrit par la comtesse de Ségur, née Sophie Rostopchine, c'est un livre pour enfants publié en 1858 aux éditions Hachette et encore lu de nos jours. Les livres \textit{Les Petites filles modèles} et \textit{Les Vacances} en sont la suite.} dans lequel les bêtises de Sophie, enfant aventurière, sont dépeintes négativement, et les tempéraments plus calmes de ses amies Camille et Madeleine, petites filles modèles, valorisés.
Tentant d'expliquer pourquoi c'est Alice qui est \guil{devenue \textit{la} petite fille, plutôt que Sophie en ses malheurs\footcite[7]{Lecercle1998}}, Jean-Jacques Lecercle propose qu'elle soit \guil{la première des petites filles\footcite[7]{Lecercle1998}}.
Alors que l'enfant était, jusque là, soit petit ange sage, soit petit diable à discipliner, l'enfant décrite par Lewis Carroll se pose différemment.
Se détachant de cette \guil{littérature pour enfants chrétienne et édifiante}, elle appartient à une nouvelle \guil{littérature enfantine où l'enfant, au lieu d'être objet de correction, est héros d'aventures\footcite[9]{Lecercle1998}.}
Ni petite fille sage, ni bon petit diable, elle est simplement \textit{petite fille}, rejetant à la fois moralisme et mièvrerie\footcite[10]{Lecercle1998}: \guil{Alors naît effectivement la petite fille moderne, délivrée des nécessités de la correction et de la conversion, qui peut être ce qu'est encore pour nous aujourd'hui la petite fille: un espace (mythique) de liberté\footcite[11]{Lecercle1998}.}
\par
Cette conception de la petite fille comme personnage \textit{libre} s'accompagne cependant des présupposés intrinsèquement liberticides selon lesquels \guil{l'innocence de la petite fille doit être protégée contre les dangers du vaste monde, où rôdent des loups déguisés en grand-mères\footcite[13]{Lecercle1998}.}
Une petite fille comme Alice est donc \textit{problématique} pour la littérature, en ce sens que son existence en tant que personnage n'est envisageable qu'au prix du non-respect des diktats sociaux sur les genres.
Cette perspective féministe -- d'ailleurs fort actuelle -- nous semble intéressante, notamment à la lumière des théories sur le genre de Monique Wittig, revisitées par Judith Butler\footcite[222-247]{Butler2012}.
\par
Or, ayant choisi de nous intéresser à \textit{Zazie dans le métro} de Raymond Queneau ainsi qu'à l'enfance représentée dans \textit{Borderline} de Marie-Sissi Labrèche\footnote{Puisque nos recherches portent sur l'enfance, nous nous intéresserons exclusivement aux chapitres pairs, soit ceux qui présentent une narratrice enfant.}, nous avons vite repéré un autre fil conducteur: si la petite fille pose encore problème, c'est d'abord et avant tout par son langage.
De là notre idée de rapprocher deux romans que tout oppose sur un autre plan: l'un est un roman parodique publié en 1959 par un homme de lettres français extrêmement respecté; l'autre est une autofiction publiée en 2000 et tirée de la partie \guil{création} du mémoire de maîtrise d'une jeune québécoise.
Puisque \textit{Zazie dans le métro} a déjà été beaucoup étudié -- plusieurs ont d'ailleurs fait le lien entre les écrits de Queneau et ceux de Lewis Carroll, et même posé Zazie comme \guil{héritière} naturelle d'Alice --, l'originalité de notre travail réside surtout au niveau de notre analyse du personnage d'enfant chez Labrèche.
Bien que \textit{Borderline} ait déjà été analysé, personne n'a, à notre connaissance, fait d'analyse spécifique sur le personnage d'enfant.
Ainsi, beaucoup ont examiné le rapport mère-fille ou le discours sur le corps, la sexualité et le désir, mais toujours en mettant l'accent sur le personnage adulte et en évacuant presque totalement les chapitres narrés par l'enfant Sissi.
Ceux qui se sont intéressés à l'enfant l'ont fait de concert avec le personnage adulte, utilisant une approche psychanalytique afin d'étudier principalement le rapport à la mère: ainsi \textit{Borderline} est-il parfois perçu, dans une perspective freudienne, comme un \guil{roman familial des névrosés\footcite[9-10]{Dion2010}}.
\par
Avec ce mémoire, nous comptons mettre en lumière une confrontation: non pas celle entre deux générations, mais plutôt celle entre deux \textit{mondes}, deux univers qui sont aux antipodes mais qui se nourrissent l'un et l'autre, en ce sens qu'ils se doivent de coexister et même de cohabiter.
Nous souhaitons donc dépasser la simple étude sur l'enfance et le personnage d'enfant, afin d'observer, dans les romans \textit{Zazie dans le métro} et \textit{Borderline}, la \guil{collision} entre les mondes des adultes et des enfants.
En plus de porter attention au conflit entre ces univers si différents, nous nous intéresserons à la médiation de ce conflit, à savoir comment ces deux mondes qui s'entrechoquent essaient -- et parviennent, ou pas -- à s'arrimer ou à s'harmoniser.
\par
Nous convoquerons d'abord la figure d'Alice de Lewis Carroll\footcite{Carroll2012}, enfant confrontée à un monde inquiétant, vaste métaphore du monde adulte, afin de définir une \textit{typologie} au sens où l'a entendu Philippe Hamon dans sa théorie sur la sémiologie du personnage\footcite{Hamon1977, Hamon1983}.
Puis, nous établirons des liens entre Zazie et Sissi, soit les jeunes filles des romans de notre corpus, et notre typologie nouvellement établie.
Enfin, dans une approche qui relève d'une \textit{linguistique du locuteur}, nous examinerons le rapport qu'ont ces personnages avec le langage, à savoir quel usage elles en font et leurs raisons.
Par ce mémoire, nous espérons dresser le portrait d'une nouvelle \textit{figure} de personnage d'enfant, que nous appellerons l'\textit{enfant-femme}.
