\chapter{Ces enfants de langage: une existence conditionnée par la parole}

\begin{flushright}
                                    \begin{singlespace}
                                    \epigraph{
                                    Je hais tellement l'adulte, le renie avec tant de colère, que j'ai dû jeter les fondements d'une nouvelle langue.}
                                    \par
                                    Réjean Ducharme, \textit{L'Avalée des avalés}\footcite[336-337]{Ducharme1982}
                                    \end{singlespace}
\end{flushright}

Chez l'enfant-femme, le paradigme de l'âge ne se résume pas au seul caractère du personnage.
Son langage rapprochera également des caractéristiques contradictoires, jusqu'à constituer la pierre d'assise de la spécificité de la typologie.
Ainsi, l'enfant-femme est paradoxale jusqu'au bout de la langue, en ce sens qu'elle existe et se distingue par son rapport au langage et par son usage des mots.
\par
En ce sens, nous convoquons le concept d'\guil{enfants de langage\footcite[95-103]{Vurm2014}}, mécanisme qui permet au personnage d'accéder au statut d'enfant-adulte, lequel n'est pleinement atteint que par le maniement de la langue.
C'est donc par l'agencement joueur et enfantin d'un vocabulaire avancé et adulte que se manifestera brutalement la discordance entre les divers âges du personnage d'enfant-femme, à savoir son âge biologique et son âge psychologique.
Par ce ludisme, qui consiste en fait en un jeu sur le code, l'enfant-adulte exposera à la fois son appartenance au monde adulte et sa résistance envers lui.
Ainsi, l'ambiguïté née de cette incapacité à situer le narrateur sur le plan de l'âge -- il n'est ni un enfant, ni un adulte -- transpose le paradoxe de l'âge sur un second plan, soit celui du langage, lequel est à son tour fort ambivalent quant à l'âge du personnage.
\par
Petr Vurm, dont la thèse porte sur les enfants chez Réjean Ducharme, voit l'\guil{indétermination de l'âge linguistique} -- lorsque \guil{des enfants parlent comme des adultes et les adultes glissent souvent vers le discours des enfants} --, comme élément générateur de paradoxes et donc d'un grand intérêt au niveau littéraire\footcite[97]{Vurm2014}.
L'existence en soi d'\guil{enfants de langage} est d'autant plus paradoxale que le terme d'\guil{enfant} signifie étymologiquement \guil{qui ne parle pas\footnote{Le mot enfant provient du latin \textit{infantem}, dérivé accusatif de l'\textit{infans}.}}.
\par
Le \guil{caractère ambigu de l'enfant-adulte $\left[ \dots \right]$ nous mène à une deuxième métamorphose de l'enfant, celle de l'enfant littéraire\footcite[100]{Vurm2014}}, dont l'une des particularités sera sa \guil{lutte contre le monde des adultes\footcite[102]{Vurm2014}.}
Alors que le combat entre l'enfance et l'âge adulte se transpose sur le plan du langage, l'enfant se doit d'utiliser la langue des adultes s'il veut se positionner face à lui à armes égales.
Vurm insiste d'ailleurs sur le fait que même si les enfants sont en révolte, \guil{ils utilisent en même temps la langue, la logique et le raisonnement des adultes $\left[ \dots \right]$\footcite[99]{Vurm2014}.}
Leur usage du langage des adultes relève cependant de la ruse de guerre, sorte de cryptographie militaire mise en \oe{}uvre avec les moyens dont ils disposent.


\section{Zazie et le dégonflage du mythe langagier}
Dans \textit{Zazie dans le métro}, il est évident dès l'entrée en scène du personnage de Zazie que cette dernière ne correspond pas au modèle de l'enfant sage et naïf: \guil{Son langage et son comportement ne relèvent en rien des bonnes manières, elle maîtrise parfaitement l'argot et les grossièretés\footcite[88]{Maurin2007}.}
Cependant, si la critique mentionne systématiquement la \guil{\textit{foul tongue}\footcite[118]{Bernofsky1994}} de Zazie, c'est Roland Barthes qui a véritablement mis au jour la mécanique derrière le langage zazique, y repérant un système de réactions fort différentes envers les deux catégories que sont le langage-objet et le métalangage:
\begin{quote}
  \begin{singlespace}
    \small
    L'innocence de Zazie n'est pas fraîcheur, virginité fragile, valeurs qui ne pourraient appartenir qu'au méta-langage romantique ou édifiant: elle est refus du langage chanté, science du langage transitif; Zazie circule dans son roman à la façon d'un génie ménager, sa fonction est hygiénique, contre-mythique: elle rappelle à l'ordre\footcite[129]{Barthes1964}.
    \normalsize
  \end{singlespace}
\end{quote}
Selon Barthes, la \guil{vocation} du personnage de Zazie serait de déconstruire le langage des adultes en y opposant la vulgarité afin de le remplacer par un langage-jouet enfantin.
C'est ce qui correspond pour nous à son rapport au langage, à la fois démonstratif et constitutif de son statut d'enfant-adulte.

\subsection{Les zazismes: l'empressement phonétique}
Dans le vocabulaire de Zazie, maints mots font l'objet de graphies peu orthodoxes, détournant l'orthographe conventionnelle des mots au profit d'un phonétisme enfantin qui crée un \guil{climat sonore\footcite[79]{Leon1962}} particulier.
Ainsi, Zazie veut un \guil{cacocalo} ou des \guil{bloudjinnzes} et se demande ce que peut bien être un \guil{hormosessuel}.
Version écrite de la prononciation parfois déformée des mots, les \textit{zazismes} relèvent de cette \guil{ortografe fonétik} chère à Raymond Queneau, sorte de \guil{professeur de baragouin\footcite[13]{Queneau1975}} qui avait déjà prédit depuis un moment une révolution dans l'orthographe afin de tendre vers un néo-français constituant en la \guil{notation correcte du français parlé\footcite[20]{Queneau1965}}.
Si l'usage de la langue vernaculaire et de l'oral-écrit est linguistiquement marqueur de réalisme, il semble que cette affirmation soit inexacte en ce qui concerne \textit{Zazie dans le métro}\footnote{Gillian Lane-Mercier estime d'ailleurs que \guil{la logique du texte exclut $\left[ \dots \right]$ toute \guill{logique phonématique} réaliste} puisque les déviations de la norme orthographique ne sont pas constantes. Dans \cite[30]{Lane-Mercier1989}.} alors que la malléabilité du langage y constitue le bourgeon d'une poétique singulière chez un auteur qui apprécie divers types de mots et même les \guil{mots savoureux d'être des mots}: \guil{En vérité, pour Queneau, le graphisme n'est donc pas seulement la clé du réalisme mais aussi celle de la poésie. Et cette poésie est surtout celle de la fantaisie verbale\footcite[79]{Leon1962}.}
Dans son \guil{délire verbal\footcite[82]{Leon1962}}, ce qui importe à Queneau \guil{dans [s]es personnages, ce sont moins les actes qu'ils ont à accomplir que le climat poétique dans lequel ils ont à se mouvoir\footnote{\cite[1]{Hulot1952} cité dans \cite[83]{Leon1962}.}.}
\par
Ainsi, plusieurs digressions à la grammaire semblent relever du simple plaisir de l'écrivain, du \guil{jeu de la graphie pour la graphie} dont la vocation est de détruire le \guil{système établi} afin d'en rebâtir un nouveau\footcite[80]{Leon1962}, notamment l'\guil{agglutination syllabique} -- aussi appelé \guil{\textit{concertina-words}\footcite[45]{Redfern1959}} --, dont le fameux \guil{Doukipudonktan\footnote{\guil{D'où est-ce qu'il pue donc tant.}}}~(\textit{Z}:~7) de l'incipit ou l'exclamation \guil{Lagoçamilébou\footnote{\guil{La gosse à mis les bouts}, signifiant que l'enfant est partie.}}~(\textit{Z}:~33).
Si elles semblent parfois être laissées au hasard lorsqu'elles relèvent de la narration ou de paroles prononcées par les adultes -- à cet effet, Pierre R. Léon souligne que la phonétisation opérée par Queneau est non-systématique et relève \guil{de multiples contradictions} au sein du texte, parfois même à quelques pages ou quelques lignes\footcite[72-73]{Leon1962} --, les fantaisies grammaticales attribuées à Zazie opèrent comme un engrenage bien huilé, mécanisme de réactions induites par le langage et les écarts au langage des adultes.

\subsubsection{L'apport étranger: comme un langage de proximité}
Dans sa réinterprétation orthographique des mots, plusieurs variations concernent des termes étrangers qui sont orthographiés phonétiquement selon leur prononciation avec l'accent français par un auteur qui \guil{aime lui aussi habiller à la française les mots d'emprunt étranger\footcite[80]{Leon1962}}.
Tout d'abord le mot servant à désigner ces langues étrangères, les langues \guil{forestières}~(\textit{Z}:~87, 94), constitue selon Pierre Léon un bricolage adjectival \textit{à la française} à partir du mot anglais \textit{Far East} -- \guil{langues \textit{far-eastières}\footcite[82]{Leon1962}} --, utilisé ici de façon généralisatrice pour désigner tout ce qui est étranger à la France et au français \footnote{Nous pourrions ajouter que le mot \guil{\textit{forestiero}} désigne un étranger en italien.
La manie encyclopédique de Queneau ne nous étant pas étrangère -- ou \guil{forestière}! --, nous estimons qu'il s'agit là d'un autre trésor linguistique enfoui par l'auteur à l'intention des lecteurs curieux.}.
S'ajoutent le \guil{guidenappeur\footnote{Mot-valise entre \guil{guide} (touristique) et \textit{kidnapping}, mot anglais désignant un enlèvement (d'enfant).}}, le \guil{flicmane\footnote{De \textit{policeman}, mot anglais désignant un officier de police.}}, les \guil{vécés\footnote{D'après l'anglais W-C, pour \textit{water closet}, à savoir un cabinet d'eau (et de toilette).}} et même l'expression \guil{apibeursdè touillou\footnote{Souhait d'anniversaire -- \textit{Happy birthday to you} -- phonétisé en incluant un fort accent français dans la prononciation.}}.
Cette appropriation par la \guil{noble langue françouèze\footcite[127]{Barthes1964}} de concepts américains accentue l'impression de familiarité et induit un sentiment de proximité puisqu'il s'agit d'une reproduction relativement fidèle de la sonorité que prennent les mots lorsqu'ils sont prononcés par les Français et les Parisiens.
\par
Pour le personnage de Zazie, cette idée de \guil{langage de proximité} est encore plus concrète puisqu'il s'agit en réalité de son terrain de jeu:
\begin{quote}
  \begin{singlespace}
    \small
    Le langage-objet, c'est le langage qui se fonde dans l'action même, qui \textit{agit} les choses, c'est le premier langage transitif, celui dont on peut parler mais qui lui-même transforme plus qu'il ne parle. C'est exactement dans ce langage-objet que vit Zazie, ce n'est donc jamais lui qu'elle distance ou détruit. Ce que Zazie parle, c'est le contact transitif du réel: Zazie \textit{veut} son coca-cola, son blue-jean, son métro, elle ne parle que l'impératif ou l'optatif, et c'est pour cela que son langage est à l'abri de toute dérision\footcite[128]{Barthes1964}.
    \normalsize
  \end{singlespace}
\end{quote}
Le langage-objet tel que défini par Barthes, langue du concret et du réel, suscite chez Zazie le respect puisqu'il s'agit de la langue de ce qu'elle veut.
D'un point de vue sémiologique, le mot-objet, sorte de signe linguistique zazique, se construit directement du signifiant au référent, évacuant le concept médian de signifié: Zazie ne pense pas l'objet, elle le veut et le nomme.
\par
Si elle ne la met à distance ni ne la détruit, il demeure que l'enfant profite de la malléabilité de la langue et qu'elle en dévoile une prononciation nouvelle hautement phonétique.
Constituant des accrocs à l'orthographe usuelle, les \guil{zazismes} consistent en la mise à l'écrit de prononciations enfantines ou erronées attribuées au personnage de Zazie.
Il y a d'abord le terme \guil{bloudjinnzes}, françisation accentuée de \textit{blue jeans}, mot que Zazie se refuse de prononcer devant un commerçant de la foire aux puces: \guil{Elle n'ose pas énoncer le mot disyllabique et anglo-saxon qui voudrait dire ce qu'elle veut dire.}~(\textit{Z}:~43)
Pour revenir brièvement à la sémiotique, le signe \guil{bloudjinnzes}, intégré par Zazie à son vocabulaire en tant que mot-objet désiré, semble susciter chez elle de très vives émotions, comme si le seul fait de prononcer le mot (le signifiant) impliquait directement l'objet (le référent) sans même passer par la représentation mentale (le signifié) de cet objet.
Enfin, l'inversion des voyelles mène à l'invention du mot \guil{caco-calo}, également objet du désir, et dans lequel le linguiste et phonéticien Pierre Léon lit un calembour dont le second sens serait \guil{à l'eau\footcite[80]{Leon1962}}.
Les zazismes incluent également le phonétisme erroné \guil{hormosessuel}, objet de toutes les questions et qui, en plus du jeu sur l'orthographe, relève d'une forme particulière d'humour portant sur le double-sens.

\subsubsection{L'invention de nouveaux mots: l'insidieuse moquerie}
Au-delà du zazisme, le discours sur l'homosexualité prend parfois une tournure connotée par la liaison dans la prononciation des termes \guil{un hormosessuel}, syntagme homophone d'\guil{un normosessuel}, sorte de néologisme concédant une \textit{norme sexuelle}.
C'est le personnage de Fédor Balanovitch, lequel \guil{connaissait à fond la langue française} qui relève le jeu zazique sur la sonorité.
Lorsque Zazie lui demande si Gabriel est \guil{un hormo}, l'homme lui répond: \guil{Tu veux dire un normal.}~(\textit{Z}:~117)
Paradoxalement, Gabriel serait ainsi désigné à la fois comme homosexuel et comme \guil{normosexuel}, terme dont l'actualisation en fonction des théories \textit{queer} désignerait possiblement un hétérosexuel cisgenre.
Cette norme est donc définie par un personnage dont le physique est décrit comme éminement masculin, qui pratique un métier féminisé -- danseuse de charme -- sous un prénom également féminin, Gabriella -- et dont la conjointe semble à la fin du roman être plutôt un conjoint.
La normosexualité semble associée au maître de l'inversion sexuelle par excellence, au point où la blague semble plus qu'évidente.
De même, si le questionnement de Zazie à savoir ce que peut bien être un \guil{hormosessuel} relève possiblement de la naïveté enfantine qui lui est généralement supposée, la double lecture que nous proposons va à l'opposé: Zazie, posée comme être de raison servant selon Barthes à dégonfler la \textit{doxa} et le mythe de la jeunesse\footcite{Barthes1964}, ne serait-elle pas celle qui suggère la fin de l'hétéronormativité et de la catégorisation en \guil{normosessuels} et \guil{anormaux-sessuels}?
\par
Par le biais de Zazie, sorte de marionnette qui permet à son auteur d'écrire ce que lui-même ne peut pas dire, Raymond Queneau laisse également transparaître sa propre moquerie.
De façon tout aussi humoristique que les autres zazismes, l'agglutination syllabique \guil{Singermindépré}~(\textit{Z}:~26) est écrite ainsi comme parole l'enfant, alors que le nom du quartier avait pourtant été bien orthographié lorsque mentionné par Zazie auparavant~(\textit{Z}:~15).
Contenant le nom \guil{singe} de même que le verbe \guil{singer\footnote{\guil{Imiter maladroitement ou d'une manière caricaturale, pour se moquer.} dans \cite[Entrée: « Singer »]{Robert2013}}}, nous estimons qu'il pourrait s'agir d'une critique moqueuse de l'élite intellectuelle et culturelle parisienne habitant le quartier Saint-Germain-des-Prés, où sont d'ailleurs situées plusieurs maisons d'édition, dont Gallimard qui a édité \textit{Zazie dans le métro}.
Il nous semble, au-travers de cette moquerie zazique, apercevoir l'humour de Raymond Queneau, lequel a possiblement transposé dans les paroles de sa jeune créature une part de sa propre critique envers la société bourgeoise de son époque.

\subsection{La vulgarité comme violence langagière}
Qualifiée par le tenancier du bar de \guil{petite salope qui di[t] des cochoncetés}~(\textit{Z}:~18), Zazie maîtrise habilement le langage des adultes, y compris (et surtout) les grossièretés et l'argot\footcite[88]{Maurin2007}.
Elle offense pour ainsi dire les adultes par sa vulgarité, comme l'exprime la veuve Mouaque: \guil{La preuve, vous n'avez qu'à l'écouter parler (geste), elle est d'une grossièreté, dit la dame en manifestant tous les signes d'un vif dégoût.}~(\textit{Z}:~95)
Si nous ne remettons aucunement en question le fait que Zazie soit, linguistiquement parlant, vulgaire et grossière, nous émettons l'hypothèse qu'elle ne le soit pas gratuitement.
Il nous semble plutôt que le langage soit pour elle une façon de combattre les adultes et que son l'emploi de la vulgarité et d'un vocabulaire adulte lui permette d'aborder ce combat langagier à armes égales: \oe{}il pour \oe{}il, dent pour dent, mot pour mot!

\subsubsection{Le langage de brousse comme réaction à la contrariété}
Pour Zazie, l'utilisation d'un langage subversif semble avant tout constituer une réaction face à la contrariété.
Ainsi, l'exclamation qui lui échappe est d'une sincérité sans équivoque lorsqu'elle apprend que le métro est en grève: \guil{Ah! les salauds, s'écrie Zazie, ah! les vaches. $\left[ \dots \right]$ Sacrebleu, merde alors.}~(\textit{Z}:~10)
Devant la plus grande des déceptions -- le métro étant son principal objet de convoitise --, la jeune fille expose d'un seul coup toute l'étendue de son vocabulaire vulgaire afin d'exprimer sa frustration.
\par
Ce même vocabulaire sera d'ailleurs repris lors de contrariétés subséquentes.
Par exemple, l'expression \guil{le salaud} servira à désigner le \textit{type} lorsqu'il lui reprendra \guil{ses} bloudjinnzes qu'elle avait volés~(\textit{Z}:~56) ou encore le sergent de ville dont elle se méfie et qu'elle \guil{voi[t] venir avec ses gros yéyés}~(\textit{Z}:~101).
Parlant du tarif de l'avocat de sa mère, elle commentera: \guil{Il a été gourmand, la vache.}~(\textit{Z}:~48)
L'expression revient alors qu'elle s'indigne de ce que dit le \textit{type} à son propos -- \guil{C'est vache, ça, dit Zazie}~(\textit{Z}:~108) --, comme une ritournelle face à la contrariété.
Enfin, l'expression \guil{merde alors} sera reprise comme marqueur d'impatience ou de contrariété: d'impatience d'abord, que ce soit face à la lenteur des discussions futiles entre adultes -- \guil{Alors quoi, merde, dit Zazie, on va le boire, ce verre?}~(\textit{Z}:~15) -- ou face au refus des adultes de répondre à ses questions sur la sexualité -- \guil{Tu réponds, oui ou merde, cria Zazie}~(\textit{Z}:~87); de contrariété lorsque les enfantillages des adultes l'empêchent de dormir: \guil{Alors quoi merde, on peut plus dormir?}~(\textit{Z}:~25)
Ne reprenant qu'une seule des expressions à la fois, Zazie paraît évidemment moins irritée par ces désagréments que face à la grève du métro.
Que son utilisation d'un vocabulaire vulgaire soit faite avec parcimonie ou immodérément en fonction de l'ampleur de sa contrariété, l'objectif reste toutefois le même: il s'agit pour Zazie d'un exutoire à sa frustration, laquelle est souvent causée par l'embourbement bureaucratique des adultes.
Le langage grossier constitue alors une nouvelle modalité du rejet global du monde adulte.
\par
Se \guil{foutant} tout d'abord de l'explication selon laquelle il n'y a pas que pour elle que le métro soit en grève~(\textit{Z}:~10), Zazie exprime le même détachement nonchalant face à la belle ville que Gabriel lui montre: \guil{Je m'en fous, dit Zazie, moi ce que j'aurais voulu c'est aller dans le métro.}~(\textit{Z}:~11)
Elle n'a d'intérêt que pour une sélection restreinte de choses -- la sexualité, les \textit{jeans} vendus dans les surplus américains et surtout le métro parisien -- et en fait presque une obsession, exprimant envers tout le reste la plus grande indifférence qui soit.
Que sa fixation porte sur le métro ou plus tard sur les bloudjinnzes, face auxquels elle déclarera se foutre des boussoles~(\textit{Z}:~43), Zazie manifeste un rejet de ce qui est autre d'une expression vaguement grossière: \guil{Je m'en fous.}
Plus violemment encore, son seul désintérêt laisse place à un rejet pur et simple alors qu'elle \guil{emmerde} le monde adulte en général: \guil{Les pas marants, dit Zazie, je les emmerde. $\left[ \dots \right]$ Parce que moi, le billard, ça m'emmerde.}~(\textit{Z}:~120)
Auparavant qualifiés de \guil{petits marants}~(\textit{Z}:~13), l'oncle Gabriel et son ami Charles sont possiblement exclus de cette boutade méprisante envers le monde adulte.
De l'expression un peu brusque jusqu'à l'affront direct, Zazie élève par le langage une barrière entre elle et le monde adulte, jusqu'à dire à Charles: \guil{La nouvelle génération, dit Zazie, elle t'...}~(\textit{Z}:~15)

\subsubsection{La clausule zazique: mon cul!}
Si ce \guil{langage de brousse} est régulièrement utilisé par Zazie en réaction à une action (ou une inaction) des adultes, c'est face à leur langage que sa violence atteint son paroxysme et qu'elle se pose le plus concrètement comme enfant-adulte.
Selon Barthes, c'est en opposant à certaines prises de parole des adultes une \guil{clausule zazique\footcite{Barthes1964}} que Zazie dégonfle véritablement toute la mythologie et qu'elle acquiert son statut de contre-mythe.
Si tous sont d'accord à l'effet que le langage de Zazie est inconvenant -- Charles dira d'ailleurs qu'\guil{Elle peut pas dire un mot, cette gosse, sans ajouter mon cul après.}~(\textit{Z}:~18) --, l'analyse de Barthes va beaucoup plus loin en constatant qu'il s'agit là d'une véritable \guil{technique du dégonflage} systématiquement opérée sur le méta-langage tout en épargnant le langage-objet\footcite[128]{Barthes1964}. Ainsi, ce que rejette Zazie est le \guil{méta-langage des grandes personnes}:
\begin{quote}
  \begin{singlespace}
    \small
    Ce méta-langage est celui dont on parle, non pas les choses, mais \textit{à propos} des choses (ou \textit{à propos} du premier langage). C'est un langage parasite, immobile, de fond sentencieux, qui double l'acte comme la mouche accompagne le coche; face à l'impératif et à l'optatif du langage-objet, son mode principiel est l'indicatif, sorte de degré zéro de l'acte destiné à \textit{représenter} le réel, non à le modifier. Ce méta-langage développe autour de la lettre du discours un sens complémentaire, éthique, ou plaintif, ou sentimental, ou magistral, etc.; bref, c'est un \textit{chant}: on reconnaît en lui l'être même de la Littérature\footcite[128]{Barthes1964}.
    \normalsize
  \end{singlespace}
\end{quote}
Complètement opposé au langage-objet dont font notamment partie les zazismes et autres fantaisies grammaticales, ce méta-langage représente la \guil{langue des idées}, sorte de réflexion à haute voix qu'abhorre plus que tout Zazie.
Aussi, bien au-delà de la simple grossièreté ou impolitesse, la vulgarité est-elle, pour l'enfant, un moyen de rejeter en deux petits mots toute la \textit{doxa}.
\par
Concrètement, Zazie appose d'abord sa clausule vulgaire lorsque l'adulte tente de la définir par la parole.
Par exemple, elle réagit promptement lorsque Gabriel lui ordonne de ne pas être snob -- \guil{Snob mon cul}~(\textit{Z}:~11) -- ou que \guil{Turandot, d'un air supérieur} sous-entend qu'elle est gentille -- \guil{Gentille mon cul}~(\textit{Z}:~26).
Lorsque la veuve Mouaque suggère que l'enfant puisse avoir des \guil{qualités}, la principale intéressée réplique aussitôt: \guil{Qualités mon cul, grommela Zazie.}~(\textit{Z}:~98)
La réplique est quasi-automatique lorsque l'on parle d'elle pour la qualifier: Zazie refuse de sa \guil{clausule assassine} tout étiquetage de sa personne.
\par
Sorte de tentative renouvelée de la définir par l'impératif, les règles de bienséance sont rejetées tout aussi violemment par Zazie, qui opère un rejet tout aussi catégorique lorsque l'on tente de lui faire la morale: \guil{Politesse mon cul}~(\textit{Z}:~125), décrètera-t-elle.
Par son refus de respecter ces règles que tentent de lui imposer les adultes, Zazie mène selon Barthes une importante opération contre-mythique puisqu'elle dégonfle le mythe de l'enfant sage que cachent les propos des adultes.
C'est par la violence de sa ritournelle \guil{mon cul} que l'enfant réussira à rejeter ce mythe, à \textit{ne pas être} l'enfant bien élevée que l'on attend d'elle:
\begin{quote}
  \begin{singlespace}
    \small
    -- Il ne faut pas brutaliser comme ça les grandes personnes. \\
    -- Grandes personnes mon cul, répliqua Zazie. $\left[ \dots \right]$ \\
    -- [La violence] est éminemment condamnable. \\
    -- Condamnable mon cul, répliqua Zazie, je ne vous demande pas l'heure qu'il est.\\
    -- Seize heures quinze, dit la bourgeoise. \\
    -- Vous n'allez pas laisser cette petite tranquille, dit Gabriel qui s'était assis sur un banc. \\
    -- Vous m'avez encore l'air d'être un drôle d'éducateur, vous, dit la dame. \\
    -- Éducateur mon cul, tel fut le commentaire de Zazie.~(\textit{Z}:~95)
    \normalsize
  \end{singlespace}
\end{quote}
Par sa clausule, Zazie dégonfle coup sur coup les grands mythes que sont l'autorité, la morale et l'éducation, lesquels représentent tous l'existence d'un certain ascendant des adultes sur les enfants.
Le \guil{mon cul} zazique est donc d'une efficacité remarquable en ce sens qu'il détruit tout le rapport de subordination entre l'enfant et l'adulte, libérant l'espace requis pour la naissance d'un enfant-adulte.
\par
À cet effet, le projet de carrière de Zazie démontre sa compréhension manifeste du rapport d'autorité entre les adultes et les enfants -- elle anticipe le moment où ce sera elle l'autorité --, tout en manifestant tous les signes langagiers d'un refus d'obéir à cette autorité.
Ainsi, après l'avoir dégonflé par le langage, Zazie reconstruit le grand mythe sur l'éducation:
\begin{quote}
  \begin{singlespace}
    \small
    -- Moi, déclara Zazie, je veux aller à l'école jusqu'à soixante-cinq ans. \\
    -- Jusqu'à soixante-cinq ans? répéta Gabriel un chouïa surpris. \\
    -- Oui, dit Zazie, je veux être institutrice. \\
    -- Ce n'est pas un mauvais métier, dit doucement Marceline. Y a la retraite. \\
    $\left[ \dots \right]$\\
    -- Retraite mon cul, dit Zazie. Moi c'est pas pour la retraite que je veux être institutrice. \\
    $\left[ \dots \right]$ \\
    -- Alors? Pourquoi que tu veux l'être, institutrice? \\
    -- Pour faire chier les mômes, répondit Zazie. Ceux qu'auront mon âge dans dix ans, dans vingt ans, dans cinquante ans, dans cent ans, dans mille ans, toujours des gosses à emmerder. \\
    -- Eh bien, dit Gabriel. \\
    -- Je serai vache comme tout avec elles. Je leur ferai lécher le parquet. Je leur ferai manger l'éponge du tableau noir. Je leur enfoncerai des compas dans le derrière. Je leur botterai les fesses. Parce que je porterai des bottes. En hiver. Hautes comme ça (geste). Avec des grands éperons pour leur larder la chair du derche. \\
    $\left[ \dots \right]$ \\
    -- D'ailleurs, dit Gabriel, dans vingt ans, y aura plus d'institutrices: elle seront remplacées par le cinéma, la tévé, l'électronique, des trucs comme ça. $\left[ \dots \right]$ \\
    Zazie envisagea cet avenir un instant. \\
    \guil{Alors, déclara-t-elle, je serai astronaute. \\
    -- Voilà, dit Gabriel approbattivement. Voilà, faut être de son temps. \\
    -- Oui, continua Zazie, je serai astronaute pour aller faire chier les Martiens.}~(\textit{Z}:~20-21)
    \normalsize
  \end{singlespace}
\end{quote}
Désintéressée par le monde adulte, c'est encore par sa clausule que Zazie porte le coup fatal à la Grande histoire. À Gabriel qui lui propose de l'emmener \guil{voir vraiment les Invalides et le tombeau véritable du vrai Napoléon}~(\textit{Z}:~13), elle rétorque, sans appel: \guil{Napoléon mon cul, réplique Zazie. Il m'intéresse pas du tout, cet enflé, avec son chapeau à la con.}~(\textit{Z}:~13)
Véritable mécanisme de rejet de la littérarité adulte [et] du méta-langage, la clausule zazique \guil{est comme une détonation finale qui surprend la phrase mythique $\left[ \dots \right]$, la dépouille rétroactivement, en un tour de main, de sa \textit{bonne conscience}\footcite[129]{Barthes1964}.}
Autant qu'elle déstabilise les adultes par son comportement, Zazie s'attaque cette fois-ci à leur langage, qu'elle détourne par sa courte expression vulgaire.
Les adultes étant, aux yeux des enfants, de vulgaires \guil{fonctionnaires\footcite[100]{Vurm2014}}, se pourrait-il que la vocation contre-mythique de Zazie soit en fait de démontrer la \guil{dérive bureaucratique}?
Il est permis d'y croire, puisqu'en détournant invariablement la langue de l'\guil{administration}, c'est tout le système des adultes qu'elle tourne en ridicule: \guil{Cette clausule zazique résume tous les procédés du contre-mythe, dès lors qu'il renonce à l'explication directe et se veut lui-même traîtreusement littérature\footcite[129]{Barthes1964}.}


\subsection{Zazie, Grande Inquisitrice: la poétique de la question}
S'il est possible d'alléguer la grande curiosité de Zazie, il nous apparaît pertinent, afin de comprendre le rapport du personnage à la langue, de réfléchir plus longuement sur la façon dont cette curiosité se présente puisque sa principale manifestation consiste en la construction d'une poétique de la question.
Ainsi, Zazie n'est pas \textit{décrite} comme étant curieuse; elle l'\textit{est}, principalement par le langage, lequel opère souvent sur le mode interrogatif.
\par
Selon Dominique Rioux, dont le mémoire de maîtrise porte spécifiquement sur la pratique interrogative chez les personnages de jeunes filles de Raymond Queneau, le concept de la question se définit comme une \guil{appréhension de l'altérité de ce que le sujet cherche à comprendre et [est], par conséquent, début de la levée des préjugés $\left[ \dots \right]$\footcite[20]{Rioux2006}.}
Les questions des jeunes filles \guil{port[a]nt essentiellement sur la langue et le sexe, deux puissants tabous dans la littérature quenienne\footcite[22]{Rioux2006}}, elles sont représentées parfaitement par le terme \textit{hormosessualité}, lequel réfère à l'anormalité à la fois langagière et sexuelle, allant à l'encontre autant de la grammaire française que de l'hétéronormativité.
\par
Outre le métro, l'hormosessualité supposée de son oncle semble être le principal sujet d'intérêt de Zazie, et la quête initiale du métro cédera bien vite la place à une quête plus globale, soit celle d'une réponse à sa question: \guil{Au fond, dit Zazie, je voudrais bien savoir ce xé. $\left[ \dots \right]$ Ce xé qu'un hormosessuel.}~(\textit{Z}:~97)
Bien sûr, quelques adultes tentent de la contenter en lui offrant de vagues réponses: \guil{Qu'est-ce que c'est un hormosessuel? demanda Zazie. / - C'est un homme qui met des bloudjinnzes, dit doucement Marceline.}~(\textit{Z}:~61)
Zazie n'est cependant pas rassasiée de cette réponse et poursuit sa quête.
Questionnant Charles à propos du fait que Gabriel \guil{pratique l'hormosessualité}~(\textit{Z}:~81), elle lui rapporte les propos du \textit{type} disant qu'\guil{on p[eut] aller en tôle pour ça}~(\textit{Z}:~81) mais demeure toutefois sans réponse sur la signification du mot:
\begin{quote}
  \begin{singlespace}
    \small
    -- Qu'il soit hormosessuel? Mais qu'est-ce que ça veut dire? Qu'il se mette du parfum? \\
    -- Voilà. T'as compris. \\
    -- Y a pas de quoi aller en prison. \\
    $\left[ \dots \right]$ \\
    Et vous? demanda Zazie. Vous l'êtes, hormosessuel?\\
    -- Est-ce que j'ai l'air d'une pédale?\\
    -- Non, pisque vzêtes chauffeur.~(\textit{Z}:~81)
    \normalsize
  \end{singlespace}
\end{quote}
Relevant seulement la définition d'usage du mot \guil{pédale}, Zazie souligne l'absurdité du propos de Charles  -- un chauffeur ne peut évidemment pas être une pédale! --, tout en emmagasinant dans son vocabulaire le double-sens du mot, à savoir qu'il est également synonyme d'homosexuel.
Ainsi, après avoir acquis un vaste vocabulaire de l'homosexualité et de ses synonymes, mais n'en maîtrisant toujours pas le sens, Zazie se montre plus déterminée que jamais à obtenir une réponse, abandonnant les questions délicates et déballant toutes ses interrogations d'un coup à la veuve Mouaque, en reprenant notamment le synonyme \guil{pédale} récemment appris: \guil{Qu'est-ce que c'est au juste qu'une tante? lui demanda familièrement Zazie en vieille copine. Une pédale? une lope? un pédé? un hormosessuel? Y'a des nuances?}~(\textit{Z}:~123)
En plus de vouloir savoir \guil{ce xé} qu'un hormosessuel, Zazie questionne également incessemment pour savoir si son tonton en est un ~(\textit{Z}:~87, 94, 96, 114, 117).
\par
À cette grande question relative à l'hormosessualité de son oncle s'ajoutent les presqu'aussi grandes interrogations de Zazie concernant les langues étrangères: \guil{Tonton Gabriel, dit Zazie paisiblement, tu m'as pas encore espliqué si tu étais un hormosessuel ou pas, primo, et deuzio où t'avais été pêcher toutes les belles choses en langue forestière que tu dégoisais tout à l'heure? Réponds.}~(\textit{Z}:~94)
Ces \guil{langues forestières} référant, comme nous l'avons déjà évoqué, aux langues étrangères, Zazie questionne la capacité de son oncle à les parler: \guil{C'est justement ça, ma deuxième question, dit Zazie. Quand je t'ai retrouvé aux pieds de la tour Eiffel, tu parlais l'étranger aussi bien que lui [Fédor Balanovitch]. Qu'est-ce qui t'avait pris? Et pourquoi que tu recommences plus?}~(\textit{Z}:~118)
\par
Revendiquant le droit imprescriptible de poser des questions, Zazie semble avoir une interrogation adaptée à chaque situation possible.
Ainsi, rejoignant son oncle au bas de la tour Eiffel, elle constate l'absence du taximan et l'interroge:
\begin{quote}
  \begin{singlespace}
    \small
    Elle n'y vit point Charles et le fit remarquer. \\
    \guil{Il s'est tiré, dit Gabriel. \\
    -- Pourquoi? \\
    -- Pour rien. \\
    -- Pour rien, c'est pas une réponse. \\
    -- Oh bin, il est parti comme ça. \\
    -- Il avait une raison. \\
    -- Tu sais, Charles. (geste) \\
    -- Tu veux pas me le dire?
    -- Tu le sais aussi bien que moi.} \\
    $\left[ \dots \right]$ \\
    Zazie revint à son point de départ. \\
    \guil{Tout ça ne me dit pas pourquoi charlamilébou.}~(\textit{Z}:~86-87)
    \normalsize
  \end{singlespace}
\end{quote}
Tenace, la jeune fille pose et repose les mêmes questions, insistant pour que l'on assouvisse son besoin d'être informée: \guil{Et ma question à moi, demanda-t-elle mignardement. On y répond pas?}~(\textit{Z}:~95)
Revenant incessamment à la charge avec ses interrogations -- \guil{Alors tonton, et cette réponse?}~(\textit{Z}:~98) --, Zazie jumelle le mode impératif à la violence physique: les ordres -- \guil{Réponds} et \guil{Réponds donc} -- sont ainsi entrecoupés d'\guil{un bon coup de pied sur la cheville}~(\textit{Z}:~94) d'un Gabriel peu collaboratif.
Elle va jusqu'à justifier sa propre violence envers son oncle par le fait qu'il ne voulait pas lui répondre et exige de lui qu'il le fasse avant de s'en aller.
\par
De par son usage constant du mode interrogatif, nous estimons que le personnage de Zazie se construit, du moins en partie, par le biais de ses questionnements et qu'elle se définit donc d'un point de vue langagier par une poétique de la question.
De tous les objets de son intérêt, c'est à notre avis le terme \guil{hormosessuel} qui résume le mieux cette poétique puisqu'il relève à la fois de l'anomalie langagière et de la sexualité, tous deux sujets de prédilection de la curiosité zazique.
Nous estimons donc que l'hormosessualité -- et tous les questionnements qu'elle suscite -- est le principal vecteur de la poétique langagière de Zazie.
\par
L'enfant-femme faisant montre d'une curiosité hors de l'ordinaire, sa poétique de la question s'inscrit à la limite d'une politique de la torture envers son oncle.
Zazie est Grande Inquisitrice: elle a tous les droits, celui de poser des questions certes, mais surtout celui d'obtenir des réponses.
Elle est tenace devant le silence des adultes, exigeant la vérité devant leurs réponses évasives ou manifestement fausses.
Autant dégonfle-t-elle le code des adultes, autant tient-elle à le posséder: Zazie est sans complexes, revendiquant son droit d'être informée comme une adulte, sans égards à son âge.


\section{Sissi et le langage, ce jouet comme un autre}
Dans \textit{Borderline}, que l'on rappelle être une autofiction dans laquelle la narratrice revisite -- entre autres -- son enfance, le langage est en adéquation avec le milieu défavorisé dans lequel Sissi grandit.
Ainsi, la langue y est populaire, voire vernaculaire, et ce n'est qu'au prix de cours de diction que l'enfant développe une prononciation -- académiquement -- \guil{adéquate}, laquelle sera par ailleurs raillée par ses camarades de classe:
\begin{quote}
  \begin{singlespace}
    \small
    On rit de moi parce que j'ai un accent, parce que je dis \guil{moooâââ} et que j'ar-ti-cu-leuuu quand je parle. On me traite de petite fraîche: \textit{Saleuuu, petiteuuu fraîcheuuu! Saleuuu, petiteuuu fraîcheuuu!} On rit de moi et de mes cours de diction, le mardi et le jeudi, à l'heure du midi. Mes cours de diction, pour m'empêcher de dire un \guil{sien}, un \guil{sat}, des \guil{bisous}, des \guil{zenoux}. Tous les \guil{s}, les \guil{che}, les \guil{s} et les \guil{z} se mélangent.~(\textit{B}:~62-63)
    \normalsize
  \end{singlespace}
\end{quote}
Amenant l'enfant à bien détacher les syllabes et à prononcer correctement les mots, les cours de diction contribuent également à la \guil{construction} de son arsenal langagier, à savoir sa maîtrise de la langue.
C'est une fois la langue ainsi \guil{domptée} que se déploie la poétique langagière de Sissi, mettant de l'avant une \guil{déconstruction} de ce langage à peine construit.
Concrétisant cette \guil{théorie de la violence organisée} évoquée par Jakobson\footcite[40]{Jakobson1973} et présentée comme correspondant à l'exacerbation de la fonction poétique au nom d'une \guil{stratégie ludique\footcite[65]{Seyfrid-Bommertz1999}}, Sissi s'emploie donc, à partir de sa maîtrise du code du langage, à détourner ce dernier à des fins d'amusement.
\par
Si les jeux de langage peuvent sembler à prime abord relever simplement du jeu, Laure Hesbois estime au contraire qu'ils constituent une démonstration de choix de la mécanique de la langue:
\begin{quote}
  \begin{singlespace}
    \small
    Les jeux de langage sont incontestablement un matériau de choix pour étudier le fonctionnement du langage dans toute sa complexité, c'est-à-dire sous l'angle à la fois de la communication et de l'expression, comme système de signes et comme discours symbolique\footcite[18]{Hesbois1986}.
    \normalsize
  \end{singlespace}
\end{quote}
Elle prétend que les jeux de langage résultent de l'amalgame entre l'intérêt pour la linguistique au sens strict du terme\footcite[17]{Hesbois1986}, qui voit le langage comme instrument de communication\footnote{Cette approche découle de la linguistique structuraliste de Ferdinand de Saussure et fait du \textit{signe} l'unité de base; elle se perpétue par les travaux des héritiers fidèles de la linguistique saussurienne, de même que par ceux de Noam Chomsky, qui voit la langue comme une compétence.}, et la \textit{linguistique des locuteurs}\footcite[17-18]{Hesbois1986}, qui tient compte de la présence en langue du sujet parlant\footnote{C'est une approche qui n'est pas sans lien avec la psychanalyse de Freud, laquelle ouvre la porte à l'étude linguistique du \textit{sujet}; elle est portée entre autres par Roman Jakobson ou, plus récemment, par Julia Kristeva, de même que par tous ceux qui s'intéressent aux liens entre linguistique et littérature, dont Roland Barthes.}.
\par
C'est cette mécanique du jeu -- terme qui échappe d'ailleurs à toute définition chez Hesbois -- qui se déploie discrètement dans \textit{Borderline}, où les jeux de mots de tout acabit se côtoient et finissent inévitablement par être générateurs de sens.
Enfin, les jeux de mots laissent souvent échapper quelques vulgarités -- qui ne sont pas totalement étrangères à un certain ludisme --, qu'il s'agisse de jurons ou d'insultes, et qui marquent également le discours de la narratrice en tant que moyen d'exprimer ses émotions.

\subsection{Princesse Sissi et ses jeux de langage}
Dans la bouche de Sissi, plusieurs mots prennent rapidement une tournure enfantine: elle appelle sa grand-mère \guil{mémé}~(\textit{B}:~34, 35, 87, 88, 91, 92, 116, 122, 129, 130, 132), sa mère \guil{môman}~(\textit{B}:~37, 65-67, 91, 134), \guil{[s]on chien Ponpon, [s]on chat Magamarou}~(\textit{B}:~116) et son père biologique \guil{Papa Méchant}~(\textit{B}:~123-125).
Affection ou crainte, les sentiments mis en mots par ces surnoms ont en commun d'être tous exprimés de façon enfantine: par le redoublement de la syllabe dans \guil{mémé} et dans \guil{Ponpon}; par la prononciation fautive dans \guil{môman}; par un procédé qui n'est pas sans rappeler l'agglutination syllabique zazique dans \guil{Magamarou}; et par l'emploi de la majuscule -- ayant un effet d'agrandissement -- dans \guil{Papa Méchant}.
\par
Il en va de même lorsqu'elle explique à son amie Céline qu'elle risque de faire abuser d'elle si elle est \guil{placée} en famille d'accueil.
Réticente à employer un vocabulaire adulte, elle substitue d'abord aux véritables mots une terminologie enfantine: \guil{son machin \dots sa bitte}~(\textit{B}:~38); ce n'est qu'après avoir hésité qu'elle reformule plus expressément son propos en utilisant les mots d'adultes.
Toujours aussi peu à l'aise devant le vocabulaire sexuel explicite, elle réitère sa crainte -- que lui a inculquée sa grand-mère -- que des hommes la forcent à \guil{touche[r] leur pipi}~(\textit{B}:~56).
Cette hésitation est d'autant plus symbolique que Sissi connaît les véritables mots pour désigner les organes génitaux masculins: ainsi, lorsqu'on lui demande de dessiner un mouton, elle ne dessine pas autre chose, surtout pas \guil{un pilier, un pénis ou un Vinier}~(\textit{B}:~58); elle remarque également que sa maîtresse de classe mal à l'aise \guil{se frott[e] le nez, les sourcils, le soutien-gorge}~(\textit{B}:~58).
Les employant de la façon la plus banale dans des énumérations, Sissi expose sa connaissance des mots \guil{pénis} et \guil{soutien-gorge}; son hésitation ne peut donc être que fort signifiante, en ce sens que l'enfant \textit{connaît} mais n'\textit{ose} pas prononcer les véritables mots pour désigner les choses du sexe.
\par
Si cette hésitation devant le vocabulaire sexuel est révélatrice d'enfance -- malgré une connaissance des choses qui démontre une maturité plus grande --, c'est lorsqu'elle exprime combien son père biologique suscite en elle la crainte que Sissi se révèle linguistiquement comme une enfant-adulte: \guil{Papa Méchant, Papa Méchant, Papa Méchant, et qu'il me fait plus peur que le Bonhomme Sept-Heures, plus peur que King Kong, plus peur que le streptocoque de type A.}~(\textit{B}:~123)
Jonglant aisément entre la crainte enfantine suscitée par un monstre imaginaire et celle, strictement adulte, qu'induit la mention d'une maladie au nom rare, la narratrice adjoint indistinctement le jeune âge à la maturité dans un même syntagme.

\subsubsection{L'entrelacement métaphorique des âges}
Outil langagier de prédilection de Sissi, la métaphore, comparaison informelle construite sur une analogie entre deux termes, est par ailleurs la figure de style par excellence de l'enfant-adulte puisqu'elle permet de bâtir des ponts linguistiques entre l'enfance et l'âge adulte.
C'est donc par une mise en langage imagée que se rejoindront la maturité et la jeunesse dont peut faire preuve l'enfant-femme.
Comme de raison, la lucidité démontrée par Sissi concernera surtout sa situation familiale, dont elle sait qu'elle est toxique: \guil{Chez moi, la vie n'est pas un long fleuve tranquille, mais un lac artificiel rempli de BPC. Stagnant, le lac.}~(\textit{B}:~119)
Désireuse d'échapper à cet univers, elle élabore une métaphore filée pour décrire sa volonté d'avoir une famille normale et son incapacité à y arriver: \guil{Je construis des maisons avec des cubes Lego et j'y loge mes familles de bonshommes Fisher Price et ils parlent et ils parlent! $\left[ \dots \right]$ L'autre jour, j'ai voulu me construire une maison pour moi avec mes cubes Lego. Mais je n'ai pu me rendre qu'aux chevilles. Je n'avais pas assez de cubes. Je n'ai pas été capable de me construire une maison.}~(\textit{B}:~93)
Cet échec, beaucoup plus signifiant au sens symbolique qu'au sens strict, a pour sens profond le constat métaphorique par l'enfant de l'inadéquation de son environnement familial.
Voulant illustrer le peu de respect qu'elle a pour l'inertie de sa mère, Sissi la compare aux \guil{victimes} provenant de champs les plus divers:
\begin{quote}
  \begin{singlespace}
    \small
    Ma mère est une victime. Si elle était un animal, elle serait l'agneau qui se fait bouffer par un lion. Si elle était une Russe, elle resterait à Tchernobyl à deux pas de la centrale nucléaire. Dans un film d'horreur, elle serait la première à se faire couper la tête et les quatre membres, et à se faire sortir les intestins par l'énorme monstre vert et gluant. Elle est comme ça, ma mère, elle a autant de personnalité qu'une débarbouillette.~(\textit{B}:~120)
    \normalsize
  \end{singlespace}
\end{quote}
Devant ce constat hautement déceptif, Sissi adopte une attitude de rejet envers sa mère, dont la chambre est nommée \guil{chambre des lamentations}~(\textit{B}:~126) pour illustrer qu'elle pleure tout le temps, figure anti-maternelle à laquelle elle ne veut d'ailleurs pas ressembler: \guil{Niet. No. Non.}~(\textit{B}:~31)
Comme si l'usage de langues étrangères renforçait le refus.
Son rejet est d'ailleurs exprimé lui aussi de façon imagée: \guil{J'ai regardé ma mère comme ça, jusqu'à tant qu'on tourne dans le couloir et que je ne la voie plus. J'aurais souhaité que ce soit le tournant de ma vie. Que ce soit pour de bon.}~(\textit{B}:~68)
Manifestant le désir que ce coin de corridor représente davantage qu'un simple mur, Sissi transpose à l'échelle de sa vie l'idée que sa mère s'éloigne jusqu'à être complètement hors de sa vue.
\par
Elle est parfois ouvertement hostile à sa mère, osant même lui dire: \guil{Je t'haïs, Môman! Je t'haïs, Môman! Si tu savais combien je t'haïs! J'ai mal à mon nombril tellement que je t'haïs! J'ai mal à ma naissance tellement que je t'haïs! J'aurais dû me pendre avec le cordon ombilical dès que je suis sortie de ton ventre de folle! J'aurais dû!}~(\textit{B}:~67)
En plus de la répétition des mots \guil{je t'haïs}, la métaphore \guil{J'ai mal à mon nombril}, quoique immédiatement reformulée par la narratrice elle-même, s'adjoint efficacement à son regret de ne pas s'être \guil{pend[ue] avec le cordon ombilical} dès sa naissance pour exprimer toute la douleur qu'implique la relation de l'enfant à sa mère.
Ce lien de naissance est linguistiquement illustré par le nombril, trace du cordon qui reliait la mère au nourrisson.
Naïdza Leduc estime que ce \guil{véritable ressentiment} éprouvé par Sissi est une répercussion de la maladie de sa mère, de cette \guil{psychose en héritage\footcite[47-48]{Leduc2010}}.
C'est au nom de ce ressentiment que l'enfant manifeste sa colère envers sa mère et qu'elle lui lance des \guil{insectes}, que l'on peut aisément imaginer être des paroles blessantes: \guil{Ma mère, c'est ma cible. Je m'en sers pour sortir mes bibites. Je lance mes coquerelles sur elle, mes maringouins sur elle, mes araignées sur elle. Ma mère, c'est mon Insectarium de Montréal.}~(\textit{B}:~30)
Cette colère, expression d'une ranc\oe{}ur profonde, peut également être analysée en tant que manifestation langagière de l'autonomisation de la jeune fille.
Le langage est ici instrumentalisé sous le motif de la fuite et c'est par ses mots blessants que l'enfant s'éloigne irrémédiablement de sa mère.
\par
Enfant autonome mais aussi très adaptative, Sissi ne fait pas grand cas des événements dramatiques qui se déroulent autour d'elle.
C'est donc par l'humour qu'elle raconte la tentative de suicide de sa mère, représentant l'événement comme une \guil{imitation de Marilyn Monroe}~(\textit{B}:~31), fort réussie d'ailleurs:
\begin{quote}
  \begin{singlespace}
    \small
    J'ai appelé à l'urgence de l'hôpital Notre-Dame. $\left[ \dots \right]$ \textit{Bonjour, suis-je bien à l'hôpital Notre-Dame? Oui, bon. J'aurais besoin d'une ambulance, c'est que ma mère vient tout juste de faire une imitation de Marilyn Monroe. Et c'était très réussi. On a tous applaudi. Mais là, elle ne veut plus débarquer de la scène. Alors vite, envoyez-nous une ambulance ou une équipe de tournage $\left[ \dots \right]$}.~(\textit{B}:~31)
    \normalsize
  \end{singlespace}
\end{quote}
Par cette métaphore filée à propos du cinéma hollywoodien, Sissi dédramatise la situation et laisse croire qu'elle a été peu affectée par les événements.
L'effet est comique, dégonflant dans sa quasi-totalité la tension dramatique inhérente aux circonstances.
Par le ludisme, l'enfant parvient à nous montrer sa maîtrise d'elle-même, face à laquelle on n'a d'autre choix que de déduire un certain niveau de maturité.

\subsubsection{Le langage, un jeu de pauvres}
Il y a également dans l'écriture de Marie-Sissi Labrèche un ludisme que l'on pourrait qualifier d'\guil{immotivé}, en ce qu'il ne nous semble pas sémantiquement justifié.
C'est le jeu de langage pour lui-même, sorte de corollaire linguistique du slogan parnassien de \guil{l'Art pour l'art}.
La narratrice joue donc avec les mots -- dont elle détourne l'usage ou propose des associations farfelues -- afin de s'amuser, ainsi que les autres.
Cette forme d'humour, qui met en scène le langage, nous semble relever au point de vue formel de l'enfance la plus pure puisqu'il s'agit d'un \guil{jeu} auquel Marie-Sissi Labrèche accorde une véritable valeur ludique:
\begin{quote}
  \begin{singlespace}
    \small
    $\left[\textit{Extrait de la neuvième lettre adressée par l'auteure à sa mère:}\right]$
    Je pense aussi à nos jeux, à toutes les deux, quand j'avais quatre ans, puisqu'on était pauvres et que tu ne pouvais pas m'acheter tous les jouets de la terre, on s'amusait avec les mots, \guil{Finis toutes tes phrases en A, en E, en I.} Quand je réussissais, tu me félicitais à m'en rendre toute molle. À certains moments, je me dis que ma vocation d'écrivaine a pris racine dans ces jeux de pauvres\footcite[186]{Labreche2008a}.
    \normalsize
  \end{singlespace}
\end{quote}
Elle lance avec humour avoir \guil{le cerveau en compote de pommes}~(\textit{B}:~34) et décrit ainsi son amie Céline, celle qui a de la \guil{fidélité de caniche qui coule dans [l]es veines}~(\textit{B}:~37): \guil{Céline, mon amie Raisin Bran. Mon gage de régularité dans tout ce chaos.}~(\textit{B}:~36)
Revenant à la thématique des céréales matinales et de leur effet sur le transit intestinal, elle affirme son exaspération envers sa grand-mère avec le jeu sur le double sens de \guil{faire chier}: \guil{Elle, elle me fait assez chier avec ses Raisin Bran qu'elle me force à avaler tous les matins.}~(\textit{B}:~33)
La mémoire lui faisant défaut, elle rebaptise à de multiples reprises le centre sportif où sa mère va voir de la lutte:
\begin{quote}
  \begin{singlespace}
    \small
    Je me fais garder. Ma mère est partie voir un combat de lutte au centre Guy-Robillard ou Marcel-Robillard ou Jean-Marc-Robillard. Je ne sais pas trop. Je ne retiens pas ces choses-là. $\left[ \dots \right]$ Je retiens aussi que je ne dois pas trop faire chier ma grand-mère quand ma mère va voir des combats de lutte au centre Pierre, Jean, Jacques-Robillard $\left[ \dots \right]$.~(\textit{B}:~115-116)
    \normalsize
  \end{singlespace}
\end{quote}
\par
Bien plus que la mémoire qui lui fait défaut, auquel cas elle emploierait un nom erroné (mais un seul nom), Sissi démontre d'abord, par la première énumération, le peu d'importance qu'elle accorde au véritable nom de ce \guil{centre Roland-Robitaille}~(\textit{B}:~133).
C'est cependant à la seconde énumération que l'enfant fait preuve d'une véritable bouffonnerie: en le nommant le \guil{centre Pierre, Jean, Jacques-Robillard}, elle intègre dans son discours un concept d'onomastique générique.
Pierre, Jean, Jacques: ce sont les autres, à la façon de M. Dupont chez les français ou de Mr Smith pour les anglo-saxons.
Ainsi, en plus d'être linguistiquement générateur de banalité, son jeu a un aspect purement ludique, en ce sens qu'il \textit{amuse}.
\par
De façon tout aussi drôle, Sissi résume l'évolution de ses aptitudes en dessin en référant aux grands courants artistiques, dont elle détourne les noms en réduisant les mots à leur sens le plus commun:
\begin{quote}
  \begin{singlespace}
    \small
    Après ma période cubiste, où je dessinais les cubes stupides qu'on doit introduire dans un moule stupide, je suis devenue une impressionniste. À trois ans, j'étais une impressionniste qui impressionnait tout le monde avec ses dessins. $\left[ \dots \right]$ Aujourd'hui, je suis devenue une surréaliste. J'ai dépassé les bornes de l'audace artistique pour les deuxième année, et là, ils ont peur.~(\textit{B}:~57-58)
    \normalsize
  \end{singlespace}
\end{quote}
En réduisant le cubisme aux cubes ou l'impressionnisme au fait d'être impressionnante, Sissi joue avec les mots et se moque du code qui les sous-tend.
Ce ludisme non-utilitariste -- en ce sens qu'il n'a pas de vocation autre que l'amusement -- a tout de même pour effet d'affirmer l'appartenance de Sissi à la figure de l'enfant-femme puisque son maniement du langage implique forcément une maîtrise du code; elle ne pourrait détourner ainsi les noms des grands courants en peinture si elle n'en connaissait pas la définition ni la valeur symbolique, et c'est ce savoir adulte qui lui permet d'accomplir son jeu enfantin.
\par
D'une façon qui nous semble cette fois tout aussi ludique mais également délibérée, la jeune Sissi invente la perle qu'est le néologisme suivant:
\begin{quote}
  \begin{singlespace}
    \small
    Ma grand-mère est comme Dieu. Elle est partout à la fois. Elle est omniprésente. En fait, je devrais dire femniprésente, parce que tout ce qui est homme, elle le réduit en bouillie.~(\textit{B}:~32)
    \normalsize
  \end{singlespace}
\end{quote}
Le mot \guil{femniprésente} nouvellement inventé est porteur de sens puisqu'il résume à lui seul tout le dégoût des hommes qu'a la grand-mère.
La création par Sissi d'un néologisme aussi significatif révèle d'une part une bonne compréhension du monde, et d'autre part une tout aussi bonne maîtrise du langage, deux attributs valorisés chez l'enfant-femme.
Ainsi, même dans le jeu de langage, Sissi ne peut échapper à sa \guil{nature} d'enfant-femme et aux manifestations langagières de celle-ci.

\subsubsection{Sissi Labrèche, elle-même jeu de langage}
Si sa propension à faire des jeux de mots place indiscutablement la narratrice de \textit{Borderline} comme importante génératrice de ludisme langagier, nous estimons que c'est par son nom qu'elle se positionne définitivement comme reine du langage ludique.
D'abord par son prénom, lequel est en lui-même hautement connoté comme le souligne une travailleuse sociale: \guil{Elle me demande c'est quoi mon nom. Sissi. Elle trouve ça beau, elle dit que c'est comme l'impératrice, que c'est un nom de princesse. Tout le monde le dit.}~(\textit{B}:~96)
À force d'en entendre parler, l'enfant en vient à intégrer cette sémantisation de son propre prénom.
Habituée à recevoir des compliments de sa maîtresse d'école pour ses dessins, elle joue sur la connotation royale de son prénom: \guil{J'attendais qu'elle me lance ces phrases qui grossissent ma couronne invisible de petite princesse Sissi $\left[ \dots \right]$.}~(\textit{B}:~59)
Son prénom est également source de distraction pour ses camarades de classe en raison de certaines assonances qu'ils y trouvent: \guil{On rit de moi à cause de mon prénom: Sissi. On me crie: \textit{Heille Oui! Oui! Heille Pipi!} ou encore \textit{Heille Suce!} ça, c'est les cinquième année.}~(\textit{B}:~63)
\par
C'est cependant par son nom de famille que le poids sémantique de l'onomastique du personnage est décuplé:
\begin{quote}
  \begin{singlespace}
    \small
    On rit aussi de mon nom de famille dysfonctionnelle; mon nom de famille laissé, oublié par mon grand-père Labrèche mort d'un cancer du poumon à l'hôpital Notre-Dame, un dimanche après-midi ensoleillé. On m'appelle la Broche, la Brosse, la Poche. Mais ils ne comprennent pas. Ils ne rient pas pour la bonne affaire. En fait, mon nom, c'est le trou, c'est la brèche, c'est la fente de mon petit corps.~(\textit{B}:~63)
    \normalsize
  \end{singlespace}
\end{quote}
Outre les jeux de mots basés sur la sonorité ou sur le sens du mot \guil{brèche} -- les premiers étant amusants et les seconds étant davantage signifiants pour la narratrice devenue adulte que pour celle, enfant, qui nous intéresse --, c'est par le syntagme \guil{mon nom de famille dysfonctionnelle} que se sémantise toute l'onomastique du personnage de Sissi.
Le jeu de mot, sorte d'\guil{expression valise}, est d'autant plus lourd de sens que la grammaire nie la possibilité qu'il ne s'agisse pas d'une manifestation ludique: ainsi, l'accord féminin du mot \guil{dysfonctionnel} implique qu'il se rattache obligatoirement à la famille plutôt qu'au nom, scindant du même coup le syntagme entre les idées de \guil{nom de famille} et de \guil{famille dysfonctionnelle}.
La narratrice exprime de la sorte tout son ressentiment face à sa famille, négativité qu'elle rejette sur son patronyme.
De son prénom jusqu'à son nom de famille, Sissi Labrèche \textit{est} un jeu de langage!

\subsection{L'insulte et la grossièreté}
Lorsqu'il n'est pas ludique, le langage employé par Sissi concorde généralement avec le milieu dans lequel elle vit, en ce sens qu'il est familier, caractéristique du quartier populaire d'où elle vient.
Ainsi, ses jeux de mots sont parfois entrecoupés de jurons et d'insultes, lesquels sont systématiquement proférés en réaction face à la frustration de ne pas être écoutée ni comprise.

\subsubsection{La familiarité et les jurons}
Bien que Sissi sache \guil{jouer} avec les mots et qu'elle en fasse parfois un usage relativement savant, elle maîtrise également le langage populaire.
Elle sait notamment qu'elle \guil{ne doi[t] pas trop faire chier [s]a grand-mère}~(\textit{B}:~116).
Cet emploi de l'expression \guil{faire chier} est par ailleurs repris, qu'il concerne la grand-mère~(\textit{B}:~33), Cendrillon \guil{qui se fait chier par sa famille}~(\textit{B}:~129) et même lorsque l'enfant proclame que \guil{tous ceux qui [l]'ont fait chier}~(\textit{B}:~129) paieront.
Lorsque sa grand-mère dit des \guil{niaiseries}, la jeune fille est catégorique et glisse un \guil{Je m'en fous}~(\textit{B}:~88) bien senti.
Déclinant l'expression populaire sous diverses variations, elle s'en \guil{fout} ou s'en fiche: \guil{Bah, je m'en fiche de ce qu'elle me dit, ma grand-mère $\left[ \dots \right]$.}~(\textit{B}:~118)
Si elle emploie également des expressions populaires dépourvues de vulgarité -- qu'elle dise que \guil{ça ne devait pas être jojo}~(\textit{B}:~94) ou que quelque chose \guil{fait dur en titi!}~(\textit{B}:~132) --, il demeure que sa maîtrise du langage vernaculaire passe essentiellement par le langage \guil{malpoli}; ainsi déclarera-t-elle se \guil{sen[tir] moins seule dans [s]a merde}~(\textit{B}:~40) ou lancera-t-elle sans retenue l'exclamation \guil{Une grosse journée de cul!}~(\textit{B}:~36)
\par
Si Zazie ne semblait pas inquiéter outre-mesure \guil{les rangs de l'autorité\footcite[89]{Maurin2007}}, il en va de même pour Sissi, dont le langage est tout aussi inadéquat que l'était l'attitude verbale de la première.
Elle manie le juron comme d'autres enfants récitent des comptines et aucun adulte ne semble en faire de cas: \guil{Tout le long, je sacrais: \textit{Câlice-d'ostie-de-maudit-tabarnak-de-crisse-de-sacramant-de-calvaire!} $\left[ \dots \right]$ et je resacrais: \textit{Câlice-de-saint-ciboire-de-saint-calvaire-de-maudite-marde-de-crisse!}}~(\textit{B}:~66)
On ne l'écoute pas plus lorsqu'elle jure que lorsqu'elle s'exprime normalement: \guil{Elle me fait mal et je me plains. \textit{Mémé, j'ai mal!} Mais elle n'entend rien. \textit{Mémé, tu me fais bobo, câlice!} Même si j'ai sacré, elle ne m'entend toujours pas.}~(\textit{B}:~87)
Sissi fait ainsi un usage répété du juron, élément langagier négativement connoté, socialement inconvenant même pour l'adulte, d'autant plus malséant lorsque prononcé par un enfant.
\par
Si sa maîtrise d'un tel vocabulaire exprime d'une part son acclimatation et même son appartenance au monde adulte, elle est également symptomatique du peu d'autorité qu'ont sur elle les adultes, ces derniers ne s'inquiétant aucunement de la voir employer ces mots \guil{interdits}, trop occupés qu'ils sont à vaquer à leurs propres occupations: \guil{C'est à croire qu'elles pleurent pour passer le temps, câlice!}~(\textit{B}:~62)
Ainsi, le juron est pour Sissi bien plus qu'un simple défouloir; elle l'emploie afin d'attirer l'attention, voire d'\guil{alerter} (sans grand succès) l'autorité.
Il lui sert également à établir son positionnement en regard de l'adulte, cet Autre face auquel elle n'a qu'une volonté, soit celle d'\guil{être contre}.
Par exemple, à la travailleuse sociale qui remarque qu'elle est pâle et qui lui demande ce qu'elle a, Sissi réplique férocement: \guil{C'est parce que j'ai mangé du papier de toilette quand j'avais trois ans, câlice!}~(\textit{B}:~98)
Loin de se laisser amadouer, la jeune fille appose un juron à sa phrase, laquelle était déjà très \guil{mordante} en elle-même -- animal qui refuse d'être docile, elle grogne et montre les crocs.
\par
Le juron sert aussi de marque exclamative, que ça soit un \guil{Crisse!}~(\textit{B}:~67, 68) placé avant la phrase ou un \guil{câlice}~(\textit{B}:~128) qui la termine.
Sissi l'utilise pour manifester son mécontentement à l'effet qu'\guil{on marche trop vite, calvaire!}~(\textit{B}:~92) ou pour dire à sa grand-mère combien elle est méchante: \guil{Câlice! T'as pas le droit de dire ça! C'est toi qui es méchante! Méchante! Méchante! Aussi méchante que la belle-mère de Cendrillon! Aussi méchante que la bonne femme Olson dans \textit{La Petite Maison dans la prairie}!}~(\textit{B}:~127)
Si le juron appartient évidemment au vocabulaire des adultes, il nous semble que l'usage qu'en fait Sissi soit -- paradoxalement -- enfantin: en tant que marqueur de mécontentement, il démontre bien toute l'impuissance de l'enfant face à l'adulte.
Incompris, l'enfant exprime sa contrariété par la langue:
\begin{quote}
  \begin{singlespace}
    \small
    Câlice, elle devrait le savoir. J'ai de la misère à rester concentrée plus d'une minute sur la même affaire, elle devrait se le rappeler. Maudit! Là, couchée dans le lit de ma grand-mère, je suis super fâchée, je suis hyper en câlice! Je suis bleue, je suis bleu blanc rouge de colère. J'ai mon drapeau de la France sorti, alors c'est à mon tour de bougonner. D'ailleurs, si elle se repointe, la Mémé, je vais lui vomir mon camembert sur la tête.~(\textit{B}:~122)
    \normalsize
  \end{singlespace}
\end{quote}
Empêchée de regarder la télé par sa grand-mère qui \guil{n'arrête pas de chialer}~(\textit{B}:~121), la petite Sissi de cinq ans explose de colère et intime à sa grand-mère l'ordre de se taire:
\begin{quote}
  \begin{singlespace}
    \small
    - Ferme ta gueule! \\
    - Quoi?  \\
    - J'ai dit ferme ta gueule, vieille câlice! \\
    - Quoi? est-ce que j'ai bien compris? \\
    - J'ai dit ferme ta gueule, vieille câlice!~(\textit{B}:~123)
    \normalsize
  \end{singlespace}
\end{quote}
Par un amalgame entre l'insulte et la vulgarité, Sissi s'impose langagièrement afin de revendiquer son droit de parole et d'être écoutée; employant le langage comme une arme, elle exprime sa frustration et se place comme égale à l'adulte.
Dans la guerre communicationnelle qu'elle livre à sa grand-mère -- guerre qui se joue sur le front de la fonction phatique --, Sissi ne peut s'imposer par ses armes enfantines et doit user du juron, arme d'adulte.

\subsubsection{Des insultes à volonté}
Envers sa grand-mère qui ne l'écoute pas, Sissi n'est pas toujours tendre: \guil{Elle dit tout le temps plein de niaiseries, la vieille crisse!}~(\textit{B}:~56)
Elle la traite de \guil{vieille bitche}~(\textit{B}:~91) ou de \guil{vieille peau}~(\textit{B}:~89), la désigne comme étant \guil{la vieille maudite}~(\textit{B}:~121) ou simplement \guil{la maudite}~(\textit{B}:~89), la traite de \guil{vieille pas fine}~(\textit{B}:~123) ou de \guil{vieille débile}~(\textit{B}:~87); elle exprime d'ailleurs à nouveau combien elle la \guil{trouve tellement débile}~(\textit{B}:~96).
Autant que le juron, le propos insultant est souvent un moyen d'exprimer sa frustration de ne pas être écoutée: \guil{Elle ne me répond toujours pas, la vieille pas fine. Aussi bien parler à un plat de pâté chinois, câlice!}~(\textit{B}:~92)
Pour Sissi, l'insulte sert de moyen de se faire entendre ou, à défaut d'être écoutée, permet d'évacuer le mécontentement qui découle de cette impossibilité de communiquer.
\par
L'insulte est également dirigée vers les camardes de classe de Sissi.
Si le mécanisme d'expression de la frustration est similaire à celui qu'elle déploie envers sa grand-mère, la cause de son mécontentement est renouvelée: plutôt que de vivre une situation de communication ratée, Sissi subit l'incompréhension de la part des autres écoliers quant à sa réalité familiale peu commune.
Sa réaction est toutefois la même, qu'il s'agisse de mauvaise communication ou d'incompréhension: face à la frustration, elle attaque par l'insulte la plus crue.
Ainsi, elle traite les enfants traumatisés par son dessin des yeux de sa mère de \guil{peureux}~(\textit{B}:~57), exposant leur soi-disant \guil{faiblesse} devant la folie de sa mère, qui est pour elle une réalité du quotidien: \guil{Petites natures, va! Gang de brouteux de luzerne, va!}~(\textit{B}:~56)
Devant leur incompréhension de sa réalité à elle, Sissi réagit par un langage brutal: \guil{Je voudrais leur crier: \textit{Fermez-la, gang de chieux! Gang de petits crisses pas bons en dessin. $\left[ \dots \right]$ Allez donc tous chier! Allez donc tous vous faire enculer par les mouches!}}~(\textit{B}:~70)
À ses yeux, autant les adultes avec leurs précautions que les autres enfants sont une \guil{gang de chieux}~(\textit{B}:~61) qu'elle rejette.
Même sa mère et son beau-père n'y échappent pas: \guil{C'est moi qui serai censée tenir les anneaux de mariage, c'est ce qu'ils m'auront dit, les morons, sauf que lorsque ce sera le temps de tenir les maudits anneaux de mariage, ils me les enlèveront. Des morons, que je dis.}~(\textit{B}:~116)
Tous y passent: de sa \guil{mère qui pleurait comme une vraie nulle}~(\textit{B}:~64) aux enfants de l'aile psychiatrique de l'hôpital.
\guil{Bande de petits cons}, dira-t-elle~(\textit{B}:~99).
Sissi n'oublie personne et chacun reçoit son épithète insultante.

\section{À la guerre comme à la guerre: l'arme langagière}
Point de friction par excellence entre l'enfance et l'âge adulte, le langage démontre, autant chez Zazie que chez Sissi, toutes les difficultés existentielles de l'enfant-femme.
Chacune à sa façon, les deux jeunes filles manient le langage comme une arme: d'une part afin de se définir et d'autre part dans une tentative de combattre les adultes.
Toute cette mécanique fait indiscutablement du langage un vecteur majeur d'ambiguïté quant à l'âge du personnage.
C'est toutefois sur le plan du combat contre les adultes que Zazie et Sissi se distinguent langagièrement l'une de l'autre: d'abord parce que leurs combats ne se livrent pas sur le même front, l'une menant le combat de toute une génération et l'autre menant une bataille individuelle; ensuite parce que leurs quêtes n'ont pas la même issue, l'une médusant tous les adultes dans leur conception de la jeunesse, l'autre peinant simplement à être entendue.

\subsection{Le langage comme ultime consécration de la valeur contre-mythique de Zazie}
Chez Zazie, le langage s'emploie comme une arme de précision afin de dégonfler toute la mythologie entourant la jeunesse: il se veut aiguille affutée, suffisamment précise pour atteindre et crever le mythe de la \textit{doxa}.
Les attaques de l'enfant se portent surtout contre les Grandes Idées avec une majuscule, celles des Grandes Personnes (elles aussi avec la majuscule).
\guil{Attei[gnant] la pensée ou l'absence de pensée\footcite[38]{Hoja-Lacki1963}}, ce sont des attaques de front, sans complexes et sournoises.
Comme l'a démontré Barthes, Zazie est incisive: elle maîtrise bien les concepts linguistiques et son découpage se fait au scalpel, attaquant le métalangage en portant le langage-objet comme étendard.
Le succès de son entreprise est total et l'effet est grand: sous les traits de Zazie, l'enfant-femme décontenance les adultes par son usage de la langue et acquiert, selon Roland Barthes, le statut de contre-mythe.
Elle ébranle et fait vaciller tout l'édifice des Idées, en se posant comme plus philosophe que cette pseudo-philosophie émanant des adultes.
Ce succès -- dans son usage de la langue à des fins de guerre contre les adultes -- consacre par le fait même son ascendant sur toute la typologie de l'enfant-femme: précédemment désignée comme \guil{enfant-adulte absolu}, Zazie renouvelle cette consécration lorsqu'il s'agit du langage.

\subsection{Le langage à la fois jouet et arme chez Sissi}
Chez Sissi, la langue se veut à la fois jouet et arme: jouet par ce ludisme omniprésent et arme par cette multiplication des tentatives d'atteindre l'adulte par l'usage du langage.
Si le premier objectif est définitivement atteint -- Sissi fait rire et \textit{se} fait rire --, le second usage de la langue obtient pour sa part un succès mitigé puisque cet Autre, qu'elle tente d'atteindre par ses insultes et son vocabulaire vulgaire, est hors de portée; il ne l'écoute pas et n'est donc pas dérangé par elle et par ses paroles.
Pour reprendre la métaphore guerrière, c'est comme un coup d'épée dans l'eau.
\par
Ainsi, nous revenons au constat déceptif émis par Vurm à propos des enfants ducharmiens:
\begin{quote}
  \begin{singlespace}
    \small
    Mais la déception et la chute arrivent surtout avec l'impuissance des mots. Car l'éloquence semble rompue très souvent lorsque les personnages s'aperçoivent que le langage ne suffit plus dans leur lutte contre le monde. À ce moment même, les enfants perdent leur double, symboliquement et littéralement, malgré les injures, tout juste créées pour être lancées aux adultes, malgré leurs créations lexicales, qui sont leur refuge, ils sont voués à l'échec, car les mots ne peuvent plus rien pour éviter leur isolement dans ce monde fou $\left[ \dots \right]$\footcite[101]{Vurm2014}.
    \normalsize
  \end{singlespace}
\end{quote}
Loin de remédier aux tensions entre l'enfance et le monde des adultes, le langage, en ses composantes communicationnelles, constitue pour Sissi un autre vecteur de souffrance.
Isolée malgré elle, incapable d'instaurer un dialogue convenable -- et de se faire entendre et comprendre --, elle trouve remède dans cette exacerbation de la fonction poétique que représente la \textit{fonction ludique}.
\par
Faute de mieux, elle s'amuse avec les mots.
C'est sur ce point que son rapport au langage rejoint à notre avis le caractère du personnage, nous permettant d'émettre une hypothèse audacieuse: le ludisme langagier ne pourrait-il pas être une application linguistique du concept de résilience?
Ou le langage, dans sa dynamique poétique, ne serait-il pas, pour le type d'enfant-femme qu'est Sissi, un vecteur d'apaisement?
Nous estimons que oui en nous permettant d'affirmer que la capacité d'adaptation et de résilience chez Sissi est définitivement tributaire de son rapport ludique au langage.
