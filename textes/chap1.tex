\chapter{Le personnage d'enfant-femme}

\begin{flushright}
                                    \begin{singlespace}
                                    \epigraph{
                                    \guill{What--is--this?} he
                                    said at last.\\
                                    \guill{This is a child! $\left[ \dots \right]$}
                                    %Haigha replied eagerly, coming in front of Alice to introduce her, and spreading out both of his hands towards her in an Anglo-Saxon attitude.
                                    \guill{ We only found it
                                    to-day. It's as large as life, and twice as natural!} \\
                                    \guill{I always thought
                                    they were fabulous monsters!} said the Unicorn. \guill{Is it alive?}}
                                    \par
                                    Lewis Carroll, \textit{Through the looking-glass}\footcite[189]{Carroll2012}
                                    \end{singlespace}
\end{flushright}


Constituant une relecture du personnage d'Alice Liddell, née de la plume de
Lewis Carroll en 1865, nous estimons que le personnage d'enfant-femme trouve ses
racines les plus profondes dans les grands visages de l'enfance tels qu'élaborés
par la littérature du XIXe siècle, par ailleurs très riche en personnages
d'enfants. Afin de bien établir notre propre typologie, nous reviendrons dans un
premier temps aux grands types d'enfants dans la littérature française du XIXe
siècle en reprenant la classification de Marina Bethlenfalvay. Une fois ces
ancêtres de l'enfant moderne bien décrits, nous nous attacherons en second lieu à
élaborer la typologie d'un personnage qui, à notre connaissance, n'a pas encore
été défini: l'enfant-femme.

\section{L'historique de la représentation de l'enfance en littérature} Si la
représentation de l'enfant dans la littérature a grandement varié selon les
époques, les courants, de même que les cultures, il nous apparaît être devenu un
personnage à part entière sous la plume des auteurs pré-romantiques et
romantiques. Littérairement mis au monde par Jean-Jacques Rousseau sous le nom
d'Émile, l'enfant envahit la littérature française à l'apogée du courant
romantique, lequel \guil{instaur[e] un véritable culte de
l'enfance\footcite[90]{Vurm2014}}.

\subsection{Les visages de l'enfant: la typologie de Bethlenfalvay} Selon la
classification de base de l'enfant en littérature, deux figures d'enfant se
voient opposées du tout au tout. Le premier est doux et sentimental, \guil{la
figure de l'enfant permet[ant] à l'auteur de se remémorer avec nostalgie
l'enfance perdue\footcite[90]{Vurm2014}}; le second est, au contraire,
\guil{inversé, \guill{paradoxal} -- le plus souvent un enfant avancé sur son âge
biologique, génie et révolté\footcite[90]{Vurm2014}.} Cette classification de
base est toutefois raffinée par la typologie esquissée par Marina Bethlenfalvay à
propos de l'enfant dans la littérature du XIX\up{e} siècle, dans laquelle elle
distingue trois grandes figures d'enfants.

\subsubsection{L'enfant venu d'ailleurs: l'enfant-ange de la poésie romantique}
Radicalement différent de l'adulte, essentiellement meilleur, l'enfant venu
d'ailleurs a un physique idéalisé et correspond à la métaphore de l'enfant-ange,
fréquente en poésie\footcite[20]{Bethlenfalvay1979}. Ainsi, dans ce que
Bethlenfalvay appelle le \guil{complexe de l'Enfant Romantique}, l'enfant est vu
comme un messager entre Dieu et les hommes\footcite[21-22]{Bethlenfalvay1979}.
Cet enfant, représenté surtout dans l'\oe{}uvre des poètes romantiques\footcite[19]{Bethlenfalvay1979}, suggère un monde idéal en référant à la joie édénique de la scène pastorale.
Chargé des rêves brisés du monde adulte, l'enfant appelle autant la métaphore de l'île exotique que la dichotomie entre le berceau et le tombeau\footcite[24-25, 33]{Bethlenfalvay1979}.
Il se retrouve surtout dans la poésie de Victor Hugo, de Marceline Desbordes-Valmore et celle d'Alphonse de Lamartine\footcite[19, 34]{Bethlenfalvay1979}.
En tant que personnage, c'est au \textit{Petit Prince} de Saint-Exupéry qu'il fait penser tandis que \textit{Le Grand Meaulnes} d'Alain-Fournier représente sa survivance au XX\up{e} siècle\footcite[44-47]{Bethlenfalvay1979}.

\subsubsection{L'enfant victime: le petit martyr du roman réaliste}
Bien que toujours aussi pur, innocent et aimant que l'enfant venu d'ailleurs, l'enfant victime s'en distingue par la façon dont il est traité par l'auteur, lequel ne le protège pas de la misère du monde et s'attache intentionnellement à sa misérable condition. C'est donc un enfant souffrant, victime ou martyr, souffre-douleur plutôt qu'ange ou messie\footcite[53]{Bethlenfalvay1979}.
Enfant par excellence du roman réaliste, il a un rôle didactique et moral, voire polémique, dont l'objectif est de susciter pitié et indignation relativement à l'humanité souffrante et à l'injustice sociale qu'il symbolise\footcite[53, 64, 85]{Bethlenfalvay1979}.
\par
C'est bien souvent l'enfant d'une prostituée, l'enfant priant auprès de sa mère morte, celui mis en pension, abandonné, délaissé ou maltraité. Il correspond aux personnages hugoliens de Cosette dans \textit{Les Misérables} et de Gwynplaine dans \textit{L'Homme qui rit}, à ceux mis en scène dans \textit{Jack} d'Alphonse Daudet et \textit{Poil de Carotte} de Jules Renard\footcite[56, 71, 76-77]{Bethlenfalvay1979}.
Cependant, le personnage le plus représentatif de l'enfant victime se trouve du côté de la littérature anglaise: il s'agit du protagoniste d'\textit{Oliver Twist} de Charles Dickens\footcite[110]{Bethlenfalvay1979}.

\subsubsection{L'enfant du monde: le \guil{bon sauvage} du roman naturaliste}
Bien enraciné dans la vie terrestre, l'enfant du monde est robuste et vif. Optimiste et débrouillard, il se tourne vers l'avenir et s'adapte à sa vie de misère.
S'il lui arrive de souffrir, il n'est jamais victimisé, subissant un sort ni pire ni meilleur que celui des grandes personnes qui l'entourent\footcite[85-86]{Bethlenfalvay1979}.
Découlant de la foi dans le progrès, du déterminisme et de la glorification de la force vitale, cette vision place l'enfant comme un être originel, pré-civilisé voire animal. Cet enfant, généralement rattaché à un cadre rustique, incarne le mythe du \guil{bon sauvage} duquel est évacuée la dimension mystique\footcite[85, 87-88, 92]{Bethlenfalvay1979}.
\par
Correspondant à une promesse d'avenir optimiste quant à la réalisation de l'harmonie entre l'homme et l'univers\footcite[92]{Bethlenfalvay1979}, l'enfant du monde s'épanouit avant tout dans l'\oe{}uvre d'Émile Zola, tous les enfants des \textit{Rougon-Macquart}, surtout Jeanlin, semblant correspondre à ce type\footcite[85-89]{Bethlenfalvay1979}.
C'est aussi le Gavroche des \textit{Misérables} chez Victor Hugo, lequel \guil{ne se sentait jamais si bien que dans la rue [puisque] le pavé lui était moins dur que le c\oe{}ur de sa mère\footcite[90-91]{Bethlenfalvay1979}.}
\par
Ce personnage mène, au siècle suivant, au roman d'apprentissage tel \textit{Jean-Christophe} de Romain Rolland\footcite[112]{Bethlenfalvay1979}.
La Zazie de Raymond Queneau est également mentionnée comme faisant partie de ces \guil{petits personnages que l'imagination ne trouble guère, et qui prolongent à leur manière la lignée du gamin, s'adaptant à tous les milieux, sans perdre son équilibre} en tant qu'\guil{enfant terrible et irrépressible}, moins désorientée que les adultes par tous les événements qu'elle traverse\footcite[117]{Bethlenfalvay1979}.

\subsubsection{L'enfant des autres: l'enfant aliéné par la modernité}
Dans sa typologie des enfants du XIX\up{e} siècle, Marina Bethlenfalvay déborde sur le XX\up{e} siècle pour dresser un portrait de l'enfant de la modernité.
Elle décrit un enfant faible et vulnérable, dépendant du \guil{milieu humain} auquel il est soumis et représenté surtout chez Marcel Proust, chez Jean-Paul Sartre et chez Nathalie Sarraute\footcite[128-129]{Bethlenfalvay1979}.
Voyant en lui un enfant hautement \guil{sociologisé}, voire aliéné,
Bethlenfalvay postule que les sentiments d'infériorité, d'impuissance et d'abandon sont la condition existentielle de cette figure d'enfant\footcite[128, 134-135, 138]{Bethlenfalvay1979}.


\subsection{Un enfant multiple: l'éclatement des modèles}
Dès l'évocation par Bethlenfalvay de cet \guil{enfant des autres} préfigurant le XX\up{e} siècle, il est évident que la typologie du personnage d'enfant n'ira qu'en se complexifiant.
Ainsi, avec la modernité surviennent des personnages qui débordent des modèles jusqu'alors conçus.
C'est de ce constat d'inadéquation des \textit{types} préexistants que naîtront notamment l'enfant révolté puis l'enfant-adulte, lesquels nous amènent au plus près de cette enfant-femme qui nous intéresse.

\subsubsection{La nécessité d'un nouveau modèle: l'enfant révolté}
Cherchant à comprendre l'enfant chez l'auteur québécois Réjean Ducharme, Petr Vurm se tourne d'abord vers l'ouvrage de Bethlenfalvay, en portant une attention particulière à l'épilogue et en explicitant certaines nouvelles figures propres au XX\up{e} siècle.
D'une part, il y a l'enfant-génie ou enfant-révolté, qu'il associe aux \textit{Mots} et à \guil{L'enfance d'un chef} de Jean-Paul Sartre et à la boutade \guil{Familles, je vous hais!} lancée par André Gide dans \textit{Les nourritures terrestres}, et qui est un enfant qui remet constamment en cause les valeurs du monde adulte\footcite[93]{Vurm2014}.
D'autre part, il y a l'enfant nouveau, \guil{repérable chez les surréalistes, qui confèrent à l'enfant le pouvoir de l'imagination illimitée et d'un regard nouveau et insolite, doué d'une immense potentialité créatrice, voisine de celle de l'inconscient}, incluant l'évasion vers des mondes imaginaires et dont l'exemple donné est \textit{Les enfants terribles} de Jean Cocteau\footcite[93]{Vurm2014}.
Suite à cette analyse, Vurm propose de rajouter à la typologie de Bethlenfalvay la figure de l'\guil{enfant révolté}, à la fois aux antipodes et \guil{corollaire} de l'enfant qui subit passivement ses peines\footcite[94]{Vurm2014}, figure qui l'aidera à poser un regard critique sur l'enfant dans l'\oe{}uvre de Ducharme.

\subsubsection{L'ambiguïté identitaire: l'enfant-adulte de Vurm}
Parlant de l'enfant ducharmien, Vurm exprime la difficulté de son identité, notamment \guil{le mélange d'incertitudes concernant son âge physique, mental ou émotionnel et linguistique\footcite[97]{Vurm2014}}.
À ses yeux, l'indétermination de l'âge linguistique est la plus intéressante au plan littéraire puisqu'elle amène une confusion entre le discours des adultes et celui des enfants.
Ainsi, \guil{l'ambiguïté qui réside au c\oe{}ur des romans sur l'enfance, consiste dans l'impossibilité de dire: \guill{Je suis un enfant} ou \guill{Je suis un adulte.} }, ambiguïté qui pousse le critique à qualifier le personnage d'\guil{enfant-adulte}\footcite[99-100]{Vurm2014}.
Par la création de ce nouveau type de personnage, Vurm traduit en un néologisme toute la complexité que représente l'enfant ducharmien et qui réside dans la question identitaire de l'appartenance à un monde d'âge, que ce soit adulte ou enfant:
\begin{quote}
  \begin{singlespace}
    \small
    Malgré l'impossibilité de dire qui est celle de l'enfant, l'écrivain travaille le paradoxe littéraire du dialogue entre les mondes enfantin et adulte. Cela lui permet de situer l'enfant littéraire à la croisée de ces deux mondes et de bénéficier de ce que ces deux mondes offrent: la culture et le jeu sophistiqué de la subversion chez l'adulte, la spontanéité et le jeu chez l'enfant\footcite[103]{Vurm2014}.
    \normalsize
  \end{singlespace}
\end{quote}
Le personnage d'\textit{enfant-adulte} tel qu'élaboré par Vurm apparaît crucial pour nos recherches puisque c'est essentiellement à partir de ce dernier que nous souhaitons construire et particulariser notre propre typologie d'\textit{enfant-femme}. Nous reprendrons donc aux fins de la théorie cette idée de dialogue entre l'enfance et l'âge adulte.

\section{Le personnage d'enfant-femme: notre typologie}
Puisque le concept d'enfant-femme est de notre création, nous nous devons d'abord de le définir.
Nous reviendrons à cette fin à ce qu'on appelle le \guil{mythe} d'Alice, lequel jette les bases de notre typologie.
Puis, nous reprendrons ce que Petr Vurm a défini comme étant l'enfant-adulte et nous nous emploierons à élaborer plus profondément cette forme de cohabitation de l'enfant et de l'adulte.
Enfin, nous expliquerons en quoi l'enfant-femme est différente de la femme-enfant, principalement sur le plan des intentions en ce qui a trait à la sexualité.

\subsection{Un retour au mythe d'Alice: l'enfant-femme comme \textit{topos} moderne?}
Tel qu'évoqué dans notre titre, l'enfant-femme comme \textit{type} littéraire se veut une façon de revisiter la figure d'Alice de Lewis Carroll.
Nous affirmons à cette fin qu'Alice, tant comme personnage que par le récit de ses aventures dans le terrier du lapin, constitue un mythe littéraire moderne, voire que ce personnage représente une \guil{nouvelle mythologie de l'enfance\footcite[140]{Jousni2005}}.
Nous avançons même qu'au-delà de ce mythe, l'enfant-femme qui en découle est à son tour une figure récurrente de la littérature, à savoir un \textit{topos} contemporain.

\subsubsection{Élever Alice au rang d'une mythologie}
Comme l'a dit Marie-Hélène Inglin-Routisseau au terme de sa thèse de doctorat, la figure d'Alice \guil{gouverne incontestablement l'imaginaire littéraire du 20\up{ième} siècle}, arrivant en France avec le surréalisme et disparaissant en même temps que les auteurs de ce courant\footcite[329]{Inglin-Routisseau2006}. De la même façon, Jean-Jacques Lecercle affirme qu'Alice survit au temps \guil{parce qu'elle est devenue mythe, parce qu'elle incarne l'archétype de ce personnage éternel, la petite fille\footcite[7]{Lecercle1998}.}
Alice est donc bien ancrée dans l'imaginaire littéraire, même en France: \guil{Cette présence poétique indéfiniment prolongée l'érige au rang de mythe\footcite[330]{Inglin-Routisseau2006}.}
\par
Si la reprise de la figure d'Alice s'est quelque peu atténuée après les années 1960 et 1970 dans la littérature au sens strict, tel que le prétend Inglin-Routisseau, nous estimons qu'elle est toujours aussi présente dans la culture populaire et dans l'imaginaire social,
notamment au cinéma\footnote{Il n'y a qu'à penser à l'adaptation de Tim Burton, \textit{Alice au pays des merveilles} (2010) et à sa suite de James Bobin, \textit{Alice de l'autre côté du miroir} (2016). Du côté québécois, \textit{L'Odyssée d'Alice Tremblay} (2002) se veut une parodie du conte.},
à la télévision\footnote{Outre les adaptations plus ou moins fidèles à l'histoire originale, notamment la série \textit{Once Upon A Time In Wonderland} (2013-2014) du Studio ABC, il y a également la série \textit{Lost} (2004-2010), également de la chaîne ABC, qui s'inspire librement d'Alice, notamment dans les titres de certains épisodes.}
et dans la chanson populaire\footnote{En plus des chansons \textit{I Am the Walrus} (1967) des Beatles et \textit{White Rabbit} (1967) de Jefferson Airplane, notons les plus récentes \textit{Tweedle Dee \& Tweedle Dum} (2001) de Bob Dylan, \textit{Sunshine} (2001) d'Aerosmith, \textit{What You Waiting For?} (2004) de Gwen Stefani, \textit{Wonderland} (2014) de Taylor Swift et le vidéoclip de \textit{Brick By Boring Brick} (2009) du groupe Paramore. Il y a également les albums \textit{Cheshire Cat} (1994) de Blink-182, \textit{Alice and June} (2005) d'Indochine (album duquel est tiré la chanson du même titre), \textit{Le Cheshire Cat et moi} (2009) de Nolwenn Leroy ainsi que \textit{Eat Me, Drink Me} (2007) de Marylin Manson, dont est tirée une chanson intitulée \textit{Are You The Rabbit?}.},
sans oublier la paralittérature\footnote{Au Québec, le roman fantastique \textit{Aliss} (2000) de Patrick Sénécal est inspiré du personnage carrollien. Il y a également de nombreux mangas japonais légèrement ou fortement inspirés des aventures d'Alice.} ainsi que divers jeux vidéo.
En somme, bien que le personnage d'Alice telle qu'exploité en France par les surréalistes se soit dissipé, sa mémoire reste profondément vivante dans la culture populaire occidentale.

\subsubsection{Les autres Alice de la modernité}
Après la traduction par Louis Aragon de \textit{La Chasse au Snark} en 1929, Alice est introduite en France dans l'\oe{}uvre des surréalistes où elle devient \guil{poupée désarticulée, fragmentée, hachée\footcite[175]{Inglin-Routisseau2006}}, notamment femme-homme ou femme-mannequin et même \guil{femmes-enfants, femmes-fleurs, femmes-étoiles, femmes-flammes, flots de la mer, grandes vagues de l'amour et du rêve} selon Eluard\footcite[176]{Inglin-Routisseau2006}.
Redevenue poupée pour le poète, elle est, toujours pour Eluard, \guil{objet manipulable}, captive et fantasmatique\footcite[285]{Inglin-Routisseau2006}. \guil{Enfant brisée, misérable, et domptée}, Alice-poupée ouvre la voie à la nymphette, à la \textit{Lolita} de Vladimir Nabokov, par le biais du fantasme de l'inceste\footcite[287]{Inglin-Routisseau2006}.
C'est une \guil{Alice au pays \guill{des tristes merveilles}\footnote{\mancite \cite{Parrot1944} cité dans \cite[287]{Inglin-Routisseau2006}}}.
Véritable \textit{incarnation} de l'inconscient, cette dernière subit un \guil{morcellement sadique [de son] corps} qui mènera à la femme-enfant, icône surréaliste et \guil{ultime avatar d'Alice\footcite[177]{Inglin-Routisseau2006}}.
\par
Par la suite, Alice est reprise par de nombreux auteurs, souvent via le pastiche ou un rappel du personnage\footnote{Notamment dans la parodie \textit{Alice en France} (1945) de Raymond Queneau, dans le livre illustré \textit{De l'autre côté de la page. Alice au pays des lettres} (1968) de Roland Topor, dans \textit{Les Enfants de l'été} (1978) de Robert Sabatier où plusieurs personnages connus de la littérature jeunesse se rencontrent, dans plusieurs poèmes de Paul Gilson, entre autres dans \guil{Alerte aux rêves} du recueil \textit{À la vie à l'amour} (1943) où Alice croise Peter Pan.}.
Cependant, cette reprise de l'enfant carrollien est limitée en ce sens qu'elle ne rend pas bien compte de l'entière postérité du personnage: \guil{L'exercice de style, pour être brillant et ingénieux, oblitère en effet la spécificité de l'\oe{}uvre originale\footcite[281]{Inglin-Routisseau2006}.}


%Année 1959: parution de \textit{Zazie dans le métro}, traduction française de \textit{Lolita} (Nabokov), \textit{Laissez-moi tranquille} de Lise Deharme: \guil{Les héroïnes de ces deux romans, Lolita et Carole, ont à peu près l'âge de Zazie, et ne sont pas plus qu'elle tourmentées par les scrupules d'une conscience morale exigeante ni par le souci des bienséances. Mais la comparaison s'arrête là. [...] Ces rapprochements nous indiquent pourtant que, d'une certaine manière, l'heure de telles héroïnes avait sonné.\footcite[15]{Bigot1994}}



\subsection{Une déclinaison féminine de l'enfant-adulte}
Le personnage d'enfant-femme étant d'abord la déclinaison féminine de l'enfant-adulte, lequel a été brièvement décrit par Vurm, nous estimons utile de reprendre son travail aux fins de notre typologie, plus particulièrement son idée d'ambiguïté identitaire qui est au c\oe{}ur du concept même d'\textit{enfant-adulte}.

\subsubsection{\guil{Les âges} du personnage: quand âge biologique et âge social ne concordent pas}
À la base même du personnage d'enfant-adulte et de son ambiguïté identitaire se trouve une discordance entre \textit{les} âges d'un personnage.
Ainsi, la typologie que nous esquissons se fonde sur une discordance entre l'âge biologique et l'âge social d'un personnage: l'enfant-femme se devra d'être biologiquement une enfant, en ce sens qu'elle aura obligatoirement la physionomie d'une gamine, tout en adoptant des comportements qui rappellent inévitablement la maturité d'une adulte.
C'est le concept de dédoublement des âges d'un même personnage qui nous permet d'élaborer le concept d'enfant-femme.

\subsubsection{Tout à la fois ou ni l'un ni l'autre}
D'abord, nous tenons à poser une prémisse essentielle à la construction de notre modèle selon laquelle l'enfant-femme, ou plutôt l'enfant-adulte, n'est ni un enfant, ni un adulte, tout en réunissant à la fois des caractéristiques de l'un et de l'autre.
Si l'enfant-femme n'est ni une adulte, ni tout à fait une enfant, il est tout aussi clair pour nous qu'elle ne correspond pas à la \guil{médiane} de ces deux âges, c'est-à-dire à l'adolescente. Personnage combinant des caractéristiques appartenant aux deux extrêmes, elle n'est pas la figure de l'entre-deux, le personnage mitoyen, car nous tenons à poser comme élément essentiel de son identité cet état du ni l'un ni l'autre.
\par
L'enfant-femme est également un personnage \guil{contre} ou \textit{anti}, en ce sens qu'il implique un rejet du monde adulte: \guil{La révolte toujours placée entre les mains d'Alice, incarne à présent la rébellion contre la morale et les valeurs adultes. Le non-sens ouvre les portes du royaume de l'enfance, et l'enfance celles de la liberté\footcite[169]{Inglin-Routisseau2006}.}
Ce positionnement nous permet d'affirmer que l'enfant-femme est un personnage qui se construit par rapport à une altérité, laquelle correspond à l'adulte, ou plus précisément à l'\textit{adulte-adulte}, c'est-à-dire à celui dont à la fois les âges biologique et comportemental sont ceux d'un adulte. Ainsi, l'enfant-adulte se positionne par rapport à un \guil{eux}, dans une attitude de détachement par rapport à ces \guil{fonctionnaires} que représentent pour lui les adultes\footcite[100]{Vurm2014}, s'établissant hors de la portée de leur autorité.

\subsubsection{Un enfant adaptatif: l'autonomisation comme outil de survie}
Un peu à la façon de l'\guil{enfant du monde} qu'a décrit Marina Bethlenfalvay en référant au personnage du \guil{bon sauvage} dans le roman naturaliste, nous estimons que l'enfant-femme est un enfant qui s'adapte bien aux aléas de la vie.
Alors que l'enfant du monde du XIX\up{e} siècle est décrit comme acceptant son sort, nous croyons bon de nuancer cette capacité d'\guil{acceptation} pour en faire plutôt une qualité d'\guil{adaptation}: l'enfant-femme est à nos yeux un enfant extrêmement adaptatif qui, sans se complaire dans ses difficultés, s'efforcera de les surmonter ou de \guil{faire avec}.
\par
À l'origine de cette haute capacité d'adaptation chez l'enfant, nous retrouverons toujours une situation personnelle complexe.
Si l'on revient au mythe d'Alice, l'acclimatation concerne surtout le \textit{nonsense} des personnages et de l'environnement avec qui elle doit interagir suite à son entrée dans le terrier:
\begin{quote}
  \begin{singlespace}
    \small
    \textit{On the one hand, Alice encounters characters, in relation to whom she is like a child in the dogmatic and perplexing adult world, and on the other hand, the characters she meets behave like self-centred, stubborn children in pursuance of their enthusiasm for childish word playing\footnote{\guil{D'un côté, Alice rencontre des personnages face auxquels elle est comme une enfant dans le monde dogmatique et déconcertant des adultes, et d'un autre côté, les personnages qu'elle rencontre se comportent comme des enfants têtus et égoïstes par leur enthousiasme pour les jeux de mots enfantins.}~(notre traduction) dans \mancite \cite[39]{Katajamaki2005}.}.}
    \normalsize
  \end{singlespace}
\end{quote}
Dans le cas de l'enfant-femme, nous pensons davantage à un enfant laissé à lui-même et qui pallie cette absence d'autorité parentale en s'autonomisant à sa façon, soit en vieillissant prématurément.
Dépourvu de repères dans le monde adulte, cet enfant adapte son comportement à sa condition et se mue en enfant-adulte afin de survivre, développant des mécanismes de ruse et même de rejet de l'autorité dont il a été privé.

\subsection{Un enfant de langage: la parole comme condition d'existence}
Outre les caractéristiques psychologiques et comportementales que nous avons déjà dégagées, le langage constitue chez l'enfant-femme un trait hautement distinctif.
Ainsi, nous ne croyons pas faire erreur en affirmant que l'enfant-femme est un personnage de langage en ce sens qu'elle est déterminée par son usage du verbe, trait qu'elle hérite d'ailleurs de son ancêtre anglaise Alice, chez qui on voit \guil{pour la première fois dans la littérature, l'affirmation d'une véritable inscription de l'enfant dans l'ordre du langage, un enfant qui par là accède au statut d'être singulier et à part entière\footcite[144]{Jousni2005}}.

\subsubsection{L'Alice française, subversive plus qu'onirique}
Bien que l'écriture de Lewis Carroll ait inspiré maints auteurs français au XX\up{e} siècle, il semblerait que certains aspects de l'\oe{}uvre aient été préférés à d'autres par les surréalistes: \guil{les aventures d'\textit{Alice} en France intéressent plus par ce qu'elles comportent de subversif que par ce qu'elles véhiculent d'onirique\footcite[167]{Inglin-Routisseau2006}}.
C'est donc davantage le \textit{nonsense} anglais, que nous apparentons au ludisme langagier, qui place Alice comme précurseur de l'enfant-femme.
Ce recours aux jeux de langage rend même possible une lecture dédoublée d'Alice puisque le ludisme mène à la coexistence de discours à la fois enfant et adulte:
\begin{quote}
  \begin{singlespace}
    \small
    \textit{And it is important to recognize that the audience for Alice, at least in the English-speaking world, consists of two groups -- children and adults. Thus far, I have been speaking of Alice as a child's book. But Alice in Wonderland is, in effect, two books: a book for children and a book for adults. Its interest, its fantasy, its humor, and its logic all operate at two levels. I know that adults often wonder why and how Alice can appeal to children. I suspect that children wonder why adults like it\footnote{\guil{C'est important de reconnaître que le lectorat d'\textit{Alice}, au moins dans le monde anglophone, consiste en deux groupes -- enfants et adultes. Jusqu'ici, j'ai parlé d'\textit{Alice} comme d'un livre pour enfants. Mais \textit{Alice au Pays des Merveilles} est, en fait, deux livres: un livre pour enfants et un livre pour adultes. Son intérêt, sa fantaisie, son humour et sa logique fonctionnent tous sur deux niveaux. Je sais que les adultes se demandent souvent pourquoi et comment \textit{Alice} peut plaire aux enfants. Je soupçonne que les enfants se demandent aussi pourquoi les adultes l'aiment.}~(notre traduction) dans \mancite \cite[7]{Weaver1964}.}.}
    \normalsize
  \end{singlespace}
\end{quote}
Cette dualité dans l'écriture constitue une forme de pervertissement du langage, subversion qui demeure tout de même assez douce dans le cas du Pays des merveilles.


\subsubsection{Le ludisme langagier: au-delà de la fonction poétique}
Dans la création de son enfant-adulte, Vurm conclut que ce personnage est hautement langagier, jumelant \guil{la culture et le jeu sophistiqué de la subversion chez l'adulte} ainsi que \guil{la spontanéité et le jeu chez l'enfant\footcite[103]{Vurm2014}}.
Cette affirmation nous renvoie au concept du ludisme langagier, notamment à ce que Brigitte Seyfrid a nommé la \guil{stratégie ludique}, qui consiste en la mise en place d'\guil{une rhétorique du jeu, rattachable à la \guill{fonction poétique} de Jakobson centrée sur le message en tant que tel\footcite[65]{Seyfrid-Bommertz1999}.}
Cette stratégie ludique consiste en l'exacerbation de la fonction poétique laquelle, poussée à l'extrême, se mue en fonction purement ludique:
\begin{quote}
  \begin{singlespace}
    \small
    À la théorie de l'adéquation absolue du vers à l'esprit de la langue, de la non-résistance de la forme au matériau, nous opposons la théorie de la violence organisée exercée par la forme poétique sur la langue. (\guil{Ce n'est pas la langue qui est maîtresse du poète, mais le poète qui est maître de la langue} selon la formule du professeur Brandt.) La forme tient compte du matériau auquel elle a affaire, mais ne peut être donnée tout entière dans ce matériau (ne peut en être tout entière déduite, coïncider avec)\footcite[40]{Jakobson1973}.
    \normalsize
  \end{singlespace}
\end{quote}
Usant allégrement de cette \guil{violence organisée} dans son expression, l'enfant-femme semble être une \guil{créature de langage}, voire un personnage purement langagier et dont l'unicité est subordonnée à son usage de la langue.
\par
Ainsi, nous estimons que la discordance dans les âges, déjà évoquée au niveau comportemental, se manifeste également dans l'expression verbale de l'enfant-adulte, qui combine un vocabulaire mature et élaboré dans un arrangement naïf et joueur typique de l'enfance.
Nous avançons même que cette stratégie ludique, soit l'usage des jeux de langage, constitue pour l'enfant-adulte une autre façon de résister au monde adulte tout en en intégrant profondément certaines particularités: si l'enfant maîtrise, en tout ou en partie, le code des adultes, il s'emploie lorsqu'il est l'émetteur à brouiller le message en subvertissant la fonction poétique jusqu'à verser dans le ludisme.

\subsection{Au contraire de la femme-enfant: le paradigme de la sexualisation}
Au c\oe{}ur de notre conception de l'enfant-femme se trouve la nécessité de la distinguer et de l'opposer au personnage de la femme-enfant.
Bien que ces figures soient toutes les deux du genre féminin, nous tenons à les poser comme des antonymes sur deux plans:  d'une part, les âges d'un personnage réfèrent à la fois à son âge réel et à son âge perçu ou représenté; d'autre part, la sexualisation d'un personnage peut être intrinsèque ou extrinsèque, volontaire ou involontaire.

\subsubsection{La femme-enfant comme séductrice}
À notre avis, la différence entre les deux figures est fondamentale puisque la femme-enfant est un personnage d'âge relativement adulte qui utilise volontairement l'enfance à des fins de séduction.
À cet effet, le Petit Robert la définit comme étant une \guil{femme qui semble avoir conservé les attributs de l'enfance, qui cultive un comportement enfantin\footcite[Entrée: « Femme-enfant »]{Robert2013}}.
\par
D'abord née sous la plume de Catulle Mendès qui publie en 1891 le roman \textit{La Femme-enfant}, elle est par la suite étudiée par le psychanalyste autrichien Fritz Wittels\footcite{Wittels1999} qui la dépeint comme hautement sexualisée, \guil{convaincu d'avoir découvert \guill{la femme primitive} au pouvoir sexuel illimité\footcite[302]{Inglin-Routisseau2006}}:
\begin{quote}
  \begin{singlespace}
    \small
    Wittels décrit, dans ce texte, une fille dont \guil{l'intense pouvoir de séduction sexuelle se déclare si tôt qu'elle est forcée de commencer à vivre sa sexualité alors qu'à tout autre égard elle n'est encore qu'une enfant. Toute sa vie durant, elle demeure surdéveloppée sexuellement et incapable de comprendre le monde adulte. Pas plus que ce monde n'est capable de la comprendre.} [...] Ce type de femme \guil{est nécessairement perverse polymorphe, sadique, lesbienne et tout ce qui s'ensuit.}
    \normalsize
  \end{singlespace}
\end{quote}
C'est cependant avec le surréalisme que la femme-enfant connaîtra son apogée: \guil{Cette jeune femme capricieuse, très belle et futile, stupide et sans tact, totalement infidèle ressemble au prototype féminin idéal de la transgression sexuelle plus tard prônée par le surréalisme\footcite[302]{Inglin-Routisseau2006}.}
Cependant, quoique personnage central du surréalisme, la femme-enfant n'y constitue pas une figure à la définition stable. Son visage est multiple: vierge immatérielle chez René Crevel (\textit{Babylone}), jeunesse éternelle aux accents \oe{}dipiens chez André Breton (\textit{Arcane 17}), prisme unificateur des genres, des espèces et des âges dans une sculpture de Salvador Dali où elle surgit au centre d'un buste aux têtes de vieillard, de chérubin, d'oiseau, d'homme et de chat à la fois\footcite[303]{Inglin-Routisseau2006}. Elle est opposée à la sorcière par Breton mais aussi par Benjamin Péret, \guil{porteuse d'un amour rédempteur [...] quand la sorcière est une femme fatale qui déchaîne les passions\footcite[306]{Inglin-Routisseau2006}}.
Malgré sa multiplicité, la femme-enfant est sacralisée en raison \guil{de sa beauté merveilleuse et poétique, de son pouvoir de subversion, de la nature ou de la vie\footcite[307]{Inglin-Routisseau2006}}.
\par
Hors du surréalisme, la femme-enfant s'est également développée en mythe social et désigne un ensemble de traits de personnalité chez la femme, dont l'actrice Brigitte Bardot constitue l'exemple vivant le plus équivoque:
\begin{quote}
  \begin{singlespace}
    \small
    Mêlant innocence et sensualité, B.B. condensera malicieusement l'enfant et la femme désirable, indomptable et inaccessible. La femme-enfant se voit, à ce stade, investie d'un pouvoir de séduction et d'une attraction érotique inégalables\footcite[307]{Inglin-Routisseau2006}.
    \normalsize
  \end{singlespace}
\end{quote}
C'est donc une femme qui joue à l'enfant et qui utilise la gaminerie pour séduire, et dont le rappel de la sexualité est essentiel à la construction du personnage.
Pour la femme-enfant, le recours à la sensualité et à la sexualité est intrinsèque et volontaire et constitue un outil dans son arsenal de séduction.
Davantage qu'à la nymphe surréaliste, c'est à cette vision de la femme-enfant que nous désirons plus que tout opposer notre propre conception du personnage d'enfant-femme.

\subsubsection{L'enfant-femme comme adulte involontaire}
Au contraire de la femme-enfant, l'enfant-femme est un personnage d'enfant à l'âge clairement défini comme tel et qui comporte certaines caractéristiques d'adulte.
C'est donc un enfant dont la précocité rappelle sur certains points la maturité d'un adulte.
Cette précocité n'est toutefois pas de nature sexuelle puisque, pour ce personnage, la sexualité est toujours apportée par l'Autre.
À l'époque victorienne, contexte dans lequel évoluait Alice, alors que le petit garçon est un homme en devenir, \guil{la petite fille $\left[ \dots \right]$ n'est pas une petite femme\footcite[14]{Lecercle1998}}.
Ainsi, la figure de l'enfant \guil{n'est pas sexuée: elle concerne autant le petit garçon que la petite fille\footcite[11]{Lecercle1998}.}
Par lui-même, le personnage d'enfant-femme est donc indistinct de celui de l'enfant-adulte puisqu'il n'est ni masculin, ni féminin: c'est simplement un enfant, sans distinction de genre.
\par
Ce caractère non-genré que l'on présuppose à l'enfant est cependant impossible chez la petite fille dès que l'on prend en compte la façon dont elle est perçue, puisqu'elle est invariablement et bien involontairement sexuée par l'Autre, qui voit en elle une femme:
\guil{La fillette est érotisée par le désir de l'homme. Voici Alice transformée en une enfant avertie et aguicheuse, annonçant \textit{Zazie}, \textit{Lolita}, et la femme-enfant\footcite[328]{Inglin-Routisseau2006}.}
Alors que l'enfant est dépourvu de genre, la petite fille est un personnage invariablement genré.
\par
D'ailleurs, Dissard exprime dans son texte \textit{Alice contre les garçons} que l'enfant plongé dans le terrier du lapin ne peut être que petite fille puisque \guil{le garçon tue le texte carrollien\footcite[116]{Dissard2005}}.
En réponse, Chiara Lagani émet l'idée que l'enfant carrollien échoue à être un garçon parce que ce dernier est considéré comme un adulte dès l'enfance\footcite[204]{Lagani2005}.
Elle reprend l'explication de Lecercle selon laquelle la petite fille victorienne, contrairement à ses frères, n'avait pas accès à la scolarisation d'état et était éduquée par une gouvernante, ce qui la préservait du monde adulte et du \guil{futur regard plein de désir de l'adulte\footcite[204]{Lagani2005}}.
Par cette éducation en vase clos qui la rend \guil{à la fois plus réprimée et plus libre}, la fillette victorienne évoque une conception de l'enfant dont on doit à tout prix protéger l'innocence, voire plus crûment l'ignorance, à savoir celle de la sexualité\footcite[13-14]{Lecercle1998}.
\par
C'est surtout sur le plan de la volonté sexuelle que l'enfant-femme se distingue le plus de la femme-enfant et qu'elle correspond en quelque sorte à son inverse puisque la sensualité n'est en rien susceptible d'intéresser l'enfant-femme, si ce n'est que pour la curiosité terminologique d'ajouter de nouveaux mots à son vocabulaire.
L'enfant-femme est donc tout l'opposé de la nymphette, \guil{très jeune fille  au physique attrayant, aux manières aguicheuses, à l'air faussement candide\footcite[Entrée: « Nymphette »]{Robert2013}}, que l'on nomme également \guil{lolita\footnote{Cette synonymie tient son origine d'une lecture particulière du roman \textit{Lolita}, de Vladimir Nabokov, voulant que la jeune Dolorès \textit{séduise} Humbert Humbert, plutôt que de considérer ce dernier comme un \guil{nympholepte}, à savoir un pédophile.}}.
\par
Enfin, nous estimons que l'existence même de l'enfant-femme est tributaire de sa distinction face aux personnages littéraires de la femme-enfant et de la lolita.
La figure de l'enfant-femme se construit ainsi par opposition à ces grandes figures dont elle se distingue complètement en matière de maturité et de volonté sexuelle.
Dès lors s'ajoutera aux particularités du personnage le rapport qu'elle entretient avec le langage: d'abord celui des adultes, puis celui, hybride entre l'enfance et l'âge adulte, qui lui est propre.
