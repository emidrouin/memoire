\chapter{Au croisement des âges: les personnages d'enfant-femme}
%
\begin{flushright}
                                    \begin{singlespace}
                                    \epigraph{
                                    L'adulte est mou. L'enfant est dur. Il faut éviter l'adulte comme on évite le sable mouvant. Un baiser qu'on met sur un adulte s'y enfonce, y germe, y fait éclore des tentacules qui prennent et ne vous lâchent plus. Rien ne pénètre un enfant; une aiguille s'y briserait, une francisque s'y briserait, une hache s'y briserait. L'enfant n'est pas mou, visqueux et fertile, il est dur, sec et stérile comme un bloc de granit.}
                                    \par
                                    Réjean Ducharme, \textit{L'Avalée des avalés}\footcite[336-337]{Ducharme1982}
                                    \end{singlespace}
\end{flushright}

Tel que nous l'avons explicité précédemment, le personnage d'enfant-femme comme nous le définissons est avant tout une figure d'enfant. Ainsi, ses traits adultes ne constituent pas l'essence même du personnage et existent en toute innocence, d'abord et surtout dans le regard de l'Autre, à savoir dans sa façon de la percevoir.
Façon de revisiter Alice Liddell et ses aventures au Pays des Merveilles, l'enfant-femme est certes plus autonome que ce que l'on attend généralement d'un enfant de son groupe d'âge, mais il s'agit tout de même d'une autonomie enfantine, en ce sens qu'elle est contextuelle, constituée de méfiance et d'adaptation plutôt que de réelle volonté de liberté.
\par
Personnage dont la filiation avec la jeune fille des écrits carrolliens est déjà établie\footnote{D'autant plus que Queneau avait auparavant fait paraître le pastiche \textit{Alice en France}.}, il ne fait aucun doute que \textit{Zazie dans le métro\footcite{Queneau1959}} de Raymond Queneau contribue à enrichir notre typologie d'enfant-femme, voire même qu'elle en constitue l'emblème dans la littérature française.
Si d'autres personnages de jeunes filles pourraient à notre avis être rattachés, à notre typologie\footnote{Nous pensons ici au personnage de Bérénice Einberg de \textit{L'avalée des avalés} de Réjean Ducharme, laquelle a été régulièrement associée à Zazie par la critique.}, nous croyons que la narratrice enfant du roman \textit{Borderline\footcite{Labreche2003}} de Marie-Sissi Labrèche constitue l'une des plus fortes manifestations récentes de cette figure en littérature québécoise.
Cette lecture peut sembler hautement improbable, nous en convenons, surtout que l'\oe{}uvre de Labrèche a d'abord été analysée en fonction de son contexte de publication\footnote{En tant qu'autofiction impudique, Labrèche s'inscrit dans la lignée des françaises Christine Angot (\textit{L'inceste}, 1999), Catherine Millet (\textit{La Vie sexuelle de Catherine M.}, 2001) et Chloé Delaume (\textit{Le Cri du sablier}, 2001), et précède la québécoise Nelly Arcan (\textit{Putain}, 2001 et \textit{Folle}, 2004).} sous les angles de la théorie psychanalytique ou des études féministes, perspectives mettant en lumière soit les rapports au corps, à la sexualité et à la mère.
Si ces théories se prêtent effectivement très bien à l'étude des romans de Labrèche en tant qu'autofictions narrées par une adulte, nous croyons que le roman \textit{Borderline} ouvre également la porte à une toute autre analyse, celle-ci portée exclusivement sur la narratrice enfant.
Ainsi, profitant de cette \textit{brèche} inexplorée par la critique, nous proposons d'établir l'appartenance de la narratrice enfant de \textit{Borderline} au grand type de l'enfant-femme, jusqu'à en faire à son tour un modèle de cette typologie dans la littérature québécoise du nouveau millénaire.


\section{Zazie, l'enfance à tout vent}
Petite fille d'une dizaine d'années\footcite[88]{Maurin2007}, Zazie est un personnage à la fois archifictionnel et hautement réaliste.
D'abord fictionnelle, parce qu'elle est un personnage très \guil{construit}, particulièrement sur le plan linguistique, elle est néanmoins rattachable au monde réel par le réalisme de son comportement, qui rappelle sans nul doute celui d'un  véritable enfant:
\guil{\textit{Zazie's behavior is an exaggerated but recognizable form of what one might expect from a child her age placed (dépaysée) in an unfamiliar adult world; her frustrations are familiar. $\left[ \dots \right]$ Zazie displays a true child's wonder\footnote{\guil{Le comportement de Zazie est une forme exagérée mais reconnaissable de ce qu'on pourrait s'attendre de la part d'une enfant de son âge placée (\textit{dépaysée}) dans le monde inconnu des adultes; ses frustrations sont familières.}~(notre traduction) dans \mancite \cite[121-122]{Bernofsky1994}.}.}}
De la même façon, Aurélie Maurin décrit Zazie comme une enfant \guil{de son temps}, représentante de sa génération:
\guil{En digne enfant du baby-boom, elle aime les \guill{bloudjins} et le soda, autrement dit le rêve américain, mais elle est provinciale, issue d'un milieu modeste qu'on imagine ouvrier\footcite[88]{Maurin2007}.}
Ainsi, Zazie est à la fois caricaturale et fidèle à la réalité des enfances françaises de l'après-guerre.


\subsection{Zazie la mouflette}
Quoique son comportement et sa façon de parler nous en fassent parfois douter, Zazie est d'abord et avant tout une enfant et son \guil{jeune âge} lui permet notamment d'être instantanément excusée pour une mauvaise plaisanterie~(\textit{Z}:~98).
Tel que nous l'avons déjà évoqué au chapitre précédent, elle s'inscrit, selon la typologie de Bethlenfalvay, dans la \guil{lignée du gamin}, héritier direct de l'enfant du monde du XIX\up{e} siècle\footcite[117]{Bethlenfalvay1979}.
D'un point de vue physique, autant sa démarche galopante~(\textit{Z}:~10) que le fait que Trouscaillon et la veuve Mouaque la soulèvent chacun par un bras pour la traîner~(\textit{Z}:~106) trahissent sa petite taille.

\subsubsection{Zazie affublée de noms: la ptite, la gosse, la mouflette}
Tout au long du roman, Zazie est affublée de surnoms ou désignations, dont plusieurs la réduisent à son jeune âge: la petite ou la \guil{ptite}, l'enfant, la mouflette, la gosse, la fillette ou la petite fille, la môme, la gamine et autres variations sur un même thème.
On dit d'elle qu'elle est mineure~(\textit{Z}:~89) et on l'appelle \guil{mademoiselle}~(\textit{Z}:~44, 95, 126, 127, 128, 143, 161, 174), \guil{petite}~(\textit{Z}:~11, 13, 20, 22, 30, 110) voire \guil{petite demoiselle}~(\textit{Z}:~126) ou même \guil{princesse des djinns bleus}~(\textit{Z}:~76).
On la traite de \guil{gosseline}~(\textit{Z}:~151), de \guil{petite voleuse}~(\textit{Z}:~53), de \guil{petite salope}~(\textit{Z}:~18, 25) ou encore de \guil{petit être stupide}~(\textit{Z}:~99).
Son oncle Gabriel, qui lui est malgré tout plutôt sympathique, la qualifie pour sa part de \guil{ptite vache}~(\textit{Z}:~92), de \guil{vraie petite mule}~(\textit{Z}:~98) et, non sans ironie, de \guil{petit ange}~(\textit{Z}:~23).
En somme, Zazie est constamment ramenée à son statut de \guil{petite}, c'est-à-dire à son enfance: que ce soit en la qualifiant d'enfant ou en l'interpellant par un surnom infantilisant -- comme le \guil{Eh petite}~(\textit{Z}:~30) répété par le satyre --, tout ce qui la désigne est linguistiquement connoté, soit de façon péjorative, soit dans une perspective réductrice.
Elle n'est pas une fille et encore moins une femme, comme le lui rappelle rapidement Charles lorsque Zazie se compare à elles: \guil{Toutes les femmes, voyez-vous ça, toutes les femmes. Mais tu n'es qu'une mouflette.}~(\textit{Z}:~83) Incidemment à son statut d'enfant et dans un rappel constant de sa petitesse, Zazie est avant tout désignée comme étant une fillette ou une \textit{petite} fille.

\subsubsection{Zazie cinématographique: l'\guil{in-désirable}}
Lors de sa transposition au cinéma sous la direction de Louis Malle\footcite{Malle1960} et avec l'accord de l'auteur, Zazie est rajeunie de quatre ans afin d'éviter toute ambiguïté sur le fait qu'elle soit une enfant:
\begin{quote}
  \begin{singlespace}
    \small
    Nous avons carrément rajeuni le personnage de quatre ans. Je voulais éviter tout côté \guil{Lolita}. Notre Zazie est donc une petite fille de dix ans qui dit n'importe quoi, sans équivoque, qui est absolument hors du monde des adultes et qui n'a jamais tort devant lui\footcite{Gilson1960}.
    \normalsize
  \end{singlespace}
\end{quote}
Dans le roman, sous son alias d'inspecteur Bertin Poirée, le \textit{type} nie avoir de l'intérêt pour Zazie afin de séduire Marceline: \guil{On s'en fout de Zazie. Les gosselines, ça m'éc\oe{}ure, c'est aigrelet, beuhh. Tandis qu'une belle personne comme vous\dots \ crènom.}~(\textit{Z}:~151)
L'opposition entre Zazie et Marceline exacerbe la jeunesse de la première et dénote qu'elle soit tout le contraire d'une femme, qu'elle ne soit pas \textit{désirable} comme l'est une femme.
Comme le réitère Louis Malle en entrevue, c'est dans l'objectif d'éviter l'effet \guil{lolita} ou femme-enfant que l'actrice Catherine Demongeot, âgée de dix ans lors du tournage, a été sélectionnée:
\begin{quote}
  \begin{singlespace}
    \small
    Dans le livre, Zazie est plus âgée que dans le film. J'ai demandé à Queneau la permission de la rajeunir. Au cinéma, on n'a guère le choix entre la petite fille et le petit monstre: une starlette de quatorze ans a l'air d'une femme. Pour moi, Zazie n'a plus le côté un peu trouble, un peu \guill{Lolita} de la Zazie de Queneau\footcite[227]{Bigot1994}.
    \normalsize
  \end{singlespace}
\end{quote}
Ce choix d'une actrice plus jeune démontre une volonté très claire de montrer un personnage dont l'allure physique rappelle résolument son statut d'enfant, avant tout dans l'objectif d'évacuer toute ambiguïté sexuelle qui serait exacerbée par l'importance de l'aspect visuel du médium qu'est le cinéma.

\subsubsection{Zazie et l'\textit{asessualité}: curieuse mais pas provoc'}
Bien qu'elle soit quelque peu confuse et surtout curieuse à propos de la sexualité, Zazie n'est définitivement pas une séductrice: \guil{\textit{This barely pubescent fictional creature is no seductress $\left[ \dots \right]$; she's as confused about sex and sexuality as about most everything else\footnote{\guil{Cette créature fictive à peine pubère n'est aucunement séductrice $\left[ \dots \right]$; elle est aussi confuse à propos du sexe et de la sexualité qu'à propos d'à peu près tout le reste.}~(notre traduction) dans \mancite \cite[119]{Bernofsky1994}.}.}}
Elle a certes acquis un certain vocabulaire à propos de la sexualité en écoutant les adultes, mais elle ne connaît pas forcément \guil{la chose} et ne maîtrise pas l'usage de ce vocabulaire qu'elle ne fait que reproduire: \guil{\textit{But Zazie knows only the words for things, not what they signify\footnote{\guil{Zazie connaît seulement les mots, pas leur signification.}~(notre traduction) dans \mancite \cite[119]{Bernofsky1994}.}.}}
Elle ne correspond donc définitivement pas au mythe de l'enfant provocatrice\footcite[90]{Maurin2007} et ne devrait pas être assimilée à une lolita\footnote{\cite[118-119]{Bernofsky1994}; \cite[90]{Maurin2007}}. En fait, \guil{Zazie petite fille aurait aussi bien pu être née garçon\footcite[90]{Maurin2007}} puisqu'elle \guil{semble encore loin d'être \guill{sexuée}\footcite[90]{Maurin2007}}.
En ce qui a trait à son apparence physique asexuée, la surprise d'un concitoyen sanctimontronais des Lalochère lorsqu'il rencontre Zazie est sans équivoque: \guil{Tiens, dit le conducteur, mais c'est la fille de Jeanne Lalochère. Je l'avais pas reconnue, déguisée en garçon.}~(\textit{Z}:~106)
\par
Chez Zazie, la sexualité n'est qu'un autre sujet à découvrir, objet de curiosité à l'instar du métro: \guil{Elle les pioche avant tout dans un insatiable appétit de la nouveauté. Zazie est curieuse de tout, ses principaux intérêts portent sur les choses de la modernité: le métro, les produits américains, mais aussi et surtout sur les questions de sexualité\footcite[89-90]{Maurin2007}.}
Ayant lu dans le journal que \guil{les chauffeurs de taxi izan voyaient sous tous les aspects et dans tous les genres, de la sessualité}, Zazie demande à Charles si ça lui est arrivé souvent de voir des \guil{clientes qui veulent payer en nature.}~(\textit{Z}:~83)
Se posant du côté du questionnement curieux plutôt qu'indécent, Zazie nous semble avant tout avoir une grande soif de savoir, plus précisément de \textit{tout} savoir.
Ainsi, lorsque Zazie questionne le taximan pour savoir si elle lui plaît -- \guil{Et moi, dit Zazie, je vous plairais?}~(\textit{Z}:~82) --, c'est davantage par provocation que par intention de séduction\footcite[90]{Maurin2007}, surtout qu'elle nie tout naturellement la simple éventualité qu'il puisse lui-même lui plaire lorsque Charles lui retourne la question: \guil{Alors? moi? je te plairais? / Bien sûr que non, répondit Zazie avec simplicité.}~(\textit{Z}:~82)
Enfin, la curiosité de Zazie s'inscrit en particulier dans sa poétique langagière, à savoir son usage démesuré de l'interrogation lorsqu'elle se fait Grande Inquisitrice au sujet de l'hormosessualité.


\subsection{Au Pays des grandes personnes: quand l'enfance mène le monde}
Bien que Zazie soit hors de tout doute une enfant, il demeure qu'elle possède plusieurs traits rappelant l'adulte. Mature et peu impressionnable, elle a une bonne compréhension du monde qui l'entoure, notamment des nombreux sujets de discussion des adultes.

\subsubsection{Les Aventures de Zazie au Pays du métro}
\guil{Remarquablement vive pour son âge\footcite[191]{Ajame1981}}, Zazie fait parfois figure d'enfant-adulte posée dans un monde d'adultes-enfants.
Décrite comme une petite fille \guil{effrontée, dégourdie et curieuse}, elle n'hésite pas à s'affirmer, comme lorsqu'elle détourne la question de Gabriel qui cherche à savoir l'heure de son couvre-feu chez sa mère: \guil{Ici et là-bas ça fait deux, j'espère.}~(\textit{Z}:~20)
D'une nature peu craintive, Zazie ne perd pas une seconde et, dès son réveil dans l'appartement silencieux de Gabriel, part à l'aventure dans Paris.
Soucieuse de ne pas se faire prendre, elle est prudente dans sa fugue, employant des \guil{précautions considérables} puis de \guil{non moins grandes précautions que précédemment} afin d'éviter de faire du bruit~(\textit{Z}:~28-29).
\par
Rusée, Zazie utilise parfois l'enfance afin de parvenir à ses fins.
Bonne comédienne, capable de \guil{doubl[er] le volume de ses larmes}~(\textit{Z}:~41) pour émouvoir, elle expose une vulnérabilité feinte, notamment lorsqu'elle requiert l'intervention de son oncle: \guil{Faut  mdéfendre, tonton Gabriel. Faut mdéfendre.}~(\textit{Z}:~62)
Fière de sa ruse, elle se félicite d'ailleurs d'avoir été \guil{aussi bonne que Michèle Morgan dans \textit{La Dame aux Camélias}.}~(\textit{Z}:~62)
Cependant, son plus impressionnant fait d'armes comme actrice est probablement la feinte qu'elle emploie pour se défaire de Turandot, lequel tentait simplement de rattraper la petite fugueuse.
Jouant la carte de l'enfance, Zazie détourne l'attention d'elle en accusant le \guil{meussieu}, Turandot, de lui avoir \guil{dit des choses sales}, se posant en fillette trop gênée pour répéter ce qu'on lui a dit: \guil{C'est trop sale, murmure Zazie.}~(\textit{Z}:~31)
La foule de badauds étant trop occupée à discuter de cette histoire scandaleuse, personne ne se rend compte que Zazie a fui à nouveau.
\par
Décisionnaire, autonome et opportuniste\footcite[88]{Maurin2007}, Zazie proteste lorsqu'il le faut, hurlant et glapissant, \guil{folle de rage}~(\textit{Z}:~11, 90, 106).
Elle estime ne pas avoir besoin de son oncle, représentant de l'autorité parentale, et affirme que sa tante et elle peuvent sortir seules et se passer de lui~(\textit{Z}:~22).
Elle revendique ses droits -- \guil{J'aime pas qu'on me traite comme ça}~(\textit{Z}:~106) -- et demande ce qu'elle veut avant même qu'on le lui offre, que ça soit une glace fraise-chocolat~(\textit{Z}:~114) ou son café qu'elle prend \guil{avec de la peau dessus}~(\textit{Z}:~174).
Ayant déterminé que l'assiette qu'on lui avait servie \guil{était de la merde}, elle tient à sa liberté d'expression -- \guil{Vous m'empêcherez tout de même pas de dire, dit Zazie, que c' (geste) est dégueulasse.} Ainsi, refusant de bouffer \guil{cette saloperie} et réclamant impérieusement \guil{ottchose} puisque ce qu'on lui avait servi était \guil{tout simplement de la merde}~(\textit{Z}:~125-127), Zazie revendique en toute impolitesse son autonomie et son droit de décider pour elle-même.

\subsubsection{Le monde adulte, Pays des cauchemars?}
Parfois qualifiée de \guil{gamine délurée, sans vergogne dans sa curiosité insolente\footcite[114]{Pestureau1983}}, Zazie est comparée à Alice et à Ulysse, d'abord pour son audace mais surtout pour son \guil{goût de la découverte} qui l'entraîne dans une \guil{descente aux Enfers} vers le métro, \guil{terrier moderne pour lapins humains\footcite[113]{Pestureau1983}}.
Dans la lignée des \guil{fééries inquiétantes} de Lewis Carroll, Raymond Queneau \guil{écrit le monde à partir d'un regard enfantin qui en révèle l'envers, l'absurde, le burlesque ou l'horreur\footcite[113]{Pestureau1983}.}
Le Paris de Zazie, sorte de Pays des merveilles urbain et moderne, est avant tout un monde d'adultes, féroce, instable, violent et ambigu\footcite[113]{Pestureau1983}.
Comme Alice avant elle, Zazie se retrouve plongée ou même perdue dans un univers inquiétant qui n'a rien d'enfantin: \guil{\textit{Like \textit{Alice's Adventures in Wonderland}, \textit{Zazie dans le métro} centers around a single child in the midst of adults whose doings bewilder her\footnote{\guil{Comme \textit{Les Aventures d'Alice au Pays des Merveilles}, \textit{Zazie dans le métro} tourne autour d'une enfant seule au milieu d'adultes dont les agissements la déconcertent.}~(notre traduction) dans \mancite \cite[119]{Bernofsky1994}.}.}}
Entraînée dans ce \guil{\textit{clearly unreal environment} (environnement clairement irréel)\footcite[119]{Bernofsky1994}}, elle échoue à trouver sa voie, en ce sens qu'elle ne parvient pas à son but puisqu'elle est endormie lorsqu'elle prend enfin le métro et qu'elle n'obtient pas de réponse sur ce qu'est un \textit{hormosessuel}.
Elle quitte donc la ville aussi désorientée qu'à son arrivée\footcite[119]{Bernofsky1994}, sans réponse ni à sa quête du métro, ni à son questionnement sur les choses adultes de la sexualité et du langage.

\subsubsection{\textit{Essméfie}, en aventurière avertie}
Plongée dans un tel monde, Zazie montre maints traits de caractère que l'on retrouve normalement chez les personnes plus âgées: au-delà de sa curiosité, elle est aussi méfiante, questionnant longuement les adultes et doutant de ce qu'ils lui disent:
\begin{quote}
  \begin{singlespace}
    \small
    On peut $\left[ \dots \right]$ affirmer que la jeune héroïne n'a de l'enfance que l'âge et le questionnement. Elle possède des caractéristiques attribuées habituellement aux adultes: elle est décisionnaire et revendique une certaine autonomie, elle n'est en rien naïve et sait même se montrer soupçonneuse; ses prises de risques sont le plus souvent opportunistes. Du haut de ses 10 ans, elle mène une troupe d'adultes en mal de repères. Sans montrer aucune faiblesse, elle clame à qui veut bien l'entendre qu'elle n'a peur de rien, car, elle, Zazie, en a vu d'autres\footcite[88]{Maurin2007}!
    \normalsize
  \end{singlespace}
\end{quote}
Elle est aussi difficile à berner, surtout en ce qui concerne son objet fétiche qu'est le métro:
\begin{quote}
  \begin{singlespace}
    \small
    Zazie fronce le sourcil. Essméfie.
    \guil{Le métro? qu'elle répète. Le métro, ajoute-t-elle avec mépris, le métro, c'est sous terre, le métro. Non mais.\\
    -- Çui-là, dit Gabriel, c'est l'aérien.\\
    -- Alors, c'est pas le métro.\\
    -- Je vais t'esspliquer, dit Gabriel. Quelquefois, il sort de terre et ensuite il y rerentre.\\
    -- Des histoires.} ~(\textit{Z}:~12)
    \normalsize
  \end{singlespace}
\end{quote}
Suspicieuse et rationnelle, elle \guil{réserv[e] son opinion}~(\textit{Z}:~17), un peu comme le ferait un adulte.
\par
Au-delà de son seul caractère en avance sur son âge, Zazie a également une compréhension étonnamment profonde du monde des adultes: \guil{Elle fait parfois des raccourcis mais cerne bien la complexité des \guill{choses de la vie}\footcite[90]{Maurin2007}.}
Elle ne correspond ainsi pas au standard de l'enfant naïf, faible et pur\footcite[90]{Maurin2007}, pour l'essentiel ignorant des choses du monde des \guil{grandes personnes}.
Au contraire, Zazie possède un imaginaire des choses adultes particulièrement étoffé pour son âge: \guil{C'est hun dégueulasse qui m'a fait des propositions sales, alors on ira devant les juges tout flic qu'il est, et les juges je les connais moi, ils aiment les petites filles, alors le flic dégueulasse, il sera condamné à mort et guillotiné et moi j'irai chercher sa tête dans le panier de son et je lui cracherai sur sa sale gueule, na.}~(\textit{Z}:~62)
Pédophilie, système judiciaire, peine de mort: rien ne lui échappe!
\par
Ainsi, Zazie est précoce et lucide, voire \guil{affranchie, jetée dans un monde féroce\footcite[113]{Pestureau1983}}.
C'est également un personnage d'une grande curiosité, \guil{\textit{a fictional creature filled with curiosity, desire and a profound, never-relieved perplexity\footnote{\guil{une créature fictive remplie de curiosité, de désir et d'une profonde perplexité inassouvie}~(notre traduction) dans \mancite \cite[114-115]{Bernofsky1994}.}.}}
Ayant en commun avec son ancêtre anglaise \guil{l'audace et le goût de la découverte\footcite[113]{Pestureau1983}}, Zazie est présentée par Pestureau comme une sorte d'\guil{Alice amère et insolente, $\left[ \dots \right]$, à la fois rêveuse et cynique, voulant pénétrer le souterrain des adultes, soumise à leur jeu mais retournant leurs armes contre eux, [qui] tente de les déchiffrer au cours du bref cycle de ses aventures, d'un train l'autre\footcite[114]{Pestureau1983}.}
\par
Outre sa grande curiosité envers le monde des adultes et ce qu'il représente, Zazie n'est pas dupe des intentions de ces derniers: \guil{En effet, elle semble avoir une conscience aiguë des désirs malveillants de certains adultes, elle en use parfois, mais ne se laisse pas abuser par leurs intentions\footcite[90]{Maurin2007}.}
Elle flaire donc les dangers potentiels et y oppose une attitude méfiante.
Au satyre qui lui demande de lui faire confiance, elle répond sournoisement par la question \guil{Pourquoi que j'aurais confiance en vous?}, l'accusant ensuite d'être \guil{un vieux salaud} lorsqu'il affirme aimer les enfants, petits garçons comme petites filles~(\textit{Z}:~42).
Quoiqu'elle soit alléchée à l'idée d'avoir enfin des \textit{bloudjinnzes}, Zazie ne se laisse pas leurrer par les fausses bonnes intentions du \textit{type}: \guil{C'est sûrement un sale type, pas un dégoûtant sans défense, mais un vrai sale type. Faut sméfier, faut sméfier, faut sméfier.}~(\textit{Z}:~45-46)\
Quand ce dernier raconte à Gabriel le délit de Zazie, elle met en garde son oncle: \guil{Méfie-toi, tonton Gabriel.}~(\textit{Z}:~56)
Ayant préalablement conseillé à son oncle de ne répondre que devant son avocat~(\textit{Z}:~56), Zazie réitère son conseil lorsque le même \textit{type} questionne Marceline: \guil{Faut parler que devant ton avocat, dit Zazie. Tonton a pas voulu m'écouter, tu vois comme il est emmerdé maintenant.}~(\textit{Z}:~63)
\par
Outre sa méfiance prudente, Zazie possède également~(\textit{Z}:~63) une franche lucidité face à l'insolente bouffonnerie des autres personnages du roman.
Étant extérieure au monde des adultes, l'héroine de Queneau en met ainsi au jour le caractère ridiculement burlesque, stylistiquement amplifié au cinéma par le choix de la trame sonore:
\begin{quote}
  \begin{singlespace}
    \small
    Ce monde lui paraît rigoureusement absurde, fait de gens qui ne savent rien d'eux-mêmes et qui vivent dans le chaos. Son entrée se fait sur une musique de western et elle a un côté justicier de western : elle arrive dans la ville et se désolidarise de ses habitants. Plus elle les provoque, se moque d'eux, les injurie et par là augmente encore le chaos. $\left[ \textit{sic} \right]$ Elle les juge sévèrement. Jamais elle ne joue leur jeu; mais, à la fin, il était temps, elle allait commencer à se laisser prendre: « J'ai vieilli ! »\footcite{Gilson1960}
    \normalsize
  \end{singlespace}
\end{quote}
Ainsi, Zazie serait, selon Bernofsky, un personnage anti-anti-réel, en ce sens qu'elle n'est pas purement réaliste, mais qu'elle est plutôt en combat contre l'anti-réel:
\begin{quote}
  \begin{singlespace}
    \small
    \textit{In fact she is unique among the characters in the novel, a realist figure placed within a subversively self-reflective anti-realist milieu. This double inversion within the novel has considerable ontological and aesthetic implications both in the context of Queneau's work and for the novel in general; when the anti-real is taken as the norm, the realist tendency becomes itself a subversion: the anti-anti-real\footnote{\guil{En fait, elle est unique parmi les personnages du roman, figure réaliste placée dans un milieu anti-réaliste subversivement auto-réflexif. Cette double inversion a des implications ontologiques et esthétiques considérables, à la fois dans le contexte de l'\oe{}uvre de Queneau et du roman en général; lorsque l'anti-réel est pris comme norme, le réalisme devient lui-même une subversion: l'anti-anti-réel.}~(notre traduction) dans \mancite \cite[115]{Bernofsky1994}.}.}
    \normalsize
  \end{singlespace}
\end{quote}
Shérif de la Raison à la poursuite des cowboys du \textit{nonsense}, Zazie est donc, comme Alice, posée en gardienne de la rationalité dans un monde en déroute:
\begin{quote}
  \begin{singlespace}
    \small
    Zazie n'est pas dupe. Et elle évolue également dans une société où les adultes ont vécu la guerre, l'occupation, et la privation. Ils prennent, tant qu'ils peuvent, leur revanche sur la liberté. Il reconstruisent leurs mythes à grand renfort de frasques. Seule parmi des adultes fantaisistes, et parfois absents à eux-mêmes, Zazie doit trouver son ancrage, son étayage dans la réalité, par elle-même\footcite[89]{Maurin2007}.
    \normalsize
  \end{singlespace}
\end{quote}
D'où notre hypothèse selon laquelle Zazie est sa propre bouée de sauvetage, ne pouvant se fier que sur elle-même dans ce monde d'adultes pour la plupart dépourvus de bon sens.

\subsubsection{Les adultes, tous des cons!}
Aux yeux de Zazie, la plupart des adultes sont méprisables, d'où son attitude globalement peu respectueuse envers eux.
Ainsi, tandis que Charles le chauffeur de taxi conduit une \guil{charrette dégueulasse} dont il faudrait selon la jeune fille faire baisser le tarif~(\textit{Z}:~14), le bar de Turandot est minable puisqu'on n'y danse pas~(\textit{Z}:~26), de même que l'est le \textit{type} pour sa méconnaissance d'un journal régional~(\textit{Z}:~48).
Quand ce dernier lui dit s'intéresser au sport, et plus particulièrement au catch, Zazie ricane en se moquant de lui, jugeant à la vue de son \guil{gabarit médiocre} qu'il doit être \guil{dans la catégorie spectateurs}~(\textit{Z}:~49).
De la même façon, elle s'esclaffe lorsque Gabriel prétend avoir déjà eu à \guil{repousser les assauts de ces gens-là\footcite[93]{Queneau1959}}.
Enfin, à son oncle Gabriel qui la questionne, Zazie rétorque, à la limite de la condescendance, qu'il \guil{trouverai[t] pas tout seul, hein?}~(\textit{Z}:~20)
\par
Devant la déraison généralisée des adultes qui l'entourent, Zazie en vient presqu'à douter qu'ils puissent ne pas tous être complètement idiots: \guil{Dis donc, tonton, demande Zazie, quand tu déconnes comme ça, tu le fais esprès ou c'est sans le vouloir?}~(\textit{Z}:~14)
Acquiéscant sereinement aux propos de Charles qui décrète que \guil{[l]es passants $\left[ \dots \right]$ c'est tous des cons}~(\textit{Z}:~12-13), Zazie affirme que la foule qui fréquente la rue est constituée de \guil{braves gens avec des têtes de cons}~(\textit{Z}:~53), se demandant plus tard à propos des gens jugeant le parfum de Gabriel \guil{ce que ces cons-là peuvent bien y connaître}~(\textit{Z}:~165).
À son oncle qui décrète que Saint-Germain-des-Prés est démodé, Zazie rétorque promptement qu'elle pourrait lui répondre à lui qu'il \guil{n'es[t] qu'un vieux con}~(\textit{Z}:~15), de la même façon que l'avocat parisien célèbre de sa mère était lui aussi \guil{un con, quoi}~(\textit{Z}:~47).
La veuve Mouaque est pour sa part \guil{moins conne que [Zazie] le croyai[t]}~(\textit{Z}:~100), quoiqu'on se doute bien qu'elle le soit malgré tout encore un peu aux yeux de la jeune fille: \guil{Ce qu'elle peut déconner, dit Zazie faiblement.}~(\textit{Z}:~169)
On pourrait donc résumer la pensée zazique en cette formule choc: les adultes, tous des cons!
\par
Traitant Gabriel et Charles de \guil{ptits marants}~(\textit{Z}:~13), Zazie se pose parfois en adulte face à eux, les rappelant à l'ordre en leur intimant de ne pas recommencer~(\textit{Z}:~79, 91) et soulignant leur immaturité d'un commentaire incisif -- \guil{Les petits farceurs de votre âge, dit Zazie, ils me font de la peine.}~(\textit{Z}:~80)
Dans son rejet des enfantillages des adultes, elle fait preuve de maturité et figure en personne raisonnée face à des adultes en perdition.
Ainsi, face au Sanctimontronais qui prétend qu'il y a le métro à Saint-Montron, la fillette est incisive: \guil{Ça alors, dit Zazie, c'est le genre de déconnances qui m'éc\oe{}urent particulièrement.}~(\textit{Z}:~110)
Démasquant le mensonge et remarquant l'incohérence des adultes, la jeune héroïne confirme son statut d'enfant-adulte par le renversement de perspectives suivant: \guil{Zazie, par sa précocité, met en relief le fait qu'elle soit une enfant se comportant comme une adulte, dans un monde d'adultes se comportant comme des enfants\footcite[91]{Maurin2007}.}

\subsubsection{En-deça de l'amour: hors des enfantillages des grands}
Lucide face aux technicalités de la vie des adultes, notamment à propos de la raison qui pousse sa mère à la confier à Gabriel -- \guil{C'est comme ça qu'elle est quand elle a un jules, dit Zazie, la famille ça compte plus pour elle}~(\textit{Z}:~9) --, Zazie n'en est pas moins dédaigneuse du sentiment amoureux en général, qu'elle juge absurde et dont elle se moque sans gêne.
Aussi raille-t-elle les émois de la veuve Mouaque qui attend Trouscaillon:
\begin{quote}
  \begin{singlespace}
    \small
    \guil{- C'est le flicard qui vous met dans cet état? \\
    - Ah! l'amour\dots \ quand tu connaîtras\dots \\
    - Je me disais bien qu'au bout du compte vous alliez me débiter des cochonneries. Si vous continuez, j'appelle un flic\dots \ un autre\dots \\
    - Cest cruel}, dit la veuve Mouaque amèrement. \\
    Zazie haussa les épaules. \\
    \guil{Pauv'vieille\dots \ $\left[ \dots \right]$} \\
    Avant que la Mouaque utu le temps de répondre, Zazie avait ajouté: \\
    \guil{Tout de même\dots \ un flicard. Moi, ça me débecterait.}~(\textit{Z}:~121)
    \normalsize
  \end{singlespace}
\end{quote}
Loin de mal comprendre le sentiment amoureux comme on le supposerait de la part d'une enfant, Zazie parvient à \guil{lire} l'agitation de la veuve Mouaque et à verbaliser avec justesse l'émotion qui habite cette dernière.
Elle se place cependant au-dessus de cette futilité qu'est l'amour en rejetant rapidement le postulat bienveillant de la veuve qui suppose que la petite ne le connaît pas encore: ce n'est pas qu'elle ne comprend pas le sentiment, c'est plutôt qu'elle ne l'estime pas digne de son intérêt et qu'il ne constitue surtout pas, à ses yeux, une raison justifiable de se mettre \guil{dans cet état}.
\par
Directe et sans pitié, Zazie signale finalement sa lassitude par rapport à la discussion et son mépris de l'amour à la veuve Mouaque, mettant abruptement fin aux doléances de cette dernière:
\begin{quote}
  \begin{singlespace}
    \small
    Zazie haussa les épaules. \\
    \guil{Tout ça, c'est du cinéma, qu'elle dit. Vous auriez pas un autre sujet de conversation? \\
    - Non, dit énergiquement la veuve Mouaque. \\
    - Eh bien alors, dit non moins énergiquement Zazie, je vous annonce que la semaine de bonté est terminée. A rvoir.}~(\textit{Z}:~122)
    \normalsize
  \end{singlespace}
\end{quote}
Indélicate mais non moins diplomate, Zazie signifie ainsi à la veuve Mouaque son désintérêt pour les choses de l'amour, s'élevant au-delà de ce sujet futile et le mettant à distance puisqu'il est sans aucun doute indigne d'elle.
\par
Enfin, à la limite du mépris, Zazie se moque des effets de l'amour sur Mme Mouaque: \guil{Elle est suprême, celle-là. $\left[ \dots \right]$ C'est l'amour qui rend comme ça?}~(\textit{Z}:~122)
Soulignant sans complexe l'ignorance de la veuve par l'interrogation rhétorique \guil{Qu'est-ce que vous en savez? Il vous a fait des confidences? Déjà?}~(\textit{Z}:~123), Zazie-moqueuse en rajoute avec l'adverbe \guil{[d]éjà?} qui a pour effet de décupler la portée de son mépris pour les \guil{enfantillages} des adultes.
Enfin, revenue à de meilleures dispositions envers la veuve Mouaque sur le point de partir rejoindre son Trouscaillon, Zazie lui souhaite en toute simplicité de \guil{bonnes fleurs bleues}~(\textit{Z}:~124), montrant par son usage de cette expression toute l'étendue de sa compréhension des relations amoureuses des adultes, qu'elle résume même à son oncle Gabriel en affirmant que la dame a \guil{un fleurte terrible avec le flicmane}~(\textit{Z}:~124).
De façon générale, Zazie est donc plutôt dédaigneuse face à l'irrationnel et surtout aux histoires de princesses. Après avoir divagué jusqu'à s'en être fait un conte de fées à elle-même, Zazie, \guil{récupérant son esprit critique, $\left[ \dots \right]$ finit par se déclarer que c'est drôlement con les contes de fées et décide de sortir.}~(\textit{Z}:~29)

\subsubsection{Une République zazique: hors de l'autorité parentale}
Si Zazie peut aller et venir aussi librement dans Paris, c'est avant tout parce qu'elle ne subit pas les contraintes de l'autorité parentale: \guil{Plutôt que l'émanation d'un modèle, il semblerait qu'elle incarne le fait que des enfants puissent être acteurs de leur quotidien et vivre en dehors de l'autocratie parentale\footcite[91]{Maurin2007}.}
Dès les premières pages du roman, Zazie est confiée à son oncle Gabriel par une mère \guil{désinvolte mais émancipatrice\footcite[114]{Pestureau1983}}, davantage pressée d'aller rejoindre son prétendant que de se soucier du bien-être de son enfant: \guil{Tu vois l'objet, dit Jeanne Lalochère s'amenant enfin. T'as bien voulu t'en charger, eh bien, le voilà.}~(\textit{Z}:~9)
Qualifiée d'\guil{objet} par sa mère, qui réitère son détachement envers son enfant lorsqu'elle vient \guil{récupérer la fille} plutôt que \guil{sa} fille~(\textit{Z}:~180), Zazie semble être aux yeux des adultes un simple encombrement: \guil{Et qu'est-ce que c'est que cette môme que tu trimbales avec toi?}~(\textit{Z}:~88)
Bien que l'oncle Gabriel prenne théoriquement le relais de Jeanne Lalochère et qu'il fasse figure de substitut de parent, il n'a en réalité que très peu d'ascendant sur sa nièce, que ce soit sur ses déplacements, son comportement ou son langage:
\begin{quote}
  \begin{singlespace}
    \small
    On pourrait plutôt parler d'un enfant sujet, qui semble mener sa vie en dehors de toute autorité ou considérations parentales. Zazie fait son éducation, seule, par elle-même et pour elle-même. Les figures masculines du roman (son oncle, Charles le taximan, Pedro-surplus le flicman, le cafetier) revendiqueront ce souci d'éducation, mais ne parviendront jamais à imposer à la demoiselle aucune marque d'obéissance. Zazie est, certes, indisciplinée, mais personne ne tente réellement de maintenir auprès d'elle quelques règles qui en soient vraiment. $\left[ \dots \right]$ Zazie et son impétuosité inquiètent les rangs de l'autorité, mais elle divertit aussi\footcite[89]{Maurin2007}.
    \normalsize
  \end{singlespace}
\end{quote}
D'ailleurs, il ne semble pas réellement souhaiter avoir sur elle une quelconque autorité, notamment lorsqu'il apprend sa fugue et qu'il suppose qu'elle \guil{se retrouvera bien toute seule}~(\textit{Z}:~36).
Ainsi, bien qu'il revendique le droit de l'élever comme il le veut -- \guil{Vous, dit Gabriel au flicmane, je vous prie de me laisser élever cette môme comme je l'entends. C'est moi qui en ai la responsibilitas. Pas vrai, Zazie?}~(\textit{Z}:~125) --, la question finale ajoutée à son affirmation, à savoir qu'il demande à Zazie d'approuver son propos, inverse à notre avis la relation d'autorité entre l'oncle et la nièce.
N'est-ce pas Zazie qui est de cette façon posée comme responsable de Gabriel et, par le fait même, de sa propre personne?
\par
Au final, Zazie évolue hors de toute autorité sans même avoir eu à se libérer d'une quelconque forme de tutelle puisque personne ne semble en fait avoir de contrôle sur elle ou même chercher à en avoir.
Devant cette absence quasi-totale d'encadrement, Zazie \guil{grandit} en s'auto-responsabilisant: \guil{L'autonomie et la rébellion deviennent alors une question de survie, ne pouvant compter sur la fiabilité des adultes, elle doit faire preuve d'autres ressources\footcite[89]{Maurin2007}.}
Disposant de sa liberté afin d'assouvir sa soif de découvertes et affirmant son autonomie par sa curiosité\footcite[91]{Maurin2007}, Zazie démontre tout au long du roman l'étendue de sa débrouillardise.
Conséquemment, elle s'autonomise, sans qu'il ne soit possible de déterminer si cette précocité est innée ou acquise chez elle: \guil{Devant le peu de constance du monde adulte, on ne sait donc pas si Zazie est autonome par disposition ou par obligation\footcite[89]{Maurin2007}.}


\subsection{Un enfant adaptatif: la philosophie du \guil{tout va de soi!}}
En plus de sa très grande disposition pour la liberté et l'aventure, ce qui distingue Zazie de la plupart des autres enfants est sa capacité d'adaptation et sa résilience face aux événements qu'elle vit.
À la façon de Gavroche ou des \guil{enfants du monde} zoliens, elle représente la \guil{vraie vie} et ne correspond pas au mythe de l'enfant martyr\footcite[90]{Maurin2007}.
Inébranlable malgré son histoire familiale particulière, Zazie n'est également pas décontenancée par les choses rationnelles de la vie, à l'égard desquelles elle nourrit généralement une certaine curiosité.
Ainsi, même si elle ne comprend pas le mot \guil{hormosessuel}, elle n'a pas de réaction particulière lorsqu'elle rencontre Marceline ou bien Marcel, compagne-compagnon de son oncle Gabriel~(\textit{Z}:~181).
Malgré leur aspect inédit pour elle, ce n'est ni l'homosexualité ni le travestissement de son oncle qui arriveront à perturber l'inébranlable Zazie.
\par
Ayant échappé de peu à son père qui tentait de l'agresser, pour ensuite être témoin de l'assassinat de ce dernier par sa mère, Zazie semble très peu affectée par son histoire familiale, la racontant comme s'il s'agissait du récit d'événements anodins:
\begin{quote}
  \begin{singlespace}
    \small
    Zazie a une histoire de vie complexe et douloureuse, mais n'en fait ni étalage ni gloriole. Pour elle, tout semble aller de soi, il lui suffit d'une tirade pour exposer les faits, presque sans affects: après que son père, alcoolique, tente de la violer, sa mère l'assassine d'un coup de hache, s'ensuivent d'innombrables procédures judiciaires et pénales\footcite[88]{Maurin2007}.
    \normalsize
  \end{singlespace}
\end{quote}
C'est donc avec une attitude détachée que Zazie entame son récit, davantage intéressée par sa bière que par le drame qu'elle raconte:
\begin{quote}
  \begin{singlespace}
    \small
    \guil{Vous en mettez du temps pour écluser votre godet. Papa, lui, il en avalait dix comme ça en autant de temps.\\
    -- Il boit beaucoup ton papa?\\
    -- I buvait, qu'il faut dire. Il est mort.\\
    -- Tu as été bien triste quand il est mort?\\
    -- Pensez-vous (geste). J'ai pas eu le temps avec tout ce qui se passait (silence).\\
    -- Et qu'est-ce qui se passait?\\
    -- Je boirais bien un autre demi, mais pas panaché, un vrai demi de vraie bière.}~(\textit{Z}:~47)
    \normalsize
  \end{singlespace}
\end{quote}
Poursuivant son histoire, Zazie explicite la mort de son père jusque dans les moindres détails, la hache comprise:
\begin{quote}
  \begin{singlespace}
    \small
    \guil{Vous lisez les journaux?\\
    -- Des fois.\\
    -- Vous vous souvenez de la couturière de Saint-Montron qu'a fendu le crâne de son mari d'un coup de hache? En bien, c'était maman. Et le mari, naturellement, c'était papa.\\
    -- Ah! dit le type.\\
    -- Vous vous en souvenez pas?}\\
    Il n'en a pas l'air très sur. Zazie est indignée.\\
    \guil{Merde, pourtant, ça a fait assez de foin.}~(\textit{Z}:~47)
    \normalsize
  \end{singlespace}
\end{quote}
Avec un naturel improbable, Zazie semble davantage perturbée par la possibilité que le \textit{type} n'ait pas entendu parler de son histoire que par l'histoire elle-même.
Toujours aussi imperturbable, elle raconte par la suite la quasi-agression à l'origine de toute la séquence d'événements:
\begin{quote}
  \begin{singlespace}
    \small
    Papa, il était donc tout seul à la maison, tout seul qu'il attendait, il attendait rien de spécial, il attendait tout de même, et il était tout seul, ou plutôt il se croyait tout seul, attendez, vous allez comprendre. Je rentre donc, faut dire qu'il était noir comme une vache, papa, il commence donc à m'embrasser ce qu'était normal puisque c'était mon papa, mais voilà qu'il se met à me faire des papouilles zozées, alors je dis ah! non parce que je comprenais où c'est qu'il voulait en arriver le salaud, mais quand je lui ai dit ah! non ça jamais, lui il saute sur la porte et il la ferme à clef et il met la clef dans sa poche et il roule les yeux en faisant ah ah ah tout à fait comme au cinéma, c'était du tonnerre. Tu y passeras à la casserole qu'il déclamait, tu y passeras à la casserole, il bavait même un peu quand il proférait ces immondes menaces et finalement immbondit dssus. J'ai pas de mal à l'éviter. Comme il était rétamé, il se fout la gueule par terre. Isrelève. Ircommence à me courser, enfin bref, une vraie corrida. Et voilà qu'il finit par m'attraper. Et les papouilles zozées de recommencer.~(\textit{Z}:~50)
    \normalsize
  \end{singlespace}
\end{quote}
De façon enfantine, notamment par l'emploi de l'expression \guil{papouilles zozées}, qui relève d'une terminologie approximative de la sexualité, Zazie expose toujours aussi calmement le drame de sa famille.
Ne parlant pas d'agression sexuelle -- saurait-elle seulement ce que c'est? --, elle compare les événements à une \guil{corrida} afin d'illustrer l'assaut, dont elle sait toutefois qu'il n'était pas \guil{normal} comme pouvait l'être l'embrassade initiale que lui a faite son père, qu'elle qualifie ultimement de \guil{salaud}.
\par
Dans le détachement le plus complet et après quelques considérations plutôt techniques sur le prix exorbitant de l'avocat qui a défendu sa mère, Zazie explique que c'est Georges, le \guil{coquin de maman}, qui \guil{avait refilé [à cette dernière] la hache (silence) pour couper son bois (léger rire).}~(\textit{Z}:~48)
Elle salue d'ailleurs la ruse de sa mère -- \guil{Pas bête, la guêpe, hein?} --, soulignant avec dégoût l'horreur visuelle de la scène du crime:
\begin{quote}
  \begin{singlespace}
    \small
    -- Mais, à ce moment, la porte s'ouvre tout doucement, parce qu'il faut vous dire que maman elle lui avait dit comme ça, je sors, je vais acheter des spaghetti et des côtes de porc, mais c'était pas vrai, c'était pour le feinter, elle s'était planquée dans la buanderie où c'est que c'est qu'elle avait garé la hache et elle s'était ramenée en douce et naturellement elle avait avec elle son trousseau de clefs. Pas bête, la guêpe, hein?\\
    -- Eh oui, dit le type.\\
    -- Alors donc elle ouvre la porte en douce et elle entre tout tranquillement, papa lui il pensait à autre chose le pauvre mec, il faisait pas attention quoi, et c'est comme ça qu'il a eu le crâne fendu. Faut reconnaître, maman elle avait mis la bonne mesure. C'était pas beau à voir. Dégueulasse même. De quoi mdonner des complexes. Et c'est comme ça qu'elle a été acquittée. J'ai eu beau dire que c'était Georges qui lui avait refilé la hache, ça n'a rien fait, ils ont dit que quand on a un mari qu'est un salaud de skalibre, y a qu'une chose à faire, qu'à lbousiller. Jvous ai dit, même qu'on l'a félicitée. Un comble, vous trouvez pas?\\
    $\left[ \dots \right]$\\
    -- Et après? demanda le type.\\
    -- Bin après c'est Georges qui s'est mis à tourner autour de moi. Alors maman a dit comme ça qu'elle pouvait tout de même pas les tuer tous quand même, ça finirait par avoir l'air drôle, alors elle l'a foutu à la porte, elle s'est privée de son jules à cause de moi. C'est pas bien, ça? C'est pas une bonne mère?~(\textit{Z}:~50-51)
    \normalsize
  \end{singlespace}
\end{quote}
Enfant de l'immédiat, si Zazie est d'abord indignée de la clémence du juge, elle s'accommode rapidement de l'acquittement de sa mère, allant jusqu'à être reconnaissante envers cette dernière d'avoir quitté Georges pour la protéger.
Avec ce récit, Zazie ne cherche aucunement la pitié et semble accepter son sort sans broncher:
\begin{quote}
  \begin{singlespace}
    \small
    On peut surprendre Zazie comme faisant preuve de subjectivité et d'objectivité, elle est un sujet, elle est partie prenante de l'action qu'elle vit. Elle ne se déresponsabilise pas de son vécu, ne se pose pas en victime de son drame familial, pas plus que ses parents ne passent pour des bourreaux\footcite[90]{Maurin2007}.
    \normalsize
  \end{singlespace}
\end{quote}
C'est donc son acceptation de tout ce qui lui arrive, sorte de philosophie du \guil{tout va de soi!}, qui justifie à nos yeux l'appartenance de Zazie au type de l'enfant-femme tel que nous l'avons précédemment décrit.


\section{Sissi, une enfance à réparer}
Dans \textit{Borderline}\footcite{Labreche2003}, l'auteure québécoise Marie-Sissi Labrèche fictionnalise son \guil{enfance de coquerelles\footcite{Labreche2008b}}, vécue dans la pauvreté entre une mère souffrant de maladie mentale et une grand-mère autoritaire.
Alternant de chapitre en chapitre entre une narratrice enfant et la jeune-adulte hautement sexualisée qu'elle est devenue, Labrèche met au monde le personnage de Sissi Labrèche~(\textit{B}:~63).
Dans \textit{La Brèche}\footcite{Labreche2008}, la narratrice Émilie-Kiki raconte l'époque entourant l'écriture du premier roman. L'enfance n'y est que très peu évoquée, mais reprend, lorsqu'elle l'est, les mêmes marqueurs, à savoir la pauvreté matérielle, la folie de la mère et les reproches incessants de la grand-mère.
\par
En 2008, Lyne Charlebois et Marie-Sissi Labrèche cosignent le scénario du film \textit{Borderline}\footcite{Charlebois2008}, adaptation cinématographique des deux romans amalgamés et dont le personnage principal, Kiki, est montrée à trois différents moments de sa vie: sous les traits de l'enfant de dix ans élevée dans la misère et la folie, puis de la jeune femme archisexuelle du roman \textit{Borderline}, ainsi que de la quasi-trentenaire qui rédige ce qui sera son premier roman à l'instar de la narratrice dans \textit{La Brèche}\footcite[85]{Ledoux-Beaugrand2009}.
Clôturant ce que nous qualifierions de \guil{grand cycle romanesque des enfances délabrées}, \textit{La lune dans un HLM}\footcite{Labreche2008b} a été écrit pendant la période de production du film.
Y alternent des lettres de l'auteure à sa mère et des chapitres purement fictionnels mettant en scène Léa, jeune adulte dont le contexte familial n'est pas sans rappeler les personnages précédents -- mère psychiatrisée et décès d'une grand-mère responsabilisante --, à l'exception notable toutefois que son comportement ne verse pas dans l'excès, ni de substances, ni de sexe, ni d'amour.


\subsection{Sissi la \guil{fillette de l'est}}
Si la construction du roman \textit{Borderline} départage clairement deux temporalités différentes dans la vie d'une même narratrice, la phrase liminaire de chacun des chapitres pairs rend non-équivoque le fait qu'il y soit question d'une enfant puisqu'on y fait mention de son âge exact: \guil{J'ai onze ans}~(\textit{B}:~29), \guil{J'ai huit ans et je suis en deuxième année}~(\textit{B}:~55), \guil{J'ai sept ans}~(\textit{B}:~87) et \guil{J'ai cinq ans}~(\textit{B}:~115).
La narratrice parle d'ailleurs d'elle-même comme d'une \guil{petite fille}~(\textit{B}:~30-31, 121), se décrivant comme une enfant maigre avec des \guil{cheveux comme des spaghettis} et des \guil{yeux trop grands, grands comme des yeux de chien piteux}~(\textit{B}:~63).
La Sissi des chapitres pairs est ainsi un personnage d'enfant, d'autant plus qu'elle se reconnaît lorsqu'il est question de jeunesse: \guil{Et l'enfant, c'est moi, je le sais $\left[ \dots \right]$.}~(\textit{B}:~94)
\par
De son enfance dans le quartier Centre-Sud, \guil{à l'est de la rue Papineau}~(\textit{B}:~62), la Sissi de huit ans, en deuxième année du primaire, tire une illustration plutôt sombre, la comparant au \guil{\textit{Russian way of life}, le \textit{concentration camp way of life}, version rue Dorion} et l'opposant à l'enfance de type \guil{\textit{American way of life}, version québécoise} que dessinent, au sens propre du terme lorsque la maîtresse leur fait faire l'exercice artistique de représenter leurs familles en dessin, ses camarades de classe~(\textit{B}:~56).
C'est, selon Virginie Doucet, une enfance \guil{déterminée par la peur, le manque, le solitude (\textit{sic})\footcite[79]{Doucet2007}.}
Reprenant à la façon d'une comptine un court poème de Lucie Delarue-Mardrus\footnote{Les vers \guil{\textit{Je suis une petite fille / Et tout le monde m'aime bien / Quelquefois je pleure et je rage / On ne peut pas toujours être sage comme une image / C'est tout!}}, quoique retranscrits de façon approximative avec certains mots inversés, sont tirés du poème \guil{Une petite fille} du recueil \textit{Poèmes mignons} de Lucie Delarue-Mardrus, publié en 1929.}, Émilie-Kiki évoque son passé de \guil{fillette de l'est\footcite[58]{Labreche2008}}, dans un récit qui n'est pas sans rappeler celui de Sissi et qui se déroule dans le même Montréal défavorisé, dans l'appartement rempli de coquerelles~(\textit{B}:~34-35, 38, 119) du \guil{2020 de la rue Dorion où [s]a grand-mère habitait et où [elle] demeurai[t] avec elle huit mois par année parce que [s]a mère était internée\footcite[149]{Labreche2008}}.
\par
Décrite par la travailleuse sociale comme étant hyperactive, agitée et très créative~(\textit{B}:~99), Sissi est une \guil{petite fille nerveuse}~(\textit{B}:~121), trop \guil{paquet de nerfs} pour qu'on lui laisse tenir les alliances lors du mariage imaginé de sa mère~(\textit{B}:~116).
Enfant au tempérament nerveux, elle a des gestes brusques et pousse des cris aigus, constamment à la recherche d'attention: \guil{qu'on se taise et m'écoute}~(\textit{B}:~118).
Comme enfant, Sissi est donc énervée et volontairement énervante: \guil{C'est [s]a marque de commerce.}~(\textit{B}:~117)

\subsubsection{À l'âge de la contradiction}
Pour l'enfant Sissi, l'enfance semble se construire de façon complètement antagoniste à l'âge adulte, qu'elle refuse.
Ainsi, elle exprime sous quelques formes son rejet de l'adulte, \textit{a priori} de sa mère: \guil{Je ne veux pas lui ressembler et je me bats. Tout ce qu'elle aime, je ne l'aime pas. Tout ce qu'elle fait, je ne le fais pas. Je ne veux pas être elle. Niet. No. Non. Je ne suis pas elle.}~(\textit{B}:~30-31)
Moins brutal que la haine, Sissi utilise également la contradiction comme mécanisme de mise à distance de l'adulte: \guil{Moi, quand je serai plus vieille, je contredirai ma grand-mère. Je me marierai avec tous les hommes de la terre juste pour la faire chier.}~(\textit{B}:~32-33)
Elle retire donc une satisfaction de dire ou de faire l'opposé de ce que dit ou fait sa grand-mère.
\par
Son rejet de l'adulte, bien qu'il soit principalement dirigé à l'encontre de ces deux figures maternelles, est également présent envers d'autres représentants de l'autorité:  \guil{Alors là, je l'ai traumatisée, la maîtresse, avec mon dessin. Je l'ai traumatisée, sauf qu'elle a été obligée de l'accrocher quand même, la vieille chipie. $\left[ \dots \right]$ Traumatisée, la prof. Je m'en fiche! $\left[ \dots \right]$ Même la directrice, qui me serre toujours dans ses bras lorsqu'elle voit mes beaux dessins, a été traumatisée. Quand elle l'a vu, elle n'a pas su quoi dire.}~(\textit{B}:~59)
Non seulement Sissi ne semble pas avoir envie de plaire aux intervenants adultes, mais elle semble également retirer une fierté de leur déplaire ou, selon ses mots, de les traumatiser.
Cette constante contradiction des adultes semble être, pour elle, une façon bien réfléchie de se positionner hors du monde des adultes et d'y échapper.

\subsubsection{Sissi et l'asexualité: ne sait ni ne veut savoir}
Pour la jeune Sissi, la sexualité n'est pas le sujet d'un grand savoir, ni même un sujet suscitant un quelconque intérêt.
Quoique l'on suppose qu'elle prononcerait, au contraire de Zazie et de ses innombrables questions sur la \guil{sessualité}, le terme \guil{sexualité} correctement, le fait est que ce n'est pas un terme qu'on l'imagine vraiment utiliser; Sissi verse dans l'asexualité la plus totale, en ce que le sexe ne semble même pas piquer sa curiosité.
Ainsi, lorsque sa grand-mère lui décrit des sévices que subissent certains enfants et bifurque vers les agressions de nature sexuelle, Sissi semble ne pas trop comprendre, ou peut-être ne pas vouloir comprendre, où elle veut en venir:
\begin{quote}
  \begin{singlespace}
    \small
    -- S'amuser avec eux? C'est le fun, s'amuser avec quelqu'un? \\
    -- Pas s'amuser avec des bébelles, s'amuser avec quelqu'un. Tu sais toucher quelqu'un partout, même où c'est pas supposé. \\
    -- C'est où que c'est pas supposé? \\
    -- Dans les fesses!~(\textit{B}:~129-130)
    \normalsize
  \end{singlespace}
\end{quote}
Lasse des insinuations de sa grand-mère avant même que cette dernière en vienne à ce qu'elle voulait véritablement savoir, Sissi plaide à nouveau l'ignorance enfantine afin d'éviter le sujet:
\begin{quote}
  \begin{singlespace}
    \small
    Mais elle revient à la charge en s'accrochant après mes mots comme une journaliste en plein milieu d'une entrevue qui veut faire cracher le morceau. Elle veut poursuivre jusqu'à ce que j'avoue son intention à elle. \\
    -- Est-ce qu'il t'a déjà grattée, le nouveau chum de ta mère? \\
    -- Bien oui, il m'a déjà grattée! \\
    -- Hon oui, comment? \\
    -- Bien comme ça, avec les doigts. \\
    -- Oh oui, et où ça? \\
    -- Dans le dos pis sur la tête aussi. \\
    -- T'a-t-il déjà grattée ailleurs? \\
    -- Je ne me rappelle pas. \\
    -- Ou plutôt tu ne veux pas me le dire. \\
    -- T'es tannante! Je viens de te le dire, il m'a déjà grattée dans le dos et sur la tête. Je ne raconte pas de mensonges, moi. ~(\textit{B}:~130-131)
    \normalsize
  \end{singlespace}
\end{quote}
Hésitante et mal à l'aise face au sujet de la sexualité abordé par sa grand-mère, la Sissi de cinq ans semble comprendre ce qui est attendu d'elle, à savoir qu'elle accuse son beau-père de l'avoir attouchée, sans cependant manifester de curiosité face à la sexualité.
L'hésitation de Sissi semble d'ailleurs exacerbée par son ambivalence entre une dualité de valeurs: d'une part, il y a sa volonté de ne pas mentir et d'autre part son désir de contenter sa grand-mère en lui disant ce qu'elle veut entendre.
Sissi semble donc en mesure de conceptualiser grossièrement la sexualité, quoiqu'elle ne s'y intéresse pas vraiment et ne serait pas en mesure de l'expliquer dans les détails; si elle sait qu'il y a des touchers \guil{pas supposés}, le reste de son éducation sexuelle reste à faire.
\par
Malgré la mise en garde virulente de sa grand-mère contre les attouchements de nature sexuelle, le contact physique peut tout de même, pour la jeune Sissi, prendre les traits d'un vecteur d'apaisement.
Il n'est alors porteur ni de sensualité ni de sexualité, apportant plutôt un état de bien-être.
Ainsi, associé à sa grande gentillesse, le toucher du professeur d'éducation physique contribue à apaiser la douleur que cause à Sissi son sentiment de solitude:
\begin{quote}
  \begin{singlespace}
    \small
    Je ne voulais pas rater mon cours d'éducation physique avec mon professeur barbu. Mon professeur qui est si gentil avec moi. Je ne voulais pas arriver en retard parce que je voulais courir en rond et que mon professeur barbu vienne attacher mon lacet. Qu'il s'approche de moi et qu'il me regarde avec ses grands yeux noirs gentils, encadrés par ses gros sourcils noirs gentils. Qu'il se penche vers moi et que son épaule frôle mon petit corps. Que je sois moins seule, pour quelques secondes.~(\textit{B}:~63)
    \normalsize
  \end{singlespace}
\end{quote}
Ayant rapidement compris que Sissi ne se sentait pas bien en la voyant insulter sa mère, le professeur d'éducation physique tente de la rassurer en établissant une proximité physique de type paternel avec elle: il la tient par la main, l'aide à se changer au vestiaire puis lui fait un câlin.
\begin{quote}
  \begin{singlespace}
    \small
    Une fois que la maîtresse et la classe de deuxième année ont commencé à s'éloigner, il m'a regardée longtemps dans les yeux, s'est penché vers moi puis m'a serrée dans ses bras fort, fort, fort. Et il a dit: \textit{Je sais que c'est dur pour toi, Sissi. Je le sais.} Mon estomac vide a grimpé le long de mon corps et s'est remis à sa place, tout comme mon c\oe{}ur, mon pancréas, mes intestins, mon foie et mes reins. Les choses ont repris un semblant de normalité.~(\textit{B}:~69)
    \normalsize
  \end{singlespace}
\end{quote}
Cette douceur est également celle de la main \guil{douce et tiède}, \guil{comme une caresse dans [s]a main} d'Aline, la jeune intervenante sociale qui lui \guil{tient tout doucement la main sans la comprimer, sans [lui] écraser les doigts} et qu'elle oppose à la main \guil{bourrée d'arthrite et prête à [lui] broyer les jointures}~(\textit{B}:~98) de sa grand-mère.
Chez le personnage d'enfant créé par Labrèche, le toucher peut appartenir à diverses catégories.
Il y a bien évidemment la proximité physique dangereuse que pourrait imposer un agresseur à l'enfant, qui prend racine dans l'imaginaire de Sissi par les mises en garde de sa grand-mère.
Tout à l'opposé, il y a également le contact physique réconfortant, agréable et non-sexuel qui prend la forme d'une main tendue ou d'une caresse de la part d'un adulte compréhensif face aux difficultés familiales de l'enfant.
Ainsi, bien que les rapports entre Sissi et son professeur barbu aient une dimension physique, ils sont, en raison du jeune âge de la fillette, dépourvus de tout ce qui touche de près ou de loin à la sexualité.
Cette asexualité dans les rapports entre Sissi et son professeur, alors que le câlin se distingue de la caresse et le déshabillage de l'effeuillage, est génératrice d'enfance en ce sens que de tels rapports physiques entre adultes supposent forcément une connotation affective différente, ce qui n'est pas le cas avec un enfant.


\subsection{Au Pays des adultes: une enfant seule au monde}
Enfant unique dont la famille se résume à sa mère et à sa grand-mère, Sissi construit sa conception du \guil{nous} autour de ces deux femmes et à l'exclusion du reste du monde: \guil{Contre nous, moi, ma grand-mère et ma mère.}~(\textit{B}:~100)
Isolée par ses deux mamans~(\textit{B}:~40, 88, 124-125), lesquelles exercent l'autorité dans une sorte de co-parentalité -- \guil{mes parents, mes deux mamans}~(\textit{B}:~119) --, Sissi est seule au monde: \guil{De toute façon, si je meurs, qui ça va déranger? Je suis toute seule au monde. Toute seule. Je vois déjà mon épitaphe d'ici: \textit{Ci-gît Sissi, la plus toute seule des toutes seules. Va en paix, petite souris sans queue. Fin.}}~(\textit{B}:~61)
Cette solitude s'alourdit également du \guil{perpétuel silence qui [l]'entoure}~(\textit{B}:~93) imposé par la folie de la mère, auquel la petite tente de remédier en le brisant de sa propre parole, \guil{pass[ant] [s]es journées à soliloquer}~(\textit{B}:~93).
\par
Face à ce climat familial claustrophobique, Sissi développe une aversion pour la vie, ou plutôt pour \textit{cette} vie qui est la sienne:
\guil{À vrai dire, j'ai plutôt envie de la dégueuler, la vie. J'ai huit ans et j'en fais déjà une indigestion, de la vie.}~(\textit{B}:~59)
En découle d'abord un désintérêt envers les membres de sa famille, à commencer par sa grand-mère qui \guil{dit tout le temps des niaiseries}~(\textit{B}:~92) et qui \guil{marmonne des choses} incompréhensibles qui n'ont \guil{pas d'importance} pour elle~(\textit{B}:~88).
Sa solitude d'enfant égarée dans le monde des adultes, accrue par le peu d'intérêt qu'elle porte aux adultes qui l'entourent, développe chez Sissi une haine envers sa famille, sentiment qu'elle extériorise par un violent rejet: \guil{Oui, si je le pouvais, je détruirais tout. Je piétinerais cette maudite maison de carton remplie de coquerelles. J'écrabouillerais la chambre de ma mère qui renferme un drame. Je réduirais en mille miettes tous ces vilains meubles décatis qui m'empêchent de courir jusqu'à l'infini.}~(\textit{B}:~119)
Isolée du reste du monde, avec pour seule compagnie celle de ses \guil{deux folles de mamans}~(\textit{B}:~88) qu'elle honnit, Sissi est en quelque sorte orpheline, dépourvue de sa famille qu'elle a rejetée violemment.

\subsubsection{Deux mamans, pas de maman}
Dans l'\oe{}uvre de Marie-Sissi Labrèche, l'autonomisation de l'enfant passe avant tout par le délaissement et le blâme qu'elle subit.
Chez la petite Sissi, l'autorité parentale est surtout assumée par la grand-mère, vu l'incapacité récurrente de la mère d'assurer son rôle maternel en raison de ses problèmes mentaux:
\begin{quote}
  \begin{singlespace}
    \small
    Ma mère, c'était une folle. Une vraie folle avec des yeux qui fixent, un comportemnet désaxé et des milliers de pilules à prendre tous les jours. Une vraie folle avec un vrai certificat médical en bonne et due forme, qui devait passer le test de Rorschach très souvent, si souvent qu'à la vue d'une tache elle ne pouvait s'empêcher de dire à quoi ça lui faisait penser: \textit{Une tulipe! Un éléphant! Un nuage! Un utérus éventré! Des Chinois qui mangent du riz!}~(\textit{B}:~16)
    \normalsize
  \end{singlespace}
\end{quote}
Même dans ses moments de lucidité, la mère de Sissi est inapte à assumer son rôle, notamment parce qu'elle craint absolument tout, sa fille incluse: \guil{Ma mère a peur de moi. Faut dire que ma mère a peur de tout. $\left[ \dots \right]$ Elle a peur de mes cris et de mes pleurs. Quand je pleure et je crie, elle a peur que les autres disent qu'elle n'est pas une bonne mère et qu'elle me bat. Elle a peur que les voisins envoient la DPJ et que la DPJ me mette dans son fourgon d'enfants maltraités.}~(\textit{B}:~120)
Souvent déconnectée de la réalité, la mère de la narratrice est absente bien que présente physiquement, ce qui cause chez l'enfant un sentiment d'abandon:
\begin{quote}
  \begin{singlespace}
    \small
    Ma mère, j'ai toujours pensé qu'elle ne tenait pas à moi. J'ai toujours pensé que, parce qu'elle se réfugiait trop souvent quelque part dans sa tête où je n'avais pas accès, elle ne tenait pas à moi. Ma mère pouvait passer des semaines comme ça, dans sa tête, à me regarder avec ses yeux bleus braqués sur moi, sans expression, ses yeux remplis de dépression qui me rendaient malade.~(\textit{B}:~19-20)
    \normalsize
  \end{singlespace}
\end{quote}
C'est ce que Virginie Doucet appelle la \guil{non-vie} de la mère\footcite[79]{Doucet2007}, qui fait d'elle une mère instable incapable d'endosser son rôle: \guil{Une mère instable est défaillante en ceci qu'elle est incapable d'offrir à ses proches -- et notamment à ceux qui dépendent d'elle -- des réactions suffisamment prévisibles pour faire fonction de référence, de repère, d'appui\footnote{\cite[212]{Eliacheff2003} cité dans \cite[31]{Leduc2010}}.}
Or, la mère de Sissi ne la \guil{comprenait même pas à cause de ses maudites hormones défectueuses, ses maudites hormones passées date}~(\textit{B}:~20) et était pour ainsi dire une figure absente en raison de sa maladie.
Katherine Dion abonde dans le même sens, à savoir que la folie de la mère la \guil{tue} symboliquement: \guil{Absente à sa fille autant qu'elle l'est à elle-même, la mère folle est telle une mère morte\footcite[33]{Dion2010}.}
Cette \guil{mère-spectrale} n'assure donc, sur le plan affectif, ni stabilité ni sécurité\footcite[31]{Leduc2010}, et l'enfant doit apprendre à vivre malgré l'absentéisme maternel.
Outre cette absence mentale et affective de la mère qui se retranche fréquemment dans sa tête, il ne faut pas négliger les périodes d'hospitalisation qui l'amènent parfois à s'absenter totalement, à la fois sur les plans physique, mental et affectif, accentuant le deuil symbolique dont a parlé Dion\footcite[36]{Dion2010}.
\par
En l'absence d'une véritable figure maternelle, Sissi est prise en charge par sa grand-mère maternelle qui fait office d'autorité parentale.
Cette substitution de la mère par la \textit{mémé} opère un déplacement symbolique\footcite[33]{Leduc2010} alors que Sissi et sa mère se retrouvent au même niveau générationnel: \guil{Tout se mélange dans ma tête, comme les rôles familiaux dans la maison: ma mère est ma soeur, ma grand-mère est ma mère $\left[ \dots \right]$.}~(\textit{B}:~63)

\subsubsection{Une main de fer sans le gant de velours}
Cependant, la grand-mère est une figure très peu maternelle et représente encore moins la douceur associée traditionnellement à la mère, éduquant Sissi dans la crainte de tout, à grands coups d'histoires d'horreur et de récits des malheurs d'enfants du monde entier: \guil{Il y a des enfants qui se font maltraiter. On ne leur donne pas à manger. On ne les habille pas. On les attache et on les brûle avec des cigarettes.}~(\textit{B}:~129)
Ainsi, dès son plus jeune âge, Sissi est consciente qu'elle \guil{ne doi[t] pas trop faire chier [s]a grand-mère}~(\textit{B}:~116) sous peine de représailles:
\begin{quote}
  \begin{singlespace}
    \small
    Par exemple, quand j'étais tannante, elle avait coutume de me dire: \textit{Si t'es pas gentille, un fifi va entrer par la fenêtre et te violer} ou \textit{Je vais te vendre à un vilain qui fera la traite des Blanches avec toi} ou encore \textit{Un assassin va venir te découper en petits morceaux avec un scalpel, c'est ça que tu veux? Hein?}~(\textit{B}:~11)
    \normalsize
  \end{singlespace}
\end{quote}
Bien au-delà de la seule incapacité maternelle de la mère biologique, la violence verbale subie par Sissi est, à notre avis, le principal vecteur de la dépossession de l'enfance de la narratrice. Cette violence verbale à laquelle se livre la \textit{Mémé} se manifeste dans l'hybridation des trois formes que sont le reproche, la menace et l'insulte.
\par
Laissée à elle-même sur le plan affectif, Sissi est très jeune accablée par ce blâme quasi-litanique que fait peser sur elle une grand-mère toute-puissante, inflexible et même despotique\footcite[34]{Leduc2010}: \guil{D'ailleurs, combien de fois ma grand-mère m'a cassé les oreilles avec ça? \textit{T'es bonne pour dire des niaiseries, toi. T'es bonne en crisse pour dire des niaiseries qui inquiètent ta mère}.}~(\textit{B}:~15)
Abreuvant constamment Sissi de ses reproches, sa grand-mère lui impose la responsabilité de ne pas inquiéter sa mère: \guil{\textit{Sissi, n'oublie pas ta bombonne de Ventolin, sinon tu risques de faire une crise d'asthme, et ça va inquiéter ta mère. Et quand ta mère s'inquiète, elle devient folle et on doit l'enfermer. Tu ne veux pas rendre ta mère folle, hein?}}~(\textit{B}:~60)
Et si ce n'est pas l'asthme, c'est autre chose: la grand-mère trouve rapidement un autre motif pour lui faire porter la culpabilité de la folie de sa mère -- \guil{Ma grand-mère me l'a toujours dit: \textit{T'es bonne en crisse pour raconter des histoires. T'es bonne en crisse pour raconter des histoires qui rendent ta mère folle.} Des histoires qui rendent folle ma mère folle\dots}~(\textit{B}:~76) --, reproches qui font par ailleurs également partie des souvenirs d'enfance d'Émilie-Kiki, narratrice du second livre: \guil{\textit{ÉMILIE-KIKI, TU RENDS FOLLE TA MÈRE! C'EST DE TA FAUTE SI ON DOIT TOUT LE TEMPS L'INTERNER!}}~(\textit{B}:~40)
\par
Au reproche s'ajoute parfois la menace, notamment dans la formule répétitive du \guil{si tu fais ceci il arrivera tel malheur}, la réalisation du malheur étant toujours conditionnelle et imputée au mauvais comportement de Sissi: \guil{Elle me dit: \textit{Si t'arrête pas de courir comme ça, les voisins d'en bas vont se plaindre et on va nous jeter à la porte. On n'aura plus de maison et on devra rester dans la rue. C'est-tu ça que tu veux?}}~(\textit{B}:~118)
Notons que pour culpabiliser l'enfant davantage et renforcer sa menace, la grand-mère utilise une formulation qui fait porter le blâme à l'enfant, en lui demandant si c'est vraiment ce qu'elle veut.
Accablant Sissi d'une responsabilité supplémentaire, la grand-mère l'accuse de ne pas vouloir le bien de sa mère:
\guil{Ça va faire ton affaire si ta mère meurt. Mais je vais te dire \dots si elle meurt, je vais peut-être être obligée de te placer dans une famille d'accueil, et je t'ai conté ce qui arrive dans ces familles, c'est pas drôle.}~(\textit{B}:~35)
Ainsi, toute son enfance durant, il plane sur Sissi une perpétuelle menace d'abandon:
\begin{quote}
  \begin{singlespace}
    \small
    Si tu n'es pas gentille, je vais te placer. Si tu ne manges pas toute ton assiette, je vais te placer. Si tu racontes des menteries, je vais te placer. Si tu te fouilles dans le nez, je vais te placer! Si tu déplaces le pot de jus, je vais te placer. Je vais te placer! Je vais te placer! Je vais te placer! Je vais te placer comme ça ne se peut pas! Je vais te placer jusque sur une autre planète. Tiens, Pluton, c'est la plus loin!~(\textit{B}:~36)
    \normalsize
  \end{singlespace}
\end{quote}
De façon récurrente et toujours sous la formule du \guil{si tu fais ceci ou celà}, la grand-mère menace d'imposer à Sissi des conséquences en raison de son comportement jugé répréhensible: \guil{Si t'es méchante comme ça, je vais appeler ton vrai papa, Papa Méchant. Il va t'emmener chez lui, là où il reste avec sa pute, et ça ne va pas être drôle.}~(\textit{B}:~123)
Le rituel de menace que fait subir la grand-mère à Sissi implique généralement son abandon et sa prise en charge par une famille de remplacement, famille dont elle a une grande crainte en raison de ce qui lui a été raconté à son propos.
\par
Ajoutant l'injure à la menace d'abandon, la grand-mère rappelle constamment à Sissi qu'elle est une mauvaise fille:
\begin{quote}
  \begin{singlespace}
    \small
    Je suis fatiguée de la voir s'énerver, parce que quand elle s'énerve, elle s'en prend toujours à moi et j'en prends plein la gueule. Elle me dit que je suis méchante, que je ne pense qu'à faire mal aux autres, que je suis une petite débauchée et qu'un jour elle va me placer.~(\textit{B}:~36)
    \normalsize
  \end{singlespace}
\end{quote}
Tout devient ainsi prétexte au reproche, que ça soit envers ce que Sissi dit ou fait, ne dit pas ou ne fait pas, voire même le fait qu'elle n'aime pas porter certains vêtements:
\begin{quote}
  \begin{singlespace}
    \small
    Que l'imperméable lui a coûté un bras et que si je ne le porte pas, c'est parce que je suis méchante et que je veux juste lui faire de la peine, que je veux juste lui faire gaspiller son argent, que l'argent ne pousse pas dans les arbres, que j'exagère tout le temps sur le pain pis le beurre, que j'en demande toujours trop, que je ne suis jamais contente de rien, que je suis paresseuse, que je suis traîneuse, que je suis débauchée, que je n'arriverai à rien dans la vie, que je vais finir sur l'aide sociale avec un mari qui me bat et quatre enfants sur les bras, et blablabla.~(\textit{B}:~90)
    \normalsize
  \end{singlespace}
\end{quote}
L'imperméable semble ici n'être qu'un point de départ pour une cascade de reproches tout aussi variés les uns que les autres et qui s'accumulent.
Partant de la seule ingratitude d'une enfant qui ne veut pas porter un vêtement qu'elle trouve laid, la grand-mère termine sa pluie de reproches par une mention prophétique à l'effet que Sissi aura une vie de misère.
Les accusations de débauche ne sont d'ailleurs pas isolées et font partie du rituel dénigrant de la grand-mère: \guil{Ça aussi ma grand-mère me l'a souvent dit: \textit{T'es juste une petite vicieuse et une petite cochonne!}}~(\textit{B}:~76)
Bref, aux encouragements normalement attendus de la part d'un parent se substituent pour la petite Sissi les récriminations et injures de sa \textit{Mémé}.
Élevée dans la menace et le reproche constants, et bien que sa grand-mère ne la \guil{place} pas vraiment en famille d'accueil, Sissi est victime d'une véritable forme d'abandon sur les plans affectif et émotionnel dont résultent \guil{un moi criblé de trous\footnote{\cite[63]{Haineault2006} cité dans \cite[33]{Dion2010}.}} et un vieillissement émotionnel précoce: \guil{Crisse! Je n'ai même pas besoin de me grandir, malgré ma petite taille, j'ai déjà mille ans.}~(\textit{B}:~68)
Enfin, si la grand-mère remplace effectivement la mère et représente à ce titre l'autorité parentale, il faut souligner qu'il s'agit davantage d'autorité que de parentalité: la main de fer est raide et nue, sans gant de velours pour l'adoucir.

\subsubsection{De l'autre côté du miroir du monde adulte}
Réduisant chez Sissi la part laissée à l'enfance, sa compréhension de la \guil{chose adulte} est surtout illustrée dans un spectre précis de sa vie, soit sa conscience de sa situation familiale peu orthodoxe: \guil{\textit{Bien oui, ma mère n'est pas comme la vôtre! Bien oui, ma mère est folle! Qu'est-ce que ça peut bien vous foutre! Allez donc tous chier!}}~(\textit{B}:~70)
À seulement cinq ans, la jeune fille a conscience que sa mère prend des antidépresseurs qui la rendent \guil{peace and love}~(\textit{B}:~116):
\guil{J'ai beau avoir cinq ans, mais je m'en aperçois, que tout est au ralenti, je ne suis pas con, je ne suis pas stupide. Je ne suis pas sur les antidépresseurs, moi. J'ai un plafond dans la tête, moi.}~(\textit{B}:~118)
Puis, à sept ans, elle explique à sa façon la maladie de sa mère:
\begin{quote}
  \begin{singlespace}
    \small
    Elle est froide, ma mère. Froide et effacée. Mais ce n'est pas de sa faute. C'est à cause de son manque de petits ponts dans le cerveau. C'est ce qu'un médecin m'a raconté. Ça a l'air qu'il y a plein de petits ponts dans notre tête qui font passer les mots d'un endroit à l'autre. Ma mère, elle, quelques fois durant l'année, il lui en manque.~(\textit{B}:~91)
    \normalsize
  \end{singlespace}
\end{quote}
Exposant métaphoriquement son interprétation enfantine de la réalité adulte, Sissi démontre sa compréhension précise de la maladie mentale de sa mère.
Malgré son jeune âge, Sissi semble donc bien saisir la complexité de la vie des adultes qui l'entourent, monoparentalité et maladie mentale incluses.
\par
En plus de la compréhension strictement matérielle de certaines réalités, Sissi fait également montre d'une perspicacité quant il s'agit d'évaluer les motivations des agissements des adultes.
Fine observatrice, l'enfant décode rapidement la crainte de la travailleuse sociale: \guil{Vu le ton qu'elle emploie pour me parler, je sens qu'elle a peur que je me mette à crier $\left[ \dots \right]$.}~(\textit{B}:~96)
Sissi, pas dupe, comprend que les questions de la jeune femme ne sont pas anodines et visent avant tout à lui \guil{occuper l'esprit}~(\textit{B}:~95) afin de la distraire, relevant d'ailleurs à un moment qu'elle trouve \guil{louche} de ne pas avoir vu sa grand-mère depuis un moment~(\textit{B}:~97).
D'un naturel méfiant, elle ne croit donc pas tout ce qu'on lui dit, notamment lorsque les adultes lui ont raconté que son chien s'était sauvé pour retrouver sa maman.
Cynique, elle estime qu'il s'agissait là d'un pur mensonge: \guil{Façon détournée de me dire qu'ils l'avaient fait tuer à la SPCA $\left[ \dots \right]$.}~(\textit{B}:~118)
Sissi décèle souvent les intentions détournées des adultes, y compris celles de sa propre grand-mère: \guil{Là, ma grand-mère commence son baratin. Elle me tartine, elle me cuisine, si elle pouvait me faire cuire, elle le ferait. $\left[ \dots \right]$ Ma grand-mère veut me faire dire des choses, elle veut que je lui avoue des choses qu'elle a inventées.}~(\textit{B}:~130)
Comprenant dans une certaine mesure le \guil{code} des adultes, elle n'a pas, du moins pas entièrement, la naïveté que l'on suppose généralement aux enfants.
Bien que ne faisant pas explicitement partie du monde des adultes, elle n'y est pas tout à fait étrangère et y trouve une certaine aisance puisqu'elle en saisit une partie du fonctionnement.
\par
Outre sa bonne compréhension de certains agissements des adultes, Sissi est également au fait des désirs moins nobles qui motivent certains d'entre eux puisqu'elle y a été initiée dès son plus jeune âge par les histoires que lui racontait sa grand-mère: \guil{À quatre ans, je n'avais pas droit au croque-mitaine ou au Bonhomme Sept-Heures, mais au serial killer.}~(\textit{B}:~11)
Elle est donc conditionnée à craindre un peu tout, et surtout les hommes.
Par exemple, elle ne peut pas se rendre à l'école seule: \guil{Ma grand-mère dit que, parce que je suis petite, les vieux ivrognes de la taverne vont me rentrer de force dans les toilettes pour que je touche leur pipi, et après on ne va plus jamais me revoir\footcite[56]{Labreche2003}.}
Cette même grand-mère qui menaçait Sissi de l'abandonner lui fait miroiter le danger qu'elle se fasse enlever: \guil{Si on ne l'arrête pas, il va vous enlever, toi et ta mère, vous emmener très loin pour vous violer et vous tuer. Et personne ne va vous retrouver, puisqu'il va vous laisser dans un champ, la nuit, sans lumière.}~(\textit{B}:~133)
À entendre de telles histoires, Sissi en développe un vaste imaginaire quant à la méchanceté des adultes et les sévices qu'ils peuvent faire subir aux enfants.
La maltraitance est d'ailleurs une thématique qu'elle réutilise afin d'impressionner Céline lorsqu'elle évoque la possibilité d'être prise en charge par une famille d'accueil suite à la tentative de suicide de sa mère.
Fort bien informée sur la méchanceté potentielle des adultes à l'égard des enfants, Sissi dresse alors un portrait sans nuances des familles d'accueil:
\begin{quote}
  \begin{singlespace}
    \small
    -- Ma grand-mère m'a dit que si je tombe sur une bonne famille, ça peut aller. J'aurai plein de belles robes. Une limousine viendra me reconduire à l'école et j'aurai toutes les Barbies de la terre. Par contre, si je tombe sur une bande de salauds qui adoptent des enfants uniquement pour l'argent, ils vont me donner à manger des toasts pas de beurre. Et je porterai le vieux linge des autres enfants, tout usé, plein de trous. On me laissera me laver juste à l'eau froide et pas de savon. Peut-être aussi que le papa de la maison voudra s'amuser avec moi. \\
    -- S'amuser avec toi? \\
    -- Oui, tu sais\dots \ il va me montrer son machin\dots \ sa bitte. \\
    -- Oh! sa bitte\dots \\
    -- Il va vouloir que je la mette dans ma bouche, au complet dans ma petite bouche. Et lui, le gros dégueulasse, il va l'enfoncer si loin dans ma gorge qu'il va m'étouffer. Et moi, je ne pourrai pas me libérer parce qu'il va tenir ma tête solidement avec ses deux grosses mains sales aux ongles crottés. Comme il fera noir, il ne verra pas que je suis toute bleue, en train de mourir.~(\textit{B}:~38)
    \normalsize
  \end{singlespace}
\end{quote}
Dissertant ensuite sur ce qu'elle ferait subir à son agresseur -- \guil{la lui mordre, sa grosse bitte}, et \guil{même la lui arracher au complet} --, elle en conclut qu'elle devrait vivre recluse dans les bois, \guil{à cause des flics qui seront à [s]es trousses parce qu'[elle] aur[a] mutilé le papa de la famille}~(\textit{B}:~39).
Tout ça, bien évidemment, afin d'éviter la prison pour délinquants: \guil{Ma grand-mère m'a dit que c'est horrible ce qui se passe là, c'est pire que dans les familles d'accueil!}~(\textit{B}:~39)
Du test de résistance imposé par les gardiens aux petites filles, de l'appartenance des fillettes ayant réussi le test à certains de leurs gardiens, jusqu'à l'échec du test qui mène à l'éclatement du ventre qui libère par le nombril les objets qu'on y avait mis, Sissi répète ce que sa grand-mère lui a appris~(\textit{B}:~39-40).
Ainsi, du haut de ses onze ans, elle possède un imaginaire qui inclut déjà de nombreuses notions du monde adulte, quoique sa perception en soit exagérément cauchemardesque.
Elle a conscience du danger que représentent les adultes et vit dans une crainte immodérée des abus sexuels et physiques.


\subsection{Un enfant adaptatif: malgré tout, on survit}
Lorsque la psychologue demande à Émilie-Kiki, narratrice de \textit{La Brèche}, de parler de sa naissance ou plutôt de sa conception, la jeune femme expose et résume à la fois tout le drame de son enfance:
\begin{quote}
  \begin{singlespace}
    \small
    Un viol rue Wurtele, d'après ce qu'elle m'a raconté, ma mère folle à lier, un viol pour ne pas avoir à assumer l'envie de se faire tripoter, les vêtements déchirés, qu'elle m'a déjà dit, ma mère, une femme à qui la sensualité est refusée, qu'elle a essayé de me faire croire, ma mère. Et mes premiers mois, à moi! \textit{Ce n'est pas ma fille!} criait-elle sur tous les toits, ma mère. Petite puce en plein milieu d'une cuisine sale remplie de coquerelles, \textit{Ce n'est pas ma fille! C'est la fille d'une autre!} et ses pleurs à ma mère, ses longs pleurs comme des guimauves au-dessus d'un feu de tristesse \textit{CE N'EST PAS MA FILLE! J'AVAIS PRIS DES VALIUMS, CRISSE!}\footcite[38]{Labreche2008}
    \normalsize
  \end{singlespace}
\end{quote}
La monoparentalité, la pauvreté\footnote{La rue Wurtele est située dans le quartier Centre-Sud à Montréal, faubourg populaire abritant une population majoritairement défavorisée.}, la maladie mentale de la mère et son rejet de sa maternité: tout y passe!
Quelques lignes suffisent à mettre en place la tragédie des enfances racontées dans les romans de Marie-Sissi Labrèche.
\par
Si le seul rejet de Sissi par sa mère a déjà le potentiel de participer à son autonomisation et à sa transition accélérée vers l'âge adulte, son exposition à des événements dramatiques y contribue d'autant plus.
Ainsi, lorsque la narratrice de \textit{Borderline} a onze ans, sa mère tente de se suicider: \guil{Ma mère vient de se suicider. Elle a pris du Lithium Carbonate, des Luvox, des Dalmane et des Valium; toutes ses pilules en même temps. Puis, elle a crié: \textit{JE VOUS AIME TOUS!}}~(\textit{B}:~30)
Cynique, Sissi estime qu'il s'agit là d'une \guil{drôle de façon d'aimer le monde}~(\textit{B}:~30).
Réagissant de façon mature, elle compose le numéro d'urgence afin d'obtenir des secours, exposant ironiquement la situation dans toute sa complexité:
\begin{quote}
  \begin{singlespace}
    \small
    J'ai appelé à l'urgence de l'hôpital Notre-Dame. Je ne sais pas comment j'ai fait. Je ne me rappelle rien. En fait, je pense que c'est une autre petite fille qui l'a fait pour moi. Une autre petite fille blonde comme moi qui m'a souri et qui a pris sa main pour composer le numéro. Elle a parlé aussi: \textit{Bonjour, suis-je bien à l'hôpital Notre-Dame? Oui, bon. J'aurais besoin d'une ambulance $\left[ \dots \right]$.}~(\textit{B}:~31)
    \normalsize
  \end{singlespace}
\end{quote}
Derrière cet humour se cache cependant une détresse évoquée par l'amnésie et l'idée du dédoublement de la personnalité dans cette \guil{autre petite fille}.
Réexposant son intérêt pour la télévision, l'enfant démontre surtout sa volonté d'avoir une vie \guil{normale}, à tout le moins dépourvue de drames: \guil{Ensuite, la petite fille m'a dit: \textit{Viens, on va aller voir qui a gagné.} Alors moi et la petite fille, on a regardé la télé.}~(\textit{B}:~32)
Ainsi, si l'enfant réagit de façon mature et avec beaucoup de sang-froid, il n'en est pas moins perturbé par la tragédie qui se joue autour de lui: \guil{J'ai beaucoup de misère à me concentrer. Il y a comme une boule dans ma gorge, une boule qui est en train de devenir une pastèque tellement elle grossit. Je ne sais pas si j'ai envie de pleurer ou de vomir, tout est confus.}~(\textit{B}:~29-30)
Quand un policier parle à Sissi, cette dernière a \guil{envie de [s]e faire toute petite}~(\textit{B}:~32) afin qu'il l'avale, indiquant une volonté de disparition.
Point de départ de son autonomisation, le drame familial de Sissi est également une grande source de solitude:
\begin{quote}
  \begin{singlespace}
    \small
    Je ne sais pas si ma mère est morte. Je ne sais pas ce qui va m'arriver. $\left[ \dots \right]$ J'ai faim, mais je ne mangerai pas. Il n'y a personne pour me faire la bouffe, et de toute façon mon ventre est déjà trop rempli. Le vide m'habite. Il s'infiltre dans chacune de mes cellules à une vitesse vertigineuse $\left[ \dots \right]$. Je suis couchée par terre dans le salon, le plancher est froid et me glace le dos. Je m'en fous. Je ne me lèverai pas d'ici. Je n'ai plus envie de bouger. Le vide est tellement lourd.~(\textit{B}:~33)
    \normalsize
  \end{singlespace}
\end{quote}
Si la jeune fille comprend bien ce qui se passe, à savoir que sa mère a tenté de se suicider, elle est également placée dans un état d'ignorance puisqu'on ne l'informe pas de la suite des événements et qu'elle ne sait pas si sa mère est toujours vivante ou non.
D'abord autonomisée par la gravité intrinsèque des événements -- il est peu commun qu'une enfant de cet âge soit témoin d'une tentative de suicide --, Sissi est doublement laissée à elle-même par le fait que personne ne la prenne en charge pour lui offrir un soutien psychologique.
S'il est mention de son oncle, supposé venir la garder, et d'un policier qui lui \guil{dit des mots} qu'elle ne comprend pas~(\textit{B}:~32), elle est finalement laissée à elle-même: \guil{Moi, on m'a laissée là, seule.}~(\textit{B}:~33)
\par
Avec toute cette tragédie qui l'entoure, l'enfant semble parfois adopter une attitude d'ouverture et d'acceptation des événements.
Elle dit par exemple avoir hâte de raconter le suicide de sa mère: \guil{Un événement pareil, ça me donne de l'importance, ça fait de moi un point de mire.}~(\textit{B}:~37)
Comédienne, elle utilise ses talents de conteuse afin de bien divertir le public auquel elle fait son récit de l'histoire:  \guil{Pour l'annoncer à Céline, je prends mon air tragique. Je dis ça parce que j'ai l'impression de vivre dans un film.}~(\textit{B}:~37)
Cette fausse sérénité est cependant un simple mécanisme de protection servant à camoufler l'angoisse de l'enfant: \guil{C'est tellement gros ce qui arrive qu'il faut que je me force pour avoir l'air dedans. Quand des choses comme ça m'arrivent, je me divise en deux: une partie fait semblant, pendant que l'autre se cache et tremble.}~(\textit{B}:~37)
Cette scission de l'enfant en deux parties, l'une adulte et l'autre enfant, relève à notre avis de l'instinct de survie: la façade de calme a pour but d'éviter l'écroulement autrement inévitable.
\par
Ainsi, les caractéristiques de l'adulte présentes chez Sissi sont sollicitées et développées par le drame qui se joue autour d'elle; puis, elles sont exacerbées à des fins de protection de ses caractéristiques d'enfant.
Le développement de son autonomie sert alors à masquer la solitude qui lui est imposée tandis que sa méfiance à l'égard de l'autre tend à protéger ce qu'elle a de pureté face au monde des adultes.
Nous estimons que l'autonomisation de l'enfant chez Marie-Sissi Labrèche n'est pas exclusivement dûe à sa volonté, mais qu'elle est surtout induite par son environnement et le contexte de son existence.
Sissi n'est pas une enfant-adulte parce qu'elle apprécie la liberté et l'autonomie, mais plutôt parce que les événements qui forment sa vie relèvent à ce point de l'expérience adulte qu'elle en ressort mûrie voire flétrie.
Du point de vue strict du personnage, bien que Sissi soit assurément une enfant-femme, il ne s'agit pas pour elle d'un constat positif: elle appartient à ce type parce que sa vie est une succession de difficultés et que sa maturation se place dans la lignée d'une philosophie de simple survie.


\section{Retour à l'enfant-femme: Zazie et Sissi comme adultes involontaires}
Suite à nos lectures des romans \textit{Zazie dans le métro} et \textit{Borderline}, nous évoquons l'hypothèse qu'il n'y ait pas qu'une seule et unique façon de correspondre au modèle de l'enfant-femme.
Au-delà des évidentes distinctions de genre entre Zazie et Sissi -- la première est l'héroïne d'une parodie burlesque publiée à l'issue de la Seconde Guerre mondiale, tandis que la seconde est une des multiples présences d'une même narratrice dans un roman autofictionnel contemporain --, la différence la plus marquante au regard de notre analyse est dans la manière dont chacune combine les traits de l'enfance et de l'âge adulte: comme une aventure ou comme une épreuve, avec détachement ou dans la souffrance, avec facilité ou difficilement.
Ainsi, si Zazie et Sissi sont toutes les deux des enfant-adultes et des enfant-femmes, il demeure qu'elles ne vivent pas l'expérience de la même façon: l'une en sortira grandie -- \guil{J'ai vieilli}, dira-t-elle~(\textit{Z}:~181) -- et l'autre meurtrie: \guil{Crisse! Je n'ai même pas besoin de me grandir, malgré ma petite taille, j'ai déjà mille ans.}~(\textit{B}:~68)


\subsection{Zazie comme enfant-adulte absolu: le contre-mythe de Barthes}
Dans son court article \textit{Zazie et la littérature\footcite{Barthes1964}}, le sémiologue et critique français Roland Barthes présente Zazie comme personnage à vocation contre-mythique, à savoir qu'elle défait le mythe de la jeunesse de par son inadéquation à ce même mythe.
Outre son analyse détaillée sur le rapport zazique au langage, également de nature à déconstruire la mythologie et qui constitue un sujet qui en lui seul mérite d'être traité en profondeur, Barthes propose une lecture du personnage de Zazie sous l'angle de la subversion des âges.
\par
Zazie, comme personnage, pose dès le début la question de l'appartenance à un âge ou à un autre: \guil{De cette propension à mener, tout en déroutant, le monde adulte, Barthes élèvera Zazie au rang de contre-mythe en évoquant sa jeunesse comme une abstraction, puisqu'elle réunit l'enfance et la maturité, et en postulant que sa fonction est de \guill{dégonfler} la mythologie qui l'environne\footcite[88-89]{Maurin2007}.}
L'incapacité de Zazie d'être une enfant ou une adulte mène donc Barthes à lui accoler le statut de contre-mythe.
Sa réflexion repose sur le constat que Zazie est à la fois réaliste tout en étant impossible à imaginer dans la réalité, en se posant comme \guil{un être irréel, magique, faustien, puisqu'il est contraction surhumaine de l'enfance et de la maturité, du \guill{Je suis jeune, hors du monde des adultes} et du \guill{J'ai énormément vécu}\footcite[129]{Barthes1964}.}
\par
Dans ce sens, Zazie nous apparaît comme représentant l'enfant-adulte absolue, parfaite incarnation de cette créature qui joint l'enfance à l'âge adulte.
Le choix du terme \guil{enfant-adulte} au détriment d'\guil{enfant-femme} n'est pas ici anodin et soutient cette idée que Zazie n'est à proprement parler \guil{genrée} que par autrui et que cette assignation de genre que lui impose l'Autre ne la définit ni ne consacre son appartenance à un genre en particulier.
Zazie demeure donc, à notre avis, un personnage d'enfant bien avant d'en être un de femme, et est donc par excellence l'enfant-adulte.

\subsection{Sissi comme enfant-femme contextuelle: la résilience}
La situation ne se présente pas tout à fait de la même façon chez le personnage de Sissi, pour qui le jumelage de l'enfance et de l'âge adulte ne se fait pas aussi facilement.
Nous avançons l'hypothèse que Sissi n'est une enfant-adulte et une enfant-femme que par \textit{obligation}, à savoir que son autonomisation serait le résultat d'un déterminisme socio-familial et constituerait une forme de résilience.
\par
Désignant étymologiquement la résistance au choc des matériaux, le terme \guil{résilience} s'est avec le temps transporté dans les sciences sociales, pour y prendre une définition plus adaptée à la nature de ces disciplines: \guil{Capacité d’une personne ou d’un groupe à se développer bien, à continuer à se projeter dans l’avenir en dépit d’événements déstabilisants, de conditions de vie difficiles, de traumatismes sévères\footcite{Manciaux2001a}.}
Cette résilience se fonde sur plusieurs facteurs, lesquels se regroupent en trois grandes catégories: d'abord les facteurs individuels, comme l'autonomie, la capacité de distanciation ou la sociabilité; ensuite les facteurs familiaux, référant à la qualité des interactions avec la famille proche; enfin les facteurs sociaux, lesquels touchent au support extérieur à la famille, notamment à l'école et aux ressources communautaires\footcite{Anaut2005}.
Psychiatre et éthologue français ayant vulgarisé le concept de résilience, Boris Cyrulnik accorde une grande importance à l'environnement familial -- biologique ou substitutif -- dans la construction de la résilience chez l'enfant, allant même jusqu'à parler de \guil{tuteurs de résilience\footcite{Cyrulnik1998}} comme manifestations actives de ces facteurs familiaux et sociaux.
Dans ses recherches, lesquelles portaient initialement sur les enfants ayant vécu la guerre, Cyrulnik évoque le tempérament de l'enfant en tant que facteur individuel mais l'explique comme étant inévitablement forgé par la famille et l'entourage\footnote{Il évoque que l'aptitude à la résilience d'un individu s'acquiert par l'affection qu'il reçoit lorsqu'il est nouveau-né.}.
Ainsi, la personnalité résiliente ne peut qu'être acquise, puis développée.
En l'absence d'un environnement soutenant et d'un milieu affectif adéquat, l'enfant développe donc un tempérament résilient, lequel se manifeste par les attitudes de protection que sont la révolte, le rêve, la mégalomanie, le déni ou même l'humour\footcite[3]{Taubes2001}.
\par
C'est à la lumière de ces théories sur la résilience que nous parvenons à établir la véritable nature de l'enfant-femme qu'incarne Sissi.
Nous estimons que la coexistence chez elle de caractéristiques adultes et enfantines relève du développement d'une personnalité résiliente.
Nous retrouvons d'ailleurs chez Sissi plusieurs de ces éléments constitutifs du tempérament résilient évoqués par Cyrulnik: elle se révolte contre l'autorité notamment en l'insultant, rêve d'une vraie famille, fait le projet plutôt ambitieux d'un \guil{come-back démentiel}~(\textit{B}:~129), nie être une victime et formule un grand nombre de jeux de langage.
Sissi emploie ces mécanismes de protection et parvient à se développer dans une relative normalité malgré tous les événements perturbants qui marquent son enfance.
Puisque ni sa mère -- folle -- ni sa grand-mère -- contrôlante -- ne lui offrent un véritable milieu affectif, nous croyons que sa résilience relève quasi-exclusivement de facteurs individuels, quoique certains adultes de son environnement scolaire, notamment son professeur d'éducation physique, puissent être considérés comme un élément extérieur propice au développement de la résilience.
\par
Puisqu'il s'agit d'une réaction face à un traumatisme, ce qui implique forcément une certaine forme de souffrance, l'acquisition du statut d'enfant-femme n'est pas, pour Sissi, une chose naturelle qui \guil{va de soi}.
Si cette résilience et la grande adaptabilité qui en découle sont pour nous des facteurs qui confirment le statut d'enfant-adulte de Sissi, il est indéniable qu'elle appartient au modèle d'abord et avant tout parce que son environnement familial défavorable lui a imposé un parcours de vie difficile.
Elle n'est donc une enfant-femme que parce que son contexte de vie le requiert, en ce sens que son autonomie et sa maturité ne sont pas le fruit de simples prédispositions, mais bien des caractéristiques développées face à divers facteurs familiaux, économiques et sociaux.
\par
Enfin, tout comme Zazie, il est clair pour nous que Sissi n'a de \guil{femme} que l'étiquette qui lui est donnée par l'Autre: bien qu'elle se perçoive \guil{fille} -- à savoir de genre féminin -- dans la crainte des hommes que lui a inculquée sa grand-mère, elle n'a d'\guil{adulte} que quelques caractéristiques induites par des facteurs externes.
Ainsi, Sissi se positionne selon nous comme une enfant-femme contextuelle, dont l'adéquation au modèle est conditionnelle à ce qu'elle évolue dans un certain milieu.
