\chapter*{Conclusion}
\addcontentsline{toc}{chapter}{Conclusion}

\begin{flushright}
                                    \begin{singlespace}
                                    \epigraph{
                                    Regardez tous les délais comme des avantages: c’est gagner beaucoup que d’avancer vers le terme sans rien perdre; laissez mûrir l’enfance dans les enfants. Enfin, quelque leçon leur devient-elle nécessaire? gardez-vous de la donner aujourd’hui, si vous pouvez différer jusqu’à demain sans danger.}
                                    \par
                                    Jean-Jacques Rousseau, \textit{Émile, ou De l'éducation}\footcite[441]{Rousseau1852}
                                    \end{singlespace}
\end{flushright}

Enfant-femme ou enfant-adulte, mais fondamentalement enfant avant tout.
Nous avons examiné comment s'entrecroisaient l'enfance et l'âge adulte dans \textit{Zazie dans le métro} de Raymond Queneau et \textit{Borderline} de Marie-Sissi Labrèche, mais avons-nous véritablement étudié l'âge adulte?
N'avons-nous pas étudié que l'enfance, en réalité?
Il nous semble que ces apparitions de traits adultes touchent encore au sujet de l'enfance en ce qu'ils la froissent, l'abîment et la marquent à jamais.
Autonomisée par les événements, l'enfant-femme s'acclimate en observant et en analysant les adultes, pour ensuite adopter un comportement qui les imite ou qui les offusque.
Les réalités sont donc les mêmes pour Zazie et pour Sissi: elles sont toutes deux plongées dans un monde peu fiable, entourées d'adultes sur lesquels elles ne peuvent pas compter.
Toutefois, les enjeux diffèrent: alors que Zazie livre un combat ouvert contre les adultes, Sissi tente avant tout de survivre et de se protéger du monde adulte.
Ces différences sont visibles autant sur le plan du caractère que celui du langage: l'une est brutale tandis que l'autre est résiliente; l'une manie l'arme langagière dans un \guil{combat de brousse}, l'autre n'usant de cette arme-parole que pour se protéger.
\par
Avec notre typologie, nous avons tenté de présenter un type d'enfant plutôt moderne, que nous estimons être le résultat d'une société et d'une époque données.
Selon Denise Lemieux, qui a étudié l'enfant dans la littérature québécoise, la littérature sur l'enfance est \guil{étroitement lié[e] aux transformations du statut de l'enfant dans les sociétés en voie de modernisation\footcite[11]{Lemieux1984}.}
Ainsi, le personnage enfantin est tributaire de \guil{l'avènement de modes de vie où l'enfant est de plus en plus retiré de la société des adultes\footcite[9]{Lemieux1984}}, alors que l'on ne s'efforce plus autant de l'intégrer à la vie quotidienne des adultes\footcite[9]{Lemieux1984}.
C'est un constat adéquat en ce qui concerne Zazie et Sissi, lesquelles vivent certes dans le même monde que les adultes, mais ne vivent pas \textit{avec} eux.
\par
Conciliation travail-famille ardue, parents et enseignants débordés, multiplication des problèmes de santé mentale infantiles et ressources insuffisantes: si la négligence envers les enfants est loin d'être un fait nouveau, elle est désormais plus insidieuse et souvent involontaire, en ne produisant pas des petits martyrs aux blessures visibles mais plutôt des enfants carencés dont les stigmates sont toutefois tout aussi graves.
D'ailleurs, comme l'avait annoncé Brigitte Seyfrid-Bommertz il y a plus de vingt ans, l'enfance en littérature ne va pas en se simplifiant:
\begin{quote}
  \begin{singlespace}
    \small
    De façon générale, il semble que plus on s'avance vers la période contemporaine, plus le profil passionnel de l'enfant se fait complexe, déroutant, voire inquiétant. L'enfant acquiert de multiples visages, se mue en un être polymorphe plus difficile à saisir. [...] L'enfant est aussi saisi à travers ses zones d'ombre, son côté satanique, ou encore il se fait étrange, irréel, fantastique, être incompréhensible sur lequel on n'a plus de prise \footcite[20]{Seyfrid-Bommertz1999}.
    \normalsize
  \end{singlespace}
\end{quote}
Plusieurs fictions de l'extrême contemporain se rattachent aisément à cette complexification de l'enfant en littérature\footnote{Nous pensons notamment aux romans \textit{Et au pire, on se mariera} (2011), \textit{Chercher Sam} (2014) et \textit{Autour d'elle} (2016) de Sophie Bienvenu, \textit{La déesse des mouches à feu} (2014) de Geneviève Pettersen, \textit{À l'abri des hommes et des choses} (2016) de Stéphanie Boulay, ainsi qu'à la trilogie \textit{La bête à sa mère} (2015), \textit{La bête et sa cage} (2016) et \textit{Abattre la bête} (2017) de David Goudreault.}, alors que les jeunes personnages semblent de moins en moins \textit{aimés}\footnote{En ce sens que les parents, bien qu'ils comblent leurs besoins physiques, ne comblent pas leurs besoins affectifs.} et qu'ils deviennent de plus en plus violents, de plus en plus troublés.
Ils sont perturbés, mais également perturbants et percutants, et leur rejet de ce monde adulte peu accueillant n'est pas sans nous rappeler les \textit{prisons nostalgiques}, que l'on rattache aux enfances anglaises de \textit{Peter Pan}\footcite{Barrie1987} et d'\textit{Alice au pays des merveilles}\footcite{Carroll2012} et qui correspondent à un détachement et à un retrait volontaires du monde adulte.
Ainsi, la nostalgie s'installe du fait de l'échec ou du refus volontaire d'arriver à un état d'harmonie quant aux réalités culturelles de l'époque\footcite[241]{Coveney1967}.
Tandis qu'Alice s'acclimate quelque peu à la folie inhérente au Pays des Merveilles et en ressort \textit{grandie} ou \textit{vieillie} -- d'où l'idée que l'enfant-femme en constitue une actualisation --, Peter Pan refuse catégoriquement de vieillir et s'emmure dans le monde de l'enfance.
Si la prophétie de Brigitte Seyfrid-Bommertz s'avère vraie, et que cette complexification de l'enfant passe également par son éloignement et son rejet de l'adulte, l'enfant contemporain ne peut que se rapprocher de la figure de Peter Pan.
Ce déplacement de la problématique de l'enfance s'accompagne aussi d'un changement de paradigme: alors qu'Alice est \textit{tombée} dans le terrier et qu'elle se retrouve bien involontairement au Pays des Merveilles, Peter Pan se rend, volontairement et consciemment, au Pays Imaginaire\footnote{Aussi appelé Pays de Nulle part ou Pays du Jamais, d'après son nom anglais \textit{Neverland}. Il s'agit d'un lieu qui n'est pas soumis au temps et où le jeu est la seule activité possible.}.
D'Alice, Zazie et Sissi qui subissent le monde adulte et tentent d'y trouver leurs repères, nous passons ainsi à Peter Pan et aux enfants contemporains qui rejettent le monde adulte afin de créer leur propre univers. \guil{[L]aissez mûrir l'enfance dans les enfants}, disait Jean-Jacques Rousseau; ce à quoi nous aimerions ajouter: \guil{aimez les enfants, aimez-les bien et assez}.
